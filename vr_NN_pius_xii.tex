% !TEX TS-program = LuaLaTeX+se

\documentclass[vesperale_romanum.tex]{subfiles}

\ifcsname preamble@file\endcsname
  \setcounter{page}{\getpagerefnumber{M-vr_NN_1960}}
\fi

%%this code when \customsubfiles is used should allow for continuous pagination when subfiles are compiled individually.

\begin{document}

\chapter[Additiones et Variationes.]{ADDITIONES ET VARIATIONES.}\header{additiones et variationes.}
\thispagestyle{empty}

\section[Commune Summorum Pontificum.]{COMMUNE UNUS AUT PLURIUM SUMMORUM PONTIFICUM.}

\rubrique{Omnia de Communi unius aut plurimorum Martyrum vel Confessoris Pontificis, juxta qualitatem festi, præter Orationem \normaltext{Gregem tuum,} ut infra.}\

\oratio

\lettrine{G}{r}egem tuum, Pastor ætérne, placátus inténde:~† et per beátum \textit{N.} (Mártyrem tuum atque) Summum Pontíficem, perpétua protectióne custódi;~* quem totíus Ecclésiæ præstitísti esse pastórem. Per Dóminum.

\smalltitle{Pro pluribus Pontificibus.}

\lettrine{G}{r}egem tuum, Pastor ætérne, placátus inténde:~† et per beátos \textit{N.} et \textit{N.} (Mártyres tuos atque) Summos Pontífices, perpétua protectióne custódi;~* quem totíus Ecclésiæ præstitísti esse pastóres. Per Dóminum.

\newpage

\litdate{31 Maii}

\festum{B. Mariæ Virginis Reginæ.} %%need a separate  style for Doubles II class…

\rank{Duplex II classis.}

\invesperis{i}

\rubrique{Antiphonæ et Psalmi ut in Festis B.M.V., \normaltext{\pageref{Festis_BVM},} præter sequentia:}

\capitulum
\scripture{Eccli. 24, 5 et 7.}

\lettrine{E}{go} ex ore Altíssimi prodívi, primogénita ante omnem creatúram:~† ego in altíssimis habitávi,~* et thronus meus in colúmna nubis.

\rubrique{Hymnus \normaltext{Ave maris stella.}} %% need to add pageref for hymn

\vv Salve, Regina misericordiae. \tpalleluia

\rr Ex qua natus est Christus, Rex noster. \tpalleluia

\admagnificat
\gscore[]{8. G}{an_beata_quae_credidisti_solesmes_1961}

\oratio
\lettrine{C}{o}ncéde nobis, quǽsumus, Dómine:~† ut, qui solemnitátem beátæ Maríæ Vírginis Regínæ nostræ celebrámus;~* ejus muníti præsídio, pacem in præsénti et glóriam in futúro cónsequi mereámur. Per Dóminum.

\invesperis{ii}

\rubrique{Omnia ut in I Vesperis præter sequentia:}

\vv María Virgo cælos ascéndit. \tpalleluia

\rr Cum Christo regnat in ætérnum. \tpalleluia

\admagnificat
\gscore[]{2. D}{an_beata_mater_regina_mundi_intercede_solesmes_1961}

\litdate{15 Augusti}

\festum{IN ASSUMPTIONE B.M.V.}
\rank{Duplex I classis cum Octava communi.}

\rubrique{Omnia ut in die, præter sequentia:}

\smalltitle{Capitulum.}
\scripture{Judith 13, 22–23.}

\lettrine{B}{e}nedíxit te Dóminus in virtúte sua, quia per te ad níhilum redégit inimícos nostros.~† Benedícta es tu, fília, a Dómino Deo excélso,~* præ ómnibus muliéribus super terram.

\rubrique{(In I Vesperis tantum.)}

\hymnus

\gscore[]{2.}{hy_o_prima_virgo_prodita_solesmes_1961}

\vv Exaltáta est sancta Dei Génitrix.

\rr Super choros Angelórum ad cæléstia regna.

\oratio

\lettrine{O}{m}nípotens sempitérne Deus, qui Immaculátam Vírginem Maríam, Fílii tui Genetrícem, córpore et ánima ad cæléstem glóriam assumpsísti:~† concéde, quǽsumus; ut ad supérna semper inténti,~* ipsíus glóriæ mereámur esse consórtes. Per eúmdem Dóminum.
 
\litdate{22 Augusti}

\festum{Festum Immaculati Cordis B.M.V.}
\rank{Duplex II classis.}
 
\rubrique{Omnia ut in Festis B.M.V., \normaltext{\pageref{Festis_BVM},} præter sequentia:}

\subsection{in i vesperis.}

\vv Dignáre me laudáre te Virgo sacráta. %% can replace with a file since this appears so many times…

\noindent \rr Da mihi virtútem contra hostes tuos.

\smalltitle{Ad Magnificat, Antiphona.}
\gscore[]{1. g}{an_exsultavit_cor_meum_solesmes_1961}

\oratio
\lettrine{O}{m}nípotens sempitérne Deus, qui in Corde beátæ Maríæ Vírginis dignum Spíritus Sancti habitáculum præparásti:~† concéde propítius; ut eiúsdem immaculáti Cordis festivitátem devóta mente recoléntes,~* secúndum Cor tuum vívere valeámus. Per Dóminum.

\rubrique{Comm. præced.} %%not in 1960, hence change to Pius XII

\invesperis{ii}

\rubrique{Omnia ut in I Vesperis. Comm. sequentis.}

\litdate{3 Septembris} 

\bigtitle{S. Pii X, Papæ.}

\rank{Duplex. \mtv} 

\oratio
\lettrine{D}{eus}, qui ad tuéndam cathólicam fidem, et univérsa in Christo instauránda sanctum Pium, Summum Pontíficem, cælésti sapiéntia et apostólica fortitúdine replevísti:~† concéde propítius; ut, ejus institúta et exémpla sectántes,~* prǽmia consequámur ætérna.
Per eúmdem Dóminum nostrum.


\end{document}