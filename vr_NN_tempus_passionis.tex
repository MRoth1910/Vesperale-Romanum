% !TEX TS-program = lualatexmk
% !TEX parameter =  --shell-escape

\documentclass[vesperale_romanum.tex]{subfiles}

\ifcsname preamble@file\endcsname
  \setcounter{page}{\getpagerefnumber{M-vr_NN_tempus_passionis}}
\fi

%%this code when \customsubfiles is used should allow for continuous pagination when subfiles are compiled individually.

\begin{document}

\thispagestyle{empty}

%%header and sectioning commands to be determined

\bigtitle{Sabbato ante Dominicam Passionis.}

\capitulum{Heb. 9, 11 – 12}

\lettrine{F}{r}atres: Christus assístens Póntifex futurórum bonórum, per ámplius et perféctius tabernáculum non manu factum, id est, non huius creatiónis:~† neque per sánguinem hircórum aut vitulórum, sed per próprium sánguinem introívit semel in Sancta,~* ætérna redemptióne invénta.

\hymnus \label{vexilla_regis_passionis}%%getting these references right is a problem

\gscore[]{1.}{hy_vexilla_regis_prodeunt_solesmes} %%italicized a for elision is too close to d for my liking

\vv Eripe me Dómine ab hómine malo.\label{vv_passionis}

\rr A viro iníquo éripe me.

\gscore[Ad Magnif.]{Ant. 8. G}{an_ego_sum_qui_testimonium_solesmes_1961}

\oratio

\lettrine[depth=1]{Q}{u}ǽsumus omnípotens Deus, famíliam tuam propítius réspice:~† ut te largiénte, regátur in córpore;~* et te servánte, custodiátur in mente. Per Dóminum. %%may want to adjust Q with package option depth default=1, must be an integer

\rubrique{Ab his Vesperis usque ad Festum Ss. Trinitatis inclusive, omittitur Commemoratio seu Suffragium de Omnibus Sanctis.}

\hymnusadcompletorium

\gscore[]{2.}{hy_te_lucis_ante_terminum_passiontide_solesmes_1961}

\rubrique{Hic tonus servatur ad Completorium usque ad Feriam V in Cœna Domini, etiam in Festis Sanctorum, nisi aliter notetur.}

\rubrique{Ab hac die usque ad Feriam V in Cœna Domini, ad Cornpletorium quando fit Officium de Ternpore in ℟. brevi non dicitur \normaltext{℣. Glória Patri,} sed ejus loco statim repetitur \normaltext{℟. In manus tuas,} etc.}

\bigtitle{Dominica Passionis.}

\rubrique{Antiphonæ et Psalmi de Dominica. Capitulum \normaltext{Fratres: Christus assístens} ut supra. Hymnus \normaltext{Vexílla regis, \pageref{vexilla_regis_passionis}.}  \normaltext{\vv Eripe me, \pageref{vv_passionis}.}} %%supra probably needs pageref

\gscore[Ad Magnif.]{Ant. 2. D}{an_abraham_pater_vester_solesmes_1961} %%Solesmes uses Ad Magnificat, Antiphona. for Sat/Sunday.

\bigtitle{Feria II.}

\rubrique{Antiphonæ et Psalmi de Psalterio.}

\capitulum{Jer. 11, 20}

\lettrine{T}{u} autem, Dómine Sábaoth, qui júdicas juste, et probas renes et corda,~† vídeam ultiónem tuam ex eis:~* tibi enim revelávi causam meam, Dómine Deus meus.

\gscore[Ad Magnif.]{Ant. 4. A*}{an_si_quis_sitit_solesmes_1961}

\oratio

\lettrine{D}{a} quǽsumus Dómine, pópulo tuo salútem mentis et córporis:~* ut bonis opéribus inhæréndo, tua semper mereátur prote\-ctióne deféndi.
Per Dóminum.

\rubrique{Capitulum prædictum dicitur in feriali Officio usque ad Feriam IV Majoris Hebdomadæ. Item Hymnus et \vv}

\bigtitle{Feria III.}

\gscore[Ad Magnif.]{Ant. 1. D2}{an_vos_ascendite_solesmes_1961}

\oratio

\lettrine{D}{a} quǽsumus Dómine, perseverántem in tua voluntáte famulátum:~* ut in diébus nostris, et mérito et número pópulus tibi sérviens augeátur.
Per Dóminum.

\bigtitle{Feria IV.}

\gscore[Ad Magnif.]{Ant. 4. A*}{an_multa_bona_opera_solesmes_1961}

\oratio

\lettrine{A}{d}ésto supplicatiónibus nostris, omnípotens Deus:~* et quibus fidúciam sperándæ pietátis indúlges, consuétæ misericórdiæ tríbue benígnus effé\-ctum. Per Dóminum.

\bigtitle{Feria V.}

\gscore[Ad Magnif.]{Ant. 4. A*}{an_desiderio_desideravi_solesmes_1961}

\oratio

\lettrine{E}{s}to, quǽsumus Domine, propitius plebi tuæ:~* ut quæ tibi non placent respuentes, tuorum potius repleantur delectationibus mandatorum. Per Dóminum.

\bigtitle{Feria VI.}

\gscore[Ad Magnif.]{Ant. 1. g}{an_principes_consilium_solesmes_1961}

\textes{versiculus_passionis}

\oratio

\lettrine{C}{o}ncéde, quǽsumus omnipotens Deus:~* ut qui protectionis tuæ gratiam quærimus, liberati a malis omnibus, secura tibi mente serviamus.
Per Dóminum.

\bigtitle{Sabbato ante Dominicam in Palmis.} %%neither antiphonal gives this title, just Sabbato, but it's consistent

\capitulum{Phil. 2, 5 – 7}

\lettrine{F}{r}atres: Hoc enim sentíte in vobis, quod et in Christo Jesu: qui, cum in forma Dei esset, non rapínam arbitrátus est esse se æquálem Deo:~† sed semetípsum exinanívit, formam servi accípiens, in similitúdinem hóminum factus,~* et hábitu invéntus ut homo.

\rubrique{Hymnus \normaltext{Vexílla regis, \pageref{vexilla_regis_passionis}.}  \normaltext{\vv Eripe me, \pageref{vv_passionis}.}}

\gscore[Ad Magnif.]{Ant. 4. E}{an_pater_juste_solesmes_1961} %%Solesmes uses Ad Magnificat, Antiphona. for Sat/Sunday.

\oratio

\lettrine{O}{m}nípotens sempitérne Deus, qui humáno géneri, ad imitándum humilitátis exémplum, Salvatórem nostrum carnem súmere, et crucem subíre fecísti:~† concéde propítius; ut et patiéntiæ ipsíus habére documénta,*  et resurrectiónis consórtia mereámur. Per eúmdem Dóminum nostrum.

\bigtitle{Dominica in Palmis.}

\rubrique{Antiphonæ et Psalmi de Dominica. Capitulum \normaltext{Fratres: Hoc enim sentíte} ut supra.}

\rubrique{Hymnus \normaltext{Vexílla regis, \pageref{vexilla_regis_passionis}.}}

\textes{versiculus_passionis}

\gscore[Ad Magnif.]{Ant. 8. G*}{an_scriptum_est_enim__percutiam_solesmes_1961.1} %%Solesmes uses Ad Magnificat, Antiphona. for Sat/Sunday.

\rubrique{Ab hac die usque ad Octavam Paschæ, si occurrat aliquod Festum IX Lectionum quod transferri valeat, transfertur post Octavam; secus de eo fit commemoratio, præterquam tribus diebus Pascha præcedentibus et duobus consequentibus. Festi III Lectionum item a Feria V in Cœna Domini usque ad Feriam III Pascha, nulla fit Commemoratio.}

\bigtitle{Feria II Majoris Hebdomadæ.}

\gscore[Ad Magnif.]{Ant. 4. A*}{an_non_haberes_in_me_solesmes}

\oratio

\lettrine{A}{d}juva nos Deus salutáris noster:~* et ad benefícia recolénda quibus nos instauráre dignátus es, tríbue veníre gaudéntes.
Per Dóminum

\bigtitle{Feria III Majoris Hebdomadæ.}

\gscore[Ad Magnif.]{Ant. 4. A*}{an_potestatem_habeo_solesmes}

\oratio

\lettrine{T}{u}a nos misericórdia, Deus, et ab omni subreptióne vetustátis expúrget:~* et capáces sanctæ novitátis effíciat. Per Dóminum.

\bigtitle{Feria IV Majoris Hebdomadæ.}

\gscore[Ad Magnif.]{Ant. 1. g}{an_ancilla_dixit_petro_solesmes}

\oratio

\lettrine[findent=-0.3pt]{R}{e}spice, quǽsumus, Dómine, super hanc famíliam tuam,~* pro qua Dóminus noster Jesus Christus non dubitávit mánibus tradi nocéntium, et crucis subíre torméntum: Qui tecum vivit et regnat. %%no asterisk and with silent conclusion during Triduum.

%%Tenebrae book has package option [finindent=-0.3pt]

\rubrique{Tribus sequentibus diebus, dicto secreto \normaltext{Pater noster et Ave María,} omnibus aliis prætermissis, absolute incipitur Oflicium ad Vesperas ab Antiphona\footnote{Juxta morem hodiernum Vesperæ hodie et die sequenti dicuntur sine cantu. Attamen hic ponuntur Antiphonæ cum cantu proprio, ne pereat; et in gratiam Ecclesiarum quæ morem cantandi olim ubique observatum adhuc retinent.} primi Psalmi; et Antiphonæ duplicantur sicut in Festi Duplici.}

\rubrique{In fine Psalmorum, non dicitur \normaltext{Glória Patri.}}

\bigtitle{Feria V in Cœna Domini.}

\gscore[1 Ant.]{2. D}{an_calicem_salutaris_et_nomen_solesmes}%% Vaticana and LA have EUOUAE but this is confusing as GP isn't sung during Triduum.

\psalmus{115}{115_2}{115_2}

\gscore[2. Ant.]{8. G}{an_cum_his_qui_oderunt_solesmes_1961}

\psalmus{119}{119_8G}{119_8}

\gscore[3. Ant.]{8. G}{an_ab_hominibus_iniquis_solesmes}

\psalmus{139}{139_8G}{139_8}

\gscore[4. Ant.]{7. a}{an_custodi_me_a_laqueo_solesmes_1961}

\psalmus{140}{140_7a}{140_7}

\gscore[5. Ant.]{7. a}{an_considerabam_ad_dexteram_solesmes_1961}

\psalmus{141}{141_7a}{141_7}
%
\gscore[Ad Magnif.]{Ant. 1. f}{an_coenantibus_autem_solesmes} %

\rubrique{Capitulum, Hymnus et \vv non dicuntur.}

\rubrique{Sine cantu} \vv Christus factus est pro nobis obédiens usque ad mortem.

\rubrique{Secunda die additur:} Mortem autem crucis.

\rubrique{Cum incipitur ℣. \normaltext{Christus factus est} omnes genuflectunt, et eo finito, dicitur \normaltext{Pater noster} totum sub silentio. Postea Ps. \normaltext{Miserére} aliquantulum altius.}

%%Miserére has to go somewhere (perhaps just the office of the dead? with pageref) including it here is one possibility but the line length makes it awkward

%\textes{miserere_no_bold}

\rubrique{Quo finito, sine \normaltext{Orémus} dicitur simili voce:}

\lettrine[findent=-0.3pt]{R}{e}spice, quǽsumus, Dómine, super hanc famíliam tuam, pro qua Dóminus noster Jesus Christus non dubitávit mánibus tradi nocéntium, et crucis subíre torméntum:

\rubrique{Et sub silentio concluditur:}
 
Qui tecum vivit et regnat in unitáte Spíritus Sancti, Deus, per ómnia sǽcula sæculórum. %%lifted from the LU not the Vaticana or LA

Ad Completorium. \rubrique{Non dicitur \normaltext{Jube domne,} nec Lectio brevis, nec \normaltext{Adjutórium,} neque Oratio Dominica, sed facta Confessione et Absolutione, incipitur a Ps. \normaltext{Cum invocárem;} et dicuntur Psalmi de Dominica. Dictis Psalmis, dicitur Nunc dimíttis. Deinde \normaltext{\vv Christus factus est} ut supra.}

%\oratio
%
%\lettrine{P}{o}puli tui Deus institútor et rector, peccáta quibus impugnátur, expélle:~* ut semper tibi plácitus, et tuo munímine sit secúrus. Per Dóminum.
%
\bigtitle{Feria VI in Parasceve.}

\rubrique{Antiphonæ et Psalmi præteriti diei.}
%
\gscore[Ad Magnif.]{Ant. 1. f}{an_cum_accepisset_acetum_solesmes}

\bigtitle{Sabbato Sancto.}%% these files should be moved from Easter and the file should conclude after Compline. %% see LA rubric

\rubrique{Vesperas hujus diei ut in Graduali.} %%Solesmes LA

%\gscore[Ant.]{6. f}{an_alleluia_paschal_vigil_vespers_solesmes_1961}
%
%\rubrique{Et repetitur antiphona \normaltext{Allelúia, Allelúia, Allelúia.}}
\bigtitle{Ad Completorium.}

\rubrique{Jube domne benedícere. Lectio brevis \normaltext{Fratres: Sóbrii estóte. \vv Adjutórium. Pater noster.} Et facta Confessione et Absolutione dicitur \normaltext{\vv Convérte nos. \vv Deus in adjutórium. Glória Patri.} Et amplius non dicitur \normaltext{Laus tibi Dómine Rex ætérne glóriæ,} sed ejus loco deinceps dicitur \normaltext{Allelúia.}}

\rubrique{Deinde sine Antiphona dicentur Psalmi de Dominica, in tono sequenti:}

\psalmus{4}{ps_4_ad_compl_in_sab_san}{ps_4_ad_compl_in_sab_san}

\psalmus{90}{ps_90_ad_compl_in_sab_san}{ps_90_ad_compl_in_sab_san}

\psalmus{133}{ps_133_ad_compl_in_sab_san}{ps_133_ad_compl_in_sab_san}

\rubrique{Hymnus, Capitulum, \rr breve et \vv non dicuntur.}

\smalltitle{Ad Nunc Dimittis, Antiphona.}

\gscore[]{8. G}{an_vespere_autem_sabbati_solesmes}

\canticum{Simeonis}{Luc. 2, 29 – 32}{Nunc_dimittis_8G_c3.gabc}{3}{Nunc_dimittis_8G}

\rubrique{\vv \normaltext{Dóminus vobíscum}. Oratio \normaltext{Vísita} et reliqua ut in Psalterio.}

\rubrique{In fine, Antiphona \normaltext{Regína cæli.}}

\biggerrule

\end{document}