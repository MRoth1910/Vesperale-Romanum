
\bigtitle{Preface to the \textit{Vesperale Romanum.}}

\begin{enpars}
The \textit{Vesperale Romanum} is the book containing the chants and texts of the evening offices, that is, of Vespers and of Compline. The Vatican commission established by Pope Saint Pius \textsc{x} to create a unified edition of chant for the entire Roman church published the official edition of the \textit{Antiphonale Romanum} for the day hours, excluding Matins, in 1912; a vesperale followed in 1913. This edition however suffers from some problems, namely that the psalter of 1911 is included, but the texts, calendar, and rubrics are still those of the office of Saint Pius \textsc{v} as amended; it does not therefore conform to the \textit{editio typica} issued by Saint Pius \textsc{x}.

The monks of the abbey of Saint-Pierre de Solesmes, whose research of medieval manuscripts led to the creation of the Vatican commission in the first place, published their own vesperale, including the rhythmic signs favored by Dom André Mocquereau, longtime choirmaster of the abbey, but this edition was apparently less popular than other Solesmes editions, namely the \textit{Liber Usualis} which contains the chant for Sundays, feasts, and Holy Week, and the offices of the Sacred Triduum, but for not ferial days otherwise.

\smalltitle{Why a new vesperale?}

Above all, this new edition is intended as a replacement for print copies, given that neither the Vatican nor Solesmes editions of the full antiphonal nor the vesperale are in print. Further, we take advantage of the Gregorio software package, which allows for computer typesetting of Gregorian chant via LuaLaTeX; this is superior to relying on scans, since even if a high-quality scan were available, the quality of the printed score would deterioate over time, which is the case with the republished editions of chant, where the neumes such as the \textit{quilisma} or the rhythmic signs, notably the \textit{ictus} or \textit{vertical episema,} are not always identifiable.

However, this is not simply a reproduction. We aim to improve upon the traditional books to make the book as easy to use as possible, although a certain familiarity with the office is required due to the nature of a printed book.

\smalltitle{Editorial Principles.}

The vesperale project follows at its heart the office before the changes of Venerable Pope Pius \scspace{xii}, while allowing nevertheless for some discretion, e.g. we include Ss.\@ Anthony and Lawrence Brindisi among the Doctors; one is thus free to choose the antiphon for a Doctor or the antiphon from the respective common, according to the rubrics.

The Common of Popes and the new offices of the Queenship of the Blessed Virgin Mary (May 31), of the Assumption (August 15), and of the Immaculate Heart Blessed Virgin Mary (August 22), introduced by Pope Pius \scspace{xii} before 1954 (obligatory as of 1955), of Saint Joseph the Worker (May 1) and Vespers of Holy Saturday introduced in 1955 to be celebrated in the spring 1956, and four minor feasts introduced by Pope Pius \scspace{xii} and Saint John \scspace{xxiii} are included in an appendix. Note, however, that the office of the Queenship of the Blessed Virgin Mary was only published in May 1955 and does not follow the Divino Afflatu rubrics, already with simplified rubrics that were used universally from January 1956 to July 1960.

For the most part, it is possible to use the \textit{Divino Afflatu} offices for the 1960 office, which is already the case for most people who must use a vintage \textit{Liber Usualis} published before 1960. There are two unusual cases: the feast of Saint Lawrence of Brindisi was added to the general calendar on July 21 after he was named a Doctor of the Church in 1960. The feast already on the calendar is commemorated at Lauds and low Masses.

This aims to be a practical book to be put out as soon as possible, and we do not include any melodic restitutions, which also avoids upseting the faithful. Therefore, while we include the ancient hymns in the appendix, due to great demand and interest, which is a decision that follows the Vatican Edition, we will keep mode 3 psalms with a dominant of Do, the newer mode 6 tone, the psalm termini and mode assignments as in the Vatican Edition and in the Solesmes editions. However, we deviate slightly because custom has changed with respect to endings for which this character * indicates an alternative of a podatus (4A* and 8G*): the endings will be given at the end of the first verse. This is a change even from Solesmes practice coming to us from Dom Suñol and his teachers: the podatus was to be sung only at the last two verses.

Since the psalm ending is provided, EUOUAE is not included for most psalm antiphons. But as this is a practical edition, the \textit{Magnificat} will be printed in one block by tone as in the \textit{Liber Usualis,} and so EUOUAE will be kept in the antiphons. It is also kept in some cases where a page turn is required.

%%do we really want to do this? est-cequ'on veut vraiment faire ça ?

The rhythmic signs are retained. As the scores from the Solesmes editions are already available, little editorial intervention is needed to make the scores usable for typesetting. Further, given the number of chanters who still use even the \textit{ictus,}  it is better to abstain from making more changes and to simply reproduce the signs, typically from the 1960 edition of the \textit{Liber antiphonarius;} although we have a slight preference for the rhythmic signs, the notation and even the melodies as found in the monastic antiphonal, we faithfully preserve the Roman version in those places where the Solesmes monks have changed the \textit{punctum mora} to a  \textit{punctum} with an \textit{episema} or a \textit{bistropha,} with only a few exceptions: there are antiphons where a \textit{punctum mora} on a first syllable particularly after a half bar, necessarily introducing even a slight pause between the two syllables of the word, violating the Golden Rule which forbids a cæsura as Latin syllables determine the word's meaning just as much in writing as in speech in ecclesiastical Latin. We therefore insert a \textit{bistropha} instead of the \textit{punctum mora.}

Nevertheless, we do not further present the rhythm, in order to not print too many pages, leaving you free to add or to even ignore rythmic signs in following another school of Gregorian rhythm.

We wish to facilitate chanting, and the direction of chanters by choirmasters, by providing pointed psalms according to the different tones (with bold text for accents or syllables to be treated as one, italic for preparatory syllables, following the \textit{Liber Usualis}) with the first verse of the psalm notated under every antiphon or, for the common and proper of saints and for the variable psalms of major feasts of the temporal cycle, at least once per division of the book so as to reduce repetition but in minimizing the turns needed and the number of ribbons or bookmarks.

For the psalter, antiphons are given as in the antiphonal (and \textit{Liber Usualis}) for Sunday and Saturday Vespers and at Compline; otherwise, the psalms and antiphons are as in the \textit{Liber Usualis,} with the full antiphon at the beginning of the corresponding psalm, in order to chant on double feasts which use the ferial psalms and antiphons. In the psalter, we give the first verse of psalms which have a different first verse from the antiphon but which share an intonation (e.g. at Friday Compline). This allows for the book to be used by those using the 1960 office, called the ``Extraordinary Form''.

One must know the rubric directing the singing of the intonation of the psalm at the second verse when the entire first verse is skipped. Proper antiphons and the psalms in the other parts of the book are given before the psalm, as is done with Matins and during the Triduum in the \textit{Liber Usualis}.

With the \textit{Liber Usualis,} the singer is forced to turn delicate pages in order to find most of the psalms of feasts. Because of choices made over a century ago, the book is very unbalanced, particularly for Sunday Vespers, which is in the first third of the book. We aim to avoid this by moving the psalter to the middle, as in postconciliar editions. This should also maximize the life of the binding. Further, the size will hew as closely as possible to that of the modern Solesmes antiphonal.

The tones of the hymns are taken from the 1912 edition as reproduced by Solesmes; the elisions are given per the \textit{Liber antiphonarius,} that is to say that, except for newer hymns first edited by Solesmes, notes of hypermetric syllables are left in, to let those chanters who sing them sing them and to let those who omit them do so as well, according to custom.

As said above, the \textit{Magnificat} will be given in full with all of the tones necessary, as in the \textit{Liber Usualis;} the antiphonal gives the first half but presumes that the chanter memorized the second half of the text and can correctly apply the tones, even though dactyls pose a problem especially in mode \scspace{iv}.

The common tones will also be reproduced in a dedicated section (as in the antiphonal, and therefore not in the psalter); at the back, there is are chants needed for benediction and the proper of France.

The verse numbers beginning at ``1'' are kept according to the form of the \textit{Liber Usualis} as a reference for choirmasters working with singers and organists not sufficiently fluent in Latin, particularly useful in finding their place in rehearsals or keeping track of the alternations between two choirs or cantors and the full choir while singing the office; this is a primary practical consideration, as the numbers are not biblical, but it is to be noted that the divisions do not correspond to the biblical divisions as it is.

Finally, we include the decrees not to claim that they are still in force or that we do so in granting to ourselves authority, but to give the context for the decisions made in the Solesmes editions.
\end{enpars}
