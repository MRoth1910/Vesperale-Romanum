% !TEX TS-program = LuaLaTeX+se
% !TEX root = Kalendarium.tex

{\centering{{\normalsize Februarius.}\par}}
\begin{longtable}{>{\centering}p{0.025\textwidth}|>{\raggedright}p{0.040\textwidth}|>{\raggedleft}p{0.025\textwidth}|>{\raggedright\arraybackslash}p{0.80\textwidth}}

d & Kal. & 1 & S. Ignatii Episc. Martyr. \textit{Duplex.}\\
e & iv & 2 & \hang \scspace{Purificatio B}. \scspace{Mariæ Virginis}. \textit{Duplex II classis.}\\
f & iij & 3 & \hang S. Blasii Episc. Martyr. \textit{Simplex.}\\
g & Prid. & 4 & \hang S. Andreæ Corsini Episc. Conf. \textit{Duplex.}\\
A & Non. & 5 & \hang S. Agathæ Virginis et Martyris. \textit{Duplex.}\\
b & viij & 6 & \hang S. Titi Episc. Conf. \textit{Duplex.} \mem{S. Dorotheæ Virginis et Martyris.} \\
c & vij & 7 & \hang S. Romualdi Abbatis. \textit{Duplex.}\\
d & vj & 8 & \hang S. Joannis de Matha Conf. \textit{Duplex.}\\
e & v & 9 & S. Cyrilli Episc. Alexandrini Conf. et Eccl. Doct. \textit{Duplex.}\\
f & iv & 10 & \hang S. Scholasticæ Virginis. \textit{Duplex.}\\
g & iij & 11 & \hang Apparitionis B.M.V. Immaculatæ. \textit{Duplex majus.}\\
A & Prid. & 12 & SS. Septem Fundatorum Ord. Servorum B.V.M., Cc. \textit{Duplex.}\\
b & Ibid. & 13 & \\
c & xvj & 14 & \hang S. Valentini Presbyteri Martyris. \textit{Simplex.}\\
d & xv & 15 & SS. Faustini et Jovitæ Martyrum. \textit{Simplex.}\\
e & xiv & 16 & \\
f & xiij & 17 & \\
g & xij & 18 & S. Simeonis Episcopi Martyris. \textit{Simplex.}\\
A & xj & 19 & \\
b & x & 20 & \\
c & ix & 21 & \\
d & viij & 22 & \hang Cathedra S. Petri Antiochæ. \textit{Dupl. maj.} \mem{S. Pauli Ap.}\\
e & vij & 23 & \hang S. Petri Damiani Episc. Conf. et Eccl. Doct. \textit{Duplex.} \mem{Vigiliæ.}\\
f & vj & 24 & \scspace{S}. \scspace{Matthiæ Apostoli}. \textit{Duplex II classis.}\\
g & v & 25 & \\
A & iv & 26 & \\
b & iij & 27 & \hang S. Gabrielis a Virgine Perdolente Conf. \textit{Duplex.}\\
c & Prid. & 28 &
%%& & & \rubrique{In anno bisextili mensis Febrarius est dierum 29, et Festum S. Matthiæ celebratur die 25 Februarii, ac Festum S. Gabrielis a Virg. Perdolente die 28 Februarii, et bis dicitur Sexto Kalendas, id est die 24 et die 25; et littera Dominicalis, quæ assumpta fuit in mense Januario, mutatur in præcedentem ut, si in Januario littera Dominicalis fuerit A, mutetur in præcedentum, quæ est \normaltext{g,} etc., littera  \normaltext{f} bis servit 24 et 25.}
\end{longtable}