% !TEX TS-program = lualatexmk
% !TEX parameter =  --shell-escape

\documentclass[vesperale_romanum.tex]{subfiles}

\ifcsname preamble@file\endcsname
  \setcounter{page}{\getpagerefnumber{M-vr_NN_commune_non_virginum}}
\fi

%%this code when \customsubfiles is used should allow for continuous pagination when subfiles are compiled individually.

\begin{document}

\section[Commune non Virginum.]{COMMUNE SANCTÆ MARTYRIS TANTUM}\header{commune non virginum.}
\subtitle{et nec Virginis nec Martyris.}
%\thispagestyle{empty}

\invesperis{i}

\primaantiphona{3. a.}

\initialscore{an_dum_esset_rex_solesmes_1961}
\psalmus{109}{109_3a}{109_3a_b}
%\espacetitre %no magic numbers

\gscore[2. Ant.]{4. A*}{an_in_odorem_solesmes_1961}
\psalmus{112}{112_4A_A_star}{112_4A_A_star}

\gscore[3. Ant.]{8. G}{an_jam_hiems_transiit_solesmes_1961}
\psalmus{121}{121_8G}{121_8}

\gscore[4. Ant.]{1. f}{an_veni_electa_mea_solesmes_1961}
\psalmus{126}{126_1f}{126_1}

\gscore[5. Ant.]{8. G*}{an_ista_est_speciosa_non_virginis_solesmes_1961}
\psalmus{147}{147_8Gstar}{147_8}

%%stop worrying and love what it does: the alignement issue is not really one cf same common in LA1949

\specialcapitulum{Pro Martyre tantum.}{Eccli. 51, 1 – 3.}

\lettrine{C}{o}nfitébor tibi, Dómine Rex, et collaudábo te Deum Salvatórem meum.~† Confitébor nómini tuo: quóniam adjútor et proté\-ctor fa\-ctus es mihi,~* et liberásti corpus meum a perditióne.

\specialcapitulum{Pro nec Virgine nec Martyre.}{Prov. 31, 10 – 11.}

\lettrine{M}{u}líerem fortem quis invéniet? procul et de últimis fínibus prétium ejus.~† Confídit in ea cor viri sui,~* et spóliis non indigébit.

%%problem with J is perhaps descending annotation with old-style figures BUT latexmk runs lead to changed position even when the hymn is not changed otherwise…

\hymnusmode{2.}

\initialscore{hy_fortem_virili_pectore_solesmes_1961}

%%

\smalltitle{Tempore Paschali.}

\gscore[]{3.}{hy_fortem_virili_pectore_tp_solesmes_1961} %%finish transcribing %%check elisions

\rubrique{Ab Ascensione ad Pentecosten, pro doxologia dicitur:}

\smallscore{dox_ascensionis_tristes_erant}

\altertonus

\gscore[]{4.}{hy_fortem_virili_pectore_paschal_time_solesmes_1961}

\vel

\smallscore{dox_ascensionis_deus_tuorum_alter_tonus_mode_4}

%%should we bother with TP Alleluia? at some point you do just have to know this rubric.

\vv Spécie tua et pulchritúdine tua. \tpalleluia

\rr Inténde próspere procéde, et regna. \tpalleluia

\admagnificat

\label{an_simile_est_negotiatori_solesmes_1961}\phantomsection
\gscore[]{8. G}{an_simile_est_negotiatori_solesmes_1961}

\specialoratio{Pro Virgine Martyre.}

\lettrine{D}{e}us, qui inter cétera poténtiæ tuæ mirácula, étiam in sexu frágili victóriam martýrii contulísti:~† concéde propítius; ut qui beatæ \textit{N.} Mártyris tuæ natalítia cóIimus,~* per ejus ad te exémpla gradiámur. Per Dóminum.

\specialoratio{Pro pluribus Martyribus tantum.}

\lettrine{D}{a} nobis, quǽsumus Dómine Deus noster, san\-ctárum Mártyrum tuárum \textit{N.} et \textit{N.} palmas incessábili devotióne venerári:~† ut quas digna mente non póssumus celebráre,~* humílibus saltem frequentémus obséquiis. Per Dóminum.

\specialoratio{Pro nec Virgine non Martyre.}

\lettrine{E}{x}áudi nos Deus salutáris noster:~† ut sicut de beátæ \textit{N.} tuæ festivitáte gaudémus;~* ita piæ devotiónis erudiámur afféctu.
Per Dóminum.

\invesperis{ii}

\omniapraeter %%not in any edition, but it's the best rubric given that it makes references to Lauds and such

\vv Diffúsa est grátia in lábiis tuis. \tpalleluia

\rr Proptérea bendíxit te Deus in ætérnum. \tpalleluia

\admagnificat
\label{an_manum_suam_solesmes_1961}\phantomsection
\gscore[]{8. G}{an_manum_suam_solesmes_1961}

\end{document}