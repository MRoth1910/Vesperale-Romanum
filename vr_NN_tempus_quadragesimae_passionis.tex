% !TEX TS-program = lualatexmk
% !TEX parameter =  --shell-escape

\documentclass[vesperale_romanum.tex]{subfiles}

\ifcsname preamble@file\endcsname
  \setcounter{page}{\getpagerefnumber{M-vr_NN_tempus_quadragesimae_passionis}}
\fi

%%this code when \customsubfiles is used should allow for continuous pagination when subfiles are compiled individually.

\begin{document}

%\thispagestyle{empty}

\addcontentsline{toc}{section}{A Sab. ante Dom. I in Quadragesimæ ad Pascha.}

%\addcontentsline{toc}{section}{A Dom. I in Quadragesimæ ad Pascha.}

%%header and sectioning commands to be determined

\bigtitle{Sabbato I in Quadragesima.}

%%** to be replaced with \pageref to Sat in the psalter

\rubrique{Psalmi et Antiphona de Sabbato ut in Psalterio, \normaltext{\pageref{M-sabbato_ad_vesperas}.}}

\capitulum{2 Cor. 6, 1 – 2.}
\lettrine{F}{r}atres: Hortámur vos, ne in vácuum grátiam Dei recipiátis.~† Ait enim: Témpore accépto exaudívi te,* et in die salútis adjúvi te.

\hymnus \label{audi_benigne}%%getting these references right is a problem

\gscore[]{2.}{hy_audi_benigne_conditor_solesmes_1961} 

\vv Angelis suis Deus mandávit de te.\label{vv_quad}

\rr Ut custódiant te in ómnibus víis túis.

\rubrique{Hymnus et ℣. hujus Dominicæ dicitur in aliis Dominicis et in Feriis usque ad Dominicam Passionis exclusive.}

\gscore[Ad Magnif.]{Ant. 7. a}{an_tunc_invocabis_solesmes_1961} %%Solesmes uses Ad Magnificat, Antiphona. for Sat/Sunday.

\oratio

\lettrine{D}{e}us, qui Ecclésiam tuam ánnua quadragesimáli observatióne puríficas:~† præsta famíliæ tuæ; ut quod a te obtinére abstinéndo nítitur,* hoc bonis opéribus exsequátur. Per Dóminum.

\hymnusadcompletorium

\gscore[]{2.}{hy_te_lucis_ante_terminum_lent_solesmes_1961}

\rubrique{Hic tonus servatur ad Completorium usque ad Dominicam Passionis, etiam in Festis Sanctorum, nisi aliter notetur.}

\bigtitle{Dominica I in Quadragesima.}

\rubrique{Antiphonæ et Psalmi de Dominica. Hymnus \normaltext{Audi benígne cónditor, \pageref{audi_benigne}.} Capitulum \normaltext{Fratres: Hortámur} ut supra. \normaltext{\vv Angelis suis, \pageref{vv_quad}.}} %%supra probably needs pageref

\gscore[Ad Magnif.]{Ant. 8. G*}{an_ecce_nunc_tempus_solesmes_1961}

\bigtitle{Feria II.}

\capitulum{Joel 2, 17.}

\lettrine{I}{n}ter vestíbulum et altáre plorábunt sacerdótes minístri Dómini, et dicent:~† Parce Dómine, parce pópulo tuo:~* et ne des hæreditátem tuam in oppróbrium, ut dominéntur eis natiónes.

\rubrique{Hymnus \normaltext{Audi benígne cónditor, \pageref{audi_benigne}.}}

\textes{versiculus_t_quadragesima}

\gscore[Ad Magnif.]{Ant. 1. f}{an_quod_uni_ex_minimis_solesmes}

\rubrique{In feriali Officio per Quadragesimam dicuntur Preces.}

\oratio

\lettrine{A}{b}sólve, quǽsumus, Dómine, nostrórum víncula peccatórum:~* et quidquid pro eis merémur, propitiátus avérte.
Per Dóminum.

\rubrique{Capitulum supradictum dicitur in feriali Officio usque ad Dominicam Passionis.}

\bigtitle{Feria III.}

\gscore[Ad Magnif.]{Ant. 8. G}{an_scriptum_est_enim_quia_solesmes_1961}

\oratio

\lettrine{A}{s}céndant ad te Dómine preces nostræ:~* et ab Ecclésia tua cunctam repélle nequítiam. Per Dóminum.

\bigtitle{Feria IV Quatuor Temporum.}

\gscore[Ad Magnif.]{Ant. 4. A*}{an_sicut_fuit_jonas_solesmes}

\oratio

\lettrine{M}{e}ntes nostras, quǽsumus Dómine, lúmine tuæ claritátis illústra:~* ut vidére possímus quæ agenda sunt; et quæ recta sunt agere valeamus. Per Dóminum.

\bigtitle{Feria V.}

\gscore[Ad Magnif.]{Ant. 4. A*}{an_o_mulier_solesmes_1961}

\oratio

\lettrine{D}{a} quǽsumus Dómine, pópulis christiánis, et quæ profiténtur agnóscere:~* et cæléste munus dilígere, quod frequéntant. Per Dóminum.

\bigtitle{Feria VI Quatuor Temporum.}

\gscore[Ad Magnif.]{Ant. 1. f}{an_qui_me_sanum_fecit_solesmes_1961}

\oratio

\lettrine{E}{x}áudi nos miséricors Deus:~* et méntibus nostris grátiæ tuæ lumen osténde. Per Dóminum.

\bigtitle{Sabbato Quatuor Temporum.}

\capitulum{1 Thess. 4, 1.}

\lettrine{F}{r}atres: Rogámus vos, et obsecrámus in Dómino Jesu:~† ut, quemádmodum accepístis a nobis, quómodo vos opórteat ambuláre, et placére Deo,* sic et ambulétis, ut abundétis magis.

\rubrique{Hymnus \normaltext{Audi benígne cónditor, \pageref{audi_benigne}. \vv Angelis suis, \pageref{vv_quad}.}}

\gscore[Ad Magnif.]{Ant. 1. f}{an_visionem_solesmes_1961}

\oratio

\lettrine{D}{e}us, qui cónspicis omni nos virtúte destítui: intérius exteriúsque custódi;~† ut ab ómnibus adversitátibus muniámur in córpore,* et a právis cogitatiónibus mundémur in mente. Per Dóminum.

\bigtitle{Dominica II in Quadragesima.}

\rubrique{Antiphonæ et Psalmi de Dominica. Capitulum \normaltext{Fratres: Rogámus vos} ut supra. Hymnus \normaltext{Audi benígne cónditor, \pageref{audi_benigne}.} \normaltext{\vv Angelis suis, \pageref{vv_quad}.}} %% does supra need pageref 

\rubrique{Ad Magnificat \normaltext{Visiónem,} ut supra.} %%Vaticana has pageref

\bigtitle{Feria II.}

\gscore[Ad Magnif.]{Ant. 1. f}{an_qui_me_misit_solesmes_1961}

\oratio

\lettrine{A}{d}ésto supplicatiónibus nostris omnípotens Deus:~* et quibus fidúciam sperándæ pietátis indúlges, consuétæ misericórdiæ tríbue benígnus efféctum. Per Dóminum.

\bigtitle{Feria III.}

\gscore[Ad Magnif.]{Ant. 4. E}{an_omnes_autem_vos_solesmes_1961}

\oratio

\lettrine{P}{r}pitiáre Dómine supplicatiónibus nostris, et animárum nostrárum medére languóribus:~* ut remissióne percépta, in tua semper benedictióne lætémur. Per Dóminum.

\bigtitle{Feria IV.}

\gscore[Ad Magnif.]{Ant. 1. f}{an_tradetur_enim_gentibus_solesmes_1961}

\oratio

\lettrine{D}{e}us innocéntiæ restitútor et amátor, dírige ad te tuórum corda servórum:~* ut spíritus tui fervóre concépto, et in fide inveniántur stábiles, et in ópere efficáces. Per Dóminum.

\bigtitle{Feria V.}

\gscore[Ad Magnif.]{Ant. 7. c2}{an_dives_ille_guttam_solesmes_1961}

\oratio

\lettrine{A}{d}ésto Dómine fámulis tuis, et perpétuam benignitátem largíre poscéntibus:~* ut iis, qui te auctóre et gubernatóre gloriántur, et congregáta restáures, et restauráta consérves. Per Dóminum.

\bigtitle{Feria VI.}

\gscore[Ad Magnif.]{Ant. 3. a}{an_quaerentes_eum_tenere_solesmes_1961}

\oratio

\lettrine{D}{a} quǽsumus Dómine, pópulo tuo salútem mentis et córporis:~* ut bonis opéribus inhæréndo, tuæ semper virtútis mereátur protectióne deféndi. Per Dóminum.

\bigtitle{Sabbato III in Quadragesima.}

\capitulum{Eph. 5, 1 – 2.}

\lettrine{F}{r}atres: Estóte imitatóres Dei, sicut fílii caríssimi:~† et ambuláte in dilectióne, sicut et Christus diléxit nos, et trádidit semetípsum pro nobis~* oblatiónem et hóstiam Deo in odórem suavitátis.

\rubrique{Hymnus \normaltext{Audi benígne cónditor, \pageref{audi_benigne}. \vv Angelis suis, \pageref{vv_quad}.}}

\gscore[Ad Magnif.]{Ant. 8. G}{an_dixit_autem_pater_ad_servos_solesmes_1961}

\oratio

\lettrine{Q}{u}ǽsumus omnípotens Deus, vota humílium réspice:~† atque ad defensiónem nostram,* déxteram tuæ majestátis exténde.
Per Dóminum.

\bigtitle{Dominica III in Quadragesima.}

\rubrique{Antiphonæ et Psalmi de Dominica. Capitulum \normaltext{Fratres: Estóte imitatóres Dei} ut supra. Hymnus \normaltext{Audi benígne cónditor, \pageref{audi_benigne}.}} %%does supra need pageref?

\textes{versiculus_t_quadragesima}

\gscore[Ad Magnif.]{Ant. 8. G}{an_extollens_solesmes_1961}

\bigtitle{Feria II.}

\gscore[Ad Magnif.]{Ant. 1. f}{an_jesus_autem_transiens_solesmes_1961}

\oratio

\lettrine{S}{u}bvéniat nobis Dómine misericórdia tua:~* ut ab imminéntibus peccatórum nostrórum perículis te mereámur protegénte éripi, te liberánte salvári.
Per Dóminum.

\bigtitle{Feria III.}

\gscore[Ad Magnif.]{Ant. 4. A*}{an_ubi_duo_vel_tres_solesmes_1961}

\oratio

\lettrine{T}{u}a nos Dómine protectióne defénde:~* et ab omni semper iniquitáte custódi. Per Dóminum.

\bigtitle{Feria IV.}

\gscore[Ad Magnif.]{Ant. 7. a}{an_non_lotis_manibus_solesmes_1961} 

\oratio

\lettrine{C}{o}ncéde, quǽsumus omnípotens Deus: ut qui protectiónis tuæ grátiam quǽrimus,~* liberáti a malis ómnibus, secúra tibi mente serviámus. Per Dóminum.

\bigtitle{Feria V.}

\gscore[Ad Magnif.]{Ant. 1. f}{an_omnes_autem_vos_solesmes_1961}

\oratio

\lettrine{S}{u}bjéctum tibi pópulum, quǽsumus, Dómine, propitiátio cæléstis amplíficet:~* et tuis semper fáciat servíre mandátis. Per Dóminum.

\bigtitle{Feria VI.}

\gscore[Ad Magnif.]{Ant. 3. a}{an_domine_ut_video_solesmes_1961}

\oratio

\lettrine{P}{r}æsta, quǽsumus, omnípotens Deus:~† ut qui in tua protectióne confídimus, cuncta nobis adversántia te adjuvánte vincámus. Per Dóminum.

\bigtitle{Sabbato IV in Quadragesima.}

\capitulum{Gal. 4, 22 – 24.}

\lettrine{F}{r}atres: Scriptum est quóniam Abraham duos fílios hábuit: unum de ancílla, et unum de líbera:~† sed qui de ancílla, secúndum carnem natus est: qui autem de líbera, per repromissiónem:~* quæ sunt per allegoríam dicta.

\rubrique{Hymnus \normaltext{Audi benígne cónditor, \pageref{audi_benigne}.}}

\textes{versiculus_t_quadragesima}

\gscore[Ad Magnif.]{3. a}{an_nemo_te_condemnavit_solesmes_1961}

\oratio

\lettrine{C}{o}ncéde, quǽsumus omnípotens Deus:~† ut, qui ex mérito nostræ actiónis afflígimur,* tuæ grátiæ consolatióne respirémus.
Per Dóminum.

\bigtitle{Dominica IV in Quadragesima.}

\rubrique{Antiphonæ et Psalmi de Dominica. Capitulum \normaltext{Fratres: Scriptum est} ut supra. Hymnus \normaltext{Audi benígne cónditor, \pageref{audi_benigne}.} \normaltext{\vv Angelis suis, \pageref{vv_quad}.}} %%does supra need pageref?

\gscore[Ad Magnif.]{Ant. 1. g}{an_subiit_ergo_solesmes_1961}

\bigtitle{Feria II.}

\gscore[Ad Magnif.]{Ant. 5. a}{an_solvite_templum_hoc_solesmes_1961}

\oratio

\lettrine{D}{e}precatiónem nostram, quǽsumus Dómine, benígnus exáudi:~* et quibus supplicándi præstas afféctum, tríbue defensiónis auxílium. Per Dóminum.

\bigtitle{Feria III.}

\gscore[Ad Magnif.]{Ant. 1. g}{an_nemo_in_eum_misit_solesmes_1961}

\oratio

\lettrine{M}{i}serére Dómine pópulo tuo:~* et contínuis tribulatiónibus laborántem, propítius respiráre concéde. Per Dóminum.

\bigtitle{Feria IV.}

\gscore[Ad Magnif.]{Ant. 1. a}{an_ille_homo_solesmes_1961}

\oratio

\lettrine{P}{a}teant aures misericórdiæ tuæ Dómine précibus supplicántium:~* et, ut peténtibus desideráta concédas, fac eos quæ tibi sunt plácita postuláre.
Per Dóminum.

\bigtitle{Feria V.}

\gscore[Ad Magnif.]{Ant. 4. A}{an_propheta_magnus_solesmes}

\oratio

\lettrine{P}{o}puli tui Deus institútor et rector, peccáta quibus impugnátur, expélle:~* ut semper tibi plácitus, et tuo munímine sit secúrus. Per Dóminum.

\bigtitle{Feria VI.}

\gscore[Ad Magnif.]{Ant. 1. g2}{an_domine_si_hic_fuisses_solesmes_1961}

\oratio

\lettrine{D}{a}  nobis, quǽsumus, omnípotens Deus:~* ut qui infirmitátis nostræ cónscii, de tua virtúte confídimus, sub tua semper pietáte gaudeámus.
Per Dóminum.

\bigtitle{Sabbato ante Dominicam Passionis.}

\capitulum{Heb. 9, 11 – 12.}

\lettrine{F}{r}atres: Christus assístens Póntifex futurórum bonórum, per ámplius et perféctius tabernáculum non manu factum, id est, non huius creatiónis:~† neque per sánguinem hircórum aut vitulórum, sed per próprium sánguinem introívit semel in Sancta,~* ætérna redemptióne invénta.

\hymnus \label{vexilla_regis_passionis}%%getting these references right is a problem

\gscore[]{1.}{hy_vexilla_regis_prodeunt_solesmes} %%italicized a for elision is too close to d for my liking

\vv Eripe me Dómine ab hómine malo.\label{vv_passionis}

\rr A viro iníquo éripe me.

\gscore[Ad Magnif.]{Ant. 8. G}{an_ego_sum_qui_testimonium_solesmes_1961}

\oratio

\lettrine[depth=1]{Q}{u}ǽsumus omnípotens Deus, famíliam tuam propítius réspice:~† ut te largiénte, regátur in córpore;~* et te servánte, custodiátur in mente. Per Dóminum. %%may want to adjust Q with package option depth default=1, must be an integer

\rubrique{Ab his Vesperis usque ad Festum Ss. Trinitatis inclusive, omittitur Commemoratio seu Suffragium de Omnibus Sanctis.}

\hymnusadcompletorium

\gscore[]{2.}{hy_te_lucis_ante_terminum_passiontide_solesmes_1961}

\rubrique{Hic tonus servatur ad Completorium usque ad Feriam V in Cœna Domini, etiam in Festis Sanctorum, nisi aliter notetur.}

\rubrique{Ab hac die usque ad Feriam V in Cœna Domini, ad Cornpletorium quando fit Officium de Ternpore in ℟. brevi non dicitur \normaltext{℣. Glória Patri,} sed ejus loco statim repetitur \normaltext{℟. In manus tuas,} etc.}

\bigtitle{Dominica Passionis.}

\rubrique{Antiphonæ et Psalmi de Dominica. Capitulum \normaltext{Fratres: Christus assístens} ut supra. Hymnus \normaltext{Vexílla regis, \pageref{vexilla_regis_passionis}.}  \normaltext{\vv Eripe me, \pageref{vv_passionis}.}} %%supra probably needs pageref

\gscore[Ad Magnif.]{Ant. 2. D}{an_abraham_pater_vester_solesmes_1961} %%Solesmes uses Ad Magnificat, Antiphona. for Sat/Sunday.

\bigtitle{Feria II.}

\rubrique{Antiphonæ et Psalmi de Psalterio.}

\capitulum{Jer. 11, 20.}

\lettrine{T}{u} autem, Dómine Sábaoth, qui júdicas juste, et probas renes et corda,~† vídeam ultiónem tuam ex eis:~* tibi enim revelávi causam meam, Dómine Deus meus.

\gscore[Ad Magnif.]{Ant. 4. A*}{an_si_quis_sitit_solesmes_1961}

\oratio

\lettrine{D}{a} quǽsumus Dómine, pópulo tuo salútem mentis et córporis:~* ut bonis opéribus inhæréndo, tua semper mereátur protectióne deféndi.
Per Dóminum.

\rubrique{Capitulum prædictum dicitur in feriali Officio usque ad Feriam IV Majoris Hebdomadæ. Item Hymnus et \vv}

\bigtitle{Feria III.}

\gscore[Ad Magnif.]{Ant. 1. D2}{an_vos_ascendite_solesmes_1961}

\oratio

\lettrine{D}{a} quǽsumus Dómine, perseverántem in tua voluntáte famulátum:~* ut in diébus nostris, et mérito et número pópulus tibi sérviens augeátur.
Per Dóminum.

\bigtitle{Feria IV.}

\gscore[Ad Magnif.]{Ant. 4. A*}{an_multa_bona_opera_solesmes_1961}

\oratio

\lettrine{A}{d}ésto supplicatiónibus nostris, omnípotens Deus:~* et quibus fidúciam sperándæ pietátis indúlges, consuétæ misericórdiæ tríbue benígnus efféctum. Per Dóminum.

\bigtitle{Feria V.}

\gscore[Ad Magnif.]{Ant. 4. A*}{an_desiderio_desideravi_solesmes_1961}

\oratio

\lettrine{E}{s}to, quǽsumus Domine, propitius plebi tuæ:~* ut quæ tibi non placent respuentes, tuorum potius repleantur delectationibus mandatorum. Per Dóminum.

\bigtitle{Feria VI.}

\gscore[Ad Magnif.]{Ant. 1. g}{an_principes..._consilium_solesmes_1961}

\textes{versiculus_passionis}

\oratio

\lettrine{C}{o}ncéde, quǽsumus omnipotens Deus:~* ut qui protectionis tuæ gratiam quærimus, liberati a malis omnibus, secura tibi mente serviamus.
Per Dóminum.

\bigtitle{Sabbato ante Dominicam in Palmis.} %%neither antiphonal gives this title, just Sabbato, but it's consistent

\capitulum{Phil. 2, 5 – 7.}

\lettrine{F}{r}atres: Hoc enim sentíte in vobis, quod et in Christo Jesu: qui, cum in forma Dei esset, non rapínam arbitrátus est esse se æquálem Deo:~† sed semetípsum exinanívit, formam servi accípiens, in similitúdinem hóminum factus,~* et hábitu invéntus ut homo.

\rubrique{Hymnus \normaltext{Vexílla regis, \pageref{vexilla_regis_passionis}.}  \normaltext{\vv Eripe me, \pageref{vv_passionis}.}}

\gscore[Ad Magnif.]{Ant. 4. E}{an_pater_juste_solesmes_1961} %%Solesmes uses Ad Magnificat, Antiphona. for Sat/Sunday.

\oratio

\lettrine{O}{m}nípotens sempitérne Deus, qui humáno géneri, ad imitándum humilitátis exémplum, Salvatórem nostrum carnem súmere, et crucem subíre fecísti:~† concéde propítius; ut et patiéntiæ ipsíus habére documénta,*  et resurrectiónis consórtia mereámur. Per eúndem Dóminum nostrum.

\bigtitle{Dominica in Palmis.}

\rubrique{Antiphonæ et Psalmi de Dominica. Capitulum \normaltext{Fratres: Hoc enim sentíte} ut supra.}

\rubrique{Hymnus \normaltext{Vexílla regis, \pageref{vexilla_regis_passionis}.}}

\textes{versiculus_passionis}

\gscore[Ad Magnif.]{Ant. 8. G*}{an_scriptum_est_enim__percutiam_solesmes_1961.1} %%Solesmes uses Ad Magnificat, Antiphona. for Sat/Sunday.

\rubrique{Ab hac die usque ad Octavam Paschæ, si occurrat aliquod Festum IX Lectionum quod transferri valeat, transfertur post Octavam; secus de eo fit commemoratio, præterquam tribus diebus Pascha præcedentibus et duobus consequentibus. Festi III Lectionum item a Feria V in Cœna Domini usque ad Feriam III Pascha, nulla fit Commemoratio.}

\bigtitle{Feria II Majoris Hebdomadæ.}

\gscore[Ad Magnif.]{Ant. 4. A*}{an_non_haberes_in_me_solesmes}

\oratio

\lettrine{A}{d}juva nos Deus salutáris noster:~* et ad benefícia recolénda quibus nos instauráre dignátus es, tríbue veníre gaudéntes.
Per Dóminum

\bigtitle{Feria III Majoris Hebdomadæ.}

\gscore[Ad Magnif.]{Ant. 4. A*}{an_potestatem_habeo_solesmes}

\oratio

\lettrine{T}{u}a nos misericórdia, Deus, et ab omni subreptióne vetustátis expúrget:~* et capáces sanctæ novitátis effíciat. Per Dóminum.

\bigtitle{Feria IV Majoris Hebdomadæ.}

\gscore[Ad Magnif.]{Ant. 1. g}{an_ancilla_dixit_petro_solesmes}

\oratio

\lettrine[findent=-0.3pt]{R}{e}spice, quǽsumus, Dómine, super hanc famíliam tuam,~* pro qua Dóminus noster Jesus Christus non dubitávit mánibus tradi nocéntium, et crucis subíre torméntum: Qui tecum vivit et regnat. %%no asterisk and with silent conclusion during Triduum.

%%Tenebrae book has package option [finindent=-0.3pt]

\rubrique{Tribus sequentibus diebus, dicto secreto \normaltext{Pater noster et Ave María,} omnibus aliis prætermissis, absolute incipitur Oflicium ad Vesperas ab Antiphona\footnote{Juxta morem hodiernum Vesperæ hodie et die sequenti dicuntur sine cantu. Attamen hic ponuntur Antiphonæ cum cantu proprio, ne pereat; et in gratiam Ecclesiarum quæ morem cantandi olim ubique observatum adhuc retinent.} primi Psalmi; et Antiphonæ duplicantur sicut in Festi Duplici.}

\rubrique{In fine Psalmorum, non dicitur \normaltext{Glória Patri.}}

\bigtitle{Feria V in Cœna Domini.}

\gscore[1 Ant.]{2. D}{an_calicem_salutaris_et_nomen_solesmes}%% Vaticana and LA have EUOUAE but this is confusing as GP isn't sung during Triduum.

\psalmus{115}{115_2}{115_2}

\gscore[2. Ant.]{8. G}{an_cum_his_qui_oderunt_solesmes_1961}

\psalmus{119}{119_8G}{119_8}

\gscore[3. Ant.]{8. G}{an_ab_hominibus_iniquis_solesmes}

\psalmus{139}{139_8G}{139_8}

\gscore[4. Ant.]{7. a}{an_custodi_me_a_laqueo_solesmes_1961}

\psalmus{140}{140_7a}{140_7}

\gscore[5. Ant.]{7. a}{an_considerabam_ad_dexteram_solesmes_1961}

\psalmus{141}{141_7a}{141_7}
%
\gscore[Ad Magnif.]{Ant. 1. f}{an_coenantibus_autem_solesmes} %

\rubrique{Capitulum, Hymnus et \vv non dicuntur.}

\rubrique{Sine cantu} \vv Christus factus est pro nobis obédiens usque ad mortem.

\rubrique{Secunda die additur:} Mortem autem crucis.

\rubrique{Cum incipitur ℣. \normaltext{Christus factus est} omnes genuflectunt, et eo finito, dicitur \normaltext{Pater noster} totum sub silentio. Postea Ps. \normaltext{Miserére} aliquantulum altius.}

%%Miserére has to go somewhere (perhaps just the office of the dead? with pageref) including it here is one possibility but the line length makes it awkward

%\textes{miserere_no_bold}

\rubrique{Quo finito, sine \normaltext{Orémus} dicitur simili voce:}

\lettrine[findent=-0.3pt]{R}{e}spice, quǽsumus, Dómine, super hanc famíliam tuam, pro qua Dóminus noster Jesus Christus non dubitávit mánibus tradi nocéntium, et crucis subíre torméntum:

\rubrique{Et sub silentio concluditur:}
 
Qui tecum vivit et regnat in unitáte Spíritus Sancti, Deus, per ómnia sǽcula sæculórum. %%lifted from the LU not the Vaticana or LA

Ad Completorium. \rubrique{Non dicitur \normaltext{Jube domne,} nec Lectio brevis, nec \normaltext{Adjutórium,} neque Oratio Dominica, sed facta Confessione et Absolutione, incipitur a Ps. \normaltext{Cum invocárem;} et dicuntur Psalmi de Dominica. Dictis Psalmis, dicitur Nunc dimíttis. Deinde \normaltext{\vv Christus factus est} ut supra.}

%\oratio
%
%\lettrine{P}{o}puli tui Deus institútor et rector, peccáta quibus impugnátur, expélle:~* ut semper tibi plácitus, et tuo munímine sit secúrus. Per Dóminum.
%
\bigtitle{Feria VI in Parasceve.}

\rubrique{Antiphonæ et Psalmi præteriti diei.}
%
\gscore[Ad Magnif.]{Ant. 1. f}{an_cum_accepisset_acetum_solesmes}

\bigtitle{Sabbato Sancto.}%% these files should be moved from Easter and the file should conclude after Compline. %% see LA rubric

\rubrique{Vesperas hujus diei ut in Graduali.} %%Solesmes LA

%\gscore[Ant.]{6. f}{an_alleluia_paschal_vigil_vespers_solesmes_1961}
%
%\rubrique{Et repetitur antiphona \normaltext{Allelúia, Allelúia, Allelúia.}}
\bigtitle{Ad Completorium.}

\rubrique{Jube domne benedícere. Lectio brevis \normaltext{Fratres: Sóbrii estóte. \vv Adjutórium. Pater noster.} Et facta Confessione et Absolutione dicitur \normaltext{\vv Convérte nos. \vv Deus in adjutórium. Glória Patri.} Et amplius non dicitur \normaltext{Laus tibi Dómine Rex ætérne glóriæ,} sed ejus loco deinceps dicitur \normaltext{Allelúia.}}

\rubrique{Deinde sine Antiphona dicentur Psalmi de Dominica, in tono sequenti:}

\psalmus{4}{ps_4_ad_compl_in_sab_san}{ps_4_ad_compl_in_sab_san}

\psalmus{90}{ps_90_ad_compl_in_sab_san}{ps_90_ad_compl_in_sab_san}

\psalmus{133}{ps_133_ad_compl_in_sab_san}{ps_133_ad_compl_in_sab_san}

\rubrique{Hymnus, Capitulum, \rr breve et \vv non dicuntur.}

\smalltitle{Ad Nunc Dimittis, Antiphona.}

\gscore[]{8. G}{an_vespere_autem_sabbati_solesmes}

\canticum{Simeonis}{Luc. 2, 29 – 32}{Nunc_dimittis_8G_c3.gabc}{3}{Nunc_dimittis_8G}

\rubrique{\vv \normaltext{Dóminus vobíscum}. Oratio \normaltext{Vísita} et reliqua ut in Psalterio.}

\rubrique{In fine, Antiphona \normaltext{Regína cæli.}}

\biggerrule

\end{document}


