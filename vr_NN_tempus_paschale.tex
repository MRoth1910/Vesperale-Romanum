% !TEX TS-program = lualatexmk
% !TEX parameter =  --shell-escape

\documentclass[vesperale_romanum.tex]{subfiles}

\ifcsname preamble@file\endcsname
  \setcounter{page}{\getpagerefnumber{M-vr_NN_tempus_pascale}}
\fi

%%this code when \customsubfiles is used should allow for continuous pagination when subfiles are compiled individually.

\begin{document}

 \thispagestyle{empty}
          
        %% need to include Compline with a pageref to psalter in addition to the other days of the octave
        % \section[Dominica Resurrectionis.]{DOMINICA RESURRECTIONIS.}
 \festum{Dominica Resurrectionis.}
% \header{dominica resurrectionis.}
 \rank{Duplex I classis cum Octava privilegiata I ordinis.}
 
\primaantiphona{8. G.}
 \initialscore{an_angelus_autem_domini_solesmes_1961}
 \psalmus{109}{109_8G}{109_8}
 
 \gscore[2. Ant.]{7. c}{an_et_ecce_terraemotus_solesmes_1961}
  \psalmus{110}{110_7c}{110_7}
 
 \gscore[3. Ant.]{8. c}{an_erat_autem_aspectus_solesmes_1961}
  \psalmus{111}{111_8c}{111_8}
 
  \gscore[4. Ant.]{7. a}{an_prae_timore_solesmes_1961}
   \psalmus{112}{112_7a}{112_7}
  
   \gscore[5. Ant.]{8. G}{an_respondens_autem_angelus_solesmes_1961}
    \psalmus{113}{113_8G}{113_8}
    
\rubrique{Et sic dicitur usque ad Vesperas Sabbati.}

\rubrique{Capitulum, Hymnus et ℣. non dicuntur; sed eorum loco:}
   
    \gscore[Ant.]{2.}{an_haec_dies_solesmes_1961}
    \label{Haec_dies}
    
     \admagnificat{3. a.}
     \initialscore{an_et_respicientes_solesmes_1961}
     
     \oratio

\lettrine{D}{e}us, qui hodiérna die per Unigénitum tuum æternitátis nobis áditum devícta morte reserásti:~† vota nostra, quæ præveniéndo aspíras,~* étiam adjuvándo proséquere.
Per eúmdem Dóminum nostrum.
    
     \smallscore{Benedicamus_domino_alleluia_solesmes_1961}
     
   \rubrique{Sic dicitur in Vesperis et Laudibus tantum, usque ad Vesperas Sabbati in Albis exclusive.}

\rubrique{vel hodie et duobus diebus seq. tantum ut sequitur:}
     
          \smallscore{Benedicamus_Dom_resurrectionis_solesmes_1961}
          
         \rubrique{Infra Octavam Paschæ usque ad Vesperas Sabbati exclusive, omnia dicuntur ut in die, præter Orationem et Antiphonam ad Magnificat. Feria II et III Officium fit duplex; aliis diebus, semiduplex.}
         
\bigtitle{Ad Completorium.}

\rubrique{Dicto \normaltext{Jube Domne, Fratres: Sóbrii estóte, ℣. Adjutórium nostrum, Pater noster}
et facta Confessione et Absolutione, post \normaltext{℣. Convérte nos, ℣. Deus in adjutórium, Glória Patri, Allelúia,} dicuntur Psalmi de Dominica cum Antiphona \normaltext{Allelúia}
prout notatur in Ordinario pro Tempore Paschali.}    

\canticum{Simeonis}{Luc. 2, 29 – 32}{Nunc_dimittis_oct_paschae.gabc}{Nunc_dimittis_oct_paschae}

\bigtitle{Feria Secunda.}

\gscore[Ad. Magnif.]{Ant. 8. G*}{an_qui_sunt_hi_sermones_solesmes_1961}

\oratio

\lettrine{D}{e}us, qui solemnitáte pascháli mundo remédia contulísti~† pópulum tuum, quǽsumus, cælésti dono proséquere;~* ut et perféctam libertátem cónsequi mereátur, et ad vitam profíciat sempitérnam. Per Dóminum nostrum.

\bigtitle{Feria Tertia.}

\gscore[Ad. Magnif.]{Ant. 8. G}{an_videte_manus_meas_solesmes_1961}

\oratio

\lettrine{D}{e}us, qui Ecclésiam tuam novo semper fœtu multíplicas:~† concéde fámulis tuis, ut sacraméntum vivéndo téneant,~* quod fide percepérunt.
Per Dóminum nostrum.

\bigtitle{Feria Quarta.}

\gscore[Ad. Magnif.]{Ant. 8. G}{an_dixit_jesus_discipulis_solesmes_1961}

\oratio

\lettrine{D}{e}us, qui nos resurrectiónis Domínicæ ánnua solemnitáte lætíficas:~† concéde propítius; ut per temporália festa quæ ágimus,~* perveníre ad gaúdia ætérna mereámur.
Per eúmdem Dóminum nostrum.

%\rubrique{Deinde dicuntur, commemorationes, si quæ faciendæ sunt, cum reliquis ut supra in die et ut in Ordinarium.}

%%this rubric is LU, I believe.

\bigtitle{Feria Quinta.}

\gscore[Ad. Magnif.]{Ant. 7. b}{an_tulerunt_dominum_solesmes_1961}

\oratio

\lettrine{D}{e}us, qui diversitátem géntium in confessióne tui nóminis adunásti:~† da, ut renátis fonte baptísmatis,~* una sit fides méntium et píetas actiónum.
Per Dóminum nostrum.

\bigtitle{Feria Sexta.}

\gscore[Ad. Magnif.]{Ant. 8. G}{an_data_est_mihi_solesmes_1961}

\oratio

\lettrine{O}{m}nípotens sempitérne Deus, qui paschále sacraméntum in reconciliatiónis humánæ fœ́dere contulísti:~† da méntibus nostris; ut quod professióne celebrámus, ~* imitémur efféctu. Per Dóminum nostrum.

\bigtitle{Sabbato in Albis.}

\rubrique{Officium fit duplex.}

\gscore[Ant.]{6. F}{an_alleluia_sat_pt_at_vespers_solesmes_1961}

%** is reference to psalter that needs to get a pageref.

\rubrique{Ps. \normaltext{Benedíctus Dóminus} cum reliquis ut in Psalterio, **.}

\capitulum \label{cap_dom_in_albis}

\scripture{1 Joan. 5, 4.}

\lettrine{C}{a}ríssimi: Omne quod natum est ex Deo, vincit mundum:~† et hæc est victória, quæ vincit mundum,~* fides nostra.

\label{hy_ad_regias_agni_dapes}

\hymnus

\gscore[]{8.}{hy_ad_regias_agni_dapes_solesmes_1961}

\altertonus

\gscore[]{4.}{hy_ad_regias_agni_dapes_another_chant_solesmes_1961}

\rubrique{Sic terminantur omnes Hymni ejusdem metri usque ad Ascensionem, nisi propriam doxologiam habeant.}

%%needs input or macro

\vv Mane nobíscum Dómine, allelúia.

\rr Quóniam advesperáscit, allelúia.

\gscore[Ad Magnif.]{Ant. 1. D}{an_cum_esset_sero_solesmes_1961}

\rubrique{Oratio ut infra ad Vesperas Dominicæ, \normaltext{\pageref{or_dom_in_albis}.}}

%\gscore[]{7.}{BD_TP}

%\rubrique{\normaltext{℣. Benedicámus Dómino} dicitur deinceps sine \normaltext{Allelúia.}}
%%this rubric may be needed

%\rubrique{Sic dicitur deinceps sine \normaltext{Allelúia,} et quidem sub hoc tono in Officio de Tempore usque ad Festurn Ss. Trinitatis exclusive, nisi sit Duplex I classis.} 

%%original rubric of antiphonal

%%I have this (as of 22 Dec 2023) in the toni communes like in the AM1934. LA1949 has it in the TP section only.

%%styled as in TC because otherwise the gabc has to change.

%%** = replace with pageref

Completorium \rubrique{dicitur ut in Psalterio cum Psalmi Sabbati et Antiphona \normaltext{Allelúia, **,} Hymnus \normaltext{Te lucis} in tono paschali, **.
Non dicuntur Preces. Et in fine ant. \normaltext{Salva nos,} additur \normaltext{Allelúia.}}

\rubrique{Sic dicitur usque ad Dominicam Ss. Trinitatis exclusive: et quando non fit Duplex aut Commemoratio Duplicis, nec infra Octavam, dicuntur Preces ut in Psalterio.}

%%this is where \vvrub and \rrrub are supposed to work but they need revision.

\rubrique{Ad ℟℟. brevia et ℣℣. per omnes Horas jungitur \normaltext{Allelúia,} usque ad Sabbatum post Pentecosten, præterquam in Preces Primæ et Completorii quando dicuntur. Et quandocurnque
\normaltext{Allelúia} jungitur ℟. brevi, duplicatur: sed in  ℣℣. dicitur semel.}

%%rubric copied from Vaticana without elimination of reference to other hours and to Prime as is typical.

\bigtitle{Dominica in Albis, in Octava Paschæ.}

\rank{Duplex majus.}

\gscore[Ant.]{7. c2}{an_alleluia_sun_tp_vespers_solesmes_1961}

\rubrique{Sub qua sola dicuntur omnes Psalmi de Dominica.}

\rubrique{Capitulum \normaltext{Caríssimi: Omne quod natum est, \pageref{cap_dom_in_albis},} ut supra.}

%%this should probably be a separate file, it repeats so much

\textes{ad_regias_vv_rr}

\gscore[Ad Magnif.]{Ant. 8. c}{an_post_dies_octo_solesmes_1961}

\oratio \label{or_dom_in_albis}

\lettrine{P}{r}æsta, quǽsumus omnípotens Deus:~† ut qui paschália festa perégimus,~* hæc, te largiénte, móribus et vita teneámus. Per Dóminum.

\bigtitle{Feria Secunda.}

\rubrique{Psalmi Feriæ sub una Antiphona \normaltext{Allelúia.}}

\capitulum

\scripture{Rom. 6, 9 – 10.}

\lettrine{C}{h}ristus resúrgens ex mórtuis jam non móritur, mors illi ultra non dominábitur.~† Quod enim mórtuus est peccáto, mórtuus est semel:~* quod autem vivit, vivit Deo.

\textes{ad_regias_vv_rr}

\gscore[Ad Magnif.]{Ant. 6. F}{an_pax_vobis_ego_sum_solesmes}

\rubrique{Capitula, Hymni, ℣℣. et ℟℟. brevia supradicta semper dicuntur in Feriali Officio usque ad Ascensionem.}

%%rubrique copied wholesale, without changes from plural.

\bigtitle{Feria Tertia.}

\gscore[Ad Magnif.]{Ant. 8. G*}{an_mitte_manum_tuam_solesmes}

\bigtitle{Feria Quarta.}

\gscore[Ad Magnif.]{Ant. 8. G}{an_quia_vidisti_me_solesmes_1961}

\bigtitle{Feria Quinta.}

\gscore[Ad Magnif.]{Ant. 8. G*}{an_misi_digitum_meum_solesmes}

\bigtitle{Feria Sexta.}

\bvmsabbato

%\bigtitle{Sabbato ante Dominicam II post Pascha.}

\bigtitle{Sabbato.}

%%** to be replaced with pageref to psalter.

\rubrique{Ant. \normaltext{Allelúia.} Ps. \normaltext{Benedíctus Dóminus,} cum reliquis, **.}

\capitulum \label{cap_dom_2_post_pascha}

\scripture{1 Petri 2, 21 – 22.}

\lettrine{C}{a}ríssimi: Christus passus est pro nobis,~† vobis relínquens exémplum ut sequámini vestígia eius.~* Qui peccátum non fecit, nec invéntus est dolus in ore ejus.

\textes{ad_regias_vv_rr}

\gscore[Ad Magnif.]{Ant. 8. G*}{an_ego_sum_pastor_ovium_solesmes_1961}

\rubrique{Oratio ut infra ad Vesperas Dominicæ, \normaltext{\pageref{or_dom_2_post_pascha}.}}

\bigtitle{Dominica II post Pascha.}

\rubrique{Ant. \normaltext{Allelúia.} Ps. de Dominica. Capitulum \normaltext{Caríssimi: Christus passus est pro nobis, \pageref{cap_dom_2_post_pascha},} ut supra.}

\textes{ad_regias_vv_rr}

\gscore[Ad Magnif.]{Ant. 3. a}{an_ego_sum_pastor_bonus_solesmes_1961}

\oratio \label{or_dom_2_post_pascha}

\lettrine{D}{e}us, qui in Fílii tui humilitáte jacéntem mundum erexísti:~† fidélibus tuis perpétuam concéde lætítiam;~* ut, quos perpétuæ mortis eripuísti cásibus, gáudiis fácias pérfrui sempitérnis. Per eúmdem Dóminum.

\bigtitle{Feria Secunda.}

\gscore[Ad Magnif.]{Ant. 3. a}{an_pastor_bonus_animam_solesmes}

\bigtitle{Feria Tertia.}

\gscore[Ad Magnif.]{Ant. 3. a}{an_mercenarius_est_cujus_solesmes}

\bigtitle{Feria Quarta.}

\gscore[Ad Magnif.]{Ant. 5. a}{an_sicut_novit_me_solesmes}

\bigtitle{Feria Quinta.}

\gscore[Ad Magnif.]{Ant. 8. G}{an_alias_oves_habeo_solesmes}

\bigtitle{Feria Sexta.}

\bvmsabbato

%\bigtitle{Sabbato ante Dominicam III post Pascha.}

\bigtitle{Sabbato.}

\rubrique{Ant. \normaltext{Allelúia.} Ps. \normaltext{Benedíctus Dóminus,} cum reliquis, **.}

\capitulum \label{cap_dom_3_post_pascha}

\scripture{1 Petri 2, 11.}
\lettrine{C}{a}ríssimi: Obsecro vos tamquam ádvenas et peregrínos,~† abstinére vos a carnálibus desidériis,~* quæ mílitant advérsus ánimam.

\textes{ad_regias_vv_rr}

\gscore[Ad Magnif.]{Ant. 6. F}{an_modicum_et_non_videbitis_me_solesmes}

\rubrique{Oratio ut infra ad Vesperas Dominicæ, \normaltext{\pageref{or_dom_3_post_pascha}.}}

\bigtitle{Dominica III post Pascha.}

\rubrique{Ant. \normaltext{Allelúia.} Ps. de Dominica. Capitulum \normaltext{Caríssimi: Obsecro vos, \pageref{cap_dom_3_post_pascha},} ut supra.}

\textes{ad_regias_vv_rr}

\gscore[Ad Magnif.]{Ant. 8. G}{an_amen_amen_quia_plorabitis_solesmes}

\oratio \label{or_dom_3_post_pascha}

\lettrine{D}{e}us, qui errántibus, ut in viam possint redíre justítiæ, veritátis tuæ lumen osténdis:~† da cunctis qui christiána professióne censéntur, et illa respúere quæ huic inimíca sunt nómini;~* et ea quæ sunt apta sectári. Per Dóminum nostrum.

\bigtitle{Feria Secunda.}

\gscore[Ad Magnif.]{Ant. 8. G}{an_tristitia_vestra_vertetur_solesmes}

\bigtitle{Feria Tertia.}

\gscore[Ad Magnif.]{Ant. 8. G}{an_tristitia_implevit_cor_solesmes}

\bigtitle{Feria Quarta.}

\gscore[Ad Magnif.]{Ant. 6. F}{an_tristitia_vestra_alleluia_solesmes}

\bigtitle{Feria Quinta.}

\gscore[Ad Magnif.]{Ant. 1. f}{an_amen_amen_dico_vobis_iterum_solesmes}

\bigtitle{Feria Sexta.}

\bvmsabbato

%\bigtitle{Sabbato ante Dominicam IV post Pascha.}

\bigtitle{Sabbato.}

\rubrique{Ant. \normaltext{Allelúia.} Ps. \normaltext{Benedíctus Dóminus,} cum reliquis, **.}

\capitulum \label{cap_dom_4_post_pascha}

\scripture{Jac. 1, 17.}
\lettrine{C}{a}ríssimi: Omne datum óptimum, et omne donum perféctum desúrsum est, descéndens a Patre lúminum,~† apud quem non est transmutátio,~* nec vicissitúdinis obumbrátio.

\textes{ad_regias_vv_rr}

\gscore[Ad Magnif.]{Ant. 1. a3}{an_vado_ad_eum_et_nemo_solesmes_1961}

\rubrique{Oratio ut infra ad Vesperas Dominicæ, \normaltext{\pageref{or_dom_4_post_pascha}.}}

\bigtitle{Dominica IV post Pascha.}

\rubrique{Ant. \normaltext{Allelúia.} Ps. de Dominica. Capitulum \normaltext{Caríssimi: Omne datum, \pageref{cap_dom_4_post_pascha},} ut supra.}

\textes{ad_regias_vv_rr}

\gscore[Ad Magnif.]{Ant. 2. D}{an_vado_ad_eum_sed_quia_solesmes_1961}

\oratio \label{or_dom_4_post_pascha}

\lettrine{D}{e}us, qui fidélium mentes uníus éfficis voluntátis:~†  da pópulis tuis id amáre quod prǽcipis, id desideráre quod promíttis;~* ut inter mundánas varietátes, ibi nostra fixa sint corda, ubi vera sunt gáudia. Per Dóminum nostrum.

\bigtitle{Feria Secunda.}

\gscore[Ad Magnif.]{Ant. 7. a}{an_ego_veritatem_solesmes}

\bigtitle{Feria Tertia.}

\gscore[Ad Magnif.]{Ant. 8. G}{an_cum_venerit_paraclitus_spiritus_solesmes}

\bigtitle{Feria Quarta.}

\gscore[Ad Magnif.]{Ant. 5. a}{an_adhuc_multa_solesmes}

\bigtitle{Feria Quinta.}

\gscore[Ad Magnif.]{Ant. 8. G}{an_non_enim_loquetur_solesmes}

\bigtitle{Feria Sexta.}

\bvmsabbato

%\bigtitle{Sabbato ante Dominicam V post Pascha.}

\bigtitle{Sabbato.}

\rubrique{Ant. \normaltext{Allelúia.} Ps. \normaltext{Benedíctus Dóminus,} cum reliquis, **.}

\capitulum \label{cap_dom_5_post_pascha}

\scripture{Jac. 1, 22 – 24.}
\lettrine{C}{a}ríssimi: Estóte factóres verbi, et non auditóres tantum: falléntes vosmetípsos.~† Quia si quis audítor est verbi, et non factor: hic comparábitur viro consideránti vultum nativitátis suæ in spéculo:~* considerávit enim se, et ábiit, et statim oblítus est qualis fúerit.

\textes{ad_regias_vv_rr}

\gscore[Ad Magnif.]{Ant. 2. D}{an_usque_modo_solesmes_1961}

\rubrique{Oratio ut infra ad Vesperas Dominicæ, \normaltext{\pageref{or_dom_5_post_pascha}.}}

\bigtitle{Dominica V post Pascha.}

\rubrique{Ant. \normaltext{Allelúia.} Ps. de Dominica. Capitulum \normaltext{Caríssimi:  Estóte factóres, \pageref{cap_dom_5_post_pascha},} ut supra.}

\textes{ad_regias_vv_rr}

\gscore[Ad Magnif.]{Ant. 8. G*}{an_petite_et_accipietis_solesmes_1961}

\oratio \label{or_dom_5_post_pascha}

\lettrine{D}{e}us, a quo bona cuncta procédunt, largíre supplícibus tuis:~†  ut cogitémus te inspiránte quæ recta sunt;~* et, te gubernánte éadem faciámus. Per Dóminum nostrum.

\bigtitle{Feria Secunda in Rogationibus.}

Vesperæ, \rubrique{nisi occurat Festum IX Lectionum, dicuntur de Psalterio cum Oratione Dominicæ præcedentis. Si vero sequatur Festum Simplex, a Capituluo fit de eo cum Commem. Cruce tantum.,t}

\gscore[Ad Magnif.]{Ant. 8. G*}{an_ipse_enim_pater_solesmes}

\rubrique{Oratio \normaltext{Deus, a quo bona, \pageref{or_dom_5_post_pascha}.}}

\bigtitle{Feria Tertia in Rogationibus.}

\gscore[Ad Magnif.]{Ant. 8. G}{an_exivi_a_patre_solesmes}

\rubrique{Oratio \normaltext{Deus, a quo bona, \pageref{or_dom_5_post_pascha}.}}

\biggerrule

\newpage

  \thispagestyle{empty}
% \section[In Ascensionis Domini.]{IN ASCENSIONIS DOMINI.}\header{in ascensionis domini.}

\festum{In Ascensionis Domini.}
 \rank{Duplex I classis cum Octava privilegiata III ordinis.}
 
% \subsection{in i vesperis.}
\invesperis{i}
 \label{i_vesperis_asc}
 \smalltitle{I Antiphona 7. a.}
 \initialscore{an_viri_galilaei_solesmes_1961}
 \psalmus{109}{109_7a}{109_7}

 \gscore[2. Ant.]{8. G*}{an_cumque_intuerentur_solesmes_1961}
  \psalmus{110}{110_8Gstar}{110_8}
 
 \gscore[3. Ant.]{4. A*}{an_elevatis_manibus_solesmes_1961}
  \psalmus{111}{111_4_alt_Astar}{111_4A_Astar}
 
  \gscore[4. Ant.]{8. G*}{an_exaltate_regem_solesmes_1961}
   \psalmus{112}{112_8Gstar}{112_8}
  
   \gscore[5. Ant.]{8. G}{an_videntibus_illis_solesmes_1961}
    \psalmus{116}{116_8G}{116_8}
    
   \capitulum
    \scripture{Act. 1, 1 – 2.}

\lettrine{P}{r}imum quidem sermónem feci de ómnibus, o Theóphile,~† quæ cœpit Jesus fácere et docére usque in diem, qua præcípiens Apóstolis per Spíritum San\-ctum, quos elégit,~* assúmptus est.
    
\hymnus \label{Salutis Humanae}\phantomsection
   \gscore[]{4.}{hy_salutis_humanae_sator_solesmes_1961}
   \smallscore{versiculus_in_vesperis_ascensionis}
   
   \rubrique{(Sic cantatur in die Festi tantum; alias in tono solito.)}
   
       \admagnificat
           \gscore[]{6. F}{an_pater_manifestavi_solesmes_1961}
           
\oratio \label{oratio_ascensionis}\phantomsection

\lettrine{C}{o}ncéde, quǽsumus omnípotens Deus:~† ut qui hodiérna die Unigénitum tuum Redemptórem nostrum ad cælos ascendísse crédimus,~* i\-psi quoque mente in cæléstibus habitémus. Per eúmdem Dóminum.        
           
\hymnusadcompletorium
    \gscore[]{4.}{hy_te_lucis_in_ascensione_solesmes}
    
\rubrique{Sic terminantur omnes Hymni ejusdem metri usque ad Pentecosten, etiam in Officio Sanctorum, nisi aliter notetur.}

\rubrique{In eodem tono cantantur Hymni ad Horas etiam in Festis Sanctorum, usque ad Pentecosten.} 
      
  \invesperis{ii}
  
\rubrique{Omnia ut in I Vesperis præter sequentia:}

%%LU rubric

\admagnificat \label{o_rex_gloriae}
 \gscore[]{2. D}{an_o_rex_gloriae_solesmes_1961}
 
\rubrique{Infra Octavam Ascensionis usque ad Pentecosten, quotidie fit Officium de Ascensione, nisi occurrat Festum IX Lectionum. De Festis vero III Lectionum tantum commemoratio.}

\bigtitle{Sabbato infra octavam Ascensionis.}

%\header{sabbato infra octavam ascensionis.}

\rubrique{Antiphonæ et Psalmi ut in I Vesperis Festi, \normaltext{\pageref{i_vesperis_asc}.}}

\capitulum \label{cap_dom_infra_oct_asc}
\scripture{I Petr. 4, 7 – 8.}

\lettrine{C}{a}ríssimi: Estóte prudéntes, et vigiláte in oratiónibus.~† Ante ómnia autem mútuam in vobismetípsis caritátem contínuam habéntes:~* quia cáritas óperit multitúdinem peccatórum.

\rubrique{Hymnus \normaltext{Salútis humánæ Sator, \pageref{Salutis Humanae}.}}

\vv Dóminus in cælo, allelúia.
      
\rr Parávit sedem suam, allelúia.

\smalltitle{Ad Magnificat, Antiphona.}\label{cum_venerit}\phantomsection
 \gscore[]{8. G}{an_cum_venerit_paraclitus_solesmes_1961}
      
\oratio \label{oratio_dom_ascensionis}

\lettrine{O}{m}nípotens sempitérne Deus:~† fac nos tibi semper et devótam gérere voluntátem,~* et majestáti tuæ sincéro corde servíre. Per Dóminum.
      
\textes{com_ascensionis}
      
      \bigtitle{Dominica infra octavam Ascensionis.}
      
%      \header{dominica infra octavam ascensionis.}
      
\rubrique{Antiphonæ et Psalmi ut in I Vesperis Festi, \normaltext{\pageref{i_vesperis_asc}.}}

 \rubrique{Capitulum \normaltext{Caríssimi: Estóte prudéntes,} ut supra, \normaltext{\pageref{cap_dom_infra_oct_asc}.}}
   
   \rubrique{Hymnus \normaltext{Salútis humánæ Sator, \pageref{Salutis Humanae}.}}

      \gscore[Ad Magnif.]{Ant. 8. G}{an_haec_locutus_sum_vobis_solesmes_1961}
      
\rubrique{Oratio \normaltext{Omnipótens sempitérne} ut supra, \normaltext{\pageref{oratio_dom_ascensionis}.}}
      
      \textes{com_ascensionis} 
   
  \rubrique{Si vero in crastinum fiat Officium de Octava, Antiphona et ℣. sumuntur e I Vesperis Festi.}

Feria Quarta. \rubrique{Ad Vesperas, omnia ut in I Vesperis Festi.}

In Octava Ascensionis. \rubrique{Duplex majus. Omnia sicut in die. In II Vesperis non fit commemoratio Officii sequentis diei.}

\rubrique{Duobus sequentibus diebus Officium fit sicut infra Octavam Ascensionis, exceptis Capit., ℣℣., Ant. ad Benedictus et Magnificat, et Oratione, quæ dicuntur de Dominica infra Oct. Ascensionis, ut supra. Et non dicuntur Preces ad Primam et Completorium. Nec fit commem. de Ascensione, neque de Cruce.}

\rubrique{Feria tamen VI fit de quolibet Officio novem Lectionum occurrente vel translato, cum commem. Feriæ, quæ omiititur tantum in Duplici I vel II classis. Ad Vesperas autem, in omnibus Officiis quæ non sint Duplicia I vel II classis, fit commem. Sabbati seq. Ant. \normaltext{Cum vénerit, ℣. Dóminus in cælo,} ut supra, \normaltext{\pageref{cum_venerit}.}}

\rubrique{A Vigilia Pentecostes usque ad Festum Ss. Trinitatis inclusive, si occurrat Festum Duplex I vel II classis, transfertur post prædictum Festum Trinitatis. De aliis vero Duplicibus, de Semiduplicibus ac de Simplicibus fit tantum commemoratio, excepto triduo Pentecoste.}

\biggerrule

\newpage

%%maybe we should think about the macros needed. \newpage and 
%%\thispagestyle{empty} can be integrated into command

\thispagestyle{empty}

 \festum{Dominica Pentecostes.}
% \header{dominica pentecostes.}
 \rank{Duplex I classis cum Octava privilegiata I ordinis.}
 
% %%In Festo Pentecostes according to Vaticana
% 
\invesperis{i.} \label{1_vesperis_pentecostes}

\rubrique{Antiphonæ et psalmi ut in II Vesperis, sed loco ultimi Ps. \normaltext{Laudáte Dóminum,} ut infra.}

\psalmus{116}{116_7c2}{116_7}

%%same as on Epiphany. if we can get cross-references right, we can print it once (or not…)

\capitulum \label{cap_dom_pent}

\scripture{Act. 2, 1 – 2.}

\lettrine{C}{u}m compleréntur dies Pentecóstes, erant omnes discípuli páriter in eódem loco:~† et factus est repénte de cælo sonus, tamquam adveniéntis spíritus veheméntis,~* et replévit totam domum, ubi erant sedéntes.

\hymnusmode{8.} \label{hy_veni_creator_pentecost}
\initialscore{hy_veni_creator_spiritus_solesmes}

\smallscore{versiculus_i_vesperis_pent}

\admagnificat

\gscore[]{1. D}{an_non_vos_relinquam_solesmes_1961}

\hymnusadcompletorium \label{tonus_oct_pentecostes}

\gscore[]{1.}{hy_te_lucis_ante_terminum_pentecost_solesmes_1961}

\rubrique{Sic cantantur et terminatur Hymni per totam Octavam.}

\invesperis{ii.} \label{2_vesperis_pentecostes}

\rubrique{Antiphonæ ut infra. Psalmi de Dominica, qui dicuntur quotidie per Octavam.}

\primaantiphona{3. a2.}

\initialscore{an_dum_complerentur_solesmes_1961}

\psalmus{109}{109_3a2}{109_3a2_g}

\gscore[2. Ant.]{8. G}{an_spiritus_domini_replevit_solesmes}

%%psalm file has a hard hyphen because to correct the kerning, LaTeX loses the hyphenation info…

\psalmus{110}{110_8G}{110_8}

\gscore[3. Ant.]{8. G}{an_repleti_sunt_omnes_solesmes_1961}

\psalmus{111}{111_8G}{111_8}

\gscore[4. Ant.]{1. a3}{an_fontes_et_omnia_solesmes_1961}

\psalmus{112}{112_1a3}{112_1}

\gscore[5. Ant.]{7. c2}{an_loquebantur_ant_solesmes_1961}

\psalmus{113}{113_7c2}{113_7}

%%same as on Epiphany and in the psalter for Paschal Time

%% this rubric needs to be rewritten; it makes less sense if you have the pointed psalm and incipit on the same page.

%%this SHOULD be obvious and has no complement for the Paschal octave.

\rubrique{Capitulum ut in I Vesperis, \normaltext{\pageref{cap_dom_pent}.}}

\rubrique{Hymnus \normaltext{Veni Créator Spíritus \pageref{hy_veni_creator_pentecost}.}}

\smallscore{versiculus_ii_vesperis_pent}

\rubrique{(Sic cantatur in die Festi tantum; alias in tono communi.)}

%%versicles really shouldn't split across a page with just a few notes left
%%may need to adjust white space before (after) use of smallscore

\admagnificat

\gscore[]{1. D}{an_hodie_completi_sunt_solesmes_1961}

\oratio \label{or_dom_pent}

\lettrine{D}{e}us, qui hodiérna die corda fidélium Sancti Spíritus illustratióne docuísti:~† da nobis in eódem Spíritu recta sápere,~* et de ejus semper consolatióne gaudére.
Per Dóminum…in unitáte ejúsdem Spíritus Sancti Deus.

%%… is followed by space in Solesmes ediition
%%only Sunday has the full conclusion with Sancti Deus. Abbreviated on other days.

\rubrique{Infra Octavam Pentecostes, fit Officium ut in die, exceptis Antiphonam ad Magnificat et Oratione. Feria II et III Officium fit duplex; aliis diebus, semiduplex.}

\bigtitle{Feria Secunda.}

\gscore[]{3. a}{an_si_quis_diligit_me_solesmes_1961}

\oratio

\lettrine{D}{e}us, qui Apóstolis tuis Sanctum dedísti Spíritum:~† concéde plebi tuæ piæ petitiónis efféctum:~* ut quibus dedísti fidem, largiáris et pacem.
Per Dóminum…in unitáte ejúsdem Spíritus.

\bigtitle{Feria Tertia.}

\gscore[]{6. F}{an_pacem_relinquo_vobis_solesmes_1961}

\oratio

\lettrine{A}{d}sit nobis, quǽsumus, Dómine, virtus Spíritus Sancti:~† quæ et corda nostra cleménter expúrget,~* et ab ómnibus tueátur advérsis. Per Dóminum…in unitáte ejúsdem Spíritus.

\bigtitle{Feria Quarta Quatuor Temporum.}

%%Vaticana has cœlo, LA1949 caeli, we will go with cæli 

\gscore[]{1. f}{an_ego_sum_panis_vivus_solesmes_1961}

\oratio

\lettrine{M}{e}ntes nostras quǽsumus Dómine, Paráclitus, qui a te procédit, illúminet:~* et indúcat in omnem, sicut tuus promísit Fílius veritátem. Qui tecum vivit et regnat in unitáte ejúsdem Spíritus.

%%Solesmes does not have † here

\bigtitle{Feria Quinta.}

\gscore[Ad Magnif.]{Ant. 8. G*}{an_spiritus_qui_solesmes_1961}

\rubrique{Oratio ut in die, \normaltext{\pageref{or_dom_pent}.}}

%%reprinted in antiphonal, but a label with pageref saves space.

\bigtitle{Feria Sexta Quatuor Temporum.}

\gscore[Ad Magnif.]{Ant. 8. G}{an_paraclitus_autem_solesmes_1961}

\oratio

\lettrine{D}{a}, quǽsumus Ecclésiæ tuæ miséricors Deus:~† ut Sancto Spíritu congregáta,* hostíli nullátenus incursióne turbétur. Per Dóminum…in unitáte ejúsdem Spíritus.
         
\end{document}