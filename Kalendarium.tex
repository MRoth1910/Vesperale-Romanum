% !TEX TS-program = lualatexmk
% !TEX parameter =  --shell-escape
\documentclass[11pt]{book}
%%%%%%%%%%%%%%% STANDARD PACKAGES %%%%%%%%%%%%%%%

%% much is based on the work of Matthias Bry (hereafter Matthias or MB) for the Nocturnale Romanum project. https://github.com/Nocturnale-Romanum/nocturnale-romanum
%% for personal reasons, the commands are a mix of French and English names. Feel free to choose one or the other if you make use of them. This file is subject to change.

%%%%%%%%%%%%%%% INDICES %%%%%%%%%%%%%%%

%%\usepackage{imakeidx}
%%
%%\indexsetup{level=\section*,toclevel=section,noclearpage,othercode=\footnotesize\thispagestyle{empty}}
%%\makeindex[name=H,title=Index Hymnorum, columns=2,columnseprule]
%%\makeindex[name=A,title=Index Antiphonarum, columns=2,columnseprule]
%%%%\makeindex[name=R,title=Index Responsoriorum, columns=2,columnseprule]
%%\makeindex[name=P,title=Index Psalmorum, columns=2,columnseprule]
%%\makeindex[name=T,title=Toni Communes, columns=2,columnseprule]
%%\makeindex[name=F,title=Index Festorum, columns=2,columnseprule]

%% This is the format of the recent Solesmes books.
\usepackage[paperheight=205mm,paperwidth=135mm]{geometry}

\usepackage{fontspec}
\usepackage[english,french,ecclesiasticlatin.usej,activeacute]{babel} %%added activeacute as an option for convenience especially for Kyrie
     \babelprovide[hyphenrules=latin]{ecclesiasticlatin}
\usepackage{xspace}%% better to use {} after certain macros (or even within definitions) 
     \usepackage{multicol}
%\usepackage{paracol}
%\footnotelayout{m} %% this allows merged footnote if text is in parallel columns (mostly translations)
\usepackage{emptypage} %% The emptypage package prevents page numbers and
% headings from appearing on empty pages.
\usepackage[savepos]{zref}
\usepackage{fancyhdr}
\usepackage{needspace} %%pour réserver de l'espace en bas de page ; ça évite des séparations entre les titres et la partition ou le texte qui les suivent. %%mais  ça génère des problèmes lorsque il faut 2 ou 3 titres l'un après l'autre (date ou jour, nom d'une fête, rang liturgique, office célébré… 
\usepackage[compact]{titlesec}
\usepackage{xcolor}
%\usepackage{xstring}
\usepackage{enumitem}%% psalm textes are in list format modified via this package
%\usepackage{expl3}
\usepackage{etoolbox}
\usepackage{lettrine}
\usepackage{perpage}%%%to reset footnotes at each page%%%
\usepackage{hyperref}
\usepackage{refcount}

%%%%%%%%%%%%%%% HYPHENATION AND TYPOGRAPHICAL CONVENTIONS %%%%%%%%%%%%%%

%%%%%%%%%%%%%%% GEOMETRY %%%%%%%%%%%%%%%
\geometry{bindingoffset=5mm,
inner=10mm,
outer=10mm,
top=12mm,
bottom=15mm,
headsep=3mm,
} %%borrowed from MB, based on the Solesmes books; binding offset TBD. He has inner=15mm instead.
%%will need serious revision

%% General scale of all graphical elements. Also borrowed from MB
%% Values different from 1 are largely untested per MB
%% Used in those commands (e.g. everything FontSpec) that use a scale parameter.
\newcommand{\customscale}{1}


\AtBeginDocument{\setlength{\parindent}{1em}} %%  default is 20 pt, this reduces indent from that of Computer Modern to the font's

\sloppy %% to allow larger spaces which avoid overfull hbox (MB: %% We want to allow large inter-words space 
%% to avoid overfull boxes in two-columns rubrics.)

%%%%%%%%%%%%%%% GREGORIO CONFIG %%%%%%%%%%%%%%%

\usepackage[autocompile]{gregoriotex}

%% sets * and † to font called via fontspec
\def\GreStar{*}
\def\GreDagger{†}

%%prints asterisk instead of star normally inserted by <v>\greheightstar</v>
 \gredefsymbol{greheightstar}{EB Garamond}{asterisk}

%% should prevent flat turning custos into a clef, as Solesmes uses an altered custos in only one chant per Élie Roux.
\gresetcustosalteration{invisible}

%% any chant using gregorioscore in some way will need alteration to account for indexing chant name, psalms etc.

  \NewDocumentCommand{\initialscore}{O{}m}{%
    \grechangedim{initialraise}{0.5cm}{scalable}%
  \greannotation{#1}%
   \grechangestyle{initial}{\fontspec{EB Garamond Initials}\fontsize{45}{45}\selectfont}% larger, illustrated initial for first antiphon
 \gregorioscore{partitions/#2}%
  \grechangedim{initialraise}{0cm}{scalable}}

%produces a score with a smaller initial (default is 40 pt) and annotations like in the Solesmes books; the 1st is optional and can be omitted as needed. %%update 23/5/23 apparently
 \NewDocumentCommand{\gscore}{O{}mm}{%    %%you can also omit the mode and it typesets correctly.
 \greannotation{#1}%
\greannotation{#2}%
\grechangestyle{initial}{\fontsize{28}{28}\selectfont}%
 \gregorioscore{partitions/#3}%
 }

%% score with no initial, e.g psalms, common tones, etc.
\newcommand{\smallscore}[2][y]{
  \gresetinitiallines{0}
  \gregorioscore{partitions/#2} %% à remplacer avec NewDocumentCommand ; les arguments ne marchent pas correctement … que fait le [y] ?
  \gresetinitiallines{1}
}

%%using the new bar-spacing algorithm by default optimizes spacing. Occasionally asterisks or other signs float too far in one direction or the other.

%%Left default is 0.3cm, right default 0.15cm

\NewDocumentCommand{\zerobaroffsettextleft}{}%
{%
\grechangedim{maxbaroffsettextleft}%
{0cm}%
{scalable}%
}

\NewDocumentCommand{\zerobaroffsettextright}{}%
{%
\grechangedim{maxbaroffsettextright}%
{0cm}%
{scalable}%
}

%%%the default for left bar offset is 0.3cm. Right is 0.15cm. This needs to be applied after a score where the bar offset is modified.

\NewDocumentCommand{\resetbaroffsettextleft}{}%
{%
\grechangedim{maxbaroffsettextleft}%
{0.3cm}%
{scalable}%
}

\NewDocumentCommand{\resetbaroffsettextright}{}%
{%
\grechangedim{maxbaroffsettextright}%
{0.15cm}%
{scalable}%
}

% \NewDocumentCommand{\MagAnnotation}{}{%
% \grechangedim{annotationseparation}{-0.01cm}{fixed}%
% } %% pour les antiennes où le chiffre convient mieux aux miniscules (parfois 1, 2, 4, 7…) ; il n'est pas nécessaire pour 6 ou 8.
%%\gresetheadercapture{annotation}{greannotation}{string}

%% A-bove L-ines T-ext shall be italicized and in a smaller font
\grechangestyle{abovelinestext}{\itshape\fontsize{8}{10}\selectfont}
%% for psalm tones etc. Roman/upright text is what the Liber Usualis uses. 
\NewDocumentCommand{\altnormal}{}{\grechangestyle{abovelinestext}{\normalfont\fontsize{8}{10}\selectfont}}
%%  resets ALT font
\NewDocumentCommand{\altitshape}{}{\grechangestyle{abovelinestext}{\itshape\fontsize{8}{10}\selectfont}} 


%% fine-tuning of space between the staff and the text above lines (used for Magnificats; needs to be rethought  and worked into \smallscore
\newcommand{\altraise}{-0.4mm} %% default is -0.1cm %% needs to be fine-tuned for \rubrique command with EB Garamond font. -0.4mm is MB value, previously 1.4mm and -0.2cm  (this value works best for Magnificat but it does not work with common tones)
\grechangedim{abovelinestextraise}{\altraise}{scalable}

\newcommand{\altheight}{0.5cm} %% default is 0.3cm and value must be bigger than text height; this should fix the problem. but needs further investigation. currently 6mm in MB command
\grechangedim{abovelinestextheight}{\altheight}{scalable} %% this is a very finicky command.

%% Essentially what happens is that there is either not enough space above (Magnificat, crashing into words above) or below (common tones, crashing into staff)

%% fine-tuning of space between the staff and the lyrics

%\newcommand{\textraise}{2.8ex} %% default is 3.48471 ex MB 2.8ex
%\grechangedim{spacelinestext}{\textraise}{scalable}

%%% fine-tuning of space between the initial and the annotations
%\newcommand{\annraise}{0mm} %% default is -0.2mm MB 0mm
%\grechangedim{annotationraise}{\annraise}{scalable}

%% fine-tuning the behavior of text placed under bars. We use the so-called "new algorithm" which
%% places the bar in the middle of surrounding notes, and the text in the middle of surrounding text.
%% however, we restrict drastically the deviation of the text from the position of the bar.

%\grechangedim{maxbaroffsettextleft}{0.5mm}{scalable} MB 0.5mm
%\grechangedim{maxbaroffsettextright}{0.5mm}{scalable}


%% allows printing of glyphs from Gregorio score font as text (available glyphs are listed in the Gregorio documentation).
%% above all needed for the indications before certain Magnificat tones
\makeatletter
\def\gretextglyph#1{{\gre@font@music\csname GreCP#1\endcsname}}
\makeatother

%%% FONT %%%
\setmainfont{EB Garamond}[UprightFont=EB Garamond Regular,
ItalicFont= EB Garamond Italic,
BoldFont= EB Garamond Bold,
Ligatures=Rare,
Numbers=Proportional,
Numbers=OldStyle]

%% all \kern numbers are based on this and should be removed/adjusted if this font is abandoned.
%% as of 5.17.2024 \kern, although more semantic, is being replaced systematically with \hspace to better respect LaTeX format

\newfontfamily\symbolfont{liturgy}
  %% text must be written {\symbolfont{+}} or called via a command but may conflict as + is † in gabc. Cross should be \small or even \footnotesize. Font also supports V & R for verse/response, but EB Garamond contains appropriate matching glyphs.

%StylisticSet=6 is long Q. Abandoned but always possible to use again

%%%Roman numerals for the year %% https://tex.stackexchange.com/questions/185548/the-year-in-roman-and-the-month-in-text

\makeatletter
\newcommand{\YEAR}{\@Roman{\the\year}}
\makeatother

%%% TYPOGRAPHICAL AND SYMBOL COMMANDS  %%%%

%% \P is pilcrow and is defined in LaTeX as is.
%% Add {\textcolor{gregoriocolor} as first part of command's definition if changing to red.

%%for proper spacing of all-caps letters to prevent clashes, e.g. serif strokes touching.

\NewDocumentCommand\capspace{m}{%
{\addfontfeature{LetterSpace=5.0}%
{%
#1}%
}}

%%for smallcaps

\NewDocumentCommand\scspace{m}{\textsc{{{\addfontfeature{LetterSpace=5.0}{#1}}}}}

%% macro to print Alleluia for versicles.
\NewDocumentCommand{\tpalleluia}{}{(\textit{T.P.} Allelúia.)}

\newcommand{\specialcharhsep}{3pt} % space after invoking R/ or V/ or A/ outside rubrics from Matthias with value of 3mm
%%3mm – note unit— is value from MB; changed to pt

%\newcommand{\vv}{\textcolor{gregoriocolor}{\fontspec[Scale=\customscale]{Charis SIL}℣.\nolinebreak[4]\hspace{\specialcharhsep}\nolinebreak[4]}} %% format from MB

\newcommand{\vv}{{\normalfont ℣.\nolinebreak[4]\hspace{\specialcharhsep}\nolinebreak[4]}}
\newcommand{\rr}{{\normalfont ℟.\nolinebreak[4]\hspace{\specialcharhsep}\nolinebreak[4]}}

%\newcommand{\vv}{{\normalfont ℣.~}} %%at the beginning of a line so it shouldn't ever split from the rest but space is needed for macro to work
%\newcommand{\rr}{{\normalfont ℟.~}}

%\newcommand{\cc}{{\fontspec[Scale=\customscale]{FreeSerif}\symbol{"2720}~}}%from Matthias


%% typesets a cross pattée
\NewDocumentCommand{\cc}{}{%
    % Grouping to keep font changes local
    {%
        % Ensure we're not in italics (since liturgy.ttf doesn't have italic)
        \normalfont%
        % Select the font liturgy.ttf
        \symbolfont%
        % Set the size
        \footnotesize%
        % In liturgy.ttf, the plus (+) is a cross pattée
        +%
    }~%    
} 
%%  can replace <+> with <U+2720> (see above) if a suitable font is found (or EB Garamond font is fixed); command will be useful in lieu of typing the Unicode.
%% use <v> tag to insert into a gabc score (usually for the faithful or for blessings, e.g. the font.

%% Same special characters, for in-score use (<sp>V/ R/ A/ +</sp>)

\gresetspecial{V/}{{\fontspec[Scale=\customscale]{EB Garamond}℣.~}}
\gresetspecial{R/}{{\fontspec[Scale=\customscale]{EB Garamond}℟.~}}
\gresetspecial{+}{{\fontspec[Scale=\customscale]{EB Garamond}†~}}
\gresetspecial{*}{\gresixstar}
\gresetspecial{cross}{\cc}

%% Same special characters, for use in rubrics (no added space will keep symbol together with incipit)
%% no scale included

\newcommand{\vvrub}{{\normalfont ℣.~}}
\newcommand{\rrrub}{{\normalfont ℟.~}}

%%rubrics: black italics, smaller than body of psalms etc

\NewDocumentCommand{\rubrique}{m}{%
    {%
        \fontsize{8}{10}%
        \selectfont%
        \textit{%
        %
        #1%
    }%
    }}
    
%% macro to print normal text inside of rubric (name of a chant or prayer, etc.)

\NewDocumentCommand{\normaltext}{m}{%
    {%
        \normalfont%
        #1%
    }%
    }
    
%% in case something should be bolded inside of a rubrique
\NewDocumentCommand{\rubriquegras}{m}{%
    {%
        \normalfont%
\textbf{#1%
}%
}}

%% to print in red instead of italicizing
\NewDocumentCommand{\rouge}{m}{%
\textcolor{gregoriocolor}%
{\fontsize{8}{10}%
\selectfont{%
#1%
}%
}}

%% to print in black within rouge group.

%\NewDocumentCommand{\textenoir}{m}{%

\NewDocumentCommand{\textenoir}{m}{%
\textcolor{black}%
{%
\normalfont%
#1%
}%
}

%% rouge but in italic (this is technically not correct but is contemporary practice of Le Barroux). Here for completeness.
\NewDocumentCommand{\rougeit}{m}{%
\textcolor{gregoriocolor}%
{\fontsize{8}{10}%
\selectfont%
\textit{#1%
}%
}}

%%https://tex.stackexchange.com/questions/156540/spacing-between-single-line-paragraph-with-lettrine%
%\pretocmd{\lettrine}{\checklettrine}{}{} 
%\newcommand{\checklettrine}{%
%  \ifnum\prevgraf<2 \vspace{\baselineskip}\fi
%}

%%this value seems more balanced with this font than the definition provided in the ecclesiasticlatin documentation.

%typesets a horizontal rule like on the page with the prayers before and after the office.
%%If you're using color at all in your document, you might want to either force \myrule to use black or make it customizable. (from u/Independent-Comb-257 on Reddit)
%%thickness started at 0.4pt%%
\NewDocumentCommand{\myrule}{}{%
    \par%
    {%
        \centering%
       \textcolor{black}{\rule{0.3\textwidth}{0.6pt}}%
        \par%
    }%  
}  

%%typesets a horizontal rule like on the page with the prayers before and after the office.
%%thickness started at 0.4pt

\NewDocumentCommand{\biggerrule}{}{%
    \par%
    {%
        \centering%
        \textcolor{black}{\rule{0.75\textwidth}{0.6pt}}%
        \par%
    }%
    }

%% macro to format psalm text. Revised May 17, 2024 to use `enumerate` rather than inserting numbers in .tex file while usng `itemize` and to specify start value via `enumitem` package. 
%% 
%% these values reflect Liber Usualis but do not appear as if typed on a typewriter. Single words (particularly monosyllables) may need adjustment.
%%leftmargin=* and itemindent=* appear to be the way to a hanging indent like the modern antiphonal (but that looks like crap)
\NewDocumentCommand{\pstexte}{m}{%
    \smallskip%
    \noindent%
    \begin{enumerate}[%
			label=\arabic*.,%
%			align=left,%
			leftmargin=10pt, %
			itemindent=15pt, %
			labelsep=3pt, %
			labelwidth=0pt,
			rightmargin=0pt, %
			parsep=0pt, %
			topsep=0pt, %
			itemsep=0pt,%
			start=2]
       \input{psaumes/#1}
    \end{enumerate}}
    
    %%same as above but for the printing of the Magnificat since it always has 2 chant verses. (As of 5.17.2024) it seems more reasonable to do this unless there is an even better way to do so  to avoid 4 arguments per invocation of \psalmus, siince that 3rd argument will just about always be {2}.
    
%%  the problem is that this repeats a bunch in the Magnificat section and that the Nunc Dimittis sometimes begins on 3.
    \NewDocumentCommand{\magnificattexte}{m}{%
    \smallskip%
    \noindent%
    \begin{enumerate}[%
			label=\arabic*.,%
%			align=left,%
			leftmargin=10pt, %
			itemindent=15pt, %
			labelsep=3pt, %
			labelwidth=0pt,
			rightmargin=0pt, %
			parsep=0pt, %
			topsep=0pt, %
			itemsep=0pt,%
			start=3]
       \input{psaumes/#1}
    \end{enumerate}}
    
     %% Command de Matthias Bry modifié, prints psalm incipit score with the text
     %% #1 is psalm number (so an integer from 1 to 150
%% #2 is the gabc file somethng like #1_7a.gabc — if there is a way to append a variable mode and ending to #1, then that'd be cool
%% #3 is the psalm file name described above (and which is confusingly a different argument number when the command is nested): something like 109_7.tex

        \NewDocumentCommand{\psalmus}{mmm}{
	\needspace{4\baselineskip}%
	\smalltitle{Psalmus #1.}%            %% needspace plus smalltitle with \vspace{baselineskip} adds a lot of white space
	\smallscore[n]{#2}%
	\pstexte{#3}}
    
    %%r5.17.2024: removed obsolete, never-really-used version for multicolumn formatting inappropriate on such a small sheet
    
%% sd=semiduplex. especially for Dominica ad Vesperas in the Psalterium (the only example which comes to mind, in fact!)
 \NewDocumentCommand{\sdpsalmus}{mm}{
	\needspace{4\baselineskip}%
		\smalltitle{Psalmus #1.}%  needspace plus smalltitle with \vspace{baselineskip} adds a lot of white space
	\pstexte{#2}%
}

%%identical to \pstexte EXCEPT we need to sometimes start the canticle on 3 such as for Compline of the Paschal Vigil
\NewDocumentCommand{\cantiquetexte}{mm}{%
    \smallskip%
    \noindent%
    \begin{enumerate}[%
			label=\arabic*.,%
%			align=left,%
			leftmargin=10pt, %
			itemindent=15pt, %
			labelsep=3pt, %
			labelwidth=0pt,
			rightmargin=0pt, %
			parsep=0pt, %
			topsep=0pt, %
			itemsep=0pt,%
			start=#1]
       \input{psaumes/#2}
    \end{enumerate}}

    %% Command de Matthias Bry modifié, prints canticle incipit score with the text and scriptural reference
    
%%    \vspace{-\baselineskip} essentially removes the spacing from \smalltitle but enough versus too much space or widows/orphans is hard (as of 5.17.2024)

  %% #1 is Canticle name in Latin, e.g. (and most often) Canticum Simeonis.
%% #2 is the Scriptural citation from the Vulgate as given in the antiphonal/Solesmes editions.
%% #3 is the gabc file somethng like #1_7a.gabc — if there is a way to append a variable mode and ending to #1, then that'd be cool
%% #4 is the verse at which the list (verses wth \item) starts on  (pretty much always 2 or 3)
%% #5 is the psalm file name described above (and which is confusingly a different argument number when the command is nested): something like 109_7.tex
    \NewDocumentCommand{\canticum}{mmmmm}{%
	\needspace{4\baselineskip}%
	\smalltitle{Canticum #1.}%
	\vspace{-\baselineskip}%
	\smalltitle{\textit{#2.}}%
	\smallscore[n]{#3}%
	\cantiquetexte{#4}%
	{#5}%
}


%%in particular for canticle at OPBMV and Compline of dead for 1 November after II Vespers and Vespers of dead
  \NewDocumentCommand{\canticumspecial}{mmm}{
	\needspace{4\baselineskip}%
	\smalltitle{#1.}%    %%filename        %% needspace plus smalltitle with \vspace{baselineskip} adds a lot of white space
	\smallscore[n]{#2}% %%file
	\cantiquetexte{#3}% %%verse number, can be 2 or 3, in case of Compline of the dead
	}

% Could use, before \smallscore \grecommentary taking an argument instead but this sits too low on Nunc Dimittis
    
 %% macro to print any additional text (Capitiulum, oratio, rubrics)
 \NewDocumentCommand{\textes}{ m }{%
    \input{textes/#1}%
    }
    
    
    %% SC work for page headers and for secondary headings but not so much for things like "Dominica…" and certainly not the feast name%%
    
    %%%% Headers%%%%
    \pagestyle{fancy}
\fancyhead{}
\fancyfoot{}
\renewcommand{\headrulewidth}{0pt}

\fancyhead[RO]{\small\rightmark\hspace{0.5cm}\thepage}
\fancyhead[LE]{\small\thepage\hspace{0.5cm}\leftmark}

\newcommand{\setheaders}[2]{
	\renewcommand{\rightmark}{{\sc#2}}
	\renewcommand{\leftmark}{{\sc#1}}
}
\setheaders{}{}

%%note as of 5.17.2024: we need to learn how to get fancyhdr or titlesec or both to work in the fulll book


 %% TITLE COMMANDS %%%% had been dead in NR version of the same, but Matthias says this isn't a good idea.
 %%need to figure out multiple section headers AND a chapter header so that everything fits…
 
%%these are defined with \sc but we want full caps for the first two in particular. Also, Missing number, treated as 0 occurs when section titleformat is copied… 
 
 \titleformat{\chapter}[display]{\Huge\filcenter\sc\addfontfeature{LetterSpace=5.0}}{}{0pt}{} %%
\titleformat{\section}[display]{\huge\filcenter\sc\addfontfeature{LetterSpace=5.0}}{}{}{} %%
\titleformat{\subsection}[display]{\LARGE\filcenter\sc\addfontfeature{LetterSpace=5.0}}{}{}{} %%originally \large, then \Large
\setcounter{secnumdepth}{0}
\titlespacing*{\chapter}{0pt}{-25pt}{20pt} %%this should reduce spacing before chapter title while putting it flush with the top of the text area
%% see https://tex.stackexchange.com/questions/63390/how-to-decrease-spacing-before-chapter-title
%%\titlespacing*{\chapter}{0pt}{-50pt}{40pt}  %%this puts chapter title in HEADER which we do not want
\titlespacing{\section}{0pt}{*0}{*0}

%%need section titleformat more appropriate for weekdays of psalter and minor feasts 
% See https://tex.stackexchange.com/questions/623797/multiple-section-styles-in-same-document.

\addto\captionsecclesiasticlatin{\renewcommand\contentsname{\centerline{INDEX GENERALIS.}}}

%this command still needs work
\NewDocumentCommand{\header}{m}{\setheaders{{\scshape\addfontfeature{LetterSpace=5.0}#1}}{{\scshape\addfontfeature{LetterSpace=5.0} #1}}}

\NewDocumentCommand{\smalltitle}{m}{%
\needspace{5\baselineskip}%  %% very small if there are neumes above the staff, including flats in mode 2, e.g. O Doctor optime
\vspace{\baselineskip}%
 {\centering #1\par}%
 }
 
  %% bigtitle as written adds too much space below (or rather the combination of needspace is too large with following  text)
  
  %% possibile to have something that is a macro which takes away a fixed amount? and then I can change the definition later (though as of 17 dec 2023), cf. the screenshot from the Praglia code
 
 \NewDocumentCommand{\bigtitle}{m}{%
\needspace{5\baselineskip}%
\vspace{\baselineskip}%
 {\centering{\Large#1}\par}%
 }
 
 %%formatting date on major feasts of sanctoral cycle or moveable feasts that follow Easter where this is listed in antiphonal.

%  \NewDocumentCommand{\biglitdate}{m}{%
%\needspace{5\baselineskip}  %% very small if there are neumes above the staff, including flats in mode 2, e.g. O Doctor optime
%\vspace{\baselineskip}
% {\centering{\large #1}}\par} %}


%%formatting date of feasts where this listed at top of page
\NewDocumentCommand{\litdate}{m}{%
\smalltitle{Die #1.}%
}

%for most/ordinary feasts of Sanctorale 
%%"as long as text isn't long enough to over-print number according to David Carlisle
\NewDocumentCommand{\datefeast}{mm}{%
  \par
  \noindent
  \makebox[0pt][l]{#1.}%
  \makebox[\textwidth]{\Large #2.}%
  }
  
  %%for feasts such as Imposition of Stigmata of St Francis on Sept 17.
  %%first two arguments as above, Dedication of the Lateran on Nov 9
  %%break line as needed and the remaining text is the 3rd argument
   \NewDocumentCommand{\longdatefeast}{mmm}{%
  \par
  \noindent
  \makebox[0pt][l]{#1.}%
  \makebox[\textwidth]{\Large #2}%
\hfill\makebox[\textwidth]{\large #3.}%
  }
  
%%  alternatively
%%  \NewDocumentCommand\datefeast{mm}{%
%%\noindent #1.\hfill{}{\Large#2.}\hfill{}%
%%}


%%formatting for Sunday feasts : inter, post… name is for consistency
\NewDocumentCommand{\dominicadate}{m}{%
\smalltitle{Dominica #1.}%
}

%%formatting for feasts that fall on a weekday fixed to some other thing, like Corpus Christi, Sacred Heart, BVM in Passiontide, and the new feasts found in the appendix (Feria Roman numeral in Latin: inter, post… name is for consistency
\NewDocumentCommand{\feriadate}{m}{%
\smalltitle{Feria #1.}%
}


%%for most important feasts (as of 16 dec 2023 leaving section alone as is, but it's similar styling)
\NewDocumentCommand{\festum}{m}{%
\needspace{5\baselineskip}%
\vspace{\baselineskip}%
 {\centering{\LARGE%
 {\MakeUppercase{%
 {\capspace{#1.}}
 }%
 }%
\par}%
 }%
 }

%%for ranks of feast; easiest to use rank for duplex 1 or 2 classis because the number changes, the octaves change.

%%probably need to reduce vspace used

\NewDocumentCommand{\rank}{m}{%
\needspace{5\baselineskip}%
\vspace{\baselineskip}%
 {\centering{\textit{\large #1.}}\par}%
}

%%looks like xparse formatting changed so o just has No Default Value whereas O{} is that OR you can insert something in braces!
%%answer thanks to David Carlisle
\NewDocumentCommand{\duplexclassis}{mo}{%
\needspace{5\baselineskip}%
\vspace{\baselineskip}%
 {\centering\textit{\large Duplex #1 classis\IfValueT{#2}{ #2}.}\par}%
}


\NewDocumentCommand{\duplexmajus}{}{%
\needspace{5\baselineskip}%
\vspace{\baselineskip}%
 {\centering{\textit{\large Duplex majus.}}\par}%
}

%%is Duplex even specified.as it's the default?
\NewDocumentCommand{\duplex}{}{%
\needspace{5\baselineskip}%
\vspace{\baselineskip}%
 {\centering{\textit{\large Duplex.}}\par}%
}

\NewDocumentCommand{\semiduplex}{}{%
\needspace{5\baselineskip}%
\vspace{\baselineskip}%
 {\centering{\textit{\large Semiduplex.}}\par}%
}

\NewDocumentCommand{\simplex}{}{%
\needspace{5\baselineskip}%
\vspace{\baselineskip}%
 {\centering{\textit{\large Simplex.}}\par}%
}

%%for feasts where the third verse (third part of 1st verse in modern terms) is changed (Mutatur tertius vers.)

\NewDocumentCommand{\mtv}{}{%
{\normalfont{(m.t.v.)}%
}%
}

%%e.g. St Joachim, D2C
\NewDocumentCommand{\duplexclassismtv}{mo}{%
\needspace{5\baselineskip}%
\vspace{\baselineskip}%
 {\centering\textit{\large Duplex #1 classis.} \mtv \par}%
}

\NewDocumentCommand{\duplexmtv}{}{%
\needspace{5\baselineskip}%
\vspace{\baselineskip}%
{\centering{\textit{\large Duplex.} \mtv \par}%
}%
}

\NewDocumentCommand{\semiduplexmtv}{}{%
\needspace{5\baselineskip}%
\vspace{\baselineskip}%
{\centering{\textit{\large Semiduplex.} \mtv \par}%
}%
}

\NewDocumentCommand{\simplexmtv}{}{%
\needspace{5\baselineskip}%
\vspace{\baselineskip}%
{\centering{\textit{\large Simplex.} \mtv \par}%
}%
}

%% The needspace package inserts white space that we don't want!
%%especially for Sab/Dom PP which has sentence-case text below main title of page
\NewDocumentCommand{\subtitle}{m}{%
 {\centering{\Large #1}\par%
 }%
 }

%%to replace original idea of subsection; will need work on the spacing (the needspace and vspace both might be too big
%%leaving subsection as is for now (16 dec 2023), for reference and in case I used it and didn't remember it
%% argument here is i or ii; will need another command if the vespers are identical.
 \NewDocumentCommand{\invesperis}{m}{%
\needspace{5\baselineskip}%
\vspace{\baselineskip}%
 {\centering{\LARGE%
 {\scspace{in #1 vesperis.}%
 }%
 }\par}%
}

%%for double major and below
 \NewDocumentCommand{\invesperisminor}{m}{%
\needspace{5\baselineskip}%
\vspace{\baselineskip}%
 {\centering{\Large In #1 Vesperis.}\par}%
 }

\NewDocumentCommand{\primaantiphona}{m}{%
\smalltitle{I Antiphona. #1}%
}

%%for vel where this occurs between texts or chants

\NewDocumentCommand{\vel}{}{%
{\rubrique{Vel:}%
}%
}

%% for prayers, e.g. at beginning (and end) of office.
\NewDocumentCommand{\secreto}{}{%
Pater noster. \rubrique{secreto.}%
}

\NewDocumentCommand{\pateravedeus}{}{%
Pater noster et Ave María. ℣. Deus in adjutórium.%
}

\NewDocumentCommand{\pateravecredo}{}{%
Pater noster, Ave María, \textit{et} Credo.%
}

\NewDocumentCommand{\lectiobrevis}{m}{%
{\quad Lectio brevis.%
\hfill \textit{#1}%
\hspace*{1em}%
\par%
}%
}

%%%original commands
%\NewDocumentCommand{\capitulum}{}{%
%\smalltitle{Capitulum.}%
%}

%%for Scriptural references, originally of the chapter but now primarily Magnificat in psalter
\NewDocumentCommand{\scripture}{m}{%
{\raggedleft{\textit{#1}%
}%
\par}%
}

%%above chapter (Capitulum)
%%should indent by 1 em or a quad and then preserve space at end of line without needing to resort to table
%%imitation,sort of, of Liber antiphonarius style; AM1934 indentation is bigger
%%argument is Scriptural reference
\NewDocumentCommand{\capitulum}{m}{%
{\needspace{5\baselineskip}%
\vspace{\baselineskip}%
\quad Capitulum.%
\hfill \textit{#1.}%
\hspace*{1em}%
\par%
}%
}

%%for use in commons
%%#1 is Pro… e.g. Pro Virgine Martyre.
%%#2 is Scriptural citation Prov. 31, 10 – 11.
%%one problem is that hfill doesnt really CENTER it

\NewDocumentCommand{\specialcapitulum}{mm}{%
{\needspace{5\baselineskip}%
\vspace{\baselineskip}%
\quad #1%
\hfill Capitulum.%
\hfill \textit{#2.}
\hspace*{1em}%
\par%
}%
}


%%above Hymn if listed this way in the antiphonal (Hymnus)
\NewDocumentCommand{\hymnus}{}{%
\smalltitle{Hymnus.}%
}

%%especially for common of apostles etc. in paschal time
\NewDocumentCommand{\hymnusspecial}{mO{}}{%
\smalltitle{#1, Hymnus. #2}%
}

%%for proper of saints where the hymn is proper
\NewDocumentCommand{\hymnusbothvespers}{}{%
\smalltitle{In utrisque Vesperis, Hymnus.}%
}

\NewDocumentCommand{\hymnusmode}{m}{%
\smalltitle{Hymnus. #1}%
}

%%for 2nd tone of hymn
\NewDocumentCommand{\altertonus}{}{%
\smalltitle{Alter Tonus.}%
}

%%for 2nd tone of hymn where Alius is employed %%optional argument is in particular for Iste Confessor tones of which there are 4 per common of confessors (bishop and not a bishop)
\NewDocumentCommand{\aliustonus}{O{}}{%
\smalltitle{Alius Tonus. #1}%
}

%%above Te lucis

%%for, e.g. St Michael (not present in LA but preesnt in LU), St John Cantius. Hymn uses what would be ALT but we wish to avoid this feature. Unlike Ave Maris Stella rubric,
%% which is never in LA, this is, at least once, so we include it.

\NewDocumentCommand{\sequensconclusio}{}{%
\rubrique{Sequens conclusio numquam mutatur.}%
}

\NewDocumentCommand{\hymnusadcompletorium}{}{%
\smalltitle{Tonus Hymni ad Completorium.}%
}

%%above Magnificat antiphon where this is done in Liber antiphonarius (Or Vaticana Vesperale) (Sundays to be determined because there are so many of them; Liber Usualis and ferial format may be best)
\NewDocumentCommand{\admagnificat}{}{%
\smalltitle{Ad Magnificat, Antiphona.}%
}

%%to pair with initial score in cases where only Mag antiphon is proper e.g. on Feb 2
\NewDocumentCommand{\admagnificatmode}{m}{%
\smalltitle{Ad Magnificat, Antiphona. #1}%
}

%%for use in collects etc. %%\@ would lead to too narrow of a space (see LU)
%%needs to be followed by {}
\NewDocumentCommand{\nomen}{}{%
\textit{N.}%
}

%%above collect (Oratio)
\NewDocumentCommand{\oratio}{}{%
\smalltitle{Oratio.}%
}

%%for use in the commons
\NewDocumentCommand{\aliaoratio}{}{%
\smalltitle{\textit{Alia Oratio.}}%
}

%%for use in the commons
\NewDocumentCommand{\specialoratio}{m}{%
\smalltitle{#1 \textit{Oratio.}}%
}

%%%these are used before II Vespers
\NewDocumentCommand{\omniapraeter}{}{%
{\rubrique{Omnia ut in I Vesperis, præter sequentia.}%
}%
}

\NewDocumentCommand{\ivesperisrubrique}{}{%
{\rubrique{Antiphonæ, Psalmi, Capit.\@ et Hymnus ut in I Vesperis.}%
}%
}

\NewDocumentCommand{\sedlocoultimi}{mm}{%
{\rubrique{Antiphonæ et Psalmi ut in #1 Vesperis, sed loco ultimi Ps.\@ \normaltext{#2,} ut infra.}%
}%
} 

%%for referring to chapter and/or hymn and/or verse placed at I or II Vespers (where II has different parts)
\NewDocumentCommand{\caphymn}{m}{%
\rubrique{Capitulum et Hymnus ut in #1 Vesperis.}%
}

\NewDocumentCommand{\caphymnvv}{m}{%
\rubrique{Capitulum, Hymnus, et ℣.\@ ut in #1 Vesperis.}%
}

\NewDocumentCommand{\hymnusut}{m}{%
\rubrique{Hymnus ut in #1 Vesperis.}%
}

%%a few doubles even are permanently impeded at II Vespers by another (higher-ranking) double the next day, e.g. St Clement, St Cecilia…
\NewDocumentCommand{\procomm}{m}{%
\smalltitle{Pro comm.\@ S.\@ #1, Antiphona.}%
}

%%for commemoration of following at II Vespers of major feasts.
\NewDocumentCommand{\commemoration}{}{%
\rubrique{Et fit comm.\@ sequentis.}%
}

%%for Vespers of the following
\NewDocumentCommand{\vespsequenti}{}{%
\rubrique{Vesp.\@ de sequenti.}%
}

%%for Vespers of the following at II Vespers
\NewDocumentCommand{\commsequentis}{}{%
\rubrique{In II Vesperis, comm.\@ sequentis.}%
}

%%for Vespers of the following where the commemoration of the previous day is still made
%%fixed latin sequentis to sequenti Sept 4 2024
\NewDocumentCommand{\vespsequentiscomm}{}{%
\rubrique{Vesp.\@ de sequenti, comm.\@ præced.}%
}

%%for Vespers of the following day beginning at the chapter (doubles that concur with each other)
%%changed 4 Sept 2024 from de to a because while de is probably OK Latin, I see a more often in the books.
\NewDocumentCommand{\capitdeseq}{}{%
\rubrique{Vesp. a capit.\@ a seq., comm.\@ præced.}%
}


%%commemoration of the Lenten feria
\NewDocumentCommand{\quadcommferiae}{}{%
\rubrique{In Quadrag.\@ commem. feriæ.}%
}

\NewDocumentCommand{\commferiae}{}{%
\rubrique{Commem.\@ feriæ.}%
}

%%for Fridays where Vespers is of Our Lady on Saturday, unless the next day is a feast of IX Lessons.
\NewDocumentCommand{\bvmsabbato}{}{%
\rubrique{Ad Vesperas, nisi occurrat Festum IX Lectionum, a Capitulo fit de Sancta Maria in Sabbato.}%
}

%%for feasts of the BVM
\NewDocumentCommand{\infestisbmv}{}{%
\rubrique{Antiphonæ, Psalmi et Capitulum ut in festis B.M.V., \normaltext{\pageref{vr_NN_commune_bmv}.}%
}%
}

\NewDocumentCommand{\amsrubrique}{}{%
\rubrique{Hymnus \normaltext{Ave Maris Stella, \pageref{M-hy_ave_maris_stella_mode_1_LA1949}.}%
}%
}

\NewDocumentCommand{\vigiliarubrique}{m}{%
\rubrique{De Vigilia #1 nihil fit nisi in Missa.}%
}

%%%%%%%SPACING%%

%\NewDocumentCommand{\baselineskipvspace}{m}{%
%\vspace{#1\baselineskip}%
%}

%%does this count as a magic number?

\NewDocumentCommand{\baselineskipvspace}{m}{%
\vspace{1\baselineskip}%
}

\NewDocumentCommand{\negativebaselineskipvspace}{m}{%
\vspace{-1\baselineskip}%
}


%% for litanies and other things like antiphon O Doctor Optime

\makeatletter  
\newcounter{score}
\newcounter{tabstop}[score]
\newcommand{\grealign}{%
	\@bsphack%
	\ifgre@boxing\else%
		\kern\gre@dimen@begindifference%
		\stepcounter{tabstop}%
		\expandafter\zsavepos{stop-\thescore-\thetabstop}%
		\kern-\gre@dimen@begindifference%
	\fi%
	\@esphack%
}

\newcommand{\setstops}{%
  \gdef\nstabbing@stops{%
    \hspace*{-\oddsidemargin}\hspace{-1in}%
    \hspace*{\zposx{stop-\thescore-1} sp}\=%
  }%
  \count@=\@ne
  \loop\ifnum\count@<\value{tabstop}%
    \begingroup\edef\x{\endgroup
      \noexpand\g@addto@macro\noexpand\nstabbing@stops{%
        \noexpand\hspace{-\noexpand\zposx{stop-\thescore-\the\count@} sp}%
        \noexpand\hspace{\noexpand\zposx{stop-\thescore-\the\numexpr\count@+1} sp}\noexpand\=%
      }%
    }\x
    \advance\count@\@ne
  \repeat
  \nstabbing@stops\kill
}
\makeatother

\newenvironment{nstabbing}
  {\setlength{\topsep}{0pt}%
   \setlength{\partopsep}{0pt}%
   \tabbing%
   \setstops}
  {\endtabbing\stepcounter{score}}

%%FOOTNOTES%%%

%%from perpage package \MakePerPage{footnote} as footnotes are just per page as it says
\MakePerPage{footnote}

%%PHANTOM SECTION MANIPULATION%%
\providecommand\phantomsection{} %% should make hyperref work properly %%needed with addcontentsline, not with section etc alone (needed if running hyperref)

\setcounter{tocdepth}{2} %% probably should hide subsections from TOC of most documents

%%%% command to wrap printindex and set the headers for indices
%%\newcommand{\cprintindex}[2]{
%%	\setheaders{Indices}{#2}
%%	\pagestyle{fancy}
%%	\thispagestyle{empty}
%%	\printindex[#1]
%%}

%%preface
      \setlength{\columnsep}{3pc}
%\setlength{\columnsep}{10pt}
\setlength\columnseprule{0.3pt}

%%for hymni antiqui
\NewDocumentEnvironment{hymnusmulticol}{}
 {\begin{multicols}{2}}
    {\end{multicols}}

\NewDocumentEnvironment{enpars}{}
  {\begin{otherlanguage*}{english}}
  {\par\end{otherlanguage*}}
  
  \NewDocumentEnvironment{frpars}{}
  {\begin{otherlanguage*}{french}}
  {\par\end{otherlanguage*}}
  
  \NewDocumentCommand{\hymnusantiquusrubrique}{m}{%
  \rubrique{Ad Completorium, tonus ut in Antiphonario, \normaltext{\pageref{#1}.}%
  }%
  }
  
%%%SUBFILES%%%
\usepackage{xr} %%to allow cross-references between documents%%%
  \usepackage{subfiles}
  
  %% When we start a new subfile (new chapter or section) , 
%% we start on a new page (with blank filling on the previous page) and create a corresponding label.

\newcommand{\customsubfile}[1]{\newpage\label{#1}\thispagestyle{empty}\subfile{#1}}



\usepackage{showframe}

\raggedbottom

\usepackage{ragged2e}
%\usepackage{geometry}
\usepackage[parfill]{parskip}

\usepackage{longtable}
\usepackage{multirow,makecell}

\usepackage{colortbl}

\usepackage{layout}
\pagestyle{plain}
%\setlength{\columnsep}{3pc}
\setlength{\parindent}{0mm}
\setlength{\marginparwidth}{7mm}
\setlength{\marginparsep}{3mm}
\setlength{\headsep}{10pt}

\usepackage{microtype}
\usepackage[defaultlines=2,all]{nowidow}
\usepackage[hyphenation,lastparline,nosingleletter]{impnattypo} 
\usepackage{epsfig}
\spaceskip=1.0\fontdimen2\font plus 3\fontdimen3\font minus 0.8\fontdimen4\font %% this probably needs adjusting as I don't use the same fonts

%% Commands for tables and for rotating the cell text taken from Matthias B. Could be moved to commonheaders.

\renewcommand{\arraystretch}{1.25}
\newcommand{\boldhline}{\noalign{\global\arrayrulewidth1.5pt}\hline\noalign{\global\arrayrulewidth1pt}}
\newcommand{\thinhline}{\noalign{\global\arrayrulewidth0.5pt}\hline\noalign{\global\arrayrulewidth1pt}}
\newcommand{\whiteline}{\noalign{\global\arrayrulewidth4pt}\hline\noalign{\global\arrayrulewidth1pt}}
\newcommand{\STAB}[1]{\begin{tabular}{@{}c@{}}#1\end{tabular}}

%% these commands use Matthias B's commands but in a cleaner and shorter format. Could be moved to commonheaders.

\NewDocumentCommand\hang{}{\setlength{\hangindent}{10pt}}
\NewDocumentCommand\mem{m}{\textit{Comm.} #1}
\NewDocumentCommand\gcolor{m}{\textcolor{gregoriocolor}{#1}} %% requires gregoriotex (declared in commonheaders). Months should be in black italic type if not using any red. Dominical Letters can remain in roman type.

%\NewDocumentCommand\capspace{m}{{\addfontfeature{LetterSpace=5.0}{#1}}}
%
%\NewDocumentCommand\scspace{m}{\textsc{{{\addfontfeature{LetterSpace=5.0}{#1}}}}}

\begin{document}

%%the table is too wide. The line breaks stink, and the leading is bad. Words are in different places in the cell (top, bottom, center) so it looks like trash.

%%it probably needs to be on two pages

%%Calendar entries need better breaks; doesn't break even if a cell extends into the bottom margin. Letters peak over left or right margin (particularly in calendar)

\pagestyle{plain}
%\input{tabella_festorum_mobilium}

%\pagebreak

\normalsize

% TABLE DU CALENDRIER

%%% modeled on the table in the Liber antiphonarius 1949.
%% need a column for Pentecost and then to fill in the dates from the table

%% the original syntax for centering has been fixed. This now puts the leap-year rubric in the right place on the page. Centering each table might work if vertical rules should be aligned (the second table on each page isn't aligned)

{\centering{\large \capspace{KALENDARIUM PERPETUUM}.}\par} \thispagestyle{empty} \markboth{Calendrier}{Calendrier} \nopagebreak \par \nopagebreak\vspace{5mm}\label{kalendarium} %%will need to fix the label later
\setlength\LTleft{0pt}
\setlength\LTright{0pt}
\setlength{\tabcolsep}{5pt}
\renewcommand{\arraystretch}{1.4}
\fontsize{8}{9}\selectfont
%\begin{longtable}{>{\centering}p{0.025\textwidth}|>{\raggedleft}p{0.025\textwidth}|>{\raggedright\arraybackslash}p{0.85\textwidth}}
%\boldhline
%\multirow{1.5}{*}{\STAB{\rotatebox[origin=c]{90}{{\footnotesize \gcolor{L.D.}}}}} & \multirow{1.5}{*}{\STAB{\rotatebox[origin=c]{90}{{\footnotesize \gcolor{Jour}}}}} &  \multicolumn{1}{c}{\multirow{1.75}{*}{{\footnotesize \gcolor{Mois}}}} \\[8.5pt]
%\boldhline
%\null & \null & \null\\[2pt]
%\endfirsthead
%\boldhline
%\multirow{1.5}{*}{\STAB{\rotatebox[origin=c]{90}{{\footnotesize \gcolor{L.D.}}}}} & \multirow{1.5}{*}{\STAB{\rotatebox[origin=c]{90}{{\footnotesize \gcolor{Jour}}}}} &  \multicolumn{1}{c}{\multirow{1.75}{*}{{\footnotesize \gcolor{Mois}}}} \\[8.5pt]
%\boldhline
%\null & \null & \null\\[2pt]
%\endhead
%\endfoot
%\endlastfoot
%\end{longtable}
%% this commented-out code originated from Matthias B's version, but there is no need for headers per the Latin original.
%\myrule

% DEBUT CALENDRIER
%\null &\null &\null &\null\\[1pt]

%% longtable repeats in each month file. This means the settings need to be changed each time unless there is a cleaner way to do this.

%% Currently scshape or capshape is repeated after the abbreviation when S., B. or SS. is used in the title of a feast. This avoids awkward spacing of periods, which fall outside the braces.

%% Line breaks after Comm. are OK per the Liber antiphonarius.

% !TEX TS-program = LuaLaTeX+se
% !TEX root = Kalendarium.tex

{\centering{\normalsize Januarius.}\par}

\begin{longtable}{>{\centering}p{0.025\textwidth}|>{\raggedright}p{0.040\textwidth}|>{\raggedleft}p{0.025\textwidth}|>{\raggedright\arraybackslash}p{0.80\textwidth}}
%\begin{longtable}{>{\centering}p{0.025\textwidth}|>{\raggedleft}p{0.025\textwidth}|>{\raggedright\arraybackslash}p{0.85\textwidth}}
A &Kal. & 1 & \hang \scspace{Circumcisio Domini} et Octava Nativitatis. \textit{Duplex II classis.}\\
 &  &  &  \hang \textit{Dominica inter Circumcisionem et Epiphaniam.}  \scspace{Ss}. \scspace{Nominis Jesu}.  \textit{Duplex II classis.}\\
b &iv & 2 & \hang Octava S. Stephani Protomartyr. \textit{Simplex.}\\
c &iij & 3 & \hang Octava S. Joannis Apost. et Evang. \textit{Simplex.}\\
d &Prid. & 4 & \hang Octava SS. Innocentum Martyrum. \textit{Simplex.}\\
e &Non. & 5 & \hang Vigiliæ Epiphianiæ.  \textit{Semiduplex.} \mem{S. Telesphori Papæ Martyris.}\\
f &viij & 6 & \hang \capspace{EPIPHANIA DOMINI}. \textit{Duplex I classis cum Octava privilegiata II ordinis.}\\
 &  &  & \hang \textit{Dominica infra Octavam Epiphianiæ.} S. Familiæ Jesu, Mariæ, Joseph.  \textit{Duplex majus.} \textit{Commem.} Dominicæ et Octavæ.\\
g & vij & 7 & \hang De Octava Epiphianæ. \textit{Semiduplex.}\\
A & vj & 8 & \hang De Octava. \textit{Semiduplex.}\\
b & v & 9 &  \hang De Octava. \textit{Semiduplex.}\\
c & iv & 10 &  \hang De Octava. \textit{Semiduplex.}\\
d & iij & 11 & \hang De Octava. \textit{Semiduplex.} \mem{S. Hygini Papæ Martyris.}\\
e & Prid. & 12 &  \hang De Octava.  \textit{Semiduplex.}\\
f & Idib. & 13 & \hang Octava Epiphianiæ. \textit{Duplex majus.}\\
g & xix & 14 & \hang S. Hilarii Episc. Conf. et Eccl. Doct. \textit{Duplex.} \mem{S. Felicis Presbyteri Martyris.}\\
A & xviij & 15 & \hang S. Pauli primi Eremitæ Conf. \textit{Duplex.}\\
b & xvij & 16 & \hang S. Marcelli I Papæ Mart. \textit{Semiduplex.}\\
c & xvj & 17 & \hang S. Antonii Abbatis. \textit{Duplex.} \mem{S. Mauri Abb.}\\
d & xv & 18 & \hang Cathedra S. Petri Romæ. \textit{Duplex majus.} \mem{S. Pauli Ap., ac S. Priscæ Virginis et Martyris.}\\
e & xiv & 19 &  \hang SS. Marii, Marthæ, Audifacis et Abachum Martyrum. \textit{Simplex.}\\
f & xiij & 20 & \hang SS. Fabiani Papæ et Sebastiani Martyr. \textit{Duplex.}\\
g & xij & 21 & \hang S. Agnetis Virginis et Martyris. \textit{Duplex.}\\
A & xj & 22 & \hang SS. Vincentii et Anastasii Martyrum. \textit{Semiduplex.}\\
b & x & 23 & \hang S. Raymundi de Peñafort Conf. \textit{Semiduplex.} \mem{S. Emerentianæ Virginis et Martyris.} \\
c & ix & 24 & \hang S. Timothei Episc. Martyris. \textit{Duplex.}\\
d & viij & 25 & \hang Conversio S. Pauli Ap. \textit{Duplex majus.} \mem{S. Petri Ap.}\\
e & vij & 26 & \hang S. Polycarpi Episc. Martyris. \textit{Duplex.}\\
f & vj & 27 & \hang S. Joannis Chrysostomi Episc. Conf. et Eccl. Doct. \textit{Duplex.}\\
g & v & 28 & \hang S. Petri Nolasci Conf. \textit{Duplex.}\\
A & iv & 29 & S. Francisci Salesii Episc. Conf. et Eccl. Doct. \textit{Duplex.}\\
b & iij & 30 & S. Martinæ Virginis et Martyris.  \textit{Semiduplex.}\\
c & Prid. & 31 & \hang S. Joannis Bosco Conf. \textit{Duplex.}
\end{longtable}
%
% !TEX TS-program = LuaLaTeX+se
% !TEX root = Kalendarium.tex

{\centering{{\normalsize Februarius.}\par}}
\begin{longtable}{>{\centering}p{0.025\textwidth}|>{\raggedright}p{0.040\textwidth}|>{\raggedleft}p{0.025\textwidth}|>{\raggedright\arraybackslash}p{0.80\textwidth}}

d & Kal. & 1 & S. Ignatii Episc. Martyr. \textit{Duplex.}\\
e & iv & 2 & \hang \scspace{Purificatio B}. \scspace{Mariæ Virginis}. \textit{Duplex II classis.}\\
f & iij & 3 & \hang S. Blasii Episc. Martyr. \textit{Simplex.}\\
g & Prid. & 4 & \hang S. Andreæ Corsini Episc. Conf. \textit{Duplex.}\\
A & Non. & 5 & \hang S. Agathæ Virginis et Martyris. \textit{Duplex.}\\
b & viij & 6 & \hang S. Titi Episc. Conf. \textit{Duplex.} \mem{S. Dorotheæ Virginis et Martyris.} \\
c & vij & 7 & \hang S. Romualdi Abbatis. \textit{Duplex.}\\
d & vj & 8 & \hang S. Joannis de Matha Conf. \textit{Duplex.}\\
e & v & 9 & S. Cyrilli Episc. Alexandrini Conf. et Eccl. Doct. \textit{Duplex.}\\
f & iv & 10 & \hang S. Scholasticæ Virginis. \textit{Duplex.}\\
g & iij & 11 & \hang Apparitionis B.M.V. Immaculatæ. \textit{Duplex majus.}\\
A & Prid. & 12 & SS. Septem Fundatorum Ord. Servorum B.V.M., Cc. \textit{Duplex.}\\
b & Ibid. & 13 & \\
c & xvj & 14 & \hang S. Valentini Presbyteri Martyris. \textit{Simplex.}\\
d & xv & 15 & SS. Faustini et Jovitæ Martyrum. \textit{Simplex.}\\
e & xiv & 16 & \\
f & xiij & 17 & \\
g & xij & 18 & S. Simeonis Episcopi Martyris. \textit{Simplex.}\\
A & xj & 19 & \\
b & x & 20 & \\
c & ix & 21 & \\
d & viij & 22 & \hang Cathedra S. Petri Antiochæ. \textit{Dupl. maj.} \mem{S. Pauli Ap.}\\
e & vij & 23 & \hang S. Petri Damiani Episc. Conf. et Eccl. Doct. \textit{Duplex.} \mem{Vigiliæ.}\\
f & vj & 24 & \scspace{S}. \scspace{Matthiæ Apostoli}. \textit{Duplex II classis.}\\
g & v & 25 & \\
A & iv & 26 & \\
b & iij & 27 & \hang S. Gabrielis a Virgine Perdolente Conf. \textit{Duplex.}\\
c & Prid. & 28 &
%%& & & \rubrique{In anno bisextili mensis Febrarius est dierum 29, et Festum S. Matthiæ celebratur die 25 Februarii, ac Festum S. Gabrielis a Virg. Perdolente die 28 Februarii, et bis dicitur Sexto Kalendas, id est die 24 et die 25; et littera Dominicalis, quæ assumpta fuit in mense Januario, mutatur in præcedentem ut, si in Januario littera Dominicalis fuerit A, mutetur in præcedentum, quæ est \normaltext{g,} etc., littera  \normaltext{f} bis servit 24 et 25.}
\end{longtable}
%%
\textes{In_anno_bisextili}%% contains ad hoc vspace command which could be removed, modified, or replaced with a custom spacing command to avoid magic numbers. rubrique macro is defined in commonheaders file
%%%% In the current configuration this is commented out and the text is inserted in the calendar table itself for reasons of space. Result TBD as the table of moveable feasts needs serious work and may have to cover two pages.
%%
% !TEX TS-program = LuaLaTeX+se
% !TEX root = Kalendarium.tex

{\centering{{\normalsize Martius.}\par}}
%
\begin{longtable}{>{\centering}p{0.025\textwidth}|>{\raggedright}p{0.040\textwidth}|>{\raggedleft}p{0.025\textwidth}|>{\raggedright\arraybackslash}p{0.80\textwidth}}
d & Kal. & 1 & \\
e & vj & 2 & \\
f & v & 3 & \\
g & iv & 4 & \hang S. Casimiri Conf. \textit{Semid.} \mem{S. Lucii I Papæ Mart.}\\
A & iij & 5 & \\
b & Prid. & 6 & SS. Perpetuæ et Felicitatis Martyrum. \textit{Duplex.}\\
c & Non. & 7 & \hang S. Thomæ de Aquino Conf. et Eccl. Doct. \textit{Duplex.}\\
d & viij & 8 & \hang S. Joannis de Deo Conf. \textit{Duplex.}\\
e & vij & 9 & \hang S. Franciscæ Romanæ, Viduæ. \textit{Duplex.}\\
f & vj & 10 & SS. Quadraginta Martyrum. \textit{Semiduplex.}\\
g & v & 11 & \\
A & iv & 12 & S. Gregorii Papæ, Conf. et Eccl. Doctoris. \textit{Duplex.}\\
b & iij & 13 & \\
c & Prid. & 14 & \\
d & Idib. & 15 & \\
e & xvij & 16 & \\
f & xvj & 17 & \hang S. Patricii Episc. Conf. \textit{Duplex.}\\
g & xv & 18 & \hang S. Cyrilli Ep. Hierosolymitani, Conf. et Eccl. Doct. \textit{Duplex.}\\
A & xiv & 19 & \hang \capspace{S. JOSEPH} Sponsi B.M.V., Conf. \textit{Duplex I classis.}\\
b & xiij & 20 & \\
c & xij & 21 & S. Benedicti Abbatis. \textit{Duplex majus.}\\
d & xj & 22 & \\
e & x & 23 & \\
f & ix & 24 & S. Gabrielis Archangeli. \textit{Duplex.}\\
g & viij & 25 & \hang \capspace{ANNUNTIATIO B}. \capspace{MARIÆ VIRGINIS}. \textit{Duplex I classis.}\\
A & vij & 26 & \\
b & vj & 27 & S. Joannis Damasceni Conf. et Ecclesiæ Doctoris. \textit{Duplex.}\\
c & v & 28 & S. Joannis a Capistrano Conf. \textit{Semiduplex.}\\
d & iv & 29 & \\
e & iij & 30 & \\
f & Prid. & 31 & \\
 &  &  & \hang \textit{Feria VI post Dom. Passionis.} Septum Dolorum B. Mariæ Virginis. \textit{Duplex majus.} \mem{Feriæ.}
\end{longtable}

% !TEX TS-program = LuaLaTeX+se
% !TEX root = Kalendarium.tex

{\centering{{\normalsize Aprilis.}\par}}

\begin{longtable}{>{\centering}p{0.025\textwidth}|>{\raggedright}p{0.040\textwidth}|>{\raggedleft}p{0.025\textwidth}|>{\raggedright\arraybackslash}p{0.80\textwidth}}
g & Kal. & 1 & \\
A & iv & 2 & \hang S. Francisci de Paula Conf. \textit{Duplex.}\\
b & iij. & 3 & \\
c & Prid. & 4 & \hang S. Isidori Episc. Conf. et Eccl. Doctoris. \textit{Duplex.}\\
d & Non. & 5 & \hang S. Vincentii Ferrerii Conf. \textit{Duplex.}\\
e & viij & 6 & \\
f & vij & 7 & \\
g & vj & 8 & \\
A & v & 9 & \\
b & iv & 10 & \\
c & iij &11 & \hang S. Leonis I Papæ, Conf. et Eccl. Doctoris. \textit{Duplex.}\\
d & Prid. & 12 & \\
e & Ibid. & 13 & \hang S. Hermenegildi Martyris. \textit{Semiduplex.}\\
f & xviij & 14 & \hang S. Justini Mart. \textit{Duplex.} \mem{SS. Tiburtii, Valeriani et Maximi Martyrum.} \\
g & xvij & 15 & \\
A & xvj & 16 & \\
b & xv & 17 & \hang S. Aniceti I, Papæ Martyris. \textit{Simplex.}\\
c & xiv & 18 & \\
d & xiij & 19 & \\
e & xij & 20 & \\
f & xj & 21 & \hang S. Anselmi Episc. Conf. et Eccl. Doctoris. \textit{Duplex.}\\
g & x & 22 & \hang SS. Soteris et Caji Pontif. Mart. \textit{Semiduplex.}\\
A & ix & 23 & \hang S. Georgii Martyris. \textit{Semiduplex.}\\
b & viij & 24 & \hang S. Fidelis a Sigmaringa Martyris.\textit{Duplex.}\\
c & vij & 25 & \hang S. \scspace{Marci Evangelistæ}. \textit{Duplex II classis.}\\
d & vj & 26 & \hang SS. Cleti et Marcellini Pontif. Martyrum. \textit{Semiduplex.}\\
e & v & 27 & \hang S. Petri Canisii Conf. et Eccl. Doct. \textit{Duplex.}\\
f & iv & 28 & \hang S. Pauli a Cruce Conf. \textit{Dupl.}\\
g & iij & 29 & \hang S. Petri Martyris. \textit{Duplex.}\\
A & Prid. & 30 & \hang S. Catharinæ Senensis Virginis. \textit{Duplex.}\\
 &  &  & \hang \textit{Feria IV infra Hebdomadam II post octavam Paschæ.} \capspace{SOLEMNITAS S}. \capspace{JOSEPH}, Sponsi B.M.V., Conf. et Eccl. univers. Patroni. \textit{Duplex I classis cum Octava communi.}\\
 &  &  & \hang \textit{Feria IV infra Hebdomadam III post octavam Paschæ.} Octava S. Joseph. \textit{Duplex majus.}
\end{longtable}

% !TEX root = Kalendarium.tex
% !TEX TS-program = LuaLaTeX+se

{\centering{{\normalsize Maius.}\par}}

\begin{longtable}{>{\centering}p{0.025\textwidth}|>{\raggedright}p{0.040\textwidth}|>{\raggedleft}p{0.025\textwidth}|>{\raggedright\arraybackslash}p{0.80\textwidth}}
b & Kal. & 1 & \hang \scspace{Ss}. \scspace{Philippi et Jacobi Apostolorum}. \textit{Duplex II classis.}\\
c & vj & 2 & \hang S. Athanasii Episc. Conf. et Eccl. Doctoris. \textit{Duplex.}\\
d & v & 3 & \hang \scspace{Inventio S}. \scspace{Crucis}. \textit{Dupl. II classis.} \mem{SS. Alexandri I Papæ et Soc. Martyrum, ac S. Juvenalis Episc. Conf.}\\
e & iv & 4 & \hang S. Monicæ Viduæ. \textit{Duplex.}\\
f & iij & 5 & S. Pii V Papæ Conf. \textit{Duplex.}\\
g & Prid. & 6 & \hang S. Joannis Ap. ante Portam Latinam. \textit{Duplex majus.}\\
A & Non. & 7 & S. Stanislai Episc. Martyris. \textit{Duplex.}\\
b & viij & 8 & \hang Apparitio S. Michaelis Archangeli. \textit{Duplex majus.}\\
c & vij & 9 & \hang S. Grgeorii Nazianzni Episc. Conf. et Eccl. Doct. \textit{Duplex.}\\
d & vj & 10 & \hang S. Antonini Episc. Conf. \textit{Duplex.}\\
e & v & 11 & \\
f & iv & 12 & \hang SS. Nerei, Achillei et Domitillæ Virginis, atque Pancratii Martyrum. \textit{Semiduplex.}\\
g & iij & 13 & \hang S. Roberti Bellarmino Episc. Conf. et Eccl. Doct. \textit{Duplex.}\\
A & Prid. & 14 & \hang S. Bonifatii Martyris. \textit{Simplex.}\\
b & Idib. & 15 & \hang S. Joannis Baptistæ de la Salle Conf. \textit{Duplex.}\\
c & xvij & 16 & S. Ubaldi Episc. Conf. \textit{Semiduplex.}\\
d & xvj & 17 & S. Paschalis Baylon Conf. \textit{Duplex.}\\
e & xv & 18 & \hang S. Venantii Martyris. \textit{Duplex.}\\
f & xiv & 19 & \hang S. Petri Cœlestini Papæ Conf. \textit{Dupl.} \mem{S. Pudentianæ Virginis.}\\
g & xiij & 20 & \hang S. Bernardini Senensis Conf.\textit{Semiduplex.}\\
A & xij & 21 & \\
b & xj & 22 &  \\
c & x & 23 & \\
d & ix & 24 &\\
e & viij & 25 & \hang S. Gregorii VII Papæ Conf. \textit{Duplex.} \mem{S. Urbani I Papæ Martyris.}\\
f & vij & 26 & \hang S. Philippi Nerii Conf. \textit{Duplex.}\\
g & vj & 27 & \hang S. Bedæ Venerabilis Conf. et Eccl. Doct. \textit{Duplex.} \mem{S. Joannis I Papæ Martyris.}\\
A & v & 28 & S. Augustini Episc. Conf. \textit{Duplex.}\\
b & iv & 29 & \hang S. Mariæ Magdalenæ de Pazzis Virg. \textit{Semiduplex.}\\
c & iij & 30 & \hang S. Felicis I Papæ Martyris. \textit{Simplex.}\\
d & Prid. & 31 & \hang S. Angelæ Mericiæ Virg. \textit{Duplex.} \mem{S. Petronillæ Virg.}\\ 
 &  &  & \hang \textit{Vel.} \scspace{B}. \scspace{Mariæ Virginis Reginæ}.  \textit{Duplex II classis.}  \mem{S. Petronillæ Virg.}
\end{longtable}

% !TEX root = Kalendarium.tex
% !TEX TS-program = LuaLaTeX+se

{\centering{{\normalsize Junius.}\par}}

\begin{longtable}{>{\centering}p{0.025\textwidth}|>{\raggedright}p{0.040\textwidth}|>{\raggedleft}p{0.025\textwidth}|>{\raggedright\arraybackslash}p{0.80\textwidth}}
e & Kal. & 1 & \\
 &  &  & \hang \textit{Vel.} S. Angelæ Mericiæ Virg. \textit{Duplex.}\\
f & iv & 2 & SS. Marcellini, Petri atque Erasmi Martyrum. \hang \textit{Simplex.}\\
g & iij & 3 & \hang \\
A & Prid. & 4 & \hang S. Francisci Caracciolo Conf. \textit{Duplex.}\\
b & Non. & 5 & \hang S. Bonifatii Episc. Martyris. \textit{Duplex.}\\
c & viij & 6 & \hang S. Norberti Episc. Conf. \textit{Duplex.}\\
d & vij & 7 & \\
e & vj & 8 & \\
f & v & 9 & \hang SS. Primi et Feliciani Martyrum. \textit{Simplex.}\\
g & iv & 10 & S. Margaritæ Reginæ, Viduæ. \textit{Semiduplex.}\\
A & iij & 11 & \hang S. Barnabæ Apostoli. \textit{Duplex majus.}\\
b & Prid. & 12 & \hang S. Joannis a S. Facundo Conf.  \textit{Dupl.} \mem{Comm. SS. Basilidis, Cyrini, Naboris et Nazarii Martyrum.}\\
c & Idib. & 13 & \hang S. Antonii de Padua Conf. (et Eccl. Doct.) \textit{Duplex.}\\ \noalign{\penalty-5000} %%this marks pagebreak preferred location
d & xviij & 14 & S. Basilii Magni Episc. Conf. et Eccl. Doct. \textit{Duplex.}\\
e & xvij & 15 & SS. Viti, Modesti atque Crescentiæ Martyrum. \textit{Simplex.}\\
f & xvj & 16 & \\
g & xv & 17 & \\
A & xiv & 18 & S. Ephraem Syri Diac., Conf. et Eccl. Doct. \textit{Duplex.}\\
b & xiij & 19 & \hang S. Julianæ de Falconeriis Virginis. \textit{Dupl.} \mem{SS. Gervasii et Protasii Martyrum.}\\
c & xij & 20 & S. Silverii Papæ Martyris. \textit{Simplex.}\\
d & xj & 21 & \hang S. Aloisii Gonzagæ Conf. \textit{Duplex.}\\
e & x & 22 & \hang S. Paulini Episc. Conf. \textit{Duplex.}\\
f & ix & 23 & Vigilia.\\
g & viij & 24 & \hang \capspace{NATIVITAS S}. \capspace{JOANNIS BAPTISTÆ}. \textit{Duplex I classis cum Octava communi.}\\
A & vij & 25 & S. Gulielmi Abbatis. \textit{Duplex.} \mem{Octavæ.}\\
b & vj & 26 & SS. Joannis et Pauli Martyrum. \mem{Octavæ.}\\
c & v & 27 & \hang De Octava. \textit{Semiduplex.}\\
d & iv & 28 & \hang S. Irenæi Episc. et Martyr. \textit{Duplex.} \mem{Octavæ et Vigiliæ.}\\
e & iij & 29 & \hang \capspace{SS}. \capspace{PETRI ET PAULI APOSTOLORUM}. \textit{Duplex I classis cum Octava communi.}\\
f & Prid. & 30 & \hang Commemoratio S. Pauli Apostoli. \textit{Duplex majus.} \mem{S. Petri Apostoli et Octavæ S. Joannis Baptistæ.}
\end{longtable}

%%\pagebreak 

% !TEX TS-program = LuaLaTeX+se
% !TEX root = Kalendarium.tex

{\centering{{\normalsize Julius.}\par}}

\begin{longtable}{>{\centering}p{0.025\textwidth}|>{\raggedright}p{0.040\textwidth}|>{\raggedleft}p{0.025\textwidth}|>{\raggedright\arraybackslash}p{0.80\textwidth}}
g & Kal. & 1 & \hang \capspace{PRETIOSISSIMI SANGUINIS} D.N.J.C. \textit{Duplex I classis.} \mem{diei Octavæ S.~Joannis Baptistæ.}\\  %%~ needed with current settings
A & vj. & 2 & \scspace{Visitatio} B.M.V. \textit{Duplex II classis.} \mem{SS. Processi et Martiniani Martyrum} \\
b & v & 3 & \hang S. Leonis II Papæ et Conf. \textit{Semiduplex.} \mem{Octavæ.}\\
c & iv & 4 & \hang De Octava. \textit{Semiduplex.}\\
d & iij & 5 & \hang S. Antonii Mariæ Zaccaria Conf. \textit{Duplex.}\\
e & Prid. & 6 & \hang Octava SS. Petri et Pauli Apostolorum. \textit{Duplex majus.}\\
f & Non. & 7 & \hang SS. Cyrilli et Methodii Episc. Conf. \textit{Duplex.}\\
g & viij & 8 & \hang S. Elisabeth Reginæ, Viduæ. \textit{Semiduplex.}\\
A & vij & 9 & \\
b & vj & 10 &  \hang SS. Septem Fratrum Martyrum et SS. Rufinæ et Secundæ Virgimum et Martyrum. \textit{Semiduplex.}\\
c & v & 11 & \hang S. Pii I Papæ Martyris. \textit{Simplex.}\\
d & iv & 12 & S. Joannis Gualberti Abbatis. \textit{Duplex.} \mem{SS. Naboris et Felicis Martyrum.}\\
e & iij & 13 & \hang S. Anacleti Papæ Mart. \textit{Semiduplex.}\\
f & Prid. & 14 & \hang  S. Bonaventuræ Eepisc. Conf. et Eccl. Doct. \textit{Duplex.}\\
g & Idib. & 15 & \hang S. Henrici Imperatoris, Conf. \textit{Semiduplex.}\\
A & xvij &16 & \hang Commemoratio B. Mariæ Virginis de Monte Carmelo. \textit{Duplex majus.}\\
b & xvj & 17 &  \hang S. Alexii Conf. \textit{Semiduplex.}\\ \noalign{\penalty-5000} %%this marks pagebreak preferred location
c & xv & 18 & \hang S. Camilli de Lellis Conf. \textit{Dupl.} \mem{SS. Symphorosæ et septem Filiorum ejus Martyrum.}\\
d & xiv & 19 & \hang S. Vincentii a Paulo Conf. \textit{Duplex.}\\
e & xiij & 20 & \hang S. Hieronymi Æmiliani Conf. \textit{Dupl.} \mem{S. Margaritæ Virg. et Mart.}\\
f & xij & 21 & \hang S. Praxedis Virginis. \textit{Simplex.}\\
g & xj & 22 & \hang S. Mariæ Magdalenæ Pœnitentis. \textit{Duplex.}\\
A & x & 23 & \hang S. Apollinaris Episc. Mart. \textit{Dupl.} \mem{Comm. S. Liborii Ep. Conf.}\\
b & ix & 24 & \hang  Vigilia. \mem{S. Christinæ Virginis et Martyris.}\\
c & viij & 25 & \hang S. \scspace{Jacobi Apostoli}. \textit{Duplex II classis.} \mem{S. Christophori Martyr.}\\
d & vij & 26 & \hang S. \scspace{Annæ Matris} B.M.V. \textit{Duplex II classis.}\\
e & vj & 27 & \hang S. Pantaleonis Martyris. \textit{Simplex.}\\
f & v & 28 &  \hang SS. Nazarii et Celis Martyrum, Victoris I Papæ Mart. ac Innocentii I Papæ Conf. \textit{Semiduplex.}\\
g & iv & 29 & \hang S. Marthæ Virg. \textit{Semid.} \mem{SS. Felicis II Papæ, Simplicii, Faustini et Beatricis Martyrum.}\\
A & iij & 30 & \hang SS. Abdon et Sennen Martyrum. \textit{Simplex.}\\
b & Prid. & 31 & \hang S. Ignatii Conf. \textit{Duplex majus.}
\end{longtable}

% !TEX TS-program = LuaLaTeX+se
% !TEX root = Kalendarium.tex

{\centering{{\normalsize Augustus.}\par}}

\begin{longtable}{>{\centering}p{0.025\textwidth}|>{\raggedright}p{0.040\textwidth}|>{\raggedleft}p{0.025\textwidth}|>{\raggedright\arraybackslash}p{0.80\textwidth}}
c & Kal. & 1 & \hang S. Petri ad Vincula. \textit{Duplex majus.} \mem{S. Pauli Ap. ac SS. Machabæorum Martyrum.}\\
d & iv & 2 & \hang S. Alpphonsi Mariæ de Ligorio Episc. Conf. et Eccl. Doct. \textit{Duplex.} \mem{S. Stephani I Papæ Martyris.}\\
e & iij & 3 & \hang Inventio S. Stephani Protomartyris. \textit{Semiduplex.}\\
f & Prid. & 4 & \hang S. Dominici Conf. \textit{Duplex majus.}\\
g & Non. & 5 & \hang Dedicatio S. Mariæ ad Nives. \textit{Duplex majus.}\\
A & viij & 6 & \hang \scspace{Transfiguratio} D.N.J.C. \textit{Duplex II classis.} \mem{SS. Xysti II Papæ, Felicissimi et Agapiti Martyrum.}\\
b & vij & 7 & \hang S. Cajetani Conf. \textit{Duplex.} \mem{S. Donatii Episc. Mart.}\\
c & vj & 8 & \hang SS. Cyriaci, Largi et Smaragdi Martyrum. \textit{Semiduplex.}\\
d & v & 9 & \hang S. Joannis Mariæ Vianney Conf. \textit{Duplex.} \mem{Vigiliæ et S. Romani Martyris.}\\
e & iv & 10 & \hang S. \scspace{Laurentii Martyris}. \textit{Duplex II classis cum Octava simplici.}\\
f & iij & 11 & \hang SS. Tiburtii et Susannæ Virg., Martyrum. \textit{Simplex.}\\
 g & Prid. & 12 & \hang S. Claræ Virginis. \textit{Duplex.}\\
A & Ibid. & 13 & \hang SS. Hippolyti et Cassiani Martyrum. \textit{Simplex.}\\
b & xix & 14 & \hang Vigilia. \mem{S. Eusebii Conf.}\\
c & xviij & 15 & \hang \capspace{ASSUMPTIO B}. \capspace{MARIÆ VIRGINIS}. \textit{Duplex I classis cum Octava communi.}\\
d & xvij & 16 & \hang S. \scspace{Joachim Patris} B.M.V. \textit{Duplex II classis.}\\
e & xvj & 17 & \hang S. Hyacinthi Conf. \textit{Duplex.} \mem{Octavæ Assumptionis ac diei Octavæ S. Laurentii Mart.} \\
f & xv & 18 & \hang De Octava Assumptionis. \textit{Semiduplex.} \mem{S. Agapiti Mart.}\\
g & xiv & 19 & \hang S. Joannis Eudes. \textit{Duplex.} \mem{Octavæ.}\\
A & xiij & 20 & \hang S. Bernardi Abbatis Conf. et Eccl. Doct. \textit{Duplex.} \mem{Octavæ.}\\
b & xij & 21 & \hang S. Joannæ Franciscæ Fremiot de Chantal Viduæ. \textit{Duplex.} \mem{Octavæ.}\\
c & xj & 22 & \hang Octava Assumptionis B.M.V. \textit{Duplex majus.} \mem{SS Timothei, Hippolyti et Symphoriani Martyrum.}\\
& & & \hang \textit{Vel.} \scspace{Immaculati Cordis} B.M.V. \textit{Duplex II classis.}  \mem{SS Timothei, Hippolyti et Symphoriani Martyrum.}\\
d & x & 23 & \hang S. Philippi Benitii Conf. \textit{Duplex.} \mem{Vigiliæ.}\\
e & ix & 24 & \hang S. \scspace{Bartholomæi Apostoli}. \textit{Duplex II classis.}\\
f & vijj & 25 & \hang S. Ludovici Regis, Conf. \textit{Semiduplex.}\\
g & vij & 26 &  \hang S. Zephyrini Papæ Martyris. \textit{Simplex.}\\
A & vj & 27 & \hang S. Josephi Calasanctii Conf. \textit{Duplex.}\\
b & v & 28 & \hang S. Augustini Episc. Conf. et Eccl. Doct. \textit{Dupl.} \mem{S. Hermetis Martyris.}\\
c & iv & 29 & \hang Decollatio S. Joannis Baptistæ. \textit{Duplex majus.} \mem{S. Sabinæ Martyris.}\\
d & iij & 30 &  \hang S. Rosæ Limanæ Virginis. \textit{Duplex.} \mem{SS. Felicis et Adaucti Martyrum.}\\
e & Prid. & 31 &  \hang S. Raymundi Nonnati Conf. \textit{Duplex.}\\
\end{longtable}


% !TEX TS-program = LuaLaTeX+se
% !TEX root = Kalendarium.tex

{\centering{{\normalsize September.}\par}}

\begin{longtable}{>{\centering}p{0.025\textwidth}|>{\raggedright}p{0.040\textwidth}|>{\raggedleft}p{0.025\textwidth}|>{\raggedright\arraybackslash}p{0.80\textwidth}}
 f & Kal. & 1 & S. Ægidii Abbatis. \textit{Simplex.} \mem{SS. Duodecim Fratrum Martyrum.}\\
g & iv & 2 & \hang S. Stephani Hungariæ Regis, Conf. \textit{Semiduplex.}\\
A & iij & 3 & \\
& & & \hang \textit{Vel.} S. Pii X Papæ et Conf. \textit{Duplex.}\\
b & Prid. & 4 & \\
c & Non. & 5 & S. Laurentii Justiniani Episc. Conf. \textit{Duplex.}\\
d & viij & 6 &  \\
e & vij & 7 & \\
f & vj & 8 & \hang \scspace{Nativitas B}. \scspace{Mariæ Virginis}. \textit{Duplex II classis cum Octava simplici.} \mem{S. Hadriani Martyris.}\\
g & v & 9 & \hang S. Gorgonii Martyris. \textit{Simplex.}\\
A & iv & 10 & \hang S. Nicolai a Tolentino. \textit{Duplex.}\\
b & iij & 11 & SS. Proti et Hyacinthi Mart.  \textit{Simplex.}\\
c & Prid. & 12 & \hang Ss. Nominis Mariæ. \textit{Duplex majus.}\\
d & Idib. & 13 & \hang \\
e & xviij & 14 & \hang Exaltatio S. Crucis. \textit{Duplex majus.}\\
f & xvij & 15 & \hang \scspace{Septum Dolorum B}. \scspace{Mariæ Virginis}. \textit{Duplex II classis.} \mem{S. Nicomedis Mart.}\\
g & xvj & 16 & \hang SS. Cornelii Papæ et Cypriani Episc., Mart. \textit{Semid.} \mem{SS. Euphemiæ et Sociorum Martyrum.}\\
A & xv & 17 & \hang Impressio sacrorum Stigmatum S. Francisci Conf. \textit{Duplex.}\\
b & xiv & 18 & \hang S. Josephi a Cupertino Conf. \textit{Duplex.}\\
c & xiij & 19 & \hang SS. Januarii Episcopi et Soc. Martyrum. \textit{Duplex.}\\
d & xij & 20 & \hang SS. Eustachii et Soc. Martyrum. \textit{Duplex.} \mem{Vigiliæ.}\\
e & xj & 21 & \hang S. \scspace{Matthæi Apostoli et Evangelistæ}. \textit{Duplex II classis.}\\
f & x & 22 &  \hang S. Thomæ de Villanova Episc. Conf. \textit{Dupl.} \mem{SS. Mauritii et Sociorum Martyrum.}\\
g & ix & 23 & \hang S. Lini Papæ Mart. \textit{Semid.} \mem{S. Theclæ Virg. et Mart.}\\
A & viij & 24 &  \hang B. Mariæ Virginis de Mercede. \textit{Duplex majus.}\\
b & vij & 25 &  \\
c & vj & 26 & \hang SS. Cypriani et Justinæ Virginis, Martyrum. \textit{Simplex.}\\
d & v & 27 & \hang SS. Cosmæ et Damiani Martyrum. \textit{Semiduplex.}\\
e & iv & 28 & \hang S. Wenceslai Ducis, Martyris. \textit{Semiduplex.}\\
f & iij & 29 & \hang \capspace{DEDICATIO S}. \capspace{MICHAELIS ARCHANGELI}. \textit{Dupl. I classis.}\\
g & Prid. & 30 & \hang S. Hieronymi Presbyteri, Conf. et Eccl. Doctoris. \textit{Duplex.}
\end{longtable}


% !TEX root = Kalendarium.tex

{\centering{{\normalsize October.}\par}}

\begin{longtable}{>{\centering}p{0.025\textwidth}|>{\raggedright}p{0.040\textwidth}|>{\raggedleft}p{0.025\textwidth}|>{\raggedright\arraybackslash}p{0.80\textwidth}}
A & Kal. & 1 & \hang S. Remigii Episc. Conf. \textit{Simplex.}\\
b & vj & 2 & \hang SS. Angelorum Custodum. \textit{Duplex majus.}\\
c & v &3 & S. Teresiæ a Jesu Infante Virg. \textit{Duplex.}\\
d & iv & 4 & \hang  S. Francisci Confessoris. \textit{Duplex majus.}\\
e & iij & 5 & \hang SS. Placidi et Sociorum Martyrum. \textit{Simplex.}\\
f & Prid. & 6 & \hang S. Brunonis Confessoris. \textit{Duplex.}\\
g & Non. & 7 & \hang \scspace{Sacratissimi Rosarii} B.M.V. \textit{Duplex II classis.} \mem{S. Marci Papæ Conf. ac SS. Sergii et Sociorum Mart.}\\
A & viij & 8 & S. Birgittæ Viduæ. \textit{Duplex.} \mem{SS. Dionysii, Episc., Rustici et Eleutherii Mart.}\\
b & vij & 9 & \hang S. Joannis Leonardi Conf. \textit{Duplex.}\\
c & vj & 10 & S. Francisci Borgiæ Confessoris. \textit{Semiduplex.}\\
d & v & 11 & \hang \scspace{Maternitatis B}. \scspace{Mariæ Virginis.} \textit{Duplex II classis.}\\
e & iv & 12 & \\
f & iij & 13 & S. Eduardi Regis, Conf. \textit{Semiduplex.}\\
g & Prid. & 14 & \hang S. Callisti I Papæ Martyris. \textit{Duplex.}\\
A & Idib. & 15 & \hang S. Teresiæ Virginis. \textit{Duplex.}\\
b & xvij & 16 & \hang S. Hedwigis Viduæ. \textit{Semiduplex.}\\
c & xvj & 17 & \hang  S. Margaritæ Mariæ Alacoque Virginis. \textit{Duplex.}\\
d & xv & 18 & \hang S. \scspace{Lucæ Evanglistæ}. \textit{Duplex II classis.}\\
e & xiv &19 & \hang S. Petri de Alcantara Conf. \textit{Duplex.}\\
f & xiij & 20 & \hang S. Joannis Cantii Conf. \textit{Duplex.}\\
g & xij & 21 & \hang S. Hilarionis Abbatis. \textit{Simplex.} \mem{SS. Ursulæ ac Sociarum Virginum et Martyrum.}\\
A & xj & 22 & \\
b & x & 23 & \hang \\
c & ix & 24 & \hang S. Raphaelis Archangelis. \textit{Duplex majus.}\\
d & viij & 25 & \hang SS. Chrystani et Dariæ Martyrum. \textit{Simplex.}\\
e & vij & 26 & \hang S. Evaristi Papæ Martyris. \textit{Simplex.}\\
f & vj & 27 & \hang Vigilia.\\
g & v & 28 & \hang \scspace{Ss}. \scspace{Simonis et Judæ Apostolorum}. \textit{Duplex II classis.}\\
A & iv & 29 & \\
b & iij & 30 & \\
c & Prid. & 31 & \hang Vigilia Omnium Sanctorum.\\
& & & \hang \textit{Dominica ultima Octobris.} \capspace{FESTUM} D.N. \capspace{JESU CHRISTI REGIS}. \textit{Duplex I classis.}
\end{longtable}


% !TEX root = Kalendarium.tex

{\centering{{\normalsize November.}\par}}

\begin{longtable}{>{\centering}p{0.025\textwidth}|>{\raggedright}p{0.040\textwidth}|>{\raggedleft}p{0.025\textwidth}|>{\raggedright\arraybackslash}p{0.80\textwidth}}
d & Kal. & 1 & \hang \capspace{OMNIUM SANCTORUM}. \textit{Duplex I classis cum Octava communi.}\\
e & iv & 2 & \hang Commemoratio Omnium Fidelium Defunctorum. \textit{Duplex.}\\
f & iij & 3 & \hang De Octava Omnium Sanctorum. \textit{Semiduplex.} \mem{Octavæ ac SS. Vitalis et Agricolæ Martyrum.}\\
g & Prid. & 4 & \hang S. Caroli Episc. Conf. \textit{Duplex.}\\
A & Non. & 5 & \hang De Octava. \textit{Semiduplex.}\\
b & viij & 6 & \hang De Octava. \textit{Semiduplex.}\\
c & vij & 7 & \hang  De Octava. \textit{Semiduplex.}\\
d & vj & 8 & \hang Octava Omnium Sanctorum.  \textit{Duplex majus.} \mem{SS. Quatuor Coronatorum Martyrum.}\\
e & v & 9 & \hang \scspace{Dedicatio Archibasilicæ Ss. Salvatoris}. \textit{Duplex II classis.} \mem{S.~Theodori Martyris.}\\ %%~ needed with current settings
f & iv & 10 & \hang  S. Andreæ Avellini Conf. \textit{Duplex.} \mem{SS. Tryphonis et Sociorum Martyrum.}\\
g & iij & 11 & \hang  S. Martini Episc. Conf. \textit{Duplex.} \mem{S. Mennæ Mart.}\\
A & Prid. & 12 & \hang S. Martini I Papæ Martyris. \textit{Semiduplex.}\\
b & Idib. & 13 & S. Didaci Conf. \textit{Semiduplex.}\\
c & xviij & 14 & S. Josaphat Episc. Martyris. \textit{Duplex.}\\
d & xvij & 15 & \hang S. Alberti Magni Ep., Conf. et Eccl. Doct. \textit{Duplex.}\\
e & xvj & 16 & \hang S. Gertrudis Virginis. \textit{Duplex.}\\
f & xv &17 & \hang S. Gregorii Thaumaturgi Episc. Conf. \textit{Semiduplex.}\\
g & xiv & 18 & \hang Dedicatio Basilicarum SS. Petri et Pauli Apost. \textit{Dupl. majus.}\\
A & xiij & 19 & \hang S. Elisabeth Viduæ. \textit{Dupl.} \mem{S. Pontiani Papæ Mart.}\\
b & xij & 20 & \hang S. Felicis de Valois Conf. \textit{Duplex.}\\
c & xj & 21 & \hang Præsentatio B. Mariæ Virginis. \textit{Duplex majus.}\\
d & x & 22 & \hang S. Cæciliæ Virginis et Martyris. \textit{Duplex.}\\
e & ix & 23 & \hang S. Clementis I Papæ Martyris. \textit{Duplex.} \mem{S. Felicitatis Martyris.}\\
f & viij & 24 & \hang S. Joannis a Cruce Conf. et Eccl. Doct. \textit{Duplex.} \mem{S. Chrysogoni Mart.}\\
g & vij & 25 & \hang S. Catharinæ Virginis et Martyris. \textit{Duplex.}\\
A & vj & 26 & \hang S. Silvestri Abbatis. \textit{Duplex.} \mem{S. Petri Alexandrini Episc. Martyris.}\\
b & v & 27 & \\
c & iv & 28 & \\
d & iij & 29 & Vigilia. \mem{S. Saturnini Martyris.}\\
e & Prid. & 30 & \hang \scspace{S}. \scspace{Andreæ Apostoli}. \textit{Duplex II classis.}
\end{longtable}

% !TEX root = Kalendarium.tex

{\centering{{\normalsize December.}\par}}

\begin{longtable}{>{\centering}p{0.025\textwidth}|>{\raggedright}p{0.040\textwidth}|>{\raggedleft}p{0.025\textwidth}|>{\raggedright\arraybackslash}p{0.80\textwidth}}
f & Kal. & 1 & \\
g & iv & 2 & S. Bibianæ Virginis et Martyris. \textit{Semiduplex.}\\
A & iij & 3 & \hang S. Francisci Xaverii Conf. \textit{Duplex majus.}\\
b & Prid. & 4 & \hang S. Petri Chrysologi Episc. Conf. et Eccl. Doctoris. \textit{Duplex.} \mem{S. Barbaræ Virginis et Martyris.}\\
c & Non. & 5 & \hang \mem{S. Sabbæ Abbatis.}\\
d & viij & 6 & \hang S. Nicolai Episc. Conf. \textit{Duplex.}\\
e & vij & 7 & \hang S. Ambrosii Episc. Conf. et Eccl. Doct. \textit{Duplex.}\\ %% the sources disagree: LA1949 has no mention of the Vigil. AM1934 mentions it but has a note "de qua nihil in Officio" AR 1912 has (Vigilia.)
f & vj & 8 & \hang \capspace{CONCEPTIO IMMACULATA B}. \capspace{MARIÆ VIRGINIS}. \textit{Duplex I classis cum Octava communi.}\\ 
g & v & 9 & \hang De Octava Conceptionis. \textit{Semiduplex.}\\
A & iv & 10 & \hang De Octava. \textit{Semiduplex.} \mem{S. Melchiadis Papæ Martyr.}\\
b & iij & 11 & \hang S. Damasi I Papæ Conf. \textit{Semiduplex.} \mem{Octavæ.}\\
c & Prid. & 12 & \hang De Octava. \textit{Semiduplex.}\\
d & Idib. & 13 & \hang S. Luciæ Virginis et Martyris. \textit{Duplex.} \mem{Octavæ.}\\
e & xix & 14 & \hang De Octava. \textit{Semiduplex.}\\
f & xviij & 15 & Octava Conceptionis Immaculatæ B.M.V. \textit{Duplex majus.}\\
g & xvij & 16 & S. Eusebii Episc. Martyris.  \textit{Semiduplex.}\\
A & xvj & 17 & \\
b & xv & 18 & \\
c & xiv & 19 & \\
d & xiij & 20 & Vigilia.\\
e & xij & 21 & \hang S. \scspace{Thomæ Apostoli}. \textit{Duplex II classis.}\\
f & xj & 22 & \\
g & x & 23 & \\
A & ix & 24 & Vigilia.\\
b & viij & 25 & \hang \capspace{NATIVITAS} D.N. \capspace{JESU CHRISTI}. \textit{Duplex I classis cum Octava privilegiata III ordinis.}\\
c & vij & 26 & \hang S. \scspace{Stephani Protomartyris}. \textit{Duplex II classis cum Octava simplici.} \mem{Octavæ Nativitatis.}\\ 
d & vj & 27 & \hang S. \scspace{Joannis Apostoli et Evangelistæ}. \textit{Duplex II classis cum Octava simplici.} \mem{Octavæ Nativitatis.}\\ \noalign{\penalty-5000} %%this marks pagebreak preferred location
e & v & 28 & \hang \scspace{Ss}. \scspace{Innocentium Martyrum}. \textit{Duplex II classis cum Octava simplici.} \mem{Octavæ Nativitatis.}\\
f & iv & 29 & \hang S. Thomæ Episcopi Martyris. \textit{Duplex.}\\
g & iij & 30 & De Octava Nativitatis.  \textit{Semiduplex.}\\
A & Prid. & 31 & \hang S. Silvestri I Papæ Conf. \textit{Duplex.} \mem{Comm. Octavæ Nativitatis.}
\end{longtable}


\normalsize

\flushbottom

\end{document}