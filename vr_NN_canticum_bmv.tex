% !TEX TS-program = LuaLaTeX+se

\documentclass[vesperale_romanum.tex]{subfiles}

\ifcsname preamble@file\endcsname
  \setcounter{page}{\getpagerefnumber{M-vr_NN_canticum_bmv}}
\fi


\begin{document}

%% to replace with appropriate chapter commands when those are fixed

%\thispagestyle{empty}

\section[Toni ad Canticum B.M.V.]{TONI AD CANTICUM B.M.V.}\header{toni ad canticum b.m.v.}

%\thispagestyle{empty}

\altnormal

\smalltitle{Tonus 1 D, D, D2, f, g, g2, a, a2, a3.}

%%\noindent Mediatio dupllici constans accentu. \hfill Terminationes unico sonstantes accentu duabusque syllabis antecedentibus. (Sed pro terminatione D\textsuperscript{2}, accentui additur superveniens anticipata, cum acuta habetur antepenultima.

\smallscore{Magnificat_1}
\pstexte{Magnificat_1}

\smalltitle{Tonus 2 D \textit{vel} A.}

%%\noindent Mediatio unico constans accentu. \hfill Terminatio unico constans accentu unaque syllaba antecedente.

\noindent
\begin{minipage}[t]{0.5cm}
\begin{center}\textbf{D}\end{center}
\vspace{-0.5cm}
\gabcsnippet{(f3)}
\end{minipage}
\rubrique{vel}
\begin{minipage}[t]{0.5cm}
\begin{center}\textbf{A}\end{center}
\vspace{-0.5cm}
\gabcsnippet{(c3)}
\end{minipage}\nopagebreak
\smallscore{Magnificat_2}
\pstexte{Magnificat_2}

\smalltitle{Tonus 3 a, b}

%%\noindent Mediatio duplici constans accentu (sed ultimo accentui additur superveniens anticipata cum acuta habetur antepenultima.) \hfill Terminationes unico constantes accentu unaque syllaba antecedente.

\smallscore{Magnificat_3a_b}
\pstexte{Magnificat_3a_b}

\smalltitle{Tonus 3 a2, g.}

%%\noindent Mediatio duplici constans accentu (sed ultimo accentui additur superveniens anticipata cum acuta habetur antepenultima.) 

%%\hfill Terminationes unico constaantes accentu duabusque syllabis antecedentibus.

\smallscore{Magnificat_3a2_g}
\pstexte{Magnificat_3a2_g}

\smalltitle{Tonus 4 E, A , A*.}

%%\noindent Mediatio unico constans accentu duabusque syllabis antecedentibus. \hfill Terminationes  unico constantes accentu tribusque syllabis antecedentibus. (Sed pro terminatione E, accentui additur superveniens anticipata, cum acuta habetur antepenultima.)

\smallscore{Magnificat_4E}
%\espacevel
\noindent \rubrique{vel}
\smallscore{Magnificat_4}
\pstexte{Magnificat_4}

\smalltitle{Tonus 4 c.}

%%\noindent Mediatio unico constans accentu duabusque syllabis antecedentibus. \hfill Terminationes  unico constantes accentu tribusque syllabis antecedentibus. (Sed pro terminatione E, accentui additur superveniens anticipata, cum acuta habetur antepenultima.)

\smallscore{Magnificat_4C}
\pstexte{Magnificat_4C}
%
\smalltitle{Tonus 5 a.}

%\noindent Mediatio unico constans accentu \hfill Terminatio duplici constans accentu.

\smallscore{Magnificat_5}
\pstexte{Magnificat_5}

\smalltitle{Tonus 6 F \textit{vel} C.}

%% \noindent Mediatio unico constans accentu unaque syllaba antecedente. \hfill Terminatio unico constans accentu duabusque syllabis antecedentibus.

\noindent
\begin{minipage}[t]{0.5cm}
\begin{center}\textbf{F}\end{center}
\vspace{-0.5cm}
\gabcsnippet{(c4)}
\end{minipage}
\rubrique{vel}
\begin{minipage}[t]{0.5cm}
\begin{center}\textbf{C}\end{center}
\vspace{-0.5cm}
\gabcsnippet{(c2)}
\end{minipage}\nopagebreak

\smallscore{Magnificat_6}
\pstexte{Magnificat_6_alt}

\smalltitle{Tonus 7 a, b, c, c\textsuperscript{2}, d.}

%%\noindent Mediatio duplici constans accentu. \hfill Terminationes duplici constantes accentu.

\smallscore{Magnificat_7}
\pstexte{Magnificat_7}

\smalltitle{Tonus 8 G, G*, c.}

%%\noindent Mediatio unico constans accentu. \hfill Terminationes unco constantes accentu duabusque syllabis antecedentibus.

\smallscore{Magnificat_8}
\pstexte{Magnificat_8}
%
\bigtitle{Toni Solemnes \\ qui ad Canticum Beatæ Mariæ Virginis in Festis majoribus I vel II classis usurpari possunt.}

\smalltitle{Tonus 1 D, D, D2, f, g, g2, a, a2, a3.}

%%\noindent Mediatio unico constans accentu (cui additur superveniens anticipata, cum acuta habetur antepenultima) tribusque syllabis antecedentibus. \hfill Terminationes unico constantes accentu duabusque syllabis antecedentibus. (Sed pro terminatione 1 Toni D\textsuperscript{2}, accentui additur superveniens anticipata, cum acuta habetur antepenultima.)

\smallscore{Magnificat_1_Solemnis}
\pstexte{Magnificat_1_Solemnis}
%
\smalltitle{Tonus 2 D \textit{vel} A.}

%%\noindent Mediatio unico constans accentu tribusque syllabis antecedentibus. \hfill Terminationes unico constantes accentu unaque syllaba antecedente.

\begin{flushright} %% ça ne laisse pas d'espace pour une 2e portée, donc elle est à la page suivante, permettant ainsi une meilleure réalisation de l'ALT qui toucherait sinon le texte supérieur.
\noindent
\begin{minipage}[t]{0.5cm}
\begin{center}\textbf{D}\end{center}
\vspace{-0.5cm}
\gabcsnippet{(f3)}
\end{minipage}
\rubrique{vel}
\begin{minipage}[t]{0.5cm}
\begin{center}\textbf{A}\end{center}
\vspace{-0.5cm}
\gabcsnippet{(c3)}
\end{minipage}\nopagebreak
\end{flushright}

\smallscore{Magnificat_2_Solemnis}
\pstexte{Magnificat_2_Solemnis}

\smalltitle{Tonus 3 a, b.}

%%\noindent Mediatio duplici constans accentu (sed ultimo accentui additur superveniens anticipata, cum acuta habetur antepenultima.) \hfill Terminationes unico constantes accentu unaque syllaba antecedente.

\smallscore{Magnificat_3a_b_Solemnis}
\pstexte{Magnificat_3a_b}

\smalltitle{Tonus 3 a2, g.}
%%\noindent Mediatio duplici constans accentu (sed ultimo accentui additur superveniens anticipata, cum acuta habetur antepenultima.) \hfill Terminationes unico constantes accentu duabusque syllabis antecedentibus.

\smallscore{Magnificat_3a2_g_Solemnis}
\pstexte{Magnificat_3a2_g}
%
\smalltitle{Tonus 4 E, A , A*.}

%%\noindent Mediatio unico constans accentu tribusque syllabis antecedentibus. \hfill Terminationes unico constantes accentu tribusque syllabis antecedentibus. (Sed pro terminatione E, accentui additur superveniens anticipata, cum acuta  habetur antepenultima.)

\smallscore{Magnificat_4E_Solemnis}
\vspace{0.5cm}
\noindent \rubrique{vel}
\smallscore{Magnificat_4_Solemnis}
\pstexte{Magnificat_4_Solemnis}
%
\smalltitle{Tonus 5 a.}

%%\noindent Mediatio unico constans accentu \hfill Terminatio duplici constans accentu.

\smallscore{Magnificat_5_Solemnis}
\pstexte{Magnificat_5_Solemnis}

\smalltitle{Tonus 6 F \textit{vel} C.}

%%\noindent Mediatio unico constans accentu (cui additur superveniens anticipata, cum acuta habetur antepenultima) tribusque syllabis antecedentibus. \hfill Terminationes unico constantes accentu duabusque syllabis antecedentibus. (Sed pro terminatione 1 Toni D\textsuperscript{2}, accentui additur superveniens anticipata, cum acuta habetur antepenultima.)

\noindent
\begin{minipage}[t]{0.5cm}
\begin{center}\textbf{F}\end{center}
\vspace{-0.5cm}
\gabcsnippet{(c4)}
\end{minipage}
\rubrique{vel}
\begin{minipage}[t]{0.5cm}
\begin{center}\textbf{C}\end{center}
\vspace{-0.5cm}
\gabcsnippet{(c2)}
\end{minipage}
\vspace{-0.2cm}
\hfill{\rubrique{Pro terminatione \rubriquegras{C} cantatur sine\medspace \gretextglyph{Flat}\thinspace.}}
\smallscore{Magnificat_6_Solemnis}
\pstexte{Magnificat_1_Solemnis}
%
\smalltitle{Tonus 7 a, b, c, c\textsuperscript{2}, d.}

%%\noindent Mediatio duplici constans accentu (sed ultimo accentui additur superveniens anticipata, cum acuta habetur antepenultima). \hfill Terminationes duplici constantes accentu.

\smallscore{Magnificat_7_Solemnis}
\pstexte{Magnificat_7}

\smalltitle{Tonus 8 G, G*, c.}

%%\noindent Mediatio unico constans accentu tribusque syllabis antecedentibus. \hfill Terminationes unico constantes accentu duabusque syllabis antecedentibus.

\smallscore{Magnificat_8_Solemnis}
\pstexte{Magnificat_8_Solemnis}

\altitshape

\end{document}