%%%%%%%%%%%%%%% STANDARD PACKAGES %%%%%%%%%%%%%%%

%% much is based on the work of Matthias Bry (hereafter Matthias or MB) for the Nocturnale Romanum project. https://github.com/Nocturnale-Romanum/nocturnale-romanum
%% for personal reasons, the commands are a mix of French and English names. Feel free to choose one or the other if you make use of them. This file is subject to change.

%% This is the format of the recent Solesmes books.
\usepackage[paperheight=205mm,paperwidth=135mm]{geometry}

\usepackage{fontspec}
\usepackage[ecclesiasticlatin.usej,activeacute]{babel} %%added activeacute as an option for convenience especially for Kyrie
     \babelprovide[hyphenrules=latin]{ecclesiasticlatin}
          \usepackage{xspace}
%     \usepackage{multicol}
%\usepackage{paracol}
%\footnotelayout{m} %% this allows merged footnote if text is in parallel columns (mostly translations)
\usepackage[savepos]{zref}
\usepackage{fancyhdr}
\usepackage{needspace} %%pour réserver de l'espace en bas de page ; ça évite des séparations entre les titres et la partition ou le texte qui les suivent.
\usepackage[compact]{titlesec}
\usepackage{xcolor}
%\usepackage{xstring}
\usepackage{enumitem} %% nécessaire pour les textes des psaumes.
%\usepackage{expl3}
\usepackage{etoolbox}
\usepackage{lettrine}
\usepackage{perpage}%%%to reset footnotes at each page%%%
\usepackage{hyperref}
\usepackage{refcount}


%%%%%%%%%%%%%%% HYPHENATION AND TYPOGRAPHICAL CONVENTIONS %%%%%%%%%%%%%%

%%%%%%%%%%%%%%% GEOMETRY %%%%%%%%%%%%%%%
\geometry{bindingoffset=5mm,
inner=10mm,
outer=10mm,
top=12mm,
bottom=15mm,
headsep=3mm,
} %%borrowed from MB, based on the Solesmes books; binding offset TBD. He has inner=15mm instead.

%% General scale of all graphical elements. Also borrowed from MB
%% Values different from 1 are largely untested.
%% Used in those commands (e.g. everything FontSpec) that use a scale parameter.
\newcommand{\customscale}{1}


\AtBeginDocument{\setlength{\parindent}{1em}} %%  default is 20 pt

\sloppy %% to allow larger spaces which avoid overfull hbox (MB: %% We want to allow large inter-words space 
%% to avoid overfull boxes in two-columns rubrics.

%%%%%%%%%%%%%%% GREGORIO CONFIG %%%%%%%%%%%%%%%

\usepackage[autocompile]{gregoriotex}

%% sets * and † to font called via fontspec
\def\GreStar{*}
\def\GreDagger{†}

%% should prevent flat turning custos into a clef, as Solesmes  uses an altered custos in only one chant per Élie Roux.
\gresetcustosalteration{invisible}


  \NewDocumentCommand{\initialscore}{O{}m}{%
    \grechangedim{initialraise}{0.5cm}{scalable}%
  \greannotation{#1}%
   \grechangestyle{initial}{\fontspec{EB Garamond Initials}\fontsize{45}{45}\selectfont}% larger, illustrated initial for first antiphon
 \gregorioscore{partitions/#2}%
  \grechangedim{initialraise}{0cm}{scalable}}

%produces a score with a smaller initial (default is 40 pt) and annotations like in the Solesmes books; the 1st is optional and can be omitted as needed. %%update 23/5/23 apparently
 \NewDocumentCommand{\gscore}{O{}mm}{%    %%you can also omit the mode and it typesets correctly.
 \greannotation{#1}%
\greannotation{#2}%
\grechangestyle{initial}{\fontsize{28}{28}\selectfont}%
 \gregorioscore{partitions/#3}%
 }

%% score with no initial, e.g psalms, common tones, etc.
\newcommand{\smallscore}[2][y]{
  \gresetinitiallines{0}
  \gregorioscore{partitions/#2} %% à remplacer avec NewDocumentCommand ; les arguments ne marchent pas correctement … que fait le [y] ?
  \gresetinitiallines{1}
}

 \NewDocumentCommand{\MagAnnotation}{}{%
 \grechangedim{annotationseparation}{-0.01cm}{fixed}%
 } %% pour les antiennes où le chiffre convient mieux aux miniscules (parfois 1, 2, 4, 7…) ; il n'est pas nécessaire pour 6 ou 8.
%\gresetheadercapture{annotation}{greannotation}{string}

%% A-bove L-ines T-ext shall be italicized and in a smaller font
\grechangestyle{abovelinestext}{\itshape\fontsize{8}{10}\selectfont}
%% for psalm tones etc. Roman/upright text is what the Liber Usualis uses. 
\NewDocumentCommand{\altnormal}{}{\grechangestyle{abovelinestext}{\normalfont\fontsize{8}{10}\selectfont}}
%%  resets ALT font
\NewDocumentCommand{\altitshape}{}{\grechangestyle{abovelinestext}{\itshape\fontsize{8}{10}\selectfont}} 


%% fine-tuning of space between the staff and the text above lines (used for Magnificats; needs to be rethought (again, should 2 and 3 be raised or not?) and worked into \smallscore
\newcommand{\altraise}{-0.2cm} %% default is -0.1cm %% needs to be fine-tuned for \rubrique command with EB Garamond font. -0.4mm is MB value, previously 1.4mm
\grechangedim{abovelinestextraise}{\altraise}{scalable}

\newcommand{\altheight}{0.5cm} %% default is 0.3cm and value must be bigger than text height; this should fix the problem. but needs further investigation. currently 6mm in MB command
\grechangedim{abovelinestextheight}{\altheight}{scalable} %% this is a very finicky command.


%% fine-tuning of space between the staff and the lyrics

%\newcommand{\textraise}{2.8ex} %% default is 3.48471 ex MB 2.8ex
%\grechangedim{spacelinestext}{\textraise}{scalable}

%%% fine-tuning of space between the initial and the annotations
%\newcommand{\annraise}{0mm} %% default is -0.2mm MB 0mm
%\grechangedim{annotationraise}{\annraise}{scalable}

%% fine-tuning the behavior of text placed under bars. We use the so-called "new algorithm" which
%% places the bar in the middle of surrounding notes, and the text in the middle of surrounding text.
%% however, we restrict drastically the deviation of the text from the position of the bar.

%\grechangedim{maxbaroffsettextleft}{0.5mm}{scalable} MB 0.5mm
%\grechangedim{maxbaroffsettextright}{0.5mm}{scalable}


%% allows printing of glyphs from Gregorio score font  as text (available glyphs are listed in the Gregorio documentation).
\makeatletter
\def\gretextglyph#1{{\gre@font@music\csname GreCP#1\endcsname}}
\makeatother

%%% FONT %%%
\setmainfont{EB Garamond}[UprightFont=EB Garamond Regular,
ItalicFont= EB Garamond Italic,
BoldFont= EB Garamond Bold,
Ligatures=Rare,
Numbers=Proportional,
Numbers=OldStyle] %% all \kern numbers are based on this and should be removed/adjusted if this font is abandoned.
\newfontfamily\symbolfont{liturgy}
  %% text must be written {\symbolfont{+}} or called via a command but may conflict as + is † in gabc. Cross should be \small or even \footnotesize. Font also supports V & R for verse/response, but EB Garamond contains appropriate matching glyphs.

%StylisticSet=6 is long Q.


%%%Roman numerals for the year %% https://tex.stackexchange.com/questions/185548/the-year-in-roman-and-the-month-in-text

\makeatletter
\newcommand{\YEAR}{\@Roman{\the\year}}
\makeatother


%%% TYPOGRAPHICAL AND SYMBOL COMMANDS  %%%%

%% \P is pilcrow and is defined in LaTeX as is.
%% Add {\textcolor{gregoriocolor} as first part of command's definition if changing to red.

%%for proper spacing of all-caps letters to prevent clashes, e.g. serif strokes touching.

\NewDocumentCommand\capspace{m}{%
{\addfontfeature{LetterSpace=5.0}%
{%
#1}%
}}

%%for smallcaps

\NewDocumentCommand\scspace{m}{\textsc{{{\addfontfeature{LetterSpace=5.0}{#1}}}}}

%% macro to print Alleluia for versicles.
\NewDocumentCommand{\tpalleluia}{}{(\textit{T.P.} Allelúia.)}

\newcommand{\specialcharhsep}{3mm} % space after invoking R/ or V/ or A/ outside rubrics from Matthias with value of 3mm
%%will leave in, but this value is quite big!

%\newcommand{\vv}{\textcolor{gregoriocolor}{\fontspec[Scale=\customscale]{Charis SIL}℣.\nolinebreak[4]\hspace{\specialcharhsep}\nolinebreak[4]}} %% format from MB

%\newcommand{\vv}{{\normalfont ℣.\nolinebreak[4]\hspace{\specialcharhsep}\nolinebreak[4]}}
%\newcommand{\rr}{{\normalfont ℟.\nolinebreak[4]\hspace{\specialcharhsep}\nolinebreak[4]}}

\newcommand{\vv}{{\normalfont ℣.~}} %%at the beginning of a line so it shouldn't ever split from the rest but space is needed for macro to work
\newcommand{\rr}{{\normalfont ℟.~}}

%\newcommand{\cc}{{\fontspec[Scale=\customscale]{FreeSerif}\symbol{"2720}~}}%from Matthias


%% typesets a cross pattée
\NewDocumentCommand{\cc}{}{%
    % Grouping to keep font changes local
    {%
        % Ensure we're not in italics (since liturgy.ttf doesn't have italic)
        \normalfont%
        % Select the font liturgy.ttf
        \symbolfont%
        % Set the size
        \footnotesize%
        % In liturgy.ttf, the plus (+) is a cross pattée
        +%
    }~%    
} 
%%  can replace <+> with <U+2720> (see above) if a suitable font is found (or EB Garamond font is fixed); command will be useful in lieu of typing the Unicode.
%% use <v> tag to insert into a gabc score (usually for the faithful or for blessings, e.g. the font.

%% Same special characters, for in-score use (<sp>V/ R/ A/ +</sp>)

\gresetspecial{V/}{{\fontspec[Scale=\customscale]{EB Garamond}℣.~}}
\gresetspecial{R/}{{\fontspec[Scale=\customscale]{EB Garamond}℟.~}}
\gresetspecial{+}{{\fontspec[Scale=\customscale]{EB Garamond}†~}}
\gresetspecial{*}{\gresixstar}
\gresetspecial{cross}{\cc}

%% Same special characters, for use in rubrics (no added space will keep symbol together with incipit)

\newcommand{\vvrub}{{\normalfont ℣.~}}
\newcommand{\rrrub}{{\normalfont ℟.~}}

%%rubrics: black italics, smaller than body of psalms etc

\NewDocumentCommand{\rubrique}{m}{%
    {%
        \fontsize{8}{10}%
        \selectfont%
        \textit{%
        %
        #1%
    }%
    }}
    
%% macro to print normal text inside of rubric (name of a chant or prayer, etc.)

\NewDocumentCommand{\normaltext}{m}{%
    {%
        \normalfont%
        #1%
    }%
    }
    
%% in case something should be bolded inside of a rubrique
\NewDocumentCommand{\rubriquegras}{m}{%
    {%
        \normalfont%
\textbf{#1%
}%
}}

%% to print in red instead of italicizing
\NewDocumentCommand{\rouge}{m}{%
\textcolor{gregoriocolor}%
{\fontsize{8}{10}%
\selectfont{%
#1%
}%
}}

%% to print in black within rouge group.

%\NewDocumentCommand{\textenoir}{m}{%

\NewDocumentCommand{\textenoir}{m}{%
\textcolor{black}%
{%
\normalfont%
#1%
}%
}

%% rouge but in italic (this is technically not correct but is contemporary practice of Le Barroux).
\NewDocumentCommand{\rougeit}{m}{%
\textcolor{gregoriocolor}%
{\fontsize{8}{10}%
\selectfont%
\textit{#1%
}%
}}

%%https://tex.stackexchange.com/questions/156540/spacing-between-single-line-paragraph-with-lettrine%
%\pretocmd{\lettrine}{\checklettrine}{}{} 
%\newcommand{\checklettrine}{%
%  \ifnum\prevgraf<2 \vspace{\baselineskip}\fi
%}

%%macro to space punctuation with ecclesiasticlatin language from babel.
\NewDocumentCommand{\espaceponctuation}{}{\hspace{0.10em}}  %%probably not needed if we use commas and not colons for Scriptural citations

%%this value seems more balanced with this font than the definition provided in the ecclesiasticlatin documentation.

%typesets a horizontal rule like on the page with the prayers before and after the office.
%%If you're using color at all in your document, you might want to either force \myrule to use black or make it customizable. (from u/Independent-Comb-257 on Reddit)
%%thickness started at 0.4pt%%
\NewDocumentCommand{\myrule}{}{%
    \par%
    {%
        \centering%
       \textcolor{black}{\rule{0.3\textwidth}{0.6pt}}%
        \par%
    }%  
}  



%%typesets a horizontal rule like on the page with the prayers before and after the office.
%%thickness started at 0.4pt

\NewDocumentCommand{\biggerrule}{}{%
    \par%
    {%
        \centering%
        \textcolor{black}{\rule{0.75\textwidth}{0.6pt}}%
        \par%
    }%
    }

%% macro to format psalm text

\NewDocumentCommand{\pstexte}{ m }{%
    \smallskip%
    \noindent%
    \begin{itemize}[%
    		label=\null, %
			leftmargin=0pt, %
			itemindent=0mm, %
			labelsep=0pt, %
			labelwidth=0pt, %
			rightmargin=0pt, %
			parsep=0pt, %
			topsep=0pt, %
			itemsep=0pt]%
    \input{psaumes/#1}
    \end{itemize}}
    
%        \NewDocumentCommand{\pstexte}{ m }{%
%    \smallskip%
%    \noindent%%
%   \begin{multicols}{2}
%    \begin{itemize}[%
%    		label=\null, %
%			leftmargin=0pt, %
%			itemindent=3mm, %
%			labelsep=0pt, %
%			labelwidth=0pt, %
%			rightmargin=0pt, %
%			parsep=0pt, %
%			topsep=0pt, %
%			itemsep=0pt]%
%    \input{psaumes/#1}
%    \end{itemize}
%    \end{multicols}} %% modified from original in order to allow 2 column psalms
    
    %% Command de Matthias Bry modifié, prints psalm incipit score with the text
    \NewDocumentCommand{\psalmus}{mmm}{
	\needspace{4\baselineskip}%
	\smalltitle{Psalmus #1.}%            %% needspace plus smalltitle with \vspace{baselineskip} adds a lot of white space
	\smallscore[n]{#2}%
	\pstexte{#3}%
}

%% sd=semiduplex. especially for Dominica ad Vesperas in the Psalterium (the only example which comes to mind, in fact!)
 \NewDocumentCommand{\sdpsalmus}{mm}{
	\needspace{4\baselineskip}%
		\smalltitle{Psalmus #1.}%  needspace plus smalltitle with \vspace{baselineskip} adds a lot of white space
	\pstexte{#2}%
}

    %% Command de Matthias Bry modifié, prints canticle incipit score with the text and scriptural reference
    \NewDocumentCommand{\canticum}{mmmm}{%
	\needspace{4\baselineskip}%
	\smalltitle{Canticum #1.}%
	\smalltitle{\textit{#2.}}%
	\smallscore[n]{#3}%
	\pstexte{#4}%
}

% Could use \grecommentary{#2} but this sits too low on Nunc Dimittis
    
 %% macro to print any additional text (Capitiulum, oratio, rubrics)
 \NewDocumentCommand{\textes}{ m }{%
    \input{textes/#1}%
    }
    
    
    %% SC work for page headers and for secondary headings but not so much for things like "Dominica…" and certainly not the feast name%%
    
    %%%% Headers%%%%
    \pagestyle{fancy}
\fancyhead{}
\fancyfoot{}
\renewcommand{\headrulewidth}{0pt}

\fancyhead[RO]{\small\rightmark\hspace{0.5cm}\thepage}
\fancyhead[LE]{\small\thepage\hspace{0.5cm}\leftmark}

\newcommand{\setheaders}[2]{
	\renewcommand{\rightmark}{{\sc#2}}
	\renewcommand{\leftmark}{{\sc#1}}
}
\setheaders{}{}


 %% TITLE COMMANDS %%%% had been dead in NR version of the same, but Matthias says this isn't a good idea.
 %%need to figure out multiple section headers AND a chapter header so that everything fits…
 
 
 \titleformat{\chapter}[display]{\Huge\filcenter\sc\addfontfeature{LetterSpace=5.0}}{}{0pt}{} %%
\titleformat{\section}[display]{\huge\filcenter\sc\addfontfeature{LetterSpace=5.0}}{}{}{} %%
\titleformat{\subsection}[display]{\LARGE\filcenter\sc\addfontfeature{LetterSpace=5.0}}{}{}{} %%originally \large, then \Large
\setcounter{secnumdepth}{0}
\titlespacing*{\chapter}{0pt}{-50pt}{40pt} %%this should reduce spacing before chapter title https://tex.stackexchange.com/questions/63390/how-to-decrease-spacing-before-chapter-title
\titlespacing{\section}{0pt}{*0}{*0}

%%need section titleformat more appropriate for weekdays of psalter and minor feasts 
% See https://tex.stackexchange.com/questions/623797/multiple-section-styles-in-same-document.

%this command still needs work
\NewDocumentCommand{\header}{m}{\setheaders{{\scshape\addfontfeature{LetterSpace=5.0}#1}}{{\scshape\addfontfeature{LetterSpace=5.0} #1}}}

\NewDocumentCommand{\smalltitle}{m}{%
\needspace{5\baselineskip}%  %% very small if there are neumes above the staff, including flats in mode 2, e.g. O Doctor optime
\vspace{\baselineskip}%
 {\centering #1\par}%
 }
 
  %% bigtitle as written adds too much space below (or rather the combination of needspace is too large with following  text)
  
  %% possibile to have something that is a macro which takes away a fixed amount? and then I can change the definition later (though as of 17 dec 2023), cf. the screenshot from the Praglia code
 
 \NewDocumentCommand{\bigtitle}{m}{%
\needspace{5\baselineskip}%
\vspace{\baselineskip}%
 {\centering{\Large#1}\par}%
 }
 
 %%formatting date on major feasts of sanctoral cycle or moveable feasts that follow Easter where this is listed in antiphonal.

%  \NewDocumentCommand{\biglitdate}{m}{%
%\needspace{5\baselineskip}  %% very small if there are neumes above the staff, including flats in mode 2, e.g. O Doctor optime
%\vspace{\baselineskip}
% {\centering{\large #1}}\par} %}

%%formatting date ordinarily above names of minor feasts
\NewDocumentCommand{\litdate}{m}{%
\smalltitle{Die #1.}%
}

%%formatting for Sunday feasts : inter, post… name is for consistenc
\NewDocumentCommand{\dominicadate}{m}{%
\smalltitle{Dominica #1.}%
}

%%for most important feasts (as of 16 dec 2023 leaving section alone as is, but it's similar styling)
\NewDocumentCommand{\festum}{m}{%
\needspace{5\baselineskip}%
\vspace{\baselineskip}%
 {\centering{\LARGE%
 {\MakeUppercase{%
 {\capspace{#1}}
 }%
 }%
\par}%
 }%
 }

\NewDocumentCommand{\rank}{m}{%
\needspace{5\baselineskip}%
\vspace{\baselineskip}%
 {\centering{\textit{\large #1}}\par}%
}

%%for feasts where the third verse (third part of 1st verse in modern terms) is changed (Mutatur tertius vers.)

\NewDocumentCommand{\mtv}{}{%
{\normalfont{(m.t.v.)}%
}%
}

%%to replace original idea of subsection; will need work on the spacing (the needspace and vspace both might be too big
%%leaving subsection as is for now (16 dec 2023), for reference and in case I used it and didn't remember it
%% argument here is i or ii; will need another command if the vespers are identical.
 \NewDocumentCommand{\invesperis}{m}{%
\needspace{5\baselineskip}%
\vspace{\baselineskip}%
 {\centering{\LARGE%
 {\scspace{in #1 vesperis.}%
 }%
 }\par}%
}

\NewDocumentCommand{\primaantiphona}{m}{%
\smalltitle{I Antiphona. #1}%
}

%%for vel where this occurs between texts or chants

\NewDocumentCommand{\vel}{}{%
{\rubrique{Vel:}%
}%
}

%% for prayers, e.g. at beginning (and end) of office.
\NewDocumentCommand{\secreto}{}{%
Pater noster. \rubrique{secreto.}%
}

\NewDocumentCommand{\pateravedeus}{}{%
Pater noster et Ave María. ℣. Deus in adjutórium.%
}

\NewDocumentCommand{\pateravecredo}{}{%
Pater noster, Ave María, \textit{et} Credo.%
}

%%above chapter (Capitulum)
\NewDocumentCommand{\capitulum}{}{%
\smalltitle{Capitulum.}%
}

%%for Scriptural references of the chapter.
\NewDocumentCommand{\scripture}{m}{%
{\raggedleft{\textit{#1}%
}%
\par}%
}

%%above Hymn if listed this way in the antiphonal (Hymnus)
\NewDocumentCommand{\hymnus}{}{%
\smalltitle{Hymnus.}%
}

\NewDocumentCommand{\hymnusmode}{m}{%
\smalltitle{Hymnus. #1}%
}

%%for 2nd tone of hymn
\NewDocumentCommand{\altertonus}{}{%
\smalltitle{Alter Tonus.}%
}

%%for 2nd tone of hymn where Alius is employed %%optional argument is in particular for Iste Confessor tones of which there are 4 per common of confessors (bishop and not a bishop)
\NewDocumentCommand{\aliustonus}{O{}}{%
\smalltitle{Alius Tonus. #1}%
}

%%above Te lucis

\NewDocumentCommand{\hymnusadcompletorium}{}{%
\smalltitle{Tonus Hymni ad Completorium.}%
}

%%above Magnificat antiphon where this is done in Liber antiphonarius (Or Vaticana Vesperale) (Sundays to be determined)
\NewDocumentCommand{\admagnificat}{}{%
\smalltitle{Ad Magnificat, Antiphona.}%
}

%%above collect (Oratio)
\NewDocumentCommand{\oratio}{}{%
\smalltitle{Oratio.}%
}

%%for Fridays where Vespers is of Our Lady on Saturday, unless the next day is a feast of IX Lessons.
\NewDocumentCommand{\bvmsabbato}{}{%
\rubrique{Ad Vesperas, nisi occurrat Festum IX Lectionum, a Capitulo fit de Sancta Maria in Sabbato.}%
}

%% originally used apparently for typesetting portions of the Common of the BVM
\newcommand{\espacetitre}{\vspace{-0.5cm}}
\newcommand{\espaceps}{\vspace{-3mm}} %% ps =psaume
\newcommand{\espacecap}{\vspace{-3mm}} %% cap is capitulum
\newcommand{\espace}{\vspace{2mm}}

%% for litanies and other things like antiphon O Doctor Optime

\makeatletter  
\newcounter{score}
\newcounter{tabstop}[score]
\newcommand{\grealign}{%
	\@bsphack%
	\ifgre@boxing\else%
		\kern\gre@dimen@begindifference%
		\stepcounter{tabstop}%
		\expandafter\zsavepos{stop-\thescore-\thetabstop}%
		\kern-\gre@dimen@begindifference%
	\fi%
	\@esphack%
}

\newcommand{\setstops}{%
  \gdef\nstabbing@stops{%
    \hspace*{-\oddsidemargin}\hspace{-1in}%
    \hspace*{\zposx{stop-\thescore-1} sp}\=%
  }%
  \count@=\@ne
  \loop\ifnum\count@<\value{tabstop}%
    \begingroup\edef\x{\endgroup
      \noexpand\g@addto@macro\noexpand\nstabbing@stops{%
        \noexpand\hspace{-\noexpand\zposx{stop-\thescore-\the\count@} sp}%
        \noexpand\hspace{\noexpand\zposx{stop-\thescore-\the\numexpr\count@+1} sp}\noexpand\=%
      }%
    }\x
    \advance\count@\@ne
  \repeat
  \nstabbing@stops\kill
}
\makeatother

\newenvironment{nstabbing}
  {\setlength{\topsep}{0pt}%
   \setlength{\partopsep}{0pt}%
   \tabbing%
   \setstops}
  {\endtabbing\stepcounter{score}}

%%FOOTNOTES%%%

\MakePerPage{footnote}%%from perpage package \MakePerPage{footnote}

%%PHANTOM SECTION MANIPULATION%%
\providecommand\phantomsection{} %% should make hyperref work properly %%needed with addcontentsline, not with section etc alone (needed if running hyperref)

\setcounter{tocdepth}{2} %% probably should hide subsections from TOC of most documents

%%%SUBFILES%%%
\usepackage{xr} %%to allow cross-references between documents%%%
  \usepackage{subfiles}
  
  %% When we start a new subfile (new chapter), 
%% we start on a new page (with blank filling on the previous page) and create a corresponding label.

\newcommand{\customsubfile}[1]{\newpage\label{#1}\thispagestyle{empty}\subfile{#1}}


