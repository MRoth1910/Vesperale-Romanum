%%%%%%%%%%%%%%% STANDARD PACKAGES %%%%%%%%%%%%%%%

%% much is based on the work of Matthias Bry (hereafter Matthias or MB) for the Nocturnale Romanum project. https://github.com/Nocturnale-Romanum/nocturnale-romanum
%% for personal reasons, the commands are a mix of French and English names. Feel free to choose one or the other if you make use of them. This file is subject to change.

%%%%%%%%%%%%%%% INDICES %%%%%%%%%%%%%%%

%%\usepackage{imakeidx}

%%%\indexsetup{level=\section*,toclevel=section,othercode=\footnotesize\thispagestyle{empty}}
%%%\makeindex[name=H,title=index hymnorum., columns=2,columnseprule]
%%%\makeindex[name=V,title=index variarum., columns=2,columnseprule]
%%%\makeindex[name=R,title=index responsoriorum., columns=2,columnseprule]
%%%\makeindex[name=P,title=index psalmorum., columns=2,columnseprule]
%%%%%\makeindex[name=T,title=toni communes., columns=2,columnseprule]
%%%%%%\makeindex[name=F,title=index festorum., columns=2,columnseprule]

%% This is the format of the recent Solesmes books.
\usepackage[paperheight=205mm,paperwidth=135mm]{geometry}
%% This is the format of the 1912 Antiphonale Romanum
%\usepackage[paperwidth=160mm, paperheight=240mm]{geometry}

\usepackage{fontspec}
\usepackage[english,french,ecclesiasticlatin.usej,activeacute]{babel} %%added activeacute as an option for convenience especially for Kyrie
     \babelprovide[hyphenrules=latin]{ecclesiasticlatin}
\usepackage{xspace}%% better to use {} after certain macros (or even within definitions) 
     \usepackage{multicol}
%\usepackage{paracol}
%\footnotelayout{m} %% this allows merged footnote if text is in parallel columns (mostly translations)
\usepackage{emptypage} %% The emptypage package prevents page numbers and
% headings from appearing on empty pages.
\usepackage[savepos]{zref}
\usepackage{fancyhdr}
\usepackage{needspace} %%pour réserver de l'espace en bas de page ; ça évite des séparations entre les titres et la partition ou le texte qui les suivent. %%mais  ça génère des problèmes lorsque il faut 2 ou 3 titres l'un après l'autre (date ou jour, nom d'une fête, rang liturgique, office célébré… 
\usepackage[compact]{titlesec}
\usepackage{xcolor}
%\usepackage{xstring}
\usepackage{enumitem}%% psalm textes are in list format modified via this package
%\usepackage{expl3}
\usepackage{etoolbox}
\usepackage{lettrine}
\usepackage{perpage}%%%to reset footnotes at each page%%%
\usepackage{hyperref}
\usepackage{refcount}

%%%%%%%%%%%%%%% HYPHENATION AND TYPOGRAPHICAL CONVENTIONS %%%%%%%%%%%%%%

%%%%%%%%%%%%%%% GEOMETRY %%%%%%%%%%%%%%%
\geometry{bindingoffset=5mm,
inner=10mm,
outer=10mm,
top=12mm,
bottom=15mm,
headsep=3mm,
} %%borrowed from MB, based on the Solesmes books; binding offset TBD. He has inner=15mm instead.
%%will need serious revision

%% General scale of all graphical elements. Also borrowed from MB
%% Values different from 1 are largely untested per MB
%% Used in those commands (e.g. everything FontSpec) that use a scale parameter.
\newcommand{\customscale}{1}


\AtBeginDocument{\setlength{\parindent}{1em}} %%  default is 20 pt, this reduces indent from that of Computer Modern to the font's

\sloppy %% to allow larger spaces which avoid overfull hbox (MB: %% We want to allow large inter-words space 
%% to avoid overfull boxes in two-columns rubrics.)

%%%%%%%%%%%%%%% GREGORIO CONFIG %%%%%%%%%%%%%%%

\usepackage[autocompile]{gregoriotex}

%% sets * and † to font called via fontspec
\def\GreStar{*}
\def\GreDagger{†}

%%prints asterisk instead of star normally inserted by <v>\greheightstar</v>
 \gredefsymbol{greheightstar}{EB Garamond}{asterisk}

%% should prevent flat turning custos into a clef, as Solesmes uses an altered custos in only one chant per Élie Roux.
\gresetcustosalteration{invisible}

%% any chant using gregorioscore in some way will need alteration to account for indexing chant name, psalms etc.

 %% default should be 40, if not using, annotations need to be turned off in gabc. For the first antiphon of the office, Gospel canticle etc. (initial is both to refer to first and the large illustrated initial, used sometimes with an annotation, cf. Christus factus est of Tenebrae in the ritus monasticus. However, 
  \NewDocumentCommand{\initialscore}{m}{%
      \grechangedim{initialraise}{0.5cm}{scalable}%
%  \greannotation{#1}%
   \grechangestyle{initial}{\fontspec{EB Garamond Initials}\fontsize{45}{45}\selectfont}%
%% \gregorioscore{partitions/#2}
  \gregorioscore{partitions/#1}
   \grechangedim{initialraise}{0cm}{scalable}}
   
    \makeatletter
\NewDocumentCommand{\idxinitialscore}{mmm}{
  %% #1 : name of the score file
  %% #2 : office part if needed
  %% #3 :mode numeral
  %% #4 :office part letter T, H, A, P, R, F, V
    %% #5 : the indexed name of the piece; needs to be repeated (unfortunately — — — will be alphabetized as such, not under previous entry

  
  %% this prevents page breaks between the phantom section and its label, and the actual score.
  \needspace{4\baselineskip} 
  \protected@edef\@currentlabelname{#3}
  \phantomsection
  \label{#1}
\initialscore{#1}
  %% if #5 (indexed name) is blank, nothing is indexed.
  %% this is for pieces that are repetitions of another piece (antiphons after psalms)
%  \ifblank{#5}{}
  %% if optional arg #1 has been passed as 'n', set no initial
%%%  \ifx n#1\gresetinitiallines{0}\fi
%  \gregorioscore{partitions/#1}
   {\index[#2]{#3}}
  }
\makeatother

%produces a score with a smaller initial (default is 40 pt) and annotations like in the Solesmes books; the 1st is optional and can be omitted as needed. %%update 23/5/23 apparently
 \NewDocumentCommand{\gscore}{O{}mm}{%    %%you can also omit the mode and it typesets correctly.
 \greannotation{#1}%
\greannotation{#2}%
\grechangestyle{initial}{\fontsize{28}{28}\selectfont}%
 \gregorioscore{partitions/#3}%
 }
 
  \makeatletter
\NewDocumentCommand{\idxscore}{mO{}mmm}{
  %% #1 : name of the score file
  %% #2 : office part if needed
  %% #3 :mode numeral
  %% #4 :office part letter T, H, A, P, R, F, V
    %% #5 : the indexed name of the piece; needs to be repeated (unfortunately — — — will be alphabetized as such, not under previous entry

  
  %% this prevents page breaks between the phantom section and its label, and the actual score.
  \needspace{4\baselineskip} 
  \protected@edef\@currentlabelname{#5}
  \phantomsection
  \label{#1}
\gscore[#2]{#3}{#1}
  %% if #5 (indexed name) is blank, nothing is indexed.
  %% this is for pieces that are repetitions of another piece (antiphons after psalms)
%  \ifblank{#5}{}
  %% if optional arg #1 has been passed as 'n', set no initial
%%%  \ifx n#1\gresetinitiallines{0}\fi
%  \gregorioscore{partitions/#1}
   {\index[#4]{#5}}
  }
\makeatother


%% score with no initial, e.g psalms, common tones, etc. need not be indexed as such
\newcommand{\smallscore}[2][y]{
  \gresetinitiallines{0}
  \gregorioscore{partitions/#2} %% à remplacer avec NewDocumentCommand ; les arguments ne marchent pas correctement … que fait le [y] ?
  \gresetinitiallines{1}
}

%%using the new bar-spacing algorithm by default optimizes spacing. Occasionally asterisks or other signs float too far in one direction or the other.

%%Left default is 0.3cm, right default 0.15cm

\NewDocumentCommand{\zerobaroffsettextleft}{}%
{%
\grechangedim{maxbaroffsettextleft}%
{0cm}%
{scalable}%
}

\NewDocumentCommand{\zerobaroffsettextright}{}%
{%
\grechangedim{maxbaroffsettextright}%
{0cm}%
{scalable}%
}

%%%the default for left bar offset is 0.3cm. Right is 0.15cm. This needs to be applied after a score where the bar offset is modified.

\NewDocumentCommand{\resetbaroffsettextleft}{}%
{%
\grechangedim{maxbaroffsettextleft}%
{0.3cm}%
{scalable}%
}

\NewDocumentCommand{\resetbaroffsettextright}{}%
{%
\grechangedim{maxbaroffsettextright}%
{0.15cm}%
{scalable}%
}

% \NewDocumentCommand{\MagAnnotation}{}{%
% \grechangedim{annotationseparation}{-0.01cm}{fixed}%
% } %% pour les antiennes où le chiffre convient mieux aux miniscules (parfois 1, 2, 4, 7…) ; il n'est pas nécessaire pour 6 ou 8.
%%\gresetheadercapture{annotation}{greannotation}{string}

%% A-bove L-ines T-ext shall be italicized and in a smaller font
\grechangestyle{abovelinestext}{\itshape\fontsize{8}{10}\selectfont}
%% for psalm tones etc. Roman/upright text is what the Liber Usualis uses. 
\NewDocumentCommand{\altnormal}{}{\grechangestyle{abovelinestext}{\normalfont\fontsize{8}{10}\selectfont}}
%%  resets ALT font
\NewDocumentCommand{\altitshape}{}{\grechangestyle{abovelinestext}{\itshape\fontsize{8}{10}\selectfont}} 
%%for hymns like Vexilla Regis (May 3)
\NewDocumentCommand{\translationnormal}{}{\grechangestyle{translation}{\normalfont}}
%%  resets ALT font
\NewDocumentCommand{\translationreset}{}{\grechangestyle{translation}{\itshape}} 

%% fine-tuning of space between the staff and the text above lines (used for Magnificats; needs to be rethought  and worked into \smallscore
\newcommand{\altraise}{-0.4mm} %% default is -0.1cm %% needs to be fine-tuned for \rubrique command with EB Garamond font. -0.4mm is MB value, previously 1.4mm and -0.2cm  (this value works best for Magnificat but it does not work with common tones)
\grechangedim{abovelinestextraise}{\altraise}{scalable}

\newcommand{\altheight}{0.5cm} %% default is 0.3cm and value must be bigger than text height; this should fix the problem. but needs further investigation. currently 6mm in MB command
\grechangedim{abovelinestextheight}{\altheight}{scalable} %% this is a very finicky command.

%% Essentially what happens is that there is either not enough space above (Magnificat, crashing into words above) or below (common tones, crashing into staff)

%% fine-tuning of space between the staff and the lyrics

%\newcommand{\textraise}{2.8ex} %% default is 3.48471 ex MB 2.8ex
%\grechangedim{spacelinestext}{\textraise}{scalable}

%%% fine-tuning of space between the initial and the annotations
%\newcommand{\annraise}{0mm} %% default is -0.2mm MB 0mm
%\grechangedim{annotationraise}{\annraise}{scalable}

%% fine-tuning the behavior of text placed under bars. We use the so-called "new algorithm" which
%% places the bar in the middle of surrounding notes, and the text in the middle of surrounding text.
%% however, we restrict drastically the deviation of the text from the position of the bar.

%\grechangedim{maxbaroffsettextleft}{0.5mm}{scalable} MB 0.5mm
%\grechangedim{maxbaroffsettextright}{0.5mm}{scalable}


%% allows printing of glyphs from Gregorio score font as text (available glyphs are listed in the Gregorio documentation).
%% above all needed for the indications before certain Magnificat tones
\makeatletter
\def\gretextglyph#1{{\gre@font@music\csname GreCP#1\endcsname}}
\makeatother

%%% FONT %%%
\setmainfont{EB Garamond}[UprightFont=EB Garamond Regular,
ItalicFont= EB Garamond Italic,
BoldFont= EB Garamond Bold,
Ligatures=Rare,
Numbers=Proportional,
Numbers=OldStyle]

%% all \kern numbers are based on this and should be removed/adjusted if this font is abandoned.
%% as of 5.17.2024 \kern, although more semantic, is being replaced systematically with \hspace to better respect LaTeX format

\newfontfamily\symbolfont{liturgy}
  %% text must be written {\symbolfont{+}} or called via a command but may conflict as + is † in gabc. Cross should be \small or even \footnotesize. Font also supports V & R for verse/response, but EB Garamond contains appropriate matching glyphs.

%StylisticSet=6 is long Q. Abandoned but always possible to use again

%%%Roman numerals for the year %% https://tex.stackexchange.com/questions/185548/the-year-in-roman-and-the-month-in-text

\makeatletter
\newcommand{\YEAR}{\@Roman{\the\year}}
\makeatother

%%% TYPOGRAPHICAL AND SYMBOL COMMANDS  %%%%

%% \P is pilcrow and is defined in LaTeX as is.
%% Add {\textcolor{gregoriocolor} as first part of command's definition if changing to red.

%%for proper spacing of all-caps letters to prevent clashes, e.g. serif strokes touching.

\NewDocumentCommand\capspace{m}{%
{\addfontfeature{LetterSpace=5.0}%
{%
#1}%
}}

%%for smallcaps

\NewDocumentCommand\scspace{m}{\textsc{{{\addfontfeature{LetterSpace=5.0}{#1}}}}}

%% macro to print Alleluia for versicles.
\NewDocumentCommand{\tpalleluia}{}{(\textit{T.P.} Allelúia.)}

\newcommand{\specialcharhsep}{3pt} % space after invoking R/ or V/ or A/ outside rubrics from Matthias with value of 3mm
%%3mm – note unit— is value from MB; changed to pt

%\newcommand{\vv}{\textcolor{gregoriocolor}{\fontspec[Scale=\customscale]{Charis SIL}℣.\nolinebreak[4]\hspace{\specialcharhsep}\nolinebreak[4]}} %% format from MB

\newcommand{\vv}{{\normalfont ℣.\nolinebreak[4]\hspace{\specialcharhsep}\nolinebreak[4]}}
\newcommand{\rr}{{\normalfont ℟.\nolinebreak[4]\hspace{\specialcharhsep}\nolinebreak[4]}}

%\newcommand{\vv}{{\normalfont ℣.~}} %%at the beginning of a line so it shouldn't ever split from the rest but space is needed for macro to work
%\newcommand{\rr}{{\normalfont ℟.~}}

%\newcommand{\cc}{{\fontspec[Scale=\customscale]{FreeSerif}\symbol{"2720}~}}%from Matthias


%% typesets a cross pattée
\NewDocumentCommand{\cc}{}{%
    % Grouping to keep font changes local
    {%
        % Ensure we're not in italics (since liturgy.ttf doesn't have italic)
        \normalfont%
        % Select the font liturgy.ttf
        \symbolfont%
        % Set the size
        \footnotesize%
        % In liturgy.ttf, the plus (+) is a cross pattée
        +%
    }~%    
} 
%%  can replace <+> with <U+2720> (see above) if a suitable font is found (or EB Garamond font is fixed); command will be useful in lieu of typing the Unicode.
%% use <v> tag to insert into a gabc score (usually for the faithful or for blessings, e.g. the font.

%% Same special characters, for in-score use (<sp>V/ R/ A/ +</sp>)

\gresetspecial{V/}{{\fontspec[Scale=\customscale]{EB Garamond}℣.~}}
\gresetspecial{R/}{{\fontspec[Scale=\customscale]{EB Garamond}℟.~}}
\gresetspecial{+}{{\fontspec[Scale=\customscale]{EB Garamond}†~}}
\gresetspecial{*}{\gresixstar}
\gresetspecial{cross}{\cc}

%% Same special characters, for use in rubrics (no added space will keep symbol together with incipit)
%% no scale included

\newcommand{\vvrub}{{\normalfont ℣.~}}
\newcommand{\rrrub}{{\normalfont ℟.~}}

%%rubrics: black italics, smaller than body of psalms etc

\NewDocumentCommand{\rubrique}{m}{%
    {%
        \fontsize{8}{10}%
        \selectfont%
        \textit{%
        %
        #1%
    }%
    }}
    
%% macro to print normal text inside of rubric (name of a chant or prayer, etc.)

\NewDocumentCommand{\normaltext}{m}{%
    {%
        \normalfont%
        #1%
    }%
    }
    
%% in case something should be bolded inside of a rubrique
\NewDocumentCommand{\rubriquegras}{m}{%
    {%
        \normalfont%
\textbf{#1%
}%
}}

%% to print in red instead of italicizing
\NewDocumentCommand{\rouge}{m}{%
\textcolor{gregoriocolor}%
{\fontsize{8}{10}%
\selectfont{%
#1%
}%
}}

%% to print in black within rouge group.

%\NewDocumentCommand{\textenoir}{m}{%

\NewDocumentCommand{\textenoir}{m}{%
\textcolor{black}%
{%
\normalfont%
#1%
}%
}

%% rouge but in italic (this is technically not correct but is contemporary practice of Le Barroux). Here for completeness.
\NewDocumentCommand{\rougeit}{m}{%
\textcolor{gregoriocolor}%
{\fontsize{8}{10}%
\selectfont%
\textit{#1%
}%
}}

%%https://tex.stackexchange.com/questions/156540/spacing-between-single-line-paragraph-with-lettrine%
%\pretocmd{\lettrine}{\checklettrine}{}{} 
%\newcommand{\checklettrine}{%
%  \ifnum\prevgraf<2 \vspace{\baselineskip}\fi
%}

%%this value seems more balanced with this font than the definition provided in the ecclesiasticlatin documentation.

%typesets a horizontal rule like on the page with the prayers before and after the office.
%%If you're using color at all in your document, you might want to either force \myrule to use black or make it customizable. (from u/Independent-Comb-257 on Reddit)
%%thickness started at 0.4pt%%
\NewDocumentCommand{\myrule}{}{%
    \par%
    {%
        \centering%
       \textcolor{black}{\rule{0.3\textwidth}{0.6pt}}%
        \par%
    }%  
}  

%%typesets a horizontal rule like on the page with the prayers before and after the office.
%%thickness started at 0.4pt

\NewDocumentCommand{\biggerrule}{}{%
    \par%
    {%
        \centering%
        \textcolor{black}{\rule{0.75\textwidth}{0.6pt}}%
        \par%
    }%
    }

%% macro to format psalm text. Revised May 17, 2024 to use `enumerate` rather than inserting numbers in .tex file while usng `itemize` and to specify start value via `enumitem` package. 
%% 
%% these values reflect Liber Usualis but do not appear as if typed on a typewriter. Single words (particularly monosyllables) may need adjustment.
%%leftmargin=* and itemindent=* appear to be the way to a hanging indent like the modern antiphonal (but that looks like crap)
\NewDocumentCommand{\pstexte}{m}{%
    \smallskip%
    \noindent%
    \begin{enumerate}[%
			label=\arabic*.,%
%			align=left,%
			leftmargin=10pt, %
			itemindent=15pt, %
			labelsep=3pt, %
			labelwidth=0pt,
			rightmargin=0pt, %
			parsep=0pt, %
			topsep=0pt, %
			itemsep=0pt,%
			start=2]
       \input{psaumes/#1}
    \end{enumerate}}
    
    %%same as above but for the printing of the Magnificat since it always has 2 chant verses. (As of 5.17.2024) it seems more reasonable to do this unless there is an even better way to do so  to avoid 4 arguments per invocation of \psalmus, siince that 3rd argument will just about always be {2}.
    
%%  the problem is that this repeats a bunch in the Magnificat section and that the Nunc Dimittis sometimes begins on 3.
    \NewDocumentCommand{\magnificattexte}{m}{%
    \smallskip%
    \noindent%
    \begin{enumerate}[%
			label=\arabic*.,%
%			align=left,%
			leftmargin=10pt, %
			itemindent=15pt, %
			labelsep=3pt, %
			labelwidth=0pt,
			rightmargin=0pt, %
			parsep=0pt, %
			topsep=0pt, %
			itemsep=0pt,%
			start=3]
       \input{psaumes/#1}
    \end{enumerate}}
    
     %% Command de Matthias Bry modifié, prints psalm incipit score with the text
     %% #1 is psalm number (so an integer from 1 to 150
%% #2 is the gabc file somethng like #1_7a.gabc — if there is a way to append a variable mode and ending to #1, then that'd be cool
%% #3 is the psalm file name described above (and which is confusingly a different argument number when the command is nested): something like 109_7.tex

        \NewDocumentCommand{\psalmus}{mmm}{
	\needspace{4\baselineskip}%
	\smalltitle{Psalmus #1.}%            %% needspace plus smalltitle with \vspace{baselineskip} adds a lot of white space
	\smallscore[n]{#2}%
	\pstexte{#3}}
    
    %%r5.17.2024: removed obsolete, never-really-used version for multicolumn formatting inappropriate on such a small sheet
    
%% sd=semiduplex. especially for Dominica ad Vesperas in the Psalterium (the only example which comes to mind, in fact!)
 \NewDocumentCommand{\sdpsalmus}{mm}{
	\needspace{4\baselineskip}%
		\smalltitle{Psalmus #1.}%  needspace plus smalltitle with \vspace{baselineskip} adds a lot of white space
	\pstexte{#2}%
}

%%identical to \pstexte EXCEPT we need to sometimes start the canticle on 3 such as for Compline of the Paschal Vigil
\NewDocumentCommand{\cantiquetexte}{mm}{%
    \smallskip%
    \noindent%
    \begin{enumerate}[%
			label=\arabic*.,%
%			align=left,%
			leftmargin=10pt, %
			itemindent=15pt, %
			labelsep=3pt, %
			labelwidth=0pt,
			rightmargin=0pt, %
			parsep=0pt, %
			topsep=0pt, %
			itemsep=0pt,%
			start=#1]
       \input{psaumes/#2}
    \end{enumerate}}

    %% Command de Matthias Bry modifié, prints canticle incipit score with the text and scriptural reference
    
%%    \vspace{-\baselineskip} essentially removes the spacing from \smalltitle but enough versus too much space or widows/orphans is hard (as of 5.17.2024)

  %% #1 is Canticle name in Latin, e.g. (and most often) Canticum Simeonis.
%% #2 is the Scriptural citation from the Vulgate as given in the antiphonal/Solesmes editions.
%% #3 is the gabc file somethng like #1_7a.gabc — if there is a way to append a variable mode and ending to #1, then that'd be cool
%% #4 is the verse at which the list (verses wth \item) starts on  (pretty much always 2 or 3)
%% #5 is the psalm file name described above (and which is confusingly a different argument number when the command is nested): something like 109_7.tex
    \NewDocumentCommand{\canticum}{mmmmm}{%
	\needspace{4\baselineskip}%
	\smalltitle{Canticum #1.}%
	\vspace{-\baselineskip}%
	\smalltitle{\textit{#2.}}%
	\smallscore[n]{#3}%
	\cantiquetexte{#4}%
	{#5}%
}


%%in particular for canticle at OPBMV and Compline of dead for 1 November after II Vespers and Vespers of dead
  \NewDocumentCommand{\canticumspecial}{mmm}{
	\needspace{4\baselineskip}%
	\smalltitle{#1.}%    %%filename        %% needspace plus smalltitle with \vspace{baselineskip} adds a lot of white space
	\smallscore[n]{#2}% %%file
	\cantiquetexte{#3}% %%verse number, can be 2 or 3, in case of Compline of the dead
	}

% Could use, before \smallscore \grecommentary taking an argument instead but this sits too low on Nunc Dimittis
    
 %% macro to print any additional text (Capitiulum, oratio, rubrics)
 \NewDocumentCommand{\textes}{ m }{%
    \input{textes/#1}%
    }
    
    
    %% SC work for page headers and for secondary headings but not so much for things like "Dominica…" and certainly not the feast name%%
    
    %%%% Headers%%%%
    \pagestyle{fancy}
\fancyhead{}
\fancyfoot{}
\renewcommand{\headrulewidth}{0pt}

\fancyhead[RO]{\small\rightmark\hspace{0.5cm}\thepage}
\fancyhead[LE]{\small\thepage\hspace{0.5cm}\leftmark}

\newcommand{\setheaders}[2]{
	\renewcommand{\rightmark}{{\sc#2}}
	\renewcommand{\leftmark}{{\sc#1}}
}
\setheaders{}{}

%%note as of 5.17.2024: we need to learn how to get fancyhdr or titlesec or both to work in the fulll book


 %% TITLE COMMANDS %%%% had been dead in NR version of the same, but Matthias says this isn't a good idea.
 %%need to figure out multiple section headers AND a chapter header so that everything fits…
 
%%these are defined with \sc but we want full caps for the first two in particular. Also, Missing number, treated as 0 occurs when section titleformat is copied… 
 
 \titleformat{\chapter}[display]{\Huge\filcenter\sc\addfontfeature{LetterSpace=5.0}}{}{0pt}{} %%
\titleformat{\section}[display]{\huge\filcenter\sc\addfontfeature{LetterSpace=5.0}}{}{}{} %%
\titleformat{\subsection}[display]{\LARGE\filcenter\sc\addfontfeature{LetterSpace=5.0}}{}{}{} %%originally \large, then \Large
\setcounter{secnumdepth}{0}
\titlespacing*{\chapter}{0pt}{-25pt}{20pt} %%this should reduce spacing before chapter title while putting it flush with the top of the text area
%% see https://tex.stackexchange.com/questions/63390/how-to-decrease-spacing-before-chapter-title
%%\titlespacing*{\chapter}{0pt}{-50pt}{40pt}  %%this puts chapter title in HEADER which we do not want
\titlespacing{\section}{0pt}{*0}{*0}

%%need section titleformat more appropriate for weekdays of psalter and minor feasts 
% See https://tex.stackexchange.com/questions/623797/multiple-section-styles-in-same-document.

\addto\captionsecclesiasticlatin{\renewcommand\contentsname{\centerline{INDEX GENERALIS.}}}

%this command still needs work
\NewDocumentCommand{\header}{m}{\setheaders{{\scshape\addfontfeature{LetterSpace=5.0}#1}}{{\scshape\addfontfeature{LetterSpace=5.0} #1}}}

\NewDocumentCommand{\smalltitle}{m}{%
\needspace{5\baselineskip}%  %% very small if there are neumes above the staff, including flats in mode 2, e.g. O Doctor optime
\vspace{\baselineskip}%
 {\centering #1\par}%
 }
 
  %% bigtitle as written adds too much space below (or rather the combination of needspace is too large with following  text)
  
  %% possibile to have something that is a macro which takes away a fixed amount? and then I can change the definition later (though as of 17 dec 2023), cf. the screenshot from the Praglia code
 
 \NewDocumentCommand{\bigtitle}{m}{%
\needspace{5\baselineskip}%
\vspace{\baselineskip}%
 {\centering{\Large#1}\par}%
 }
 
 %%formatting date on major feasts of sanctoral cycle or moveable feasts that follow Easter where this is listed in antiphonal.

%  \NewDocumentCommand{\biglitdate}{m}{%
%\needspace{5\baselineskip}  %% very small if there are neumes above the staff, including flats in mode 2, e.g. O Doctor optime
%\vspace{\baselineskip}
% {\centering{\large #1}}\par} %}


%%formatting date of feasts where this listed at top of page
\NewDocumentCommand{\litdate}{m}{%
\smalltitle{Die #1.}%
}

%for most/ordinary feasts of Sanctorale 
%%"as long as text isn't long enough to over-print number according to David Carlisle
\NewDocumentCommand{\datefeast}{mm}{%
  \par
  \noindent
  \makebox[0pt][l]{#1.}%
  \makebox[\textwidth]{\Large #2.}%
  }
  
  \NewDocumentCommand{\specialdatefeast}{m}{%
  \par
  \noindent
  \makebox[0pt][l]%
  {\phantom{10.}} %
  \makebox[\textwidth]{\Large #1.}%
  }
  
  %%for feasts such as Imposition of Stigmata of St Francis on Sept 17.
  %%first two arguments as above, Dedication of the Lateran on Nov 9
  %%break line as needed and the remaining text is the 3rd argument
   \NewDocumentCommand{\longdatefeast}{mmm}{%
  \par
  \noindent
  \makebox[0pt][l]{#1.}%
  \makebox[\textwidth]{\Large #2}%
\hfill\makebox[\textwidth]{\large #3.}%
  }
  
%%  alternatively
%%  \NewDocumentCommand\datefeast{mm}{%
%%\noindent #1.\hfill{}{\Large#2.}\hfill{}%
%%}


%%formatting for Sunday feasts : inter, post… name is for consistency
\NewDocumentCommand{\dominicadate}{m}{%
\smalltitle{Dominica #1.}%
}

%%formatting for feasts that fall on a weekday fixed to some other thing, like Corpus Christi, Sacred Heart, BVM in Passiontide, and the new feasts found in the appendix (Feria Roman numeral in Latin: inter, post… name is for consistency
\NewDocumentCommand{\feriadate}{m}{%
\smalltitle{Feria #1.}%
}


%%for most important feasts (as of 16 dec 2023 leaving section alone as is, but it's similar styling)
\NewDocumentCommand{\festum}{m}{%
\needspace{5\baselineskip}%
\vspace{\baselineskip}%
 {\centering{\LARGE%
 {\MakeUppercase{%
 {\capspace{#1.}}
 }%
 }%
\par}%
 }%
 }

%%for ranks of feast; easiest to use rank for duplex 1 or 2 classis because the number changes, the octaves change.

%%probably need to reduce vspace used

\NewDocumentCommand{\rank}{m}{%
\needspace{5\baselineskip}%
\vspace{\baselineskip}%
 {\centering{\textit{\large #1.}}\par}%
}

%%looks like xparse formatting changed so o just has No Default Value whereas O{} is that OR you can insert something in braces!
%%answer thanks to David Carlisle
\NewDocumentCommand{\duplexclassis}{mo}{%
\needspace{5\baselineskip}%
\vspace{\baselineskip}%
 {\centering\textit{\large Duplex #1 classis\IfValueT{#2}{ #2}.}\par}%
}


\NewDocumentCommand{\duplexmajus}{}{%
\needspace{5\baselineskip}%
\vspace{\baselineskip}%
 {\centering{\textit{\large Duplex majus.}}\par}%
}

%%is Duplex even specified.as it's the default?
\NewDocumentCommand{\duplex}{}{%
\needspace{5\baselineskip}%
\vspace{\baselineskip}%
 {\centering{\textit{\large Duplex.}}\par}%
}

\NewDocumentCommand{\semiduplex}{}{%
\needspace{5\baselineskip}%
\vspace{\baselineskip}%
 {\centering{\textit{\large Semiduplex.}}\par}%
}

\NewDocumentCommand{\simplex}{}{%
\needspace{5\baselineskip}%
\vspace{\baselineskip}%
 {\centering{\textit{\large Simplex.}}\par}%
}

%%for feasts where the third verse (third part of 1st verse in modern terms) is changed (Mutatur tertius vers.)

\NewDocumentCommand{\mtv}{}{%
{\normalfont{(m.t.v.)}%
}%
}

%%e.g. St Joachim, D2C
\NewDocumentCommand{\duplexclassismtv}{mo}{%
\needspace{5\baselineskip}%
\vspace{\baselineskip}%
 {\centering\textit{\large Duplex #1 classis.} \mtv \par}%
}

\NewDocumentCommand{\duplexmtv}{}{%
\needspace{5\baselineskip}%
\vspace{\baselineskip}%
{\centering{\textit{\large Duplex.} \mtv \par}%
}%
}

\NewDocumentCommand{\duplexmajusmtv}{}{%
\needspace{5\baselineskip}%
\vspace{\baselineskip}%
{\centering{\textit{\large Duplex majus.} \mtv \par}%
}%
}

\NewDocumentCommand{\semiduplexmtv}{}{%
\needspace{5\baselineskip}%
\vspace{\baselineskip}%
{\centering{\textit{\large Semiduplex.} \mtv \par}%
}%
}

\NewDocumentCommand{\simplexmtv}{}{%
\needspace{5\baselineskip}%
\vspace{\baselineskip}%
{\centering{\textit{\large Simplex.} \mtv \par}%
}%
}

%% The needspace package inserts white space that we don't want!
%%especially for Sab/Dom PP which has sentence-case text below main title of page
\NewDocumentCommand{\subtitle}{m}{%
 {\centering{\Large #1}\par%
 }%
 }

%%to replace original idea of subsection; will need work on the spacing (the needspace and vspace both might be too big
%%leaving subsection as is for now (16 dec 2023), for reference and in case I used it and didn't remember it
%% argument here is i or ii; will need another command if the vespers are identical.
 \NewDocumentCommand{\invesperis}{m}{%
\needspace{5\baselineskip}%
\vspace{\baselineskip}%
 {\centering{\LARGE%
 {\scspace{in #1 vesperis.}%
 }%
 }\par}%
}

%%for double major and below
 \NewDocumentCommand{\invesperisminor}{m}{%
\needspace{5\baselineskip}%
\vspace{\baselineskip}%
 {\centering{\Large In #1 Vesperis.}\par}%
 }

\NewDocumentCommand{\primaantiphona}{m}{%
\smalltitle{I Antiphona. #1}%
}

%%for vel where this occurs between texts or chants

\NewDocumentCommand{\vel}{}{%
{\rubrique{Vel:}%
}%
}

%% for prayers, e.g. at beginning (and end) of office.
\NewDocumentCommand{\secreto}{}{%
Pater noster. \rubrique{secreto.}%
}

\NewDocumentCommand{\pateravedeus}{}{%
Pater noster et Ave María. ℣. Deus in adjutórium.%
}

\NewDocumentCommand{\pateravecredo}{}{%
Pater noster, Ave María, \textit{et} Credo.%
}

\NewDocumentCommand{\lectiobrevis}{m}{%
{\quad Lectio brevis.%
\hfill \textit{#1}%
\hspace*{1em}%
\par%
}%
}

%%%original commands
%\NewDocumentCommand{\capitulum}{}{%
%\smalltitle{Capitulum.}%
%}

%%for Scriptural references, originally of the chapter but now primarily Magnificat in psalter
\NewDocumentCommand{\scripture}{m}{%
{\raggedleft{\textit{#1}%
}%
\par}%
}

%%above chapter (Capitulum)
%%should indent by 1 em or a quad and then preserve space at end of line without needing to resort to table
%%imitation,sort of, of Liber antiphonarius style; AM1934 indentation is bigger
%%argument is Scriptural reference
\NewDocumentCommand{\capitulum}{m}{%
{\needspace{5\baselineskip}%
\vspace{\baselineskip}%
\quad Capitulum.%
\hfill \textit{#1.}%
\hspace*{1em}%
\par%
}%
}

%%for use in commons
%%#1 is Pro… e.g. Pro Virgine Martyre.
%%#2 is Scriptural citation Prov. 31, 10 – 11.
%%one problem is that hfill doesnt really CENTER it

\NewDocumentCommand{\specialcapitulum}{mm}{%
{\needspace{5\baselineskip}%
\vspace{\baselineskip}%
\quad #1%
\hfill Capitulum.%
\hfill \textit{#2.}
\hspace*{1em}%
\par%
}%
}


%%above Hymn if listed this way in the antiphonal (Hymnus)
\NewDocumentCommand{\hymnus}{}{%
\smalltitle{Hymnus.}%
}

%%especially for common of apostles etc. in paschal time
\NewDocumentCommand{\hymnusspecial}{mO{}}{%
\smalltitle{#1, Hymnus. #2}%
}

%%for proper of saints where the hymn is proper
\NewDocumentCommand{\hymnusbothvespers}{}{%
\smalltitle{In utrisque Vesperis, Hymnus.}%
}

\NewDocumentCommand{\hymnusmode}{m}{%
\smalltitle{Hymnus. #1}%
}

%%for 2nd tone of hymn
\NewDocumentCommand{\altertonus}{}{%
\smalltitle{Alter Tonus.}%
}

%%for 2nd tone of hymn where Alius is employed %%optional argument is in particular for Iste Confessor tones of which there are 4 per common of confessors (bishop and not a bishop)
\NewDocumentCommand{\aliustonus}{O{}}{%
\smalltitle{Alius Tonus. #1}%
}

%%above Te lucis

%%for, e.g. St Michael (not present in LA but preesnt in LU), St John Cantius. Hymn uses what would be ALT but we wish to avoid this feature. Unlike Ave Maris Stella rubric,
%% which is never in LA, this is, at least once, so we include it.

\NewDocumentCommand{\sequensconclusio}{}{%
\rubrique{Sequens conclusio numquam mutatur.}%
}

\NewDocumentCommand{\hymnusadcompletorium}{}{%
\smalltitle{Tonus Hymni ad Completorium.}%
}

%%above Magnificat antiphon where this is done in Liber antiphonarius (Or Vaticana Vesperale) (Sundays to be determined because there are so many of them; Liber Usualis and ferial format may be best)
\NewDocumentCommand{\admagnificat}{}{%
\smalltitle{Ad Magnificat, Antiphona.}%
}

\NewDocumentCommand{\admagnificatuv}{}{%
\smalltitle{In utrisque Vesperis, ad Magnificat, antiphona.}%
}

%%to pair with initial score in cases where only Mag antiphon is proper e.g. on Feb 2
\NewDocumentCommand{\admagnificatmode}{m}{%
\smalltitle{Ad Magnificat, Antiphona. #1}%
}

%%for use in collects etc. %%\@ would lead to too narrow of a space (see LU)
%%needs to be followed by {}
\NewDocumentCommand{\nomen}{}{%
\textit{N.}%
}

%%above collect (Oratio)
\NewDocumentCommand{\oratio}{}{%
\smalltitle{Oratio.}%
}

%%for use in the commons
\NewDocumentCommand{\aliaoratio}{}{%
\smalltitle{\textit{Alia Oratio.}}%
}

%%for use in the commons
\NewDocumentCommand{\specialoratio}{m}{%
\smalltitle{#1 \textit{Oratio.}}%
}


\NewDocumentCommand{\bmvtone}{}{%
{\rubrique{Ad Completorium, Tonus et Doxologia de B.M.V.}%
}%
}

%%%these are used before II Vespers
\NewDocumentCommand{\omniapraeter}{}{%
{\rubrique{Omnia ut in I Vesperis, præter sequentia.}%
}%
}

\NewDocumentCommand{\ivesperisrubrique}{}{%
{\rubrique{Antiphonæ, Psalmi, Capit.\@ et Hymnus ut in I Vesperis.}%
}%
}

\NewDocumentCommand{\ivesperisrubric}{}{%
{\rubrique{Antiphonæ, Psalmi et Capitulum ut in I Vesperis.}%
}%
}

\NewDocumentCommand{\sedlocoultimi}{mm}{%
{\rubrique{Antiphonæ et Psalmi ut in #1 Vesperis, sed loco ultimi Ps.\@ \normaltext{#2,} ut infra.}%
}%
} 

%%for referring to chapter and/or hymn and/or verse placed at I or II Vespers (where II has different parts)
\NewDocumentCommand{\caphymn}{m}{%
\rubrique{Capitulum et Hymnus ut in #1 Vesperis.}%
}

\NewDocumentCommand{\caphymnvv}{m}{%
\rubrique{Capitulum, Hymnus, et ℣.\@ ut in #1 Vesperis.}%
}

\NewDocumentCommand{\hymnusut}{m}{%
\rubrique{Hymnus ut in #1 Vesperis.}%
}

%%a few doubles even are permanently impeded at II Vespers by another (higher-ranking) double the next day, e.g. St Clement, St Cecilia…
\NewDocumentCommand{\procomm}{m}{%
\smalltitle{Pro comm.\@ S.\@ #1, Antiphona.}%
}

%%so many confessors on doubles

\NewDocumentCommand{\hicvir}{}{%
\rubrique{Comm.\@ præc.\@ Ant. \normaltext{Hic vir, \pageref{M-an_cs_hic_vir_despiciens_solesmes_1961}, \vv Justum dedúxit.}}%
}


%%for commemoration of following at II Vespers of major feasts.
\NewDocumentCommand{\commemoration}{}{%
\rubrique{Et fit comm.\@ sequentis.}%
}

%%for d2c feasts followed by a simple
\NewDocumentCommand{\nocommseq}{}{%
\rubrique{Et non fit comm.\@ sequentis.}%
}

%%for Vespers of the following
\NewDocumentCommand{\vespsequenti}{}{%
\rubrique{Vesp.\@ de sequenti.}%
}


%%for Vespers of the following
\NewDocumentCommand{\sinecomm}{}{%
\rubrique{Vesp.\@ de sequenti, sine ulla commemoratione.}%
}

%%for Vespers of the following at II Vespers
\NewDocumentCommand{\commsequentis}{}{%
\rubrique{In II Vesperis, comm.\@ sequentis.}%
}

%%for Vespers of the following where the commemoration of the previous day is still made
%%fixed latin sequentis to sequenti Sept 4 2024
\NewDocumentCommand{\vespsequentiscomm}{}{%
\rubrique{Vesp.\@ de sequenti, comm.\@ præced.}%
}

\NewDocumentCommand{\vespsequentiscomms}{m}{%
\rubrique{Vesp.\@ de sequenti, comm.\@ præced. et #1.}%
}

%%for Vespers of the following day beginning at the chapter (doubles that concur with each other)
%%changed 4 Sept 2024 from de to a because while de is probably OK Latin, I see a more often in the books.
\NewDocumentCommand{\capitdeseq}{}{%
\rubrique{Vesp. a capit.\@ a seq., comm.\@ præced.}%
}

\NewDocumentCommand{\capitdeseqcomms}{m}{%
\rubrique{Vesp. a capit.\@ a seq., comm.\@ præced. et #1.}%
}

\NewDocumentCommand{\capitdeseqquad}{}{%
\rubrique{Vesp. a capit.\@ a seq., comm.\@ præced. et Feriæ in Quadrag.}%
}

\NewDocumentCommand{\capitdeseqcommfer}{}{%
\rubrique{Vesp. a capit.\@ a seq., comm.\@ præced. et Feriæ.}%
}


%%commemoration of the Lenten feria
\NewDocumentCommand{\quadcommferiae}{}{%
\rubrique{In Quadrag.\@ commem. feriæ.}%
}

\NewDocumentCommand{\commferiae}{}{%
\rubrique{Commem.\@ feriæ.}%
}

\NewDocumentCommand{\odoctoroptime}{}{%
\rubrique{In utrisque Vesperis, Ant. \normaltext{O Doctor óptime, \pageref{M-an_o_doctor_optime_solesmes_1961}.}}%
}


%%for Fridays where Vespers is of Our Lady on Saturday, unless the next day is a feast of IX Lessons.
\NewDocumentCommand{\bvmsabbato}{}{%
\rubrique{Ad Vesperas, nisi occurrat Festum IX Lectionum, a Capitulo fit de Sancta Maria in Sabbato.}%
}

%%for feasts of the BVM
\NewDocumentCommand{\infestisbmv}{}{%
\rubrique{Antiphonæ, Psalmi et Capitulum ut in festis B.M.V., \normaltext{\pageref{M-vr_NN_commune_bmv}.}%
}%
}

\NewDocumentCommand{\amsrubrique}{}{%
\rubrique{Hymnus \normaltext{Ave Maris Stella, \pageref{M-hy_ave_maris_stella_mode_1_LA1949}.}%
}%
}

\NewDocumentCommand{\vigiliarubrique}{m}{%
\rubrique{De Vigilia #1 nihil fit nisi in Missa.}%
}

%%%%%%%SPACING%%

%\NewDocumentCommand{\baselineskipvspace}{m}{%
%\vspace{#1\baselineskip}%
%}

%%does this count as a magic number?

\NewDocumentCommand{\baselineskipvspace}{}{%
\vspace{\baselineskip}%
}

\NewDocumentCommand{\variablevspace}{m}{%
\vspace{#1\baselineskip}%
}

\NewDocumentCommand{\halfbskipvspace}{}{%
\vspace{0.5\baselineskip}%
}

\NewDocumentCommand{\negativebaselineskipvspace}{}{%
\vspace{-1\baselineskip}%
}

\NewDocumentCommand{\neghalfbskipvspace}{}{%
\vspace{-0.5\baselineskip}%
}

\NewDocumentCommand{\teenyhspace}{}{%
\hspace*{0.01em}%
}
\NewDocumentCommand{\tinyhspace}{}{%
\hspace*{0.03em}%
}

\NewDocumentCommand{\myhspace}{}{%
\hspace*{0.04em}%
}

%%spur because of n, u, etc.
\NewDocumentCommand{\spurhspace}{}{%
\hspace*{0.05em}%
}


%% for litanies and other things like antiphon O Doctor Optime

\makeatletter  
\newcounter{score}
\newcounter{tabstop}[score]
\newcommand{\grealign}{%
	\@bsphack%
	\ifgre@boxing\else%
		\kern\gre@dimen@begindifference%
		\stepcounter{tabstop}%
		\expandafter\zsavepos{stop-\thescore-\thetabstop}%
		\kern-\gre@dimen@begindifference%
	\fi%
	\@esphack%
}

\newcommand{\setstops}{%
  \gdef\nstabbing@stops{%
    \hspace*{-\oddsidemargin}\hspace{-1in}%
    \hspace*{\zposx{stop-\thescore-1} sp}\=%
  }%
  \count@=\@ne
  \loop\ifnum\count@<\value{tabstop}%
    \begingroup\edef\x{\endgroup
      \noexpand\g@addto@macro\noexpand\nstabbing@stops{%
        \noexpand\hspace{-\noexpand\zposx{stop-\thescore-\the\count@} sp}%
        \noexpand\hspace{\noexpand\zposx{stop-\thescore-\the\numexpr\count@+1} sp}\noexpand\=%
      }%
    }\x
    \advance\count@\@ne
  \repeat
  \nstabbing@stops\kill
}
\makeatother

\newenvironment{nstabbing}
  {\setlength{\topsep}{0pt}%
   \setlength{\partopsep}{0pt}%
   \tabbing%
   \setstops}
  {\endtabbing\stepcounter{score}}

%%FOOTNOTES%%%

%%from perpage package \MakePerPage{footnote} as footnotes are just per page as it says
\MakePerPage{footnote}

%%PHANTOM SECTION MANIPULATION%%
\providecommand\phantomsection{} %% should make hyperref work properly %%needed with addcontentsline, not with section etc alone (needed if running hyperref)

\setcounter{tocdepth}{2} %% probably should hide subsections from TOC of most documents

%%%% command to wrap printindex and set the headers for indices
%%\newcommand{\cprintindex}[2]{
%%	\setheaders{Indices}{#2}
%%	\pagestyle{fancy}
%%	\thispagestyle{empty}
%%	\printindex[#1]
%%}

%%preface
      \setlength{\columnsep}{3pc}
%\setlength{\columnsep}{10pt}
\setlength\columnseprule{0.3pt}

%%for hymni antiqui
\NewDocumentEnvironment{hymnusmulticol}{}
 {\begin{multicols}{2}}
    {\end{multicols}}

\NewDocumentEnvironment{enpars}{}
  {\begin{otherlanguage*}{english}}
  {\par\end{otherlanguage*}}
  
  \NewDocumentEnvironment{frpars}{}
  {\begin{otherlanguage*}{french}}
  {\par\end{otherlanguage*}}
  
  \NewDocumentCommand{\hymnusantiquusrubrique}{m}{%
  \rubrique{Ad Completorium, tonus ut in Antiphonario, \normaltext{\pageref{#1}.}%
  }%
  }
  
%%%SUBFILES%%%
\usepackage{xr} %%to allow cross-references between documents%%%
  \usepackage{subfiles}
  
  %% When we start a new subfile (new chapter or section) , 
%% we start on a new page (with blank filling on the previous page) and create a corresponding label.

\newcommand{\customsubfile}[1]{\newpage\label{#1}\thispagestyle{empty}\subfile{#1}}


