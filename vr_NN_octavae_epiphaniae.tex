% !TEX TS-program = LuaLaTeX+se

\documentclass[vesperale_romanum.tex]{subfiles}

\ifcsname preamble@file\endcsname
  \setcounter{page}{\getpagerefnumber{M-vr_NN_octavae_epiphaniae}}
\fi

%%this code when \customsubfiles is used should allow for continuous pagination when subfiles are compiled individually.

%% am having trouble with crossreference for ps 112

%%\labe{112_4E} is in vr_NN_in_nativitate_usque_5_jan.tex and it's not resolving crossreference between the two files, not even if I run the main file which SHOULD get the right pages otherwise.

\begin{document}

%\thispagestyle{empty}

%\addcontentsline{toc}{section}{Ab Epiphianiam ad Septuagesimam} \phantomsection
%
%\festum{In Epiphania Domini.}

 \section[Ab Epiphianiam ad Septuagesimam]{IN EPIPHANIA DOMINI.}\header{in epiphania domini.}\header{in epiphania domini et in octava.}
%%could also be section with optional argument

\duplexclassis{I}[cum Octava privilegiata II ordinis]

%\rank{Duplex I classis.}

%%octave not in LA1949?

\invesperis{i} \label{1_vesperis_epiphaniae}

\rubrique{Antiphonæ et psalmi ut in II Vesperis, sed loco ultimi Ps. \normaltext{Laudáte Dóminum,} ut infra.}

\psalmus{116}{116_7c2}{116_7}

\caphymnvv{II}

\admagnificat

\gscore[]{8. G}{an_magi_videntes_stellam_solesmes_1961}

\rubrique{Oratio \normaltext{Deus, qui hodiérna,} ut infra \normaltext{\pageref{or_epiphaniae}.}}

\hymnusadcompletorium \label{tonus_oct_epiphaniae}

\gscore[]{8.}{hy_te_lucis_ante_terminum_in_epiphania_solesmes}

\rubrique{in Festo Sanctæ Familiæ:}

\smallscore{dox_holy_family}

\rubrique{Hic tonus servatur ad Completorium per totam Octavam.}

\invesperis{ii}

\primaantiphona{2. D}

\initialscore{an_ante_luciferum_genitus_solesmes_1961}

\psalmus{109}{109_2}{109_2}

\gscore[2. Ant.]{1. g2}{an_venit_lumen_tuum_solesmes_1961}

\psalmus{110}{110_1g2}{110_1}

%%LaTeX refuses to acknowledge the line break with hspace inserted so a manual, hard hyphen was inserted

\gscore[3. Ant.]{1. g2}{an_apertis_thesauris_solesmes_1961}

\psalmus{111}{111_1g2}{111_1}

\gscore[4. Ant.]{4. E}{an_maria_et_flumina_solesmes_1961}

%%need to make psalm label distinct so that psalm can go elsewhere with a label to minimize page turning and distance

\label{112_4E}

\psalmus{112}{112_4E}{112_4E}

%\rubrique{Ps. \normaltext{Laudáte Púeri, \pageref{M-112_4E}.}}
%
%%%should point to Nativity document

\gscore[5. Ant.]{7. c2}{an_stella_ista_solesmes_1961}

\psalmus{113}{113_7c2}{113_7}

%%ps 116 could be printed alone  above(since it's much shorter)

\capitulum{Isaiæ 60, 1}
\lettrine{S}{u}rge, illumináre Jerúsalem, quia venit lumen tuum,~* et glória Dómini super te orta est.

\hymnusmode{3.}

\initialscore{hy_crudelis_herodes_deum_solesmes_1961.gabc}

\rubrique{Sic terminantur Hymni per totam Octavam, nisi aliter concluendi sint, juxta rubricas.}

\smalltitle{Alter tonus ad libitum.}

\gscore[]{8.}{hy_crudelis_herodes_deum_another_chant_solesmes_1961}

\rubrique{In hoc altero tono cantantur Hymni per totam Octavam.}

\smallscore{versiculus_vesperis_epiphianiae_domini}

\rubrique{Sic cantantur in die Festi: alias adhibetur tonus solitus.}

\admagnificat

\gscore[]{1. D}{an_tribus_miraculis_solesmes_1961}

\oratio \label{or_epiphaniae}

\lettrine{D}{e}us, qui hodiérna die Unigénitum tuum géntibus stella duce revelásti:~† concéde propítius; ut qui jam te ex fide cognóvimus,~* usque ad contemplándam spéciem tuæ celsitúdinis perducámur. Per eúmdem Dóminum.

\rubrique{Si Epiphania venerit in Sabbato, in II Vesperis fit commemoratio Dominicæ infra octavam Epiphaniæ, Ant.\@ \normaltext{Remánsit, \pageref{com_dom_in_oct_epiphaniae}.}}

%%needs macro or separate file

\vv  Omnes de Saba vénient, allelúia.

\rr Aurum et thus deferéntes, allelúia.

\rubrique{Oratio \normaltext{Vota quǽsumus, \pageref{or_dom_1_pe}.}}

\rubrique{Infra Octavam Epiphaniæ ad Vesperas et ad Completorium, omnia dicuntur ut in die Festi, præter antiphonas ad Magnificat,
quæ singulis diebus habentur propriæ, ut sequitur.}

\rubrique{De Dominica infra Octavam non dicitur Officium, sed fit tantummodo commemoratio in Festo Sanctæ Familiæ,  tum si hoc in ipsa celebretur Dominica,
tum si ob occursum diei Octavæ ipsum in diem Sabbati vel in proximiorem antecedentem Feriam anticipetur, ut infra notabitur.}
\biggerrule

\dominicadate{infra Octavam Epiphaniæ}

\bigtitle{Sanctæ Familiæ Jesu, Mariæ, Joseph.} \label{sancta_familia}
%\header {sanctæ familiæ jesu, mariæ, joseph.}
%%headers need to be fixed later…

\duplexmajus

\rubrique{Quando dies Octava Epiphaniæ in Dominicam inciderit, Sabbato præcedenti Officium fit de Sancta Familia, et Feria VI dicuntur I Vesperæ de ipsa Sancta Familia, cum commemoratiom præcedentis diei infra Octavam ac Dominicæ infra Octavam, ut infra.}

\rubrique{Sicubi tamen hoc Sabbato occurrat Festum Duplex I classis, Officium de Sancta Familia cum commemoratio de ipsius Dominicæ anticipatur in proximiorem Feriam in qua secus faciendum esset Officium de Octava; et in Officio tam Festi Duplicis I classis quam Sanctæ Familiæ fit commemoratio currentis diei infra Octavam.}

\rubrique{Denique, a die 7 ad 12 Januarii inclusive, ubi occurrerit Dominica simul et Festum Duplex 1 classis, in utrisque Vesperis Festi agitur commemoratio prius Sanctæ Familiæ, cujus Officium. cum omnibus et singulis juribus in perpetuum Officio Dominicæ subrogatum exstitit, deinde Dominicæ et Octavæ.}

\invesperisminor{I}

\primaantiphona{1. g}

%%1st antiphon  is same ant as Mar 19, needs to be exact same code and ideally gabc header.

\initialscore{an_jacob_autem_genuit_solesmes_1961}

\psalmus{109}{109_1g}{109_1}

%%2nd is not the same antiphon as Mar 19

\gscore[2. Ant.]{7. c}{an_angelus_domini_apparuit_holy_family_solesmes_1961}

\psalmus{112}{112_7c}{112_7}

\gscore[3. Ant.]{7. d}{an_pastores_venerunt_solesmes_1961}

\psalmus{121}{121_7d}{121_7}

\gscore[4. Ant.]{2. D}{an_magi_intrantes_solesmes_1961}
\label{ps_126_2}

\psalmus{126}{126_2}{126_2}

\gscore[5. Ant.]{1. f}{an_erat_pater_ejus_solesmes_1961}

\psalmus{147}{147_1f}{147_1}

\capitulum{Luc. 2, 51}

\lettrine{D}{e}scéndit Jesus cum María et Joseph, et venit Názareth,~* et erat súbditus illis.

\hymnus

\gscore[]{2.}{hy_o_lux_beata_caelitum_solesmes_1961}

\vv Beáti qui hábitant in domo tua, Dómine.

\rr In sǽcula sæculórum laudábunt te.

\admagnificat

\gscore[]{8. G}{an_verbum_caro_factum_est_solesmes_1961}

\oratio \label{or_sanctae_familiae}

\lettrine{D}{o}mine Jesu Christe, qui Maríæ et Joseph súbditus, domésticam vitam ineffabílibus virtútibus consecrásti:~† fac nos, utriúsque auxílio, Famíliæ san\-ctæ tuæ exémplis ínstrui;~* et consórtium cónsequi sempitérnum. Qui vivis et regnas cum Deo Patre.

\rubrique{Et fit commemoratio præcedentis diei infra Octavam:
Antiphona diei currentis propria, \normaltext{℣. Reges Tharsis,} oratio \normaltext{Deus, qui hodiérna.}}
\label{com_dom_in_oct_epiphaniae}
\rubrique{Deinde commemoratio Dominicæ.}

\gscore[Ant.]{2.}{an_remansit_puer_solesmes_1961}

%%needs macro or separate file

\vv  Omnes de Saba vénient, allelúia.

\rr Aurum et thus deferéntes, allelúia.

\oratio \label{or_dom_1_pe}

%%needs just the slightest shift with package option findent

\lettrine[findent=0.05em]{V}{o}ta quǽsumus Dómine, supplicántis pópuli cælésti pietáte proséquere:~† ut et quæ agénda sunt, vídeant,~* et ad implénda quæ víderint, convaléscant.
Per Dóminum.

%%probably could be a macro, for newer feasts only.

\rubrique{Completorium de Dominica.}

%%add pageref and label for tone of hymn at Compline
%%OR add only doxology in the same place.

\rubrique{¶ Ad Completorium, Hymnus cantatur in tono Octavæ \normaltext{\pageref{tonus_oct_epiphaniae}} cum doxologia \normaltext{Jesu, tuis obédiens,} ut supra.}

\invesperisminor{II}

\rubrique{Antiphonæ de I Vesperis quoties in eis de hoc Festo facta tantum fuerit commemoratio.}

\gscore[1. Ant.]{8. G}{an_post_triduum_solesmes_1961}

\psalmus{109}{109_8G}{109_8}

\gscore[2. Ant.]{4. E}{an_dixit_mater_jesu_ad_illum_solesmes_1961}
%%actually needs to go under Vespers of the Epiphany

%\psalmus{112}{112_4E}{112_4E}

\rubrique{Ps. \normaltext{Laudáte Púeri, \pageref{M-112_4E}.}}

%%does it need the prefix if it's in the same subfile. TBD!

\gscore[3. Ant.]{8. G}{an_descendit_jesus_cum_eis_solesmes_1961}

\psalmus{121}{121_8G}{121_8}

\gscore[4. Ant.]{2. D}{an_et_jesus_proficiebat_solesmes_1961}

\rubrique{Ps. \normaltext{Nisi Dóminus, \pageref{ps_126_2}.}}

\gscore[5. Ant.]{8. G}{an_et_dicebant_solesmes_1961}

\psalmus{147}{147_8G}{147_8}

\caphymn{I}

\vv Ponam univérsos fílios tuos doctos a Dómino.

\rr Et multitúdinem pacis fíliis tuis.

\admagnificat

\gscore[]{8. G}{an_maria_autem_solesmes_1961}

\rubrique{Oratio \normaltext{Dómine Jesu Christe,} ut supra, \normaltext{\pageref{or_sanctae_familiae}.}}

\rubrique{Et fit commemoratio sequentis diei infra Octavam:
Antiphona diei currentis propria, \normaltext{℣. Reges Tharsis,} oratio \normaltext{Deus, qui hodiérna.}}

\rubrique{Deinde commemoratio Dominicæ.}

\gscore[Ant.]{8.}{an_fili_quid_fecisti_solesmes_1961}

\vv  Omnes de Saba vénient, allelúia.

\rr Aurum et thus deferéntes, allelúia.

\rubrique{Oratio \normaltext{Vota quǽsumus} ut supra, \normaltext{\pageref{or_dom_1_pe}.}}

\rubrique{¶ Si dies Octava Epiphania inciderit in Dominicam, nihil fit de Dominica,
in ipsa die Octava, sed tantum in I Vesperis diei Octavæ fit commemoratio pro II Vesperis Dominicæ si die 12 Januarii, ut supra, celebratum fuerit Officium de Sancta Familia.}

%%the LA1949 follows the Pius X breviary in assigning the antiphons to DATES specifically

%%date is flush left on same line but for now (18 dec 2023) that seems too complicated.

\bigtitle{7 Jan. Secunda die infra Octavam.}

\gscore[Ad Magnif.]{Ant.\@ 7. a}{an_videntes_stellam_solesmes_1961}

\bigtitle{8 Jan. Tertia Die.}

\gscore[Ad Magnif.]{Ant.\@ 8. c}{an_lux_de_luce_solesmes_1961}

\bigtitle{9 Jan. Quarta Die.}

\gscore[Ad Magnif.]{Ant.\@ 8. G}{an_interrogabat_magos_solesmes_1961}

\bigtitle{10 Jan. Quinta Die.}

\gscore[Ad Magnif.]{Ant.\@ 8. c}{an_omnes_de_saba_solesmes_1961}

\rubrique{Commemoratio \normaltext{ S. Hygini,} Papæ et Martyris. Ant.\@ \normaltext{Iste sanctus, \pageref{M-an_cs_iste_sanctus_pro_lege_solesmes_1961}, ℣. Glória et honóre.}}

%%this is in the antiphonal as is. page for St Telesphorus refers (properly) to the commemorations section.

%\gscore[Ant.]{8.}{an_cs_iste_sanctus_pro_lege_solesmes_1961}
%
%\vv Glória et honóre coronásti eum Dómine.
%
%\rr Et constituísti eum super ópera mánuum tuárum.

\oratio

\lettrine{I}{n}firmitátem nostram réspice, omnípotens Deus:~† et quia pondus própriæ actiónis gravat,~* beáti Hygíni Mártyris tui atque Pontíficis intercéssio gloriósa nos prótegat.
Per Dóminum.

\bigtitle{11 Jan. Sexta Die.}

\gscore[Ad Magnif.]{Ant.\@ 1. f}{an_admoniti_magi_solesmes_1961}

\bigtitle{12 Jan. Septima Die.}

\rubrique{Vesperæ de sequenti die Octava.}

%\newpage

\bigtitle{In Octava Epiphaniæ.}

\duplexmajus

In I Vesperis, \rubrique{omnia dicuntur ut in I Vesperis Epiphaniæ, \normaltext{\pageref{1_vesperis_epiphaniae},} excepta oratione.}

\oratio

\lettrine{D}{e}us, cujus Unigénitus in substántia nostræ carnis appáruit:~† præsta, quǽsumus; ut per eum, quem símilem nobis foris agnóvimus,~* intus reformári mereámur. 
Qui tecum vivit et regnat in unitáte.

Ad Vesperas, \rubrique{omnia dicuntur sicut in die Epiphaniæ, excepta oratione \normaltext{Deus, cujus Unigénitus.}}

\rubrique{Quando Octava Epiphaniæ venerit in Sabbato, in II Vesperis octavæ fit commemoratio 
Dominicæ II Epiphaniæ, ut infra.}

Dominica I post Epiphaniam, \rubrique{quæ est infra octavam, Officium fit de Sancta Familia, ut supra,
\normaltext{\pageref{sancta_familia};} sed Dominica in diem 13 januarii inciderit, agitur de die 
Octava Epiphaniæ juxta rubricas.}

%%where to break these files is not always clear based on the continuous flow of original LA1949

\biggerrule

%%for now, rule to break page and start on new page
%%to be fixed when full book is finished keeping in mind how custom subfile commands work — which involves breaking page!


\end{document}