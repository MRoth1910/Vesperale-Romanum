% !TEX TS-program = lualatexmk
% !TEX parameter =  --shell-escape

\documentclass[vesperale_romanum.tex]{subfiles}

\ifcsname preamble@file\endcsname
  \setcounter{page}{\getpagerefnumber{M-vr_34_commune_virginum}}
\fi

%%this code when \customsubfiles is used should allow for continuous pagination when subfiles are compiled individually.

\begin{document}

\section[Commune Virginum]{COMMUNE VIRGINUM.}\header{commune virginum.}

%\thispagestyle{empty}

\invesperis{i} \label{1_vesperas_conf_virg}

\idxprima{Hæc est Virgo sapiens}{1}{f}{an_haec_est_virgo_sapiens_et_una_solesmes_1961}
\psalmus{109}{109_1f}{109_1}

\idxan{Hæc est Virgo sapiens quam}[2]{1}[f]{an_haec_est_virgo_sapiens_quam_solesmes_1961}
\psalmus{112}{112_1f}{112_1}

\idxan{Hæc est quæ nescivit}[3]{3}[a2]{an_haec_est_quae_nescivit_solesmes_1961}
\psalmus{121}{121_3a2}{121_3a2_g}

\idxan{Veni electa mea}[4]{1}[f]{an_veni_electa_mea_solesmes_1961}
\psalmus{126}{126_1f}{126_1}

\idxan{Ista est~01@Ista est (Comm. virg.)}[5]{3}[a]{an_ista_est_speciosa_virginum_solesmes_1961}
\psalmus{147}{147_3a}{147_3a_b}

\label{cap_communis_virginum}
\capitulum{2 Cor. 10, 17 – 18}

\lettrine{F}{r}atres: Qui gloriátur, in Dómino gloriétur.~† Non enim qui seípsum comméndat, ille probatus est:~* sed quem Deus
comméndat.

%%problem with J is perhaps descending annotation with old-style figures BUT latexmk runs lead to changed position even when the hymn is not changed otherwise…

\hymnus

\idxhy{Jesu corona virginum~01@Jesu corona virginum}{8}{hy_jesu_corona_virginum_solesmes_1961}

%% Solesmes omits this version in Liber Usualis; melody of Fortem virili pectore
\altertonus

\idxhy{Jesu corona virginum~02@– Alter Tonus}{2}{hy_jesu_corona_virginum_alter_tonus}

\smalltitle{Tempore Paschali.}

\idxhy{Jesu corona virginum~03@– \textit{T.P.}}{3}{hy_jesu_corona_virginum_tp_solesmes}

%%J initial causes problem even after multiple passes with latexmk (in fact we had multiple passes BEFORE I changed anything else in the hymn)

%%rubric from LU

\rubrique{Ab Ascensione ad Pentecosten, pro doxologia dicitur:}

\smallscore{dox_ascensionis_tristes_erant}

\altertonus

\idxhy{Jesu corona virginum~04@– \textit{T.P.} Alter Tonus}{4}{hy_jesu_corona_virginum_tp_alter_tonus_solesmes_1961}


\vel

\smallscore{dox_ascensionis_deus_tuorum_alter_tonus_mode_4}

\vv Spécie tua et pulchritúdine tua. \tpalleluia

\rr Inténde próspere procéde, et regna. \tpalleluia

\idxmag{Veni sponsa Christi~01@ Veni sponsa Christi \textit{(I Vesp.)}}{8}{G}{an_veni_sponsa_christi_i_vesperis_solesmes_1961}

\specialoratio{Pro Virgine Martyre.}

\lettrine{D}{e}us, qui inter cétera poténtiæ tuæ mirácula, étiam in sexu frágili vi\-ctóriam martýrii contulísti:~† concéde propítius; ut qui beatæ \textit{N.} Vírginis et Mártyris tuæ natalítia cólimus,~* per ejus ad te exémpla gradiámur. Per Dóminum.

\aliaoratio

\lettrine{I}{n}dulgéntiam nobis, quǽsumus Dómine, beata \textit{N.} Virgo et Martyr implóret:~† quæ tibi grata semper éxstitit et mérito castitátis,~* et tuæ professióne virtútis. Per Dóminum.

%\lettrine{E}{x}áudi, quǽsumus Dómine, preces nostras, quas in beáti \textit{N.} Confessóris tui atque Pontíficis sollemnitáte deférimus:~† et, qui tibi digne méruit famulári, ejus intercedéntibus méritis, ab ómnibus nos absólve peccátis. Per Dóminum.

\specialoratio{Pro Virgine non Martyre.}

\lettrine{E}{x}áudi nos Deus salutáris noster:~† ut sicut de beátæ \textit{N.} Vírginis tuæ festivitáte gaudémus;~* ita piæ devotiónis erudiámur affé\-ctu.
Per Dóminum.

\rubrique{¶ Si fuerint plures, in utrisque Vesperis ad Magnificat, Antiphona:}

%%need to decide if EUOUAE should be changed or not
%%changing for now (changed as of Oct 2024)
\idxscore{A}{Prudentes virgines}{4}[A*]
{an_prudentes_virgines_solesmes_1961}

\oratio

\lettrine{D}{a} nobis, quǽsumus Dómine Deus noster, san\-ctárum Vírginum et Mártyrum tuárum \textit{N.} et \textit{N.} palmas incessábili devotióne venerári:~† ut quas digna mente non póssumus celebráre,~* humílibus saltem frequentémus obséquiis. Per Dóminum.

\invesperis{ii}

\omniapraeter %%not in any edition, but it's the best rubric given that it makes references to Lauds and such

\vv Diffúsa est grátia in lábiis tuis. \tpalleluia

\rr Proptérea bendíxit te Deus in ætérnum. \tpalleluia

\idxmag{Veni sponsa Christi~02@ Veni sponsa Christi \textit{(II Vesp.)}}{7}{c}{an_veni_sponsa_christi_ii_vesperis_solesmes_1961}


\end{document}