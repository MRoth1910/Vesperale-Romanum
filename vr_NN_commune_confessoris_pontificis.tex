% !TEX TS-program = lualatexmk
% !TEX parameter =  --shell-escape

\documentclass[vesperale_romanum.tex]{subfiles}

\ifcsname preamble@file\endcsname
  \setcounter{page}{\getpagerefnumber{M-vr_NN_commune_confessoris_pontificis}}
\fi

%%this code when \customsubfiles is used should allow for continuous pagination when subfiles are compiled individually.

\begin{document}

%% the Iste Confessor seems to pose problems notably with S and p and then 3 at the beginning of lines, above the clef on the staff below

%% the Iste Confessor seems to pose problems notably with S and p and then 3 at the beginning of lines, above the clef on the staff below

\section[Commune Confessoris Pontificis.]{COMMUNE CONFESSORIS PONTIFICIS.}
\header{commune confessoris pontificis.}

%\thispagestyle{empty}

%\label{1_vesperas_conf_pont}

\invesperis{i}

\primaantiphona{7. c.}

\initialscore{an_ecce_sacerdos_magnus_solesmes_1961}
\psalmus{109}{109_7c}{109_7}
%
\gscore[2. Ant.]{7. c}{an_non_est_inventus_solesmes_1961}
\psalmus{110}{110_7c}{110_7}

\gscore[3. Ant.]{8. G*}{an_ideo_jurejurando_solesmes_1961}
\psalmus{111}{111_8Gstar}{111_8}

\gscore[4. Ant.]{7. c}{an_sacerdotes_dei_solesmes_1961}
\psalmus{112}{112_7c}{112_7}

\gscore[5. Ant.]{3. g}{an_serve_bone_et_fidelis_intra_conf_pontificis_solesmes_1961}
\psalmus{116}{116_3g}{116_3a2_g}

%\label{cap_com_conf_pont}
\capitulum{Eccl. 44, 16 – 17.}

\lettrine{E}{c}ce sacérdos magnus,~† qui in diébus suis plácuit Deo, et invéntus est justus:~* et in témpore iracúndiæ factus est reconciliátio.

\hymnusmode{8. (1)} \label{hymnus_com_conf_pont}

\initialscore{hy_iste_confessor_conf_pontificis_1_solesmes_1961}

\aliustonus[(2)]

\gscore[]{8.}{hy_iste_confessor_conf_pontificis_2_solesmes_1961}

\aliustonus[(3)] %%5. is too close to Sit. Need to watch.

\gscore[]{8.}{hy_iste_confessor_conf_pontificis_3_solesmes_1961}

\aliustonus[(4)]

\gscore[]{1.}{hy_iste_confessor_conf_pontificis_4_solesmes_1961}

\vv Amávit eum Dóminus, et ornávit eum.

\rr Stolam glóriæ índuit eum.

\label{an_sacerdos_et_pontifex_ora_solesmes_1960}\phantomsection
\admagnificat %%need to fix asterisk placement
\zerobaroffsettextleft
\gscore[]{1. D}{an_sacerdos_et_pontifex_ora_solesmes_1960}
\resetbaroffsettextleft

Pro Doctoribus, \rubrique{Ant. \normaltext{\pageref{o_doctor_com_conf_pont},} ut infra.}

\oratio

\lettrine{D}{a}, quǽsumus, omnípotens Deus:~† ut beáti \textit{N.} Confessóris tui atque Pontíficis veneránda sollémnitas, et devotiónem nobis áugeat et salútem. Per Dóminum.

\aliaoratio

\lettrine{E}{x}áudi, quǽsumus Dómine, preces nostras, quas in beáti \textit{N.} Confessóris tui atque Pontíficis sollemnitáte deférimus:~† et, qui tibi digne méruit famulári, ejus intercedéntibus méritis, ab ómnibus nos absólve peccátis. Per Dóminum.

\specialoratio{Pro Doctoribus.}

%%to modify later

\lettrine{D}{e}us, qui pópulo tuo ætérnæ salútis beátum \textit{N.} minístrum tribuísti:~† præsta, quǽsumus; ut quem Doctórem vitæ habúimus in terris,~* intercessórem habére mereámur in cælis. Per Dóminum.

%%Solesmes closes "PD nostrum" but this is atypical and not needed.

\invesperis{ii}

\rubrique{Antiphonæ et Psalmi ut in I Vesperis, sed loco ultimi Ps. \normaltext{Meménto,} ut infra.}

\psalmus{131}{131_3g}{131_3a2_g}

\rubrique{Capit. \normaltext{Ecce sacérdos} et Hymnus ut in I Vesperis.}

\vv Justum dedúxit Dóminus per vias rectas.

\rr Et osténdit illi regnum Dei.

\admagnificat
\label{an_amavit_eum_dominus_solesmes_1960}\phantomsection
\gscore[]{1. D2}{an_amavit_eum_dominus_solesmes_1960}

\rubrique{Sequens Antiphona dicitur ad Magnificat in II Vesperis pro solis summis Pontificibus.}
\label{an_dum_esset_summus_pontifex_solesmes_1960}\phantomsection
\gscore[]{1. f}{an_dum_esset_summus_pontifex_solesmes_1960}

\rubrique{¶ Si occurrat celebrari Festum plurium Confessorum Pontificum, Officium fit ut supra; sed in Oratione, quæ ponuntar in singulari, dicantur in plurali.}

\phantomsection
\label{o_doctor_com_conf_pont}

%%need to fix these doctors and get them on the same page

{\centering{Pro Doctoribus.\par}}

\rubrique{In Festis Doctorum, Officium fit de Confessore Pontifice vel non Pontifice, pro qualitate Sancti.}

\smalltitle{In utrisque Vesperis. Ad Magnificat, Antiphona.}
\label{an_o_doctor_optime_solesmes_1960}
\gscore[]{2. D}{an_o_doctor_optime_solesmes_1960}

\smalltitle{Nomina Doctorum Ecclesiæ.}

\greseteolcustos{manual}
\gresetinitiallines{0}
\gregorioscore{partitions/an_o_doctor_optime_Beda_solesmes_1960}
\begin{nstabbing}
\>\enspace{}Ephrem \>\>Ambró-\>\>si \>Cy-\>\enspace{}\hspace{0.25mm}ríl-\enspace{}le \>\>\enspace{}Hi-\>\thickspace{}lá-\>ri\\
\>\enspace{}Le-\enspace{}o \>\>An-sél-\>\>me \>Bá-\>\enspace{}\hspace{0.25mm}si-\kern0.75emli \>\>\enspace{}Jo-\>\thickspace{}án-\>nes\\
\>\enspace{}Pe-tre \>\>An-tó\>\>ni \>Lau\>\enspace{}\hspace{0.25mm}rén-ti \>\>Fran\>cí-\>sce\\
\>\enspace{}Thoma \>\>Ro-\>\medspace{}bér-te

\end{nstabbing}

\gregorioscore{partitions/an_o_doctor_optime_Athanasi_solesmes_1960}
\begin{nstabbing}
\>Augu\>\>stí-\>ne\\
\>I-\>\enspace{}si-\>dó-\>re
\end{nstabbing}

\gregorioscore{partitions/an_o_doctor_optime_Petre_solesmes_1960}
\gresetinitiallines{1}

\end{document}