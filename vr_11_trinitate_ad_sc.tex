% !TEX TS-program = lualatexmk
% !TEX parameter =  --shell-escape

\documentclass[vesperale_romanum.tex]{subfiles}

\ifcsname preamble@file\endcsname
  \setcounter{page}{\getpagerefnumber{M-vr_11_trinitate_ad_sc}}
\fi

%%this code when \customsubfiles is used should allow for continuous pagination when subfiles are compiled individually.

%%this file could be all the way to Advent including Saturdays and Sundasy after Pentecosts. The trouble is that such would be a massive, massive amount of code. Note in any case that LA1949 has Sat and Sundays with a hanging indent in the index.

%%in any case, it needs to be at least Trinity until the Octave of the Sacred Heart inclusive.

\begin{document}

%\thispagestyle{empty}

%\phantomsection
%
%\addcontentsline{toc}{section}{A Trinitate ad Adventum.}
%
\idxdominicadate{Trinitatis Sanctissimæ}{I Post Pentecosten}
%\dominicadate{I Post Pentecosten}
%
%\festum{In Festo Sanctissimæ Trinitatis.}

\section[A Trinitate ad Adventum]{IN FESTO SANCTISSIMÆ TRINITATIS.}
\header{in festo ss. trinitatis.}

\duplexclassis{I}

\invesperis{i} \label{1_vesperis_trinitatis}\phantomsection

\sedlocoultimi{II}{Laudáte Dóminum}

%% also used on Mar 19 at I Vespers and maybe elsewhere

\psalmus{116}{116_5a}{116_5}

\caphymn{II}

\vv Benedicámus Patrem et Fílium cum Sancto Spíritu.

\rr Laudémus et superexaltémus eum in sǽcula.

\idxmag{Gratias tibi Deus}{1}{D}{an_gratias_tibi_deus_solesmes_1961}

\rubrique{Oratio \normaltext{Omnípotens sempitérne Deus,} \normaltext{\pageref{or_trinitatis}.}}

\rubrique{Et fit Cornmemoratio Dominicæ I post Pentecosten, ut sequitur.}

\idxant{Loquere Domine}{1}{an_loquere_domine_solesmes_1961}

%%good example of where the verse doesn't align anywhere close to the vowel as is the case ordinarily

\textes{versiculus_sab_pa}

\label{or_dom_1_pp}\phantomsection

 \oratio

\lettrine{D}{e}us, in te sperántium fortitúdo, adésto propítius invocatiónibus nostris:~† et quia sine te nihil potest mortális infírmitas, præsta auxílium grátiæ tuæ;~* ut in exsequéndis mandátis tuis, et voluntáte tibi et actióne placeámus. Per Dóminum.

\invesperis{ii} \label{2_vesperis_trinitatis}\phantomsection

\idxprima{Gloria tibi Trinitas}{1}{f}{an_gloria_tibi_trinitas_solesmes_1961}
\label{109_1f}\phantomsection
\psalmus{109}{109_1f}{109_1}

\idxan{Laus et perennis gloria}[2]{2}[D]{an_laus_et_perennis_gloria_solesmes_1961}

 \label{110_2}\phantomsection

 \psalmus{110}{110_2}{110_2}
 
 %%v 10 has a hard dash

\idxan{Gloria laudis}[3]{3}[a2]{an_gloria_laudis_resonet_solesmes_1961}

\psalmus{111}{111_3a2}{111_3a2_g}

%\rubrique{Ps. \normaltext{Béatus vir, \pageref{M-111_3a2}.}}

%%points to Holy Name

\idxan{Laus Deo Patri}[4]{4}[E]{an_laus_deo_patri_solesmes_1961}

\psalmus{112}{112_4E}{112_4E}

%\rubrique{Ps. \normaltext{Laudáte Púeri, \pageref{M-112_4E}.}}

%%points to Epiphany

\idxan{Ex quo omnia}[5]{5}[a]{an_ex_quo_omnia_solesmes_1961}

\psalmus{113}{113_5a}{113_5}

\capitulum{Rom. 11, 33}

\textes{capitulum_sab_pa}

\hymnus

%%this needs to be watched as J is often far too low if accompanied by mode (and it is very low even if there is no annotation)

\idxhy{Jam sol recedit igneus~02@– (Ss. Trin.)}{8}{hy_jam_sol_recedit_igneus_blessed_trinity_solesmes_1961}

\vv Benedíctus es Dómine in firmaménto cæli.

\rr Et laudábilis et gloriósus in sǽcula.

\idxmag{Te Deum Patrem}{4}{E}{an_te_deum_patrem_solesmes}

\label{or_trinitatis} \phantomsection

\oratio

\lettrine{O}{m}nípotens sempitérne Deus, qui dedísti fámulis tuis in confessióne veræ fídei, ætérnæ Trinitátis glóriam agnóscere, et in poténtia majestátis adoráre Unitátem:~† quǽsumus; ut ejúsdem fídei firmitáte,~* ab ómnibus semper muniámur advérsis. Per Dóminum.

\rubrique{Pro commemoratione Dominicæ:}

\idxant{Nolite judicare}{8}{an_nolite_judicare_solesmes_1961}

\textes{versiculus_per_annum}

\rubrique{Oratio \normaltext{Deus, in te sperántium,} \normaltext{\pageref{or_dom_1_pp}.}}

%%this should be a fresh page

\newpage

\thispagestyle{empty}

\idxferiadate{Corporis Christi~01@Corporis Christi}{V infra Hebdom.\@ I post Octav.\@ Pentecostes}
%\feriadate{V infra Hebdom.\@ I post Octav.\@ Pentecostes}

\festum{In Festo Sanctissimi Corporis Christi}
\thispagestyle{empty}\header{in corporis christi et infra oct.}
%%probably better to have In Festo on one line like in LA 1949

\duplexclassis{I}[cum Octava privilegiata II ordinis]

\invesperis{i} \label{1_vesperis_cc}\phantomsection

\idxprima{Sacerdos in æternum}{1}{f}{an_sacerdos_in_aeternum_solesmes_1961}

%\psalmus{109}{109_1f}{109_1f}

\rubrique{Ps. \normaltext{Dixit Dóminus, \pageref{M-109_1f}.}}%%subfiles versiion

%%this might be a bad idea

\idxan{Miserator Dominus}[2]{2}[D]{an_miserator_dominus_solesmes_1961}

% \psalmus{110}{110_2}{110_2}
 
 %%v 10 has a hard dash
 
 \rubrique{Ps. \normaltext{Confitébor tibi, \pageref{M-110_2}.}}%%subfiles

\idxan{Calicem salutaris}[3]{3}[a2]{an_calicem_salutaris_accipiam_et_sacrificabo_solesmes_1961}
\label{115_3a2} \phantomsection
\psalmus{115}{115_3a2}{115_3a2_g}

\idxan{Sicut novellæ}[4]{4}[E]{an_sicut_novellae_solesmes_1961}
\label{127_4E}\phantomsection
\psalmus{127}{127_4E}{127_4E}

\idxan{Qui pacem}[5]{5}[a]{an_qui_pacem_ponit_solesmes_1961}
\label{147_5}\phantomsection
 \psalmus{147}{147_5}{147_5}

\capitulum{1 Cor. 11, 23 – 24}

\lettrine{F}{r}atres: Ego enim accépi a Dómino quod et trádidi vobis,~† quóniam Dóminus Jesus, in qua no\-cte tradebátur, accépit panem, et grátias agens fregit, et dixit: Accípite et manducáte: hoc est corpus meum, quod pro vobis tradétur:~* hoc fácite in meam commemoratiónem.

\hymnusmode{3.} \label{hy_pange_lingua_corpus_christi_solesmes}\phantomsection

\idxinscore{H}{Pangue lingua gloriosi}{3}{hy_pange_lingua_corpus_christi_solesmes}

\smallscore{versiculus_vesperis_cc}

\rubrique{(Sic cantatur in I et II Vesperis Festi tantum; alias in tono communi.)}
\label{an_o_quam_suavis_solesmes_1961}\phantomsection
\idxmag{O quam suavis}{6}{F}{an_o_quam_suavis_solesmes_1961}

\oratio

\lettrine{D}{e}us, qui nobis sub Sacraménto mirábili passiónis tuæ memóriam reliquísti:~† tríbue, quǽsumus, ita nos córporis, et sánguinis tui sacra mystéria venerári;~* ut redemptiónis tuæ fru\-ctum in nobis júgiter sentiámus. Qui vivis et regnas cum Deo Patre in unitáte.
 \label{or_corpus_christi}\phantomsection
\invesperis{ii}

\rubrique{Omnia sicut in I Vesperis,  \normaltext{\pageref{1_vesperis_cc}.}}

 \label{an_o_sacrum_convivium_solesmes_1961} \phantomsection

\idxmag{O sacrum convivium}{5}{a}{an_o_sacrum_convivium_solesmes_1961}

\rubrique{Sic dicitur Officium per totam Octavam.}

\bigtitle{Sabbato ad Vesperas.}

\rubrique{Antiphonæ et Psalmi ut in I Vesperis Festi, \normaltext{\pageref{1_vesperis_cc}.}}

\label{cap_dom_infra_oct_cc}\phantomsection
\capitulum{1 Joan. 3, 13 – 14}

\lettrine{C}{a}ríssimi: Nolíte mirári, si odit vos mundus.~† Nos scimus quóniam transláti sumus de morte ad vitam,~* quóniam dilígimus fratres.

\rubrique{Hymnus. \normaltext{Pangue lingua, \pageref{hy_pange_lingua_corpus_christi_solesmes}.}}

\vv Cibávit iílos ex ádipe fruménti, allelúia.

\rr Et de pétra, mélle saturávit éos, allelúia.

\idxadmagnif{Puer Samuel}{7}{a}{an_puer_samuel_solesmes_1961}

\rubrique{Oratio ut infra ad Vesperas Dominicæ, \normaltext{\pageref{or_dom_infra_oct_cc}.}}

\rubrique{Et fit commemoratio de Octava, Ant.\@ \normaltext{O sacrum convivium, \pageref{an_o_sacrum_convivium_solesmes_1961}, ℣. Panem de cælo.} Oratio \normaltext{Deus qui nobis, \pageref{or_corpus_christi}.}}

\index[F]{Corporis Christi~02@– Dominica infra oct.}
\bigtitle{Dominica infra octavam Corporis Christi.}

{\centering quæ est II post Pentecosten.\par}

\rubrique{Antiphonæ et Psalmi ut in I Vesperis Festi, \normaltext{\pageref{1_vesperis_cc}.}}

\rubrique{Capitulum \normaltext{Carissime: Nolíte.} ut supra, \pageref{cap_dom_infra_oct_cc}.}

\rubrique{Hymnus \normaltext{Pangue lingua, \pageref{hy_pange_lingua_corpus_christi_solesmes}.}}

\vv Cibávit iílos ex ádipe fruménti, allelúia.

\rr Et de pétra, mélle saturávit éos, allelúia.

\idxadmagnif{Exi cito}{1}{a}{an_exi_cito_in_plateas_solesmes_1961}

\oratio \label{or_dom_infra_oct_cc}\phantomsection

\lettrine{S}{a}ncti nóminis tui, Dómine, timórem páriter et amórem fac nos habére perpétuum:~† quia nunquam tua gubernatióne destítuis,~* quos in soliditáte tuæ dilectiónis instítuis.
Per Dóminum.

\rubrique{Deinde fit commemoratio de Octava cum Antiphona et ℣. ut in I Vesperis, \normaltext{\pageref{an_o_quam_suavis_solesmes_1961}.}}

\rubrique{Si tamen sequenti die faciendum non sit Officium de Octava, dicitur Ant.
\normaltext{O sacrum convívium,} ut in II Vesperis Festi, \normaltext{\pageref{an_o_sacrum_convivium_solesmes_1961}.}}

\index[F]{Corporis Christi~03@– Octava}
Feria Quarta. \rubrique{Ad Vesperas, omnia ut in I Vesperis Festi, \normaltext{\pageref{1_vesperis_cc}.}}

\rubrique{Si tamen sequenti die occurrat Festum Nativtatis S. Joannis Baptistae vel SS. Apostolorum Petri et Pauli in I. Vesperis pro commemoratione præcedentis diei infra Octavam sumitur Antiphona \normaltext{O sacrum convívium,} ut in II Vesperis Festi, \normaltext{\pageref{an_o_sacrum_convivium_solesmes_1961}.}}

%%this should also be a new page

\thispagestyle{empty}

\idxferiadate{Cordis Sacratissimi Jesu}{VI post Octavam Ss. Corporis Christi}
%\feriadate{VI post Octavam Ss. Corporis Christi}
% \section[Sanctissimi Cordis Jesu.]{SANCTISSIMI CORDIS JESU.}

\festum{Sacratissimi Cordis Jesu}
%\header{sanctissimi cordis jesu.}
 \duplexclassis{I}[cum Octava privilegiata III ordinis]
 \thispagestyle{empty}
 \header{sacratissimi cordis jesu et infra oct.}
%\subsection{in i vesperis.}

\invesperis{i}  \label{i_vesperis_sc}\phantomsection
\zerobaroffsettextleft
 \idxprima{Suavi jugo}{1}{g}{an_suavi_jugo_solesmes_1961}
 %% watch in case initial remains low after a couple runs
 \resetbaroffsettextleft
\psalmus{109}{109_1g}{109_1}
 
 \idxan{Misericors et misereator}[2]{2}[D]{an_misericors_et_miserator_solesmes_1961}
 
 %%this needs a hard hyphen in v 10 for now 
 
  \rubrique{Ps. \normaltext{Confitébor tibi, \pageref{M-110_2}.}}
 
% \psalmus{110}{110_2}{110_2}
 
  \idxan{Exortum est~02@ Exortum est \textit{(Ss. Cord.)}}[3]{7}[a]{an_exortum_est_sacred_heart_solesmes_1961}
  \psalmus{111}{111_7a}{111_7}
  
   \idxan{Quid retribuam}[4]{8}[c]{an_quid_retribuam_solesmes_1961}
  \psalmus{115}{115_8c}{115_8}
   
    \idxan{Apud Dominum propitiatio}[5]{4}[A*]{an_apud_dominum_propitiatio_solesmes_1961}
    \psalmus{129}{129_4A_A_star}{129_4A_A_star}
    
    \label{cap_sc}\phantomsection
\capitulum{Eph. 3, 8 – 9}

\lettrine{F}{r}ratres: Mihi ó\-mnium san\-ctórum mínimo data est grátia hæc,~† in géntibus evangelizáre investigábiles divítias Christi:~* et illumináre omnes, quæ sit dispensátio sacraménti abscónditi a sǽculis in Deo.

\hymnus \label{hy_en_ut_superba}\phantomsection
   \idxhy{En ut superba}{3}{hy_en_ut_superba_solesmes_1961}
   
   \rubrique{Sic terminantur omnes Hymni per totam Octavam.}
   
   \label{Tollite}\phantomsection
   \smallscore{versiculus_i_vesperis_ss_cordis}
   
    
 \idxmag{Ignem}{1}{D}{an_ignem_solesmes_1961}\label{Ignem}\phantomsection
 
\oratio \label{oratio_sc}\phantomsection

\lettrine{D}{e}us, qui nobis in Corde Fílii tui, nostris vulneráto peccátis, infinítos dilectiónis thesáuros misericórditer largíri dignáris:~† concéde quǽsumus; ut illi devótum pietátis nostræ præstántes obséquium,~* dignæ quoque satisfactiónis exhibeámus offícium. Per eúmdem Dóminum.
 
 \rubrique{Et nulla fit Commemoratio, nisi tantum de præcedente festo Nativitatis S. Joannis Baptistæ vel SS. Petri et Pauli App., si pridie occurrent.}
 
\rubrique{Completorium de Dominica.} %%maybe I should have a command for these frequently repeating things that are also their own paragraphs..

\hymnusadcompletorium
    \idxhy{Te lucis ante terminum~13@– in festo Ss. Cordis}{3}{hy_te_lucis_sacred_heart_solesmes}

 \invesperis{ii}
 \label{ii_vesperis_sc}\phantomsection

\idxan{Unus militum}[1]{1}[f]{an_unus_militum_solesmes_1961}

%\psalmus{109}{109_1f}{109_1f}

\rubrique{Ps. \normaltext{Dixit Dóminus, \pageref{M-109_1f}.}} %%for subfiles

\idxan{Stans Jesus}[2]{7}[c]{an_stans_jesus_solesmes_1961} %% LaTeX's tracking is HORRIBLE with this psalm. Lots of white for no reason. %%needs fixing
\psalmus{110}{110_7c}{110_7}

\idxan{In caritate}[3]{3}[a2]{an_in_caritate_solesmes_1961}

\rubrique{Ps. \normaltext{Crédidi, \pageref{M-115_3a2}.}} %%for subfiles

%\psalmus{115}{115_3a2}{115_3a2}

\idxan{Venite ad me}[4]{4}[E]{an_venite_ad_me_solesmes_1961}

\rubrique{Ps. \normaltext{Beáti omnes, \pageref{M-127_4E}.}} %%for subfiles

%\psalmus{127}{127_4E}{127_4E}

\idxan{Fili præbe}[5]{5}[a]{an_fili_praebe_solesmes_1961}

\rubrique{Ps. \normaltext{Beáti omnes, \pageref{M-147_5}.}} %%for subfiles

%\psalmus{147}{147_5}{147_5}

\rubrique{Capitulum et Hymnus ut in I Vesperis, \normaltext{\pageref{cap_sc}.}}
\label{Haurietis}\phantomsection
\smallscore{versiculus_ii_vesperis_ss_cordis}

\label{Ad Jesum}\phantomsection
\idxmag{Ad Jesum autem}{1}{f}{an_ad_jesum_autem_solesmes_1961}

\rubrique{Completorium de Dominica.}

 %%should c-t or -ct be used for non-chant texts? 

\bigtitle{Sabbato ad Vesperas.}

%\header{sabbato infra octavam ss. cordis jesu.}

\rubrique{Antiphonæ et Psalmi ut in I Vesperis Festi, \normaltext{\pageref{i_vesperis_sc}.}}
\label{cap_dom_infra_oct_sc}\phantomsection
\capitulum{1 Petri 5, 6 – 7}

\lettrine{C}{a}ríssimi: Humiliámini sub poténti manu Dei, ut vos exáltet in témpore visitatiónis;~† omnem sollicitúdinem vestram proiciéntes in eum,~* quóniam ipsi cura est de vobis.

\rubrique{Hymnus \normaltext{En ut superba, \pageref{hy_en_ut_superba}.}}

\textes{versiculus_dom_infra_oct_sc}

\idxadmagnif{Cognoverunt omnes}{1}{f}{an_cognoverunt_omnes_solesmes_1961}

\oratio \label{oratio_dom_sc}\phantomsection

\lettrine{P}{r}oté\-ctor in te sperántium Deus, sine quo nihil est válidum, nihil san\-ctum:~† multíplica super nos misericórdiam tuam; ut te re\-ctóre, te duce, sic transeámus per bona temporália,~* ut non amittámus ætérna. Per Dóminum.

\rubrique{Et fit Commemoratio Octavæ: Ant.\@ \normaltext{Ad Jesum, \pageref{Ad Jesum}, \vvrub Hauriétis Aqua \pageref{Haurietis},} et Oratio \normaltext{Deus, qui nobis in corde, \pageref{oratio_sc}.}}

\bigtitle{Dominica infra octavam SS. Cordis Jesu}
  
%  \header{dominica infra octavam ss. cordis jesu.}
  {\centering quæ est III post Pentecosten.\par}
 
\rubrique{Antiphonæ et Psalmi ut in II Vesperis Festi, \normaltext{\pageref{ii_vesperis_sc}.}}

\rubrique{Capitulum \normaltext{Caríssimi: Humiliámini} ut supra.} \rubrique{Hymnus \normaltext{En ut superba, \pageref{hy_en_ut_superba}.}}

\textes{versiculus_dom_infra_oct_sc}

      \idxadmagnif{Quae@Quæ mulier}{6}{F}{an_quae_mulier_solesmes_1961}
      
     \rubrique{Et fit Commemoratio sequentis diei infra Octavam: Ant.\@ \normaltext{Ignem, \pageref{Ignem}, \vvrub Tóllite \pageref{Tollite},} et Oratio \normaltext{Deus, qui nobis in corde, \pageref{oratio_sc}.}}
      
     \rubrique{Si autem sequenti die faciendum non sit Officium de Octava, dicitur Ant.\@ \normaltext{Ad Jesum, \pageref{Ad Jesum}, \vvrub Hauriétis Aqua, \pageref{Haurietis},} et Oratio \normaltext{Deus, qui nobis in corde, \pageref{oratio_sc}.}}
  
  \bigtitle{Octava SS. Cordis Jesu.}
  
 \duplexmajus
  
\rubrique{In I Vesperis, omnia ut in I Vesperis Festi.}

\rubrique{Ad Vesperas et ad Completorium, omnia ut in die Festi.}

\biggerrule

\end{document}