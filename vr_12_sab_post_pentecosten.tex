% !TEX TS-program = lualatexmk
% !TEX parameter =  --shell-escape

\documentclass[vesperale_romanum.tex]{subfiles}

\ifcsname preamble@file\endcsname
  \setcounter{page}{\getpagerefnumber{M-vr_12_sab_post_pentecosten}}
\fi

%%this code when \customsubfiles is used should allow for continuous pagination when subfiles are compiled individually.

\begin{document}

%\thispagestyle{empty}

%%possiibly could used starred version and then addcontentsline for a subsection but hyperref doesn't like this

\phantomsection
\section*{ANTIPHONÆ DICENDÆ IN SABBATIS}\header{sabbatis post pentecosten.}
\addcontentsline{toc}{subsection}{Sabbatis post Pentecosten}

\subtitle{post Pentecosten usque ad Adventum.}

%%change to starred version and the macro may introduce a small amount of extra white space.

%{\centering{\Large{post Pentecosten usque ad Adventum.} %%ideally this would be made semantic as a macro and therefore made reusable. The needspace package inserts white space that we don't want!

%%the second part is in all caps post Pent. etc. in LA1949

\bigtitle{Sabbato ante Dom. IV post Pentecosten.}

\gscore[Ad Magnif.]{8. G}{an_praevaluit_david_solesmes_1961}

\bigtitle{Sabbato ante Dom. V post Pentecosten.}

\gscore[Ad Magnif.]{1. D}{an_montes_gelboe_solesmes_1961}

%%Gelboë must retain its diacritic; this is in the LA1949.

\bigtitle{Sabbato ante Dom. VI post Pentecosten.}

\gscore[Ad Magnif.]{1. f}{an_obsecro_domine_aufer_solesmes_1961}

\bigtitle{Sabbato ante Dom. VII post Pentecosten.}

\gscore[Ad Magnif.]{8. G}{an_unxerunt_salomonem_solesmes_1961}

\rubrique{Nisi ponenda sit Antiphona pro Dominica I Augusti: quia tunc, omissis Antiphonis de Libris Regum, ponuntur de Libris Salomonis; quod etiam servatur in sequentibus Hebdomadis.}

\bigtitle{Sabbato ante Dom. VIII post Pentecosten.}

\gscore[Ad Magnif.]{7. a}{an_exaudisti_domine_solesmes_1961}

\bigtitle{Sabbato ante Dom. IX post Pentecosten.}

\gscore[Ad Magnif.]{8. G}{an_dum_tolleret_dominus_solesmes_1961}

%%LA 1949 has ël for Israël, but the Vatican edition does not, and it's hyphenated. It's not necessary, so it is omitted (Gregobase score omits it too)

\bigtitle{Sabbato ante Dom. X post Pentecosten.}

\gscore[Ad Magnif.]{6. F}{an_fecit_joas_solesmes_1961}

\bigtitle{Sabbato ante Dom. XI post Pentecosten.}

\gscore[Ad Magnif.]{5. a}{an_obsecro_domine_memento_solesmes_1961}

\bigtitle{Sabbato ante Dominicam I Augusti.}

\rubrique{Ea autem dicitur I Dominica mensis, qua, est in Kalendi, vel proximior Kalendis illius mensi: ita ut si Kalendæ fuerint II, III et IV. Feria, tunc I Dominica mensis, in qua Iiber Scripturæ inchoandus ponitur, est ea quæ præcedit Kalendas. Sin autem V et VI Feria, vel Sabbato, est ea quæ sequitur. Et in Sabbato præcedenti ad Magnificat ponatur Antiphona illius historiæ, omissa alia quæ forte occurreret.}

\gscore[Ad Magnif.]{7. a}{an_sapientia_aedificavit_excidit_solesmes_1961}

\rubrique{Oratio quæ contingit in ordine aliarum Dominicarum, ut infra.}

\bigtitle{Sabbato ante Dominicam II Augusti.}

\gscore[Ad Magnif.]{8. G}{an_ego_in_altissimis_solesmes_1961}

\bigtitle{Sabbato ante Dominicam III Augusti.}

\gscore[Ad Magnif.]{8. G}{an_omnis_sapientia_solesmes_1961}

\bigtitle{Sabbato ante Dominicam IV Augusti.}

\gscore[Ad Magnif.]{8. G}{an_sapientia_clamitat_solesmes_1961}

\bigtitle{Sabbato ante Dominicam V Augusti.}

\rubrique{Nisi hæc fuerit propinquior Kalendis Septembris: quia tunc ista cum sua Hebdomada omittitur; et ponitur Antiphona pro I Dominica Septembris, ut infra.}

\gscore[Ad Magnif.]{6. F}{an_observa_fili_solesmes_1961}

\bigtitle{Sabbato proximo Kalendis Septembris.}

\gscore[Ad Magnif.]{1. f}{an_cum_audisset_job_solesmes_1961}

\bigtitle{Sabbato ante Dominicam II Septembris.}

\gscore[Ad Magnif.]{1. g}{an_in_omnibus_his_solesmes_1961}

\bigtitle{Sabbato ante Dominicam III Septembris.}

\gscore[Ad Magnif.]{4. E}{an_ne_reminiscaris_mea_solesmes_1961}

\bigtitle{Feria IV Quatuor Temporum Septembris.}

%%** will be page reference to psalter
\rubrique{Ad Vesperas, si fit de Feria, dicuntur Preces feriales, \normaltext{\pageref{M-preces},} cum
Oratione Dominicæ præcedentis.}

\bigtitle{Feria VI Quatuor Temporum Septembris.}

%%** will be page reference to psalter
\rubrique{In Vesperis, nisi fiat de Festo IX Lectionum, dicuntur Preces feriales, \normaltext{\pageref{M-preces},}  et Oratio Dominicæ præcedentis.}

\bigtitle{Sabbato ante Dominicam IV Septembris.}

\gscore[Ad Magnif.]{3. g}{an_adonai_domine_deus_solesmes}

\bigtitle{Sabbato ante Dominicam V Septembris.}

\gscore[Ad Magnif.]{1. D2}{an_domine_rex_omipotens_solesmes_1961}

\bigtitle{Sabbato ante Dominicam I Octobris.}

\gscore[Ad Magnif.]{8. G}{an_adaperiat_dominus_cor_vestrum_solesmes_1961}

\bigtitle{Sabbato ante Dominicam II Octobris.}

\gscore[Ad Magnif.]{8. G}{an_refulsit_sol_solesmes_1961}

\bigtitle{Sabbato ante Dominicam III Octobris.}

\gscore[Ad Magnif.]{1. D}{an_lugebat_autem_judam_solesmes_1961}

\bigtitle{Sabbato ante Dominicam IV Octobris.}

\gscore[Ad Magnif.]{1. f}{an_exaudiat_dominus_solesmes_1961}

\bigtitle{Sabbato ante Dominicam V Octobris.}

\rubrique{Nisi sit dimittenda, quia hæc proximior sit Kalendis Novembris.}

\gscore[Ad Magnif.]{1. f}{an_tua_est_potentia_solesmes}

\bigtitle{Sabbato ante Dominicam I Novembris.}

\gscore[Ad Magnif.]{1. a}{an_vidi_dominum_sedentem_solesmes_1961}

\bigtitle{Sabbato ante Dominicam II Novembris.}

\gscore[Ad Magnif.]{1. g}{an_aspice_domine_solesmes_1961}

\bigtitle{Sabbato ante Dominicam III Novembris.}

\rubrique{Si November habeat tantum quatuor Dominicas, tunc Sabbato ante Dominicam II ad Magnificat dicetur Antiphona \normaltext{Muro tuo,} ut infra et duobus sequentibus Sabbatis dicentur antiphonæ positæ etiam infra in Sabbatis ante Dominicam IV et V.}

\gscore[Ad Magnif.]{1. g}{an_muro_tuo_solesmes_1961}

\bigtitle{Sabbato ante Dominicam IV Novembris.}

\gscore[Ad Magnif.]{1. D2}{an_qui_caelorum_solesmes_1961}

\bigtitle{Sabbato ante Dominicam V Novembris.}

\gscore[Ad Magnif.]{1. D2}{an_super_muros_tuos_solesmes_1961}

\biggerrule

\end{document}