% !TEX TS-program = LuaLaTeX+se

\documentclass[vesperale_romanum.tex]{subfiles}

\ifcsname preamble@file\endcsname
  \setcounter{page}{\getpagerefnumber{M-vr_02_kalendarium}}
\fi

%%this code when \customsubfiles is used should allow for continuous pagination when subfiles are compiled individually.

\begin{document}


{\centering{\large \capspace{KALENDARIUM PERPETUUM}.}\par} \thispagestyle{empty} \markboth{Calendrier}{Calendrier} \nopagebreak \par \nopagebreak\vspace{5mm}\label{kalendarium}








\vspace{-5ex}
\rubrique{In anno bisextili mensis Febrarius est dierum 29, et Festum S. Matthiæ celebratur die 25 Februarii, ac Festum S. Gabrielis a Virg. perdolente die 28 Februarii, et bis dicitur Sexto Kalendas, id est die 24 et die 25; et littera Dominicalis, quæ assumpta fuit in mense Januario, mutatur in præcedentem ut si in Januario littera Dominicalis fuerit A, mutetur in præcedentum, quæ est \normaltext{g,} etc., littera  \normaltext{f} bis servit 24 et 25.}

\newpage







\thispagestyle{empty}
\smalltitle{Oratio ante Divinum Officium.}
%\vspace{0.5ex}

\lettrine{A}{p}eri, Dómine, os meum ad benedicéndum nomen sanctum tuum: munda quoque cor meum ab ómnibus vanis, pervérsis et aliénis cogitatiónibus; intelléctum illúmina, affé\-ctum inflámma, ut digne, atténte ac devóte hoc Offícium recitáre váleam, et exaudíri mérear ante conspéctum divínæ Majestátis tuæ. Per Christum Dóminum nostrum. \rr Amen.

Dómine, in unióne illíus divínæ intentiónis, qua ipse in terris laudes Deo persolvísti, has tibi Horas [\textit{vel} hanc tibi Horam] persólvo.

\smalltitle{Oratio post Divinum Officium}
%\vspace{0.5ex}

\lettrine{S}{a}crosánctæ et indivíduæ Trinitáti, crucifíxi Dómini nostri Jesu Christi humanitáti, beatíssimæ et gloriosíssimæ sempérque Vírginis Maríæ fœcúndæ integritáti, et ómnium Sanctórum universitáti sit sempitérna laus, honor, virtus et glória ab omni creatúra, nobísque remíssio ómnium peccatórum, per infiníta sǽcula sæculórum. \rr Amen.

\vv Beáta víscera Maríæ Vírginis, quæ portavérunt ætérni Pátris Fílium.

\rr Et beáta úbera, quæ lactavérunt Christum Dóminum.

Pater noster et Ave María.

\myrule

\rubrique{Ante Matutinum et omnes Horas, præterquam ad Completorium, dicitur secreto \normaltext{Pater noster et Ave María.} In principio Matutini ac Primæ, et in fine Completorium, dicitur etiam \normaltext{Credo in Deum.}}

\lettrine{P}{a}ter noster, qui es in cælis. Sanctificétur nomen tuum. Advéniat regnum tuum. Fiat volúntas tua, sicut in cælo et in terra. Panem nostrum quotidiánum da nobis hódie. Et dimítte nobis débita nostra, sicut et nos dimíttimus debitóribus nostris. Et ne nos indúcas in tentatiónem: Sed líbera nos a malo. 

\lettrine{A}{ve} María, grátia plena; Dóminus tecum:  benedícta tu in muliéribus, et benedíctus fructus ventris tui, Jesus. Sancta María, Mater Dei, ora pro nobis peccatóribus nunc et in hora mortis nostræ. Amen.

\lettrine{C}{r}edo in Deum, Patrem omnipoténtem, Creatórem cæli et terræ. Et in Jesum Christum, Fílium ejus únicum, Dóminum nostrum: qui concé\-ptus est de Spíritu San\-cto, natus ex María Vírgine : passus sub Póntio Piláto, crucifíxus, mórtuus, et sepúltus: descéndit ad ínferos: tértia die resurréxit a mórtuis : ascéndit ad cælos, sedet ad déxteram Dei Patris omnipot\-éntis : inde ventúrus est judicáre vivos et mórtuos. Credo in Spíritum San\-ctum, sanctam Ecclésiam cathólicam, san\-ctórum communiónem, remissiónem peccatórum, camis resurrectiónem, vitam ætérnam. Amen.

\end{document}