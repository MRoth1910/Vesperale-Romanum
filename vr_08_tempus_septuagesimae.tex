% !TEX TS-program = lualatexmk
% !TEX parameter =  --shell-escape

\documentclass[vesperale_romanum.tex]{subfiles}

\ifcsname preamble@file\endcsname
  \setcounter{page}{\getpagerefnumber{M-vr_09_tempus_septuagesimae}}
\fi

%%this code when \customsubfiles is used should allow for continuous pagination when subfiles are compiled individually.

\begin{document}
%%tbd if this is its own file or not
%\thispagestyle{empty} %%headers to fix later

\phantomsection
\addcontentsline{toc}{section}{A Sabbato ante Septuagesimam ad Sab. ante Dom. I Quadragesimæ.}

%\header{}

%\addcontentsline{toc}{section}{A Septuagesima ad Dom. I Quadragesimæ.}

\bigtitle{Sabbato ante Septuagesimam.}

%%** to be replaced with \pageref to Sat in the psalter

\rubrique{Psalmi et Antiphona de Sabbato ut in Psalterio, \normaltext{\pageref{M-sabbato_ad_vesperas}.}}

\capitulum{1 Cor. 9, 24}

\lettrine{F}{r}atres: Nescítis quod ii qui in stádio currunt,~† omnes quidem currunt, sed unus áccipit bravíum? Sic cúrrite ut comprehendátis.

%%**hymn of Sat in psalter
\rubrique{Hymnus \normaltext{Jam sol recédit,} \normaltext{\pageref{M-hy_sab_pa}.}}

\textes{versiculus_sab_pa}

\idxadmagnif{Dixit Dominus ad Adam}{8}{G}{an_dixit_dominus_ad_adam_solesmes_1961}

\oratio

\lettrine{P}{r}eces pópuli tui, quǽsumus, Dómine, cleménter exáudi:~† ut qui juste pro peccátis nostris afflígimur,~* pro tui nóminis glória misericórditer liberémur. Per Dóminum.

\smallscore{Benedicamus_domino_alleluia}

\rubrique{Et deinceps non dicitur \normaltext{Allelúia} usque ad Sabbatum Sanctum. Sed post \normaltext{Deus in adjutórium,} ubi dicebatur \normaltext{Allelúia,} dicitur \normaltext{Laus tibi Dómine Rex ætérnæ glóriæ.}}

\rubrique{Si in Dominicis Septuagesimæ, Sexagesimæ et Quinquagesimæ occurrat Festum Duplex I classis, fit de eo cum Commem. Dominicæ. Duplex vero II classis transfertur juxta Rubricas.}

\bigtitle{Dominica in Septuagesima.}
\cheader{dominica in septuagesima.}

\rubrique{Antiphonæ et Psalmi de Dominica.}

\rubrique{Capitulum \normaltext{Fratres: Nescítis} ut supra.}

%%**hymn of Sun in psalter

\rubrique{Hymnus Lucis Creátor, \normaltext{\pageref{M-hy_dom_pa}.}}

\textes{versiculus_pa}

%%% the Vaticana reprints it for each gesima Sunday, Solesmes just gives the symbol + first word + p number for Sunday

\idxadmagnif{Dixit paterfamilias}{7}{a}{an_dixit_paterfamilias_solesmes}

\rubrique{Oratio \normaltext{Preces pópuli tui} ut supra.} %%Vaticana always repeats full chapter and collect, but this is clearly wasteful.

\bigtitle{Feria II.}

\idxadmagnif{Hi novissimæ}{1}{a3}{an_hi_novissimi_solesmes}

\bigtitle{Feria III.}
\cheader{infra hebdom. septuagesimæ.}

\idxadmagnif{Dixit autem paterfamilias}{8}{G}{an_dixit_autem_paterfamilias_solesmes}

\bigtitle{Feria IV.}

\idxadmagnif{Tolle}{8}{c}{an_tolle_quod_tuum_est_solesmes}

\bigtitle{Feria V.}

\idxadmagnif{Non licet mihi}{8}{G}{an_non_licet_mihi_solesmes}


\rubrique{Feria VI sequenti, si Vesperæ sint de Feria, ad Magnificat dicitur ultima Antiphona ex præcedentibus Feriis prætermissa; alioquin, si omnes recitatæ fuerint, sumitur de Psa!terio.
}

\bigtitle{Sabbato ante Sexagesimam.}

%%** to be replaced with \pageref to Sat in the psalter

\rubrique{Psalmi et Antiphona de Sabbato ut in Psalterio, \normaltext{\pageref{M-sabbato_ad_vesperas}.}}

\capitulum{2 Cor. 11, 19 – 20}

\lettrine{F}{r}atres: Libénter suffértis insipiéntes, cum sitis ipsi sapiéntes:~† sustinétis enim si quis vos in servitútem rédigit, si quis dévorat, si quis áccipit, si quis extóllitur,~* si quis in fáciem vos cædit.

%%**hymn of Sat in psalter

\rubrique{Hymnus \normaltext{Jam sol recédit,} \normaltext{\pageref{M-hy_sab_pa}.}}

\textes{versiculus_sab_pa}

\idxadmagnif{Dixit Dominus ad Noe}{8}{G}{an_dixit_dominus_ad_noe_solesmes_1961}

\oratio

\lettrine{D}{e}us, qui cónspicis, quia ex nulla nostra actióne confídimus:~† concéde propítius; ut contra advérsa ómnia,~* Doctóris géntium prote\-ctióne muniámur. Per Dóminum.

%%Vaticana does allow dagger to be on next line, Solesmes rarely does this.

\bigtitle{Dominica in Sexagesima.}
\cheader{dominica in sexagesima.}

\rubrique{Antiphonæ et Psalmi de Dominica.}

\rubrique{Capitulum \normaltext{Fratres: Libénter suffértis} ut supra.}

%%**hymn of Sun in psalter

\rubrique{Hymnus Lucis Creátor, \normaltext{\pageref{M-hy_dom_pa}.}}

\textes{versiculus_pa}

\idxadmagnif{Vobis datum est}{6}{F}{an_vobis_datum_est_solesmes_1961}

\rubrique{Oratio \normaltext{Deus, qui conspícis} ut supra.} %%Solesmes doesn't even give this for Sexagesima.

\bigtitle{Feria II.}
\cheader{infra hebdom. sexagesimæ.}

\idxadmagnif{Si culmen}{7}{c}{an_si_culmen_solesmes}

\bigtitle{Feria III.}

\idxadmagnif{Semen est}{3}{a}{an_semen_est_solesmes}

\bigtitle{Feria IV.}

\idxadmagnif{Quod autem}{1}{g}{an_quod_autem_solesmes}

\rubrique{Si prædictæ Antiphonæ ad Magnificat non potuerunt dici in præcedentibus Feriis et in sequenti Feria V et VI Vesperæ sint de Feria, dicitur ultima ex iis præætermissa; alioquin, si omnes recitatæ fuerint, sumitur de Psalterio.}

\bigtitle{Sabbato ante Quinquagesimam.}

%%** to be replaced with \pageref to Sat in the psalter

\rubrique{Psalmi et Antiphona de Sabbato ut in Psalterio, \normaltext{\pageref{M-sabbato_ad_vesperas}.}}

\capitulum{1 Cor. 13, 1}

\lettrine{F}{r}atres: Si linguis hóminum loquar, et Angelórum,~† caritátem autem non hábeam,~* factus sum velut æs sonans, aut cýmbalum tínniens.

%%**hymn of Sat in psalter

\rubrique{Hymnus \normaltext{Jam sol recédit,} \normaltext{\pageref{M-hy_sab_pa}.}}

\textes{versiculus_sab_pa}

\idxadmagnif{Pater fidei}{6}{F}{an_pater_fidei_nostrae_solesmes_1961}

\oratio

\lettrine{P}{r}eces nostras, quǽsumus Dómine, cleménter exáudi:~† atque a peccatórum vínculis absolútos,~* ab omni nos adversitáte custódi. Per Dóminum.

\bigtitle{Dominica in Quinquagesima.}
\cheader{dominica in quinquagesima.}

\rubrique{Antiphonæ et Psalmi de Dominica.}

\rubrique{Capitulum \normaltext{Fratres: Si linguis} ut supra.}

\rubrique{Hymnus Lucis Creátor, \normaltext{\pageref{M-hy_dom_pa}.}}

\textes{versiculus_pa}

\idxadmagnif{Stans autem Jesus}{1}{D}{an_stans_autem_jesus_solesmes_1961}

\rubrique{Oratio \normaltext{Preces nostras} ut supra.} %%Solesmes only gives this for following Sunday.

\bigtitle{Feria II.}
%\cheader{infra hebdom. quinquagesimæ.}

\idxadmagnif{Et qui præibant}{7}{a}{an_et_qui_praeibant_solesmes_1961}

\bigtitle{Feria III.}

\idxadmagnif{Miserere mei, Fili}{8}{G}{an_miserere_mei_fili_solesmes_1961}

\bigtitle{Feria IV Cinerum.}
\cheader{feria iv cinerum.}

\rubrique{Si hac die et in Dominica I Quadragesima occurrat Festum I vel II classis, transfertur juxta Rubricas. De quocumque alio Festo fit tantum Cornmemoratio in utrisque Vesperis.}

\rubrique{Dominicis II, III, et IV Quadragesimæ, si occurrat Festum I classis, fit de eo cum Commem. Dominicæ. Si occurrat Festum II classis, transfertur.}

\rubrique{Aliis diebus Quadragesimæ usque ad Majorem Hebdomadam si occurrat Festum Duplex vel Semiduplex, fit de eo cum Commm. Feria. De Festo Simplici fit tantum Commmoratio.}

\rubrique{In hac et sequentibus Feriis usque ad Nonam Sabbati sequentis inclusive, omnia dicuntur ut per annum, exceptis Antiphonis ad Magnifcat; et post Vesperas et Completorium, dicuntur Preces flexis genibus, ut in Ordinario.}

\rubrique{Hodie et duobus sequentibus diebus, dicuntur Vesperæ hora consueta. Sabbato vero, et deinceps usque ad Pascha, dicuntur ante comestionem, tam in Festis quam in Feriis, exceptis Dominicis, in quibus dicuntur hora consueta.}

\textes{versiculus_pa}

\idxadmagnif{Thesaurizate vobis}{4}{A}{an_thesaurizate_vobis_solesmes}

\oratio

\lettrine{I}{n}clinántes se, Dómine, majestáti tuæ, propitiátus inténde:~* ut qui divíno múnere sunt refécti, cæléstibus semper nutriántur auxíliis. Per Dóminum.

\bigtitle{Feria V.}
\cheader{feriæ v et vi post cineres.}

\idxadmagnif{Domine, non sum dignus}{1}{g2}{an_domine_non_sum_dignus_solesmes}

\oratio

\lettrine{P}{a}rce, Dómine, parce populo tuo:~* ut dignis flagellatiónibus castigátus, in tua miseratióne respíret. Per Dóminum.

\bigtitle{Feria VI.}

\idxadmagnif{Tu autem cum oraveris}{1}{f}{an_tu_autem_cum_oraveris_solesmes}

\oratio

\lettrine{T}{u}ére Dómine pópulum tuum et ab ómnibus peccátis cleménter emúnda:~* quia nulla ei nocébit advérsitas, si nulla ei dominétur iníquitas.
Per Dóminum.

\biggerrule

\end{document}