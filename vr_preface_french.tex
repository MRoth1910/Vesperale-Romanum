\bigtitle{Préface du \textit{Vesperale Romanum.}}

\begin{frpars}
Le \textit{Vesperale Romanum} est le livre contenant les chants et les textes des offices du soir, c’est-à-dire des vêpres et des complies. Après la publication de l'antiphonaire pour les heures diurnes, excluant ainsi les matines, celle-ci qu'on appelle l'édition Vaticane,, dans l'année 1911, un vespéral le suivit l'année prochaine. Pourtant, celui-ci présente des difficultés ; on emploie le psautier reformé, mais les textes, les rubriques et le calendrier sont ceux du bréviaire de saint Pie textsc{v}, avec les modifications introduites jusqu'en 1910. L'antiphonaire et le vespéral ne conforment pas alors à la \textit{editio typica} de saint Pie \textsc{x}.

Les moines de l'abbaye Saint-Pierre de Solesmes, dont les recherches sur les manuscrits médiévaux inspira dans un premier temps la commission Vaticane, publièrent leur propre édition avec les signes rythmiques préférés par dom André Mocquereau, maître de chœur de l'abbaye pendant des décennies, mais cette édition était moins populaires que les autres éditions solesmiennes, notamment le \textit{Liber Usualis} (le \textit{Paroissien romain} ou encore le \textit{800}) qui inclut les chants des dimanches, des fêtes et de la Semaine sainte mais sans ceux des autres jours en semaines.

\smalltitle{Pourquoi un nouveau \textit{vespéral}?}

Avant tout, cette édition est conçu pour remplacer les copies imprimées déjà en circulation, vu que on ne publie plus ni l'antiphonaire, ni le vespéral, que ce soit l'édition Vaticane ou celle de Solesmes. En plus, nous profitons du logiciel Gregorio, qui permet la réalistion d'une partion grégorienne sur ordinateur bien supérieur à l'impression dépendant d'un scan, car même si une copie numérique de haute qualité était à notre disposition, la version imprimée se dégrade au fur et à mesure, comme on trouve déjà dans les réimpressions des éditions solesmiennes, dans lesquelles on peine à identifier des neumes tels que le \textit{quilisma} ou des signes rythmiques, notamment le \textit{ictus} ou la \textit{épisème verticale.}

Toutefois, ce livre n'est pas simplement une réproduction à l'identique. Nous avons pour but d'améliorer une édition par rapport aux livres classiques afin de faciliter son emploi le plus possible, mais une certaine familiarité avec l'office divin rest indispensable, selon la nature d'un livre.

\smalltitle{Principes d'édition.}

Au cœur, le projet Vesperale Romanum suit l'office avant les changements de Pie \scspace{xii}, en se permettant tout de même une certain discretion. Par exemple, les noms de saint Antoine et de saint Laurent de Brindes se trouvent parmi les Docteurs de l'Église. On est ainsi libre de choisir l'antiphon pour un Docteur ou celle du commun respectif, selon les rubriques.

Dans l'appendice, nous inclusns le commun des Souverains Pontiffes, les nouveaux offices de saint Joseph, ouvrier (1\up{er} mai), de la Très Sainte Vierge Marie, Reine (31 mai); de l'Assomption (15 août), et du Cœur Immaculée de la Très Sainte Vierge Marie (22 août), introduits par Pie \scspace{xii} avant 1954 (obligatoire à partir de l'année 1955), ainsi que les vêpres du Samedi saint obligatoire dès l'année 1956 et finalement deux nouvelles fêtes inscrites au calendrier par saint Jean \scspace{xxiii}, dans le \textit{Codex Rubricarum} de 1960.

Pour l'essentiel, il est tout à fait possible d'employer l'office \textit{Divino Afflatu} pour celui de 1960, une situation qui se présente déjà pour la plupart des personnes utilisant un \textit{Liber Usualis} antérieur aux réformes de 1960.

Cette édition est surtout pratique. Nous souhaitons la faire publier dès que possible. Nous n'avons pas l'intention de restituer les mélodies ce qui évite aussi la gêne chez les fidèles. Ainsi, bien que nous pensions à l'inclusion des \textit{hymni antiqui,} en raison de la demande immense et l'intérêt déjà suscité chez les fidèles, en suivant l'édition Vaticane, nous voulons garder les ton psalmodique associé avec le troisième mode sur la teneur de \textit{Do,} le ton plus récent du ton 6, celui du deuxième mode qui est plus familier, selon tant l'Édition Vaticane que celle de Solesmes. Néanmoins, il nous faut dévier car les coutumes évolèrent en ce qui concerne les terminus selon lesquels A* indiquent un terminus alternatif d'un podatus  (4A* and 8G*). Nous donnons le terminus à la fin du premier verset. En effet, dom Suñol avait écrit qu'on n'était à chanter le podatus qu'\kern 0.01 emà la doxologie.

Puisque nous fournissons le terminus du psaume, nous supprimons \frquote{EUOUAE} pour la plupart des antiennes. Mais nous gardons \frquote{EUOUAE} pour l'antienne du Magnificat, car il s'agit d'une édition, pratique dans laquelle nous insérons le cantique de la Très Sainte Vierge Marie dans une section seulement. Nous gardons également \frquote{EUOUAE} quand il faut tourner la page pour retrouver l'antienne après le chant du psaume.

Nous conservons aussi les signes rythmiques. Les partitions solesmniennes étant déjà disponibles, nous avons besoin de peu d'intervention rédactionelle pour rendre les partitions utilisables afin de permettre une composition correcte. En outre, avec le nombre de chanteurs qui emploient encore même \textit{ictus,} il vaut mieux s'abstenir et reproduire les signes, dans l'idéal en utilisant ceux de \textit{Liber antiphonarius,} dans lequel la plupart, mais pas tout, des cadences redondantes comme on le trouve dans les antiennes du mode VIII, sur les syllabes pénultièmes ou ultimes, avant la demie-barre ou surtout la barre finale, où les \textit{puncta} sont munis d'un point tous les deux. Dans ce cas, dom Gajard donna aussi le point à la syllabe spondaïque ou paryoxtyonique, avec l'accent sur l'avant-dernière syllabe. En revanche, dom Mocquereau l'avait fait seulement à l'égard de la dernière dans ses éditions publiées, malgré en avoir parlé dans \textit{Le nombre musicale grégorien,} t. 2.

Voulant rendre facile le chant et la direction d'un chœur par les maîtres de chœur, on retrouvera le texte du psaume préparé pour le chant selon les divers tons (le texte en gras pour les accents, ou sinon pour les syllabes qui y sont assimilées, le texte en italique pour les syllabes de préparation, en suivant \textit{le Liber Usualis}). Le premier verset de chaque psaume est notée en-dessous de chaque antienne. ou, pour ce qui est du commun et propre du saints et du propre du temps aux fêtes majeures, au moins une fois par division du livre, pour réduire les répétitions inutiles tout en évitant le nombre de rubans nécessaires.

En ce qui concerne le psautier, on retrouvera les antiennes comme dans le \textit{le Liber Usualis} aux vêpres du dimanche et aux complies ainsi qu'aux vêpres du samedi. Sinon, on fait de même en semaine, avec l'antienne complète avant le début du psaume, pour chanter aux fêtes doubles qui emploient les antiennes et les psaumes du jour. Ce permet aussi de pouvoir chanter l'office de 1960, dite la \frquote{forme extraordinaire}. Cela dit, il faut savoir qu'il existe une rubrique selon laquelle on chante l'intonation du psaume mais seulement si on saute le premier verset car on le chante intégralement dans l'antienne. Aux dimanches et aux fêtes avec des antiennes et des psaumes propres, les psaumes et les antiennes sont comme aux matines et aux offices du Triduum dans le \textit{le Liber Usualis}, avec l'antienne, puis le psaume, sans répéter la composition de l'antienne.

Pour retrouver les psaumes aux fêtes dans le \textit{Liber Usualis,} il faut feuiller des pages délicates. En conséquence des décisions prises il y a plus d'un siècle, le livre n'est pas très équilibré, surtout aux vêpres dominicales, se trouvant au premier tiers du livre. Nous souhaitons éviter cette gêne, en déplaçant le psautier au milieu du livre, ce qu'on trouve dans les éditions post-conciliaire. Ce choix devrait aider à mieux présérver la reliure.

Les tons des hymns sont repris de l'édition Vaticane, les mêmes trouvées dans celles de Solesmes; on trouve les élisions du  \textit{Liber antiphonarius,} c’est-à-dire que la note d'une syllabe hypermétrique est laissée dans la partition, sauf pour des hymnes plus récents édités dans un premier temps par les moines de Solesmes, pour que les chanteurs qui chantent ces syllabes puissent le faire et pour permettre également aux autres de ne pas le faire selon la coutume. 

Comme nous l'avons expliqué plus haut, les Magnificats sont regroupé dans une seule division, selon les différents tons, comme dans le \textit{Liber Usualis;} l'antiphonaire donne la première partie, surtout pour le premier versets, puis pour les tons solennels, mais il faudrait mémoriser la deuxième partie, en appliquant les terminus ordinaire, en dépit de la difficulté que posent les dactyles surtout en mode IV.

Les tons communs se trouvent dans une partie spéciale, comme dans l'antiphonaire, et non pas dans le psautier comme dans  le \textit{Liber Usualis;} vers le fond du livre, on trouvera les chants du Salut du Très Saint Sacrement ainsi que le propre de France.

Les versets sont numérotés à partir de \frquote{1} selon la forme du \textit{Liber Usualis} afin d'aider les maîtres de chœur qui travaillent avec des chanteurs ou des organistes moins expérimentés, dont le niveau en latin n'est pas toujours suffisant. La numérotation est très utile aux répétitions et aux offices, pour diriger les chanteurs et pour suivre l'alternance entre les deux chœurs ou autres divisions des chanteurs. Il s'agit d'une considération pratique primaire. Force est de constater que les divisions des psaumes ne correspondent pas aux divisions bibliques ordinaires non plus.
\end{frpars}