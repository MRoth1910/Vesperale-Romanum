\smalltitle{Canticum B. Mariae Virginis.}

\scripture{Luc. 1, 46 – 55.}

\lettrine{M}agníficat * ánima mea Dóminum.\\ Et exsultávit spíritus meus * in Deo salutári meo.

 Quia respéxit humilitátem ancíllæ suæ: * ecce enim ex hoc beátam me dicent o\-mnes generatiónes.

 Quia fecit mihi magna qui potens est: * et san\-ctum nomen ejus.

 Et misericórdia ejus a progénie in progénies * timéntibus eum.

 Fecit poténtiam in bráchio suo: * dispérsit supérbos mente cordis sui.

 Depósuit poténtes de sede, * et exaltávit húmiles.

 Esuriéntes implévit bonis: * et dívites dimísit inánes.

 Suscépit Israel púerum suum, * recordátus misericórdiæ suæ.

 Sicut locútus est ad patres no\-stros, * Abraham, et sémini ejus in sǽcula.

 Glória Patri, et Fílio, * et Spirítui San\-cto.

 Sicut erat in princípio, et nunc, et semper, * et in sǽcula sæculórum. Amen.
 
\rubrique{Post Antiphonam dicitur Oratio propria, deinde commemorationes, si quæfaciendæ occurrunt; postremo, si dicendum est, Suffragium de Omnibus Sanctis, vel de Cruce, ut infra.}