% !TEX TS-program = lualatexmk
% !TEX parameter =  --shell-escape

\documentclass[vesperale_romanum.tex]{subfiles}

\ifcsname preamble@file\endcsname
  \setcounter{page}{\getpagerefnumber{M-vr_NN_commune_apostolo_evang_martyrum_tp}}
\fi

%%this code when \customsubfiles is used should allow for continuous pagination when subfiles are compiled individually.

\begin{document}

\section[Commune Apostolorum et Evangelistarum, Unius et Plurim. Martyrum Temp. Paschali]{COMMUNE APOSTOLORUM ET EVANGELISTARUM, UNIUS ET PLURIMORUM MARTYRUM}

{\centering{\Large{Tempus Paschali.}\par}} %%ideally this would be made semantic as a macro and therefore made reusable. The needspace package inserts white space that we don't want!

%%sentence case is what Vaticana has, but Solesmes has small caps
%% NO period at end of title in this case

%%quesstion of EUOUAE here…

\invesperis{i}

\primaantiphona{8. G}

\initialscore{an_sancti_tui_domine_solesmes_1961}
\psalmus{109}{109_8G}{109_8}

\gscore[2. Ant.]{7. a}{an_in_caelestibus_regnis_alleluia_solesmes_1961}
\psalmus{110}{110_7a}{110_7}

\gscore[3. Ant.]{2. D}{an_in_velamento_clamabant_solesmes_1961}
\psalmus{111}{111_2}{111_2}

\gscore[4. Ant.]{8. G*}{an_spiritus_et_animae_solesmes_1961}
\psalmus{112}{112_8Gstar}{112_8}

\gscore[5. Ant.]{2. D}{an_fulgebunt_justi_sicut_sol_solesmes_1961}
\psalmus{116}{116_2}{116_2}
%
\capitulum{Sap. 5, 1} %%Wisdom

\lettrine{S}{t}abunt justi in magna constántia~† advérsus eos qui se angustiavérunt,~* et qui abstulérunt labóres eórum.
%
\hymnusspecial{Pro Apostolis et Evangelistis}[3.]
\initialscore{hy_tristes_erant_apostoli_solesmes_1961.gabc}

\rubrique{Sic terminantur omnes Hymni ejusdem metri (nisi aliter notetur), usque
ad Ascensionem.}

%%can use pageref as in both print editions to refer to alter tonus of Deus tuorum.

\rubrique{In tono supra posito vel in alio infra descripto 2 loco pro
Uno Martyre cantantur omnes Hymni ejusdem metri in Officio
SS. Martyrum, Confessorum, Virginum et non Virginum, ad Laudes
et Vesperas, usque ad Pentecosten, nisi aliter notetur.}

\rubrique{Ab Ascensione ad Pentecosten, pro doxologia dicitur:}

\smallscore{dox_ascensionis_tristes_erant}

\hymnusspecial{Pro uno Martyre}[]

\gscore[]{3.}{hy_deus_tuorum_militum_tp_solesmes.gabc}

\altertonus %%simplifying because it should be obvious from context

\gscore[]{4.}{hy_deus_tuorum_militum_alter_tonus_tp_solesmes.gabc}

\hymnusspecial{Pro Pluribus Martyribus}[]

\gscore[]{3.}{hy_rex_gloriose_martyrum_tp_solesmes.gabc}

\altertonus

\gscore[]{4.}{hy_rex_gloriose_martyrum_alter_tonus_tp_solesmes.gabc}

\vv Sancti et justi in Dómino gaudéte, allelúia.

\rr Vos elégit Deus in hæreditátem sibi, allelúia.
%%
\admagnificat
\label{an_lux_perpetua_solesmes_1961}
\phantomsection
\gscore[]{1. g}{an_lux_perpetua_solesmes_1961}
%

\rubrique{Oratio propria, vel de Commune extra Tempus Paschale.}
%%
\invesperis{ii}

\rubrique{Antiphonæ de I Vesperis.}

\rubrique{Pro Apostolis et Evangelistis, ut infra:}

\rubrique{Ps. \normaltext{Dixit Dóminus,} ut in I Vesperis.}

\psalmus{112}{112_7a}{112_7}

\psalmus{115}{115_2}{115_2}
\label{115_2_TP} \phantomsection

\psalmus{125}{125_8Gstar}{125_8}

\psalmus{138}{138_2}{138_2}

\rubrique{Pro uno vel pluribus Martyribus, Psalmi ut in I Vesperis sed in loco ultimi, Ps. \normaltext{Crédidi} ut supra, \normaltext{\pageref{115_2_TP}.}}

\caphymn{I}

\vv Pretiósa in conspéctu Dómini, allelúia.

\rr Mors sanctórum ejus, allelúia.

\admagnificat
\label{an_sancti_et_justi_solesmes_1961}
\gscore[]{8. G}{an_sancti_et_justi_solesmes_1961}


\end{document}