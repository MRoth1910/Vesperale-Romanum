% !TEX TS-program = lualatexmk
% !TEX parameter =  --shell-escape

\documentclass[vesperale_romanum.tex]{subfiles}

\ifcsname preamble@file\endcsname
  \setcounter{page}{\getpagerefnumber{M-vr_16_festa_decembris}}
\fi

%%this code when \customsubfiles is used should allow for continuous pagination when subfiles are compiled individually.

\begin{document}

%%LA1949 does NOT give St Andrew and its vigil, plus Saturinus a separate entry in the index/TOC
\chapter*{PROPRIUM SANCTORUM.}
\addcontentsline{toc}{chapter}{Proprium Sanctorum}\phantomsection
\header{festa novembris.}
\section*{FESTA NOVEMBRIS.}
\addcontentsline{toc}{section}{Festa Novembris}\phantomsection

\datefeast{29}{Commem. S.\@ Saturnini Martyris}

\rubrique{Ant.\@ \normaltext{Iste Sanctus, \pageref{M-an_iste_sanctus_solesmes_1961}, \vv Glória et honóre.}}

\oratio

\lettrine{D}{e}us, qui nos beáti Saturníni Mártyris tui concédis natalício pérfrui:~* ejus nos tríbue méritis adjuvári. Per Dóminum.

\rubrique{De Festis occurrentibus In Adventum non fit Officium, nisi fuerit Duplex vel Semiduplex. Festum Duplex I classis occurens in Dominica I Adventus, transfertur: si occurat in aliis Dominicis fit de eo cum commemoratione Dominicae. Duplex vero II classis occurens in Dominica quacumque Adventus transfertur.}

\rubrique{Quodvis aliud Festum Duplex vel Semiduplex occurens in Dominica fit tantum commemoratio in utrisque Vesperis. De Simplicibus fit tantum commmem.\@ etiam in Feriis.}
\vspace{\baselineskip}

\myrule

\datefeast{30}{S.\@ \capspace{ANDREÆ} Apostoli}

\duplexclassis{II}

\invesperis{i}

\primaantiphona{7. c}

\initialscore{an_salve_crux_pretiosa_solesmes_1961}
\psalmus{109}{109_7c}{109_7}

\gscore[2. Ant.]{8. G}{an_beatus_andreas_solesmes_1961}
\psalmus{110}{110_8G}{110_8}

\gscore[3. Ant.]{8. G}{an_andreas_christi_famulus_solesmes_1961}
\psalmus{111}{111_8G}{111_8}

\gscore[4. Ant.]{8. G}{an_maximilla_christo_amabilis_solesmes_1961}
\psalmus{112}{112_8G}{112_8}

\gscore[5. Ant.]{7. a}{an_qui_persequebantur_justum_solesmes_1961}
\psalmus{116}{116_7a}{116_7}

\capitulum{Rom. 10, 10 – 11}

\lettrine{F}{r}atres: Corde enim créditur ad justítiam,~† ore autem conféssio fit ad salútem.~* Dicit enim Scriptúra: Omnis qui credit in illum, non confundétur.

\rubrique{Hymnus \normaltext{Exsúltet orbis gáudiis, \pageref{M-hy_exsultet_orbis_gaudiis_solesmes_1961}.  \vv In omnem terram.}}

\admagnificat

\gscore[]{1. f}{an_unus_ex_duobus_solesmes_1961}

\oratio

\lettrine{M}{a}jestátem tuam Dómine supplíciter exorámus:~† ut sicut Ecclésiæ tuæ beátus Andréas Apóstolus éxstitit prædicátor et rector;~* ita apud te sit pro nobis perpétuus intercéssor.
Per Dóminum.

\rubrique{Si venerit in Adventu, fit commemoratio Feriæ.}

\invesperis{ii}

\rubrique{Antiphonæ ut in I Vesperis. Psalmi ut in II Vesperis Apostolorum et Evangelistæ, ut infra.}

\rubrique{Ps. \normaltext{Dixit Dóminus,} ut in I Vesperis.}

\rubrique{Ps. \normaltext{Laudáte puéri,} ut in I Vesperis.}

\psalmus{115}{115_8G}{115_8}

\psalmus{125}{125_8G}{125_8}

\psalmus{138}{138_7a}{138_7}

\rubrique{Capitulum \normaltext{Fratres: Corde enim créditur} ut in I Vesperis.}

%%\caphymn{I}

\rubrique{Hymnus \normaltext{Exsúltet orbis gáudiis, \pageref{M-hy_exsultet_orbis_gaudiis_solesmes_1961}.}}

\vv Annuntiavérunt ópera Dei. \rr Et facta ejus intellexérunt.

\admagnificat

\gscore[]{1. D}{an_cum_pervenisset_beatus_andreas_solesmes_1961}

\rubrique{Si venerit in Adventu, fit commemoratio Feriæ.}

\myrule

\newpage
\thispagestyle{empty}
\section[Festa Decembris.]{FESTA DECEMBRIS.}\header{festa decembris.}

\datefeast{2}{S.\@ Bibianæ Virginis et Martyris}

\semiduplex

\oratio

\lettrine{D}{e}us, ó\-mnium largítor bonórum, qui in fámula tua Bibiána cum virginitátis flore martýrii palmam coniunxísti:~† mentes nostras ejus intercessióne tibi caritáte conjúnge;~* ut amótis perículis, prǽmia consequámur ætérna.
Per Dóminum.

\rubrique{Et fit commemoratio Feriae in Adventu.}

\rubrique{Vesp. de seq., comm. præced. et Feriæ.}

\myrule

\datefeast{3}{S.\@ Francisci Xaverii Confessoris}

\duplexmajus

\oratio

\lettrine{D}{e}us, qui Indiárum gentes beáti Francísci prædicatióne et miráculis Ecclésiæ tuæ aggregáre voluísti:~† concéde propítius; ut cujus gloriósa mérita venerámur,~* virtútum quoque imitémur exémpla. Per Dóminum.

\rubrique{Commem.\@ S.\@Bibianæ, Ant.\@ \normaltext{Veni sponsa, \pageref{M-an_veni_sponsa_christi_ii_vesperis_solesmes_1961}. \vv Diffúsa est.}
Postea feriæ.}

\rubrique{In Vesp. Comm.\@ seq.\@ et Feriæ, ac S.\@ Barbaræ Virg.\@ et Mart.}

\myrule

\datefeast{4}{S.\@ Petri Chrysologi, Ep., Conf. et Eccl.\@ Doct}

\duplexmtv

\rubrique{Ant.\@ \normaltext{O Doctor…beáte Petre Chrysóloge, \pageref{M-an_o_doctor_optime_solesmes_1961}, \vv Amávit.}}

\oratio

\lettrine{D}{e}us, qui beátum Petrum Chrysólogum Do\-ctórem egrégium divínitus præmonstrátum, ad regéndam et instruéndam Ecclésiam tuam éligi voluísti:~† præsta, quǽsumus; ut quem Do\-ctórem vitæ habúimus in terris,~* intercessórem habére mereámur in cælis. Per Dóminum.

\rubrique{Deinde comm.\@ Feriæ. Postremo, comm.\@ \normaltext{S.\@ Barbaræ} Virg.\@ et Mart.\@ Ant.\@ \normaltext{Véni sponsa, \pageref{M-an_veni_sponsa_christi_i_vesperis_solesmes_1961}. \vv Spécie tua.}}

\oratio

\lettrine{D}{e}us, qui inter cétera poténtiæ tuæ mirácula étiam in sexu frágili vi\-ctóriam martýrii contulísti:~* concéde propítius; ut qui beátæ Bárbaræ Vírginis et Mártyris tuæ natalítia cólimus, per ejus ad te exémpla gradiámur. Per Dóminum.

\rubrique{In II Vesperis, comm.\@ Feriæ et seq.}

\myrule

\datefeast{5}{Commemoratio S.\@ Sabbæ Abbatis}

\rubrique{Ant.\@ \normaltext{Similábo eum. \pageref{M-an_similabo_eum_solesmes_1961}. \vv Amávit.}}

\oratio

\lettrine{I}{n}tercéssio nos, quǽsumus Dómine, beáti Sabbæ Abbátis comméndet:~* ut quod nostris méritis non valémus, ejus patrocínio assequámur. Per Dóminum.

\myrule

\datefeast{6}{S.\@ Nicolai Episcopi, Confessoris}

\duplex

\oratio

\lettrine{D}{e}us, qui beátum Nicoláum Pontíficem innúmeris decorásti miráculis:~† tríbue, quǽsumus; ut ejus méritis et précibus~* a gehénnæ incéndiis liberémur. Per Dóminum.

\myrule

\datefeast{7}{S.\@ Ambrosii, Ep., Conf. et Eccl.\@ Doct}

\duplexmtv

\rubrique{Ant.\@ \normaltext{O Doctor óptime, \pageref{M-an_o_doctor_optime_solesmes_1961}.}}

\oratio

\lettrine{D}{e}us, qui pópulo tuo ætérnæ salútis beátum Ambrósium minístrum tribuísti:~† præsta, quǽsumus; ut quem Do\-ctórem vitæ habúimus in terris,~* intercessórem habére mereámur in cælis. Per Dóminum.

\rubrique{Comm.\@ S.\@ Nicolai. Ant.\@ \normaltext{Amávit eum, \pageref{M-an_amavit_eum_dominus_solesmes_1961}. \vv Justum.} Postea feriæ.}

%\rubrique{De Vigilia Immaculatæ Conceptionis B.M.V. nihil fit nisi in Missa.}

\vigiliarubrique{Immaculatæ Conceptionis B.M.V.}

\rubrique{Vesperæ de sequenti, comm.\@ Feriæ tantum.}

\myrule
%\newpage

%\thispagestyle{empty}

\litdate{8 Decembris}
\thispagestyle{empty}
\label{8_decembris}
%%Beatæ Mariæ Virginis is in smaller caps/type size in LA1949
\festum{Immaculatæ Conceptio\\ Beatæ Mariæ Virginis} 

\duplexclassis{I}[cum Octava communi]

\invesperis{i}

\primaantiphona{1. g2}

\initialscore{an_tota_pulchra_es_maria__solesmes_1961}
\psalmus{109}{109_1g2}{109_1}

\gscore[2. Ant.]{8. G}{an_vestimentum_tuum_solesmes_1961}
\psalmus{112}{112_8G}{112_8}

\label{an_tu_gloria_jerusalem_solesmes_1961}\phantomsection
\gscore[3. Ant.]{8. c}{an_tu_gloria_jerusalem_solesmes_1961}
\psalmus{121}{121_8c}{121_8}

\label{an_benedicta_es_tu_virgo_solesmes}\phantomsection
\gscore[4. Ant.]{7. a}{an_benedicta_es_tu_virgo_solesmes}
\psalmus{126}{126_7a}{126_7}

\gscore[5, Ant.]{3. a2}{an_trahe_nos_virgo_solesmes_1961}
\psalmus{147}{147_3a2}{147_3a2}

\capitulum{Prov. 8, 22 – 24}

\lettrine{D}{o}minus possédit me in inítio viárum suárum, ántequam quidquam fáceret a princípio.~† Ab ætérno ordináta sum, et ex antíquis ántequam terra fíeret:~* nondum erant abýssi, et ego jam concépta eram.

\amsrubrique

\vv Immaculáta Concéptio est hódie sanctæ Maríæ Vírginis.

\rr Quæ serpéntis caput virgíneo pede contrívit.

\admagnificat
\gscore[]{8. G}{an_beatam_me_dicent_quia_fecit_solesmes_1961}

\oratio

\lettrine{D}{e}us, qui per immaculátam Vírginis Conceptiónem dignum Fílio tuo habitáculum præparásti:~† quǽsumus; ut qui ex morte ejúsdem Fílii tui prævísa, eam ab omni labe præservásti,~* nos quoque mundos ejus intercessióne ad te perveníre concédas.
Per eúmdem Dóminum.

\rubrique{Et fit commemoratio Feriæ tantum.}

\rubrique{Ad Completorium., Hymnus cantatur in tono Festorum B.M.V., et in fine dicitur \normaltext{Jesu tibi sit glória, Qui natus es de Vírgine} 
Quæ doxologia servatur per totam Octavam ad omnes Hymnos ejusdem
metri, præterquam in Officio Dominicæ infra Octavam vel in die Octava
occurrentis. 
Tonus vero assignatus pro Completorio servatur per totam Octavam ad eamdem Horam tantum, etiam in Festis, 
præterquam in Dominica occurente.}

\invesperis{ii}
\omniapraeter

\admagnificat
\gscore[]{1. f}{an_hodie_egressa_est_solesmes_1961}

\myrule

\datefeast{9}{II Die infra Oct.\@ Immac.\@ Concept}

\rubrique{In Vesperis, comm.\@ Feriae, et \normaltext{S.\@Melchiadis} Papæ, Martyris. Ant.
\normaltext{Iste Sanctus, \pageref{M-an_cs_iste_sanctus_pro_lege_solesmes_1961}, \vv Gloria et honóre.}}

\oratio

\lettrine{I}{n}firmitátem nostram réspice, omnípotens Deus:~* et quia pondus própriæ actiónis gravat, beáti Melchíadem Mártyris tui atque Pontíficis intercéssio gloriósa nos prótegat. Per Dóminum.

\myrule

\datefeast{10}{III Die infra Oct.\@ Immac.\@ Concept}

\rubrique{Vesp. de seq., comm.\@ Octavæ et Feriæ.}

\myrule

\datefeast{11}{S.\@Damasi I Papæ, Confessoris}
\semiduplex

\oratio

\lettrine{E}{x}áudi Dómine preces nostras:~† et interveniénte beáto Dámaso Confessóre tuo atque Pontífice,~* indulgéntiam nobis tríbue placátus et pacem. Per Dóminum.

\rubrique{In Vesperis, comm.\@ Octavæ et Feriæ.}

\myrule

\datefeast{12}{V Die infra Oct.\@ Immac.\@ Concept}

\myrule

\rubrique{Vesp. de seq., comm.\@ Octavæ et Feriæ.}

\myrule

\datefeast{13}{S.\@ Luciæ Virginis et Martyris}

\rubrique{De Communi Virginum, \normaltext{\pageref{M-vr_34_commune_virginum},} præter sequentia.}

\invesperisminor{I}

\gscore[1. Ant.]{7. b}{an_orante_sancta_lucia_solesmes_1961}
\psalmus{109}{109_7b}{109_7}

\gscore[2. Ant.]{7. a}{an_lucia_virgo_solesmes_1961}
\psalmus{112}{112_7a}{112_7}

\gscore[3. Ant.]{8. G}{an_per_te_lucia_solesmes_1961}
\psalmus{121}{121_8G}{121_8}

\label{an_benedico_filium_solesmes_1961}
\gscore[4. Ant.]{8. G}{an_benedico_filium_solesmes_1961}
\label{126_8G_dec}
\psalmus{126}{126_8G}{126_8}

\gscore[5. Ant.]{8. G}{an_soror_mea_lucia_solesmes_1961}
\psalmus{147}{147_8G}{147_8}

%%should we use EUOUAE for Benedico

\admagnificat

\gscore[]{1. f}{an_in_tua_patientia_solesmes_1961}

\oratio

\lettrine{E}{x}áudi  nos, Deus, salutáris noster:~† ut sicut de beátæ Lúciæ Vírginis et Mártyris tuæ festivitáte gaudémus;~* ita piæ devotiónis erudiámur affé\-ctu. Per Dóminum.

\rubrique{Commem.\@ Octavæ et Feriæ.}

\invesperisminor{II}

\omniapraeter

\vv Diffúsa.

\admagnificat

\gscore[]{7. c}{an_tanto_pondere_solesmes_1961}

\rubrique{Commem.\@ Octavæ et Feriæ.}

\myrule

\datefeast{14}{VII Die infra Oct.\@ Immac.\@ Concept}

\rubrique{Ad Vesperas, Duplex majus. Omnia ut in I Vesperis Festi, \normaltext{\pageref{8_decembris},} cum commem.\@ Feriæ.}

\rubrique{Si dies Octava Immacul.\@ Concept.\@ B.M.V.\@ inciderit in Dominica III Adventus, fit Officium de Dominica hoc modo: Vesperæ in Sabbato præcedenti dicuntur cum Psalmis Sabbati et Antiphonis Dominicæ: et fit Commem.\@ diei Octavæ ut in I Vesperis Festi. Ad
Completorium non dicuntur Preces. Ipsa Dominica fit Commem.\@ diei
Octavæ in Vesperis. In Hymnis non mutatur tonus neque doxologia, nec dicuntur Preces ad Completorium.}

\myrule

\datefeast{15}{Octava Immaculatæ Concept. B.M.V}

\duplexmajus

\rubrique{Omnia ut in Festo.}

\rubrique{In II Vesperis, Commem.\@ seq.\@ et Feriæ.}

\myrule

\datefeast{16}{S.\@Eusebii Episcopi Martyris}

\semiduplex

\rubrique{Ant.\@ \normaltext{Iste sanctus, \pageref{M-an_iste_sanctus_solesmes_1961}, \vv  Glória et honóre.}}

\oratio

\lettrine{D}{e}us, qui nos beáti Eusébii Mártyris tui atque Pontíficis ánnua sole\-mnitáte lætíficas:~† concéde propítius; ut cujus natalítia cólimus,~* de ejúsdem étiam prote\-ctióne gaudeámus. Per Dóminum.

\commferiae

\vigiliarubrique{S.\@ Thomæ Apostoli}

\myrule

\datefeast{21}{In Festo S.\@\capspace{THOMÆ} Apostoli}

\duplexclassis{II}

\smalltitle{Ad Magnificat, in utrisque Vesperis.}

\gscore[]{8. G}{an_quia_vidisti_me_solesmes_1961}

\oratio

\lettrine{D}{a} nobis, quǽsumus Dómine, beáti Apóstoli tui Thomæ solemnitátibus gloriári:~† ut ejus semper et patrocíniis sublevémur;~* et fidem cóngrua devotióne sectémur. Per Dóminum.

\rubrique{In I Vesp.\@ pro Commem. Adventus, Antiphona \normaltext{O Clavis David, \pageref{M-an_o_clavis_david_solesmes_1961},} vel si transferatur in Feriam II, \normaltext{O Oriens.} — In II Vesp., \normaltext{O Oriens, \pageref{M-an_o_oriens_solesmes_1961},} vel si transferatur, \normaltext{O Rex géntium.}
}


\end{document}