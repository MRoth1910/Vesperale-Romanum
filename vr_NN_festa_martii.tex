% !TEX TS-program = lualatexmk
% !TEX parameter =  --shell-escape
\documentclass[vesperale_romanum.tex]{subfiles}

\ifcsname preamble@file\endcsname
  \setcounter{page}{\getpagerefnumber{M-vr_NN_festa_martii}}
\fi

%%this code when \customsubfiles is used should allow for continuous pagination when subfiles are compiled individually.

\begin{document}

%%all to fix

\section[Festa Martii]{FESTA MARTII.}\header{festa martii.}

\datefeast{4}{S.\@ Casimiri Confessoris}

\semiduplex

\oratio

\lettrine{D}{e}us, qui inter regáles delícias et mundi illécebras san\-ctum Casimírum virtúte constántiæ roborásti:~† quǽsumus; ut ejus intercessióne fidéles tui terréna despíciant,~* et ad cæléstia semper aspírent.
Per Dóminum.

\quadcommferiae

\rubrique{Postea \normaltext{S.\@ Lucii I}, Papæ et Martyris. Ant. \normaltext{Iste Sanctus, \pageref{M-an_cs_iste_sanctus_pro_lege_solesmes_1961}, \vv Glória et honóre.}}

\oratio

\lettrine{D}{e}us, qui nos beáti Lúcii Martýris tui atque Pontifícis tui ánnua sole\-mnitáte lætíficas:~† concéde propítius; ut cujus natalítia cólimus,~* de ejúsdem étiam prote\-ctióne gaudeámus. Per Dóminum.

\capitdeseqquad

\myrule

\datefeast{6}{SS.\@ Felicii et Perpetuæ Martyrum}

\duplex

\smalltitle{Ad Magnificat in utrisque Vesperis, Antiphona.}

\gscore[]{8. G}{an_cs_istarum_est_enim_solesmes_1961}

\vv Glória et honóre coronásti eas Dómine.

\rr Et constituísti eas super ópera mánuum tuárum.

\oratio

\lettrine{D}{a} nobis, quǽsumus Dómine Deus noster, san\-ctárum Mártyrum tuárum Perpétuæ et Felicitátis palmas incessábili devotióne venerári:~† ut quas digna mente non póssumus celebráre,~* humílibus saltem frequentémus obséquiis. Per Dóminum.

\capitdeseqquad

\myrule

\datefeast{7}{S.\@ Thomæ de Aquino Conf.\@ et Eccl.\@ Doct}

\duplex

\odoctoroptime

\oratio

\lettrine{D}{e}us, qui Ecclésiam tuam beáti Thomæ Confessóris tui mira eruditióne claríficas, et san\-cta operatióne fœcúndas:~† da nobis, quǽsumus; et quæ dócuit, intellé\-ctu conspícere, et quæ egit, imitatióne complére. Per Dóminum.

\rubrique{Comm.\@ Perpétuæ et Felicitátis. Ant. \normaltext{Istárum est, \vv Glória et honóre} ut supra. Deinde Feriæ in Quadrag.}

\capitdeseqquad

\myrule

\datefeast{8}{S.\@ Joannis de Deo Confessoris}

\duplex

\oratio

\lettrine{D}{e}us, qui beátum Joánnem tuo amóre succénsum, inter flammas innóxium incédere fecísti, et per eum Ecclésiam tuam nova prole fecundásti:~† præsta, i\-psíus suffragántibus méritis; ut igne caritátis tuæ vítia nostra curéntur,~* et remédia nobis ætérna provéniant. Per Dóminum.

\rubrique{Comm.\@ S.\@ Thomæ. Ant. \normaltext{O Doctor óptime, ℣., \pageref{M-an_cs_o_doctor_optime_solesmes_1961} \vv Justum dedúxit.} Postea Feriæ in Quadrag.}

\myrule

\newpage

\datefeast{9}{S.\@ Franciscæ Viduæ Romanæ}

\duplex

\oratio

\lettrine{D}{e}us, qui beátam Francíscam fámulam tuam, inter cétera grátiæ tuæ dona, familiári Angeli consuetúdine decorásti:~† concéde, quǽsumus; ut intercessiónis ejus auxílio,~* Angelórum consórtium cónsequi mereámur.
Per Dóminum.

\rubrique{Comm.\@ præc.\@ Ant. \normaltext{Hic vir, \pageref{M-an_cs_hic_vir_despiciens_solesmes_1961}, \vv Justum dedúxit.} Postea Feriæ in Quadrag.}

\rubrique{In II Vesperis, comm.\@ seq., et Feriæ in Quadrag.}

\myrule

\datefeast{10}{SS.\@ Quadraginta Martyrum}

\semiduplex

\rubrique{Ant. \normaltext{Istórum est, \pageref{M-an_istorum_est_enim_solesmes_1961}, \vv Lætámini.}}

\oratio

\lettrine{P}{r}æsta, quǽsumus omnípotens Deus:~† ut qui gloriósos Mártyres fortes in sua confessióne cognóvimus,~* pios apud te in nostra intercessióne sentiámus. Per Dóminum.

\commferiae

\myrule

\datefeast{12}{S.\@ Gregorii I Papæ, Conf.\@ et Eccl.\@ Doct}

\duplex

\rubrique{In utrisque Vesperis, Ant. \normaltext{O Doctor óptime, ℣., \pageref{M-an_o_doctor_optime_solesmes_1961}.}}

\oratio

\commferiae

\myrule

\newpage
\datefeast{17}{S.\@ Patricii  Episcopi, Confessoris}

\duplex

\oratio

\lettrine{D}{e}us, qui ad prædicándam géntibus glóriam tuam beátum Patrícium Confessórem atque Pontíficem míttere dignátus es: ejus méritis et intercessióne concéde; ut, quæ nobis agénda prǽcipis, te miseránte adimplére possímus. Per Dóminum.

\commferiae

\capitdeseqcommfer

\myrule

\longdatefeast{18}{S.\@ Cyrilli Episcopi Hierosolymitani}{Conf.\@ et Eccl.\@ Doct}

\duplex

\rubrique{Ant. \normaltext{O Doctor óptime, ℣., \pageref{M-an_o_doctor_optime_solesmes_1961}.}}

\oratio

\lettrine{D}{a} nobis, quǽsumus omnípotens Deus, beato Cyríllo Pontífice intercedénte: te solum verum Deum, et quem misísti Jesum Christum ita cognóscere:~† ut inter oves, quæ vocem ejus áudiunt,~* perpétuo connumerári mereámur. Per eumdem Dóminum.

\rubrique{Comm.\@ S.\@ Patritii. Ant.\@ \normaltext{Amávit eum, \pageref{M-an_cs_amavit_eum_dominus_solesmes_1961}. \vv Justum.} Deinde feriæ.} 

\rubrique{Vesp.\@ de sequenti, comm.\@ Feriæ tantum.}

\myrule

\newpage

\litdate{19 Martii}

\festum{S. Joseph, sponsi b.m.v} 

\duplexclassis{I}

\rubrique{Si hoc Festum occurrerit in Dominica Passionis, transferendum erit in Feriam II immediate sequentem; et quoties inciderit in Majorem Hebdomadam, reponendum erit in Feria IV post Dominicam in Albis, tanquam in sede propria.}

%\subsection{in i. vesperis.} %%again subsection or section may change! and we don't want this in the toc!

\invesperis{i}

%%no TP Alleluia; you just have to know to add these, and it's far less likely to be moved than Annunciation

\primaantiphona{1. g}

\initialscore{an_jacob_autem_genuit_solesmes_1961}
\psalmus{109}{109_1g}{109_1}

\gscore[2. Ant.]{2. D}{an_missus_est_angelus_gabriel_solesmes_1961}
\label{110_2_mar} \phantomsection
\psalmus{110}{110_2}{110_2}

\gscore[3. Ant.]{3. a2}{an_cum_esset_desponsata_solesmes_1961}
\label{111_3a2_mar} \phantomsection
\psalmus{111}{111_3a2}{111_3a2_g}

\gscore[4. Ant.]{4. E}{an_joseph_vir_ejus_solesmes_1961}
\label{112_4E_mar} \phantomsection
\psalmus{112}{112_4E}{112_4E}

\gscore[5. Ant.]{5. a}{an_angelus_domini_apparuit_joseph_solesmes_1961}
\psalmus{116}{116_5a}{116_5}

\phantomsection
\label{cap_19_mar}
\capitulum{Prov. 28, 20; 27, 18}

\lettrine{V}{ir} fidélis multum laudábitur.~* Et qui custos est Dómini sui, glorificábitur.

\hymnus

\label{hy_te_joseph_celebrent_solesmes}\phantomsection

\gscore[]{1.}{hy_te_joseph_celebrent_solesmes}

%% watch virga length and fœ́ as font does not actually contain œ́ so TeX does its thing, and the accent is over the wrong letter. TBD if ǽ is indeed inappropriate

\textes{Versiculus_19_martii}

\admagnificat
\gscore[]{1. g2}{an_exsurgens_joseph_solesmes_1961}

\label{or_19_mar}

\oratio
\lettrine{S}{a}nctíssimæ Genetrícis tuæ spónsi, quǽsumus Dómine, méritis adjuvémur:~† ut quod possibílitas nostra non óbtinet,~* ejus nobis intercessióne donétur.
Qui vivis et regnas cum Deo Patre.

\commferiae

\newpage

\invesperis{ii}

\primaantiphona{1. a3}

\initialscore{an_ibant_parentes_jesu_solesmes_1961}
\psalmus{109}{109_1a3}{109_1}

\gscore[2. Ant.]{2. D}{an_cum_redirent_solesmes_1961}

\rubrique{Ps. \normaltext{Confitébor, \pageref{110_2_mar}.}}

\gscore[3. Ant.]{3. a2}{an_non_invenientes_jesum_solesmes_1961}

\rubrique{Ps. \normaltext{Beátus vir, \pageref{111_3a2_mar}.}}

\gscore[4. Ant.]{4. E}{an_dixit_mater_jesu_ad_illum_solesmes_1961}

\rubrique{Ps. \normaltext{Laudáte Púeri, \pageref{112_4E_mar}.}}

\gscore[5. Ant.]{8. G}{an_descendit_jesus_cum_eis_solesmes_1961}
    \psalmus{116}{116_8G}{116_8}

\caphymn{I}

\vv Glória et divítiæ in domo ejus.

\rr Et justítia ejus manet in sǽculum sǽculi.

\admagnificat
\gscore[]{8. G}{an_ecce_fidelis_servus_solesmes_1961}

\commferiae

\myrule

\newpage

\datefeast{21}{S.\@ Benedicti Abbatis}

\duplexmajus

\oratio

\lettrine{I}{n}tercéssio nos, quǽsumus Dómine, beáti Benedí\-cti Abbátis comméndet:~† ut quod nostris méritis non valémus,~* ejus patrocínio assequámur. Per Dóminum.

\commferiae

\myrule

\datefeast{24}{S.\@ Gabrielis Archangeli}

\duplexmajus

\invesperisminor{I}

\gscore[1. Ant.]{8. G}{an_ingresso_zacharia_templum_mar_24_solesmes}
\psalmus{109}{109_8G}{109_8}

\gscore[2. Ant.]{1. f}{an_ait_autem_angelus_solesmes_1961}
 \psalmus{110}{110_1f}{110_1} 
 
 \gscore[3. Ant.]{6. F}{an_ego_sum_gabriel_solesmes_1961}
 \psalmus{111}{111_6_alt}{111_6_alt} 
 
 \gscore[4. Ant.]{4. E}{an_gabriel_angelus_ecce_solesmes_1961}
\rubrique{Ps. \normaltext{Laudáte puéri, \pageref{112_4E_mar}.}}

\gscore[5. Ant.]{3. a}{an_dixit_autem_maria_solesmes_1961}
\psalmus{137}{137_3a}{137_3a_b}

\capitulum{Dan. 9, 21 – 22}

\lettrine{E}{c}ce vir Gábriel, quem víderam in visióne a princípio, cito volans tétigit me in témpore sacrifícii vespertíni.~† Et dócuit me, et locútus est mihi,~* dixítque: Dániel, nunc egréssus sum ut docérem te, et intellégeres.

\hymnus

\gscore[]{1.}{hy_christe_sanctorum_solesmes_1961}

\vv Stetit Angelus juxta aram templi.

\rr Habens thuríbulum áureum in manu sua.

\admagnificat

\gscore[]{7. a}{an_angelus_gabriel_apparuit_solesmes_1961}

\oratio

\lettrine{D}{e}us, qui inter céteros Angelos, ad annuntiándum Incarnatiónis tuæ mystérium, Gabriélem Archángelum elegísti:~† concéde propítius; ut, qui festum ejus celebrámus in terris,~* i\-psíus patrocínium sentiámus in cælis. Qui vivis et regnas.

\commferiae

\invesperisminor{II}

\rubrique{Quando dicendæ sunt, antiphonæ et psalmi ut in I Vesperis, sed loco ultimi Ps.\normaltext{ Laudáte Dóminum,} ut infra.}

\psalmus{116}{116_3a}{116_3a_b}

\caphymn{I}

\vv In conspéctu Angelórum psallam tibi Deus meus.

\rr Adorábo ad templum sanctum tuum, et confitébor nómini tuo.

\admagnificat

\gscore[]{1. D}{an_archangelus_gabriel_solesmes_1961}

\myrule

\newpage

\litdate{25 Martii}

\festum{In annuntiatione b.m.v} 

\duplexclassis{I}

\rubrique{Si hoc Festum occurrerit in Dominica Passionis, transferendum erit in Feriam II immediate sequentem. Quod si incidat in Hebdomadam Majorem vel Paschcalem, in Feria II post Dominicam in Albis, tamquam in sedem propriam amandetur, servato ritu paschali, ac nonnisi Festo primario ejusdem ritus occurrente valeat impediri, quo in casu in sequentem diem similiter non impeditam transferatur.}

%\subsection{in i. vesperis.} %%again subsection or section may change! and we don't want this in the toc!

\invesperis{i}

\primaantiphona{8. G*}

\initialscore{an_missus_est_gabriel_solesmes_1961}
\psalmus{109}{109_8Gstar}{109_8}

\gscore[2. Ant.]{1. g}{an_ave_maria_alleluia_solesmes_1961}
\psalmus{112}{112_1g}{112_1}

\gscore[3. Ant.]{8. G}{an_ne_timeas_alleluia_solesmes_1961}
\psalmus{121}{121_8G}{121_8}

\gscore[4. Ant.]{1. f}{an_dabit_ei_dominus_solesmes_1961}
\psalmus{126}{126_1f}{126_1}

\gscore[5. Ant.]{8. c}{an_ecce_ancilla_domini_solesmes_1961}
\psalmus{147}{147_8c}{147_8}

\capitulum{Isaiæ 7, 14 – 15}

\lettrine{E}{c}ce virgo concípiet, et páriet fílium,~† et vocábitur nomen ejus Emmánuel.~* Butýrum et mel cómedet, ut sciat reprobáre malum, et elígere bonum.

\amsrubrique%%needs page reference when added to main file.

\vv Ave Maria, grátia plena.

\rr Dóminus tecum.

\admagnificat
\gscore[]{8. G}{an_spiritus_sanctus_solesmes_1961}

\label{or_25_mar} \oratio

\textes{oratio_25_mar}

%%is eu-mdem really a correct split??

\commferiae

\rubrique{Ad Completorium, Tonus et Doxologia de B.M.V.}

\invesperis{ii}

\omniapraeter

\vv Ave Maria, grátia plena. \rr Dóminus tecum.

\admagnificat
\gscore[]{7. d}{an_gabriel_angelus_ave_solesmes_1961}

\myrule

\datefeast{27}{S.\@ Joannis Damasceni Conf.\@ et Eccl.\@ Doct}

\duplexmtv

\rubrique{In utrisque Vesperis, Ant. \normaltext{O Doctor óptime…beáte Joánnes, \pageref{M-an_o_doctor_optime_solesmes_1961}.}}

\oratio

\lettrine{O}{m}nípotens sempitérne Deus, qui ad cultum sacrárum imáginum asseréndum, beátum Joánnem cælésti doctrína et admirábili spíritus fortitúdine imbuísti:~† concéde nobis ejus intercessióne et exémplo'~* ut quorum cólimus imágines, virtútes imitémur et patrocínia sentiámus.
Per Dóminum.

\commferiae

\rubrique{In II Vesperis, comm.\@ seq., et Feriæ.}

\myrule

\datefeast{28}{S.\@ Joannis a Capistrano Conf}

\semiduplexmtv

\rubrique{Ant. \normaltext{Similábo eum, \pageref{M-an_similabo_eum_solesmes_1961}, \vv Amávit eum Dóminus.}}

\oratio

\lettrine{D}{e}us, qui per beátum Joánnem fidéles tuos in virtúte san\-ctíssimi nóminis Jesu de crucis inimícis triumpháre fecísti:~† præsta, quǽsumus; ut spirituálium hóstium, ejus intercessióne, superátis insídiis,~* corónam justítiæ a te accípere mereámur. Per Dóminum.

\commferiae

\myrule

\feriadate{VI post Dominicam Passionis}
\vspace{0.25\baselineskip}

\specialdatefeast{Septum Dolorum B.M.V}

\duplexmajus

\invesperisminor{I}

\gscore[1. Ant.]{1. a3}{an_vadam_ad_montem_solesmes_1961}
\psalmus{115}{115_1a3}{115_1a3}

\gscore[2. Ant.]{2. D}{an_dilectus_meus_solesmes_1961}
\psalmus{119}{119_2}{119_2}

\gscore[3. Ant.]{3. a2}{an_quo_abiit_solesmes_1961}
\psalmus{139}{139_3a2}{139_3a2}

\gscore[4. Ant.]{4. A*}{an_fasciculus_myrrhae_solesmes_1961}
\psalmus{140}{140_4A_A_star}{140_4A_A_star}

\gscore[5. Ant.]{1. f}{an_fulcite_me_floribus_solesmes_1961}
\psalmus{144}{144_1f}{144_1f}

\capitulum{Isaiæ 53, 1 – 2}

\lettrine{Q}{u}is crédidit audítui nostro? et bráchium Dómini cui revelátum est? Et ascéndet sicut virgúltum coram eo,~* et sicut radix de terra sitiénti.

\hymnus

\gscore[]{6.}{hy_stabat_mater_dolorosa_solesmes_fer_6_passionis}

\vv Ora pro nobis, Virgo dolorosíssima.

\rr Ut digni efficiámur promissiónibus Christi.

\admagnificat

\gscore[]{6. F}{an_tuam_ipsius_animam_solesmes_1961}

\oratio

\lettrine{D}{e}us, in cujus passióne, secúndum Simeónis prophetíam, dulcíssimam ánimam gloriósæ Vírginis et Matris Maríæ dolóris gládius pertransívit:~† concéde propítius; ut qui dolóres ejus venerándo recólimus,~* passiónis tuæ effé\-ctum felícem consequámur. Qui vivis et regnas cum Deo Patre in unitáte Spíritus Sancti Deus.

\commferiae

\rubrique{Ad Completorium, Hymnus cantatur ut in Tempore Passionis, \normaltext{\pageref{hy_te_lucis_ante_terminum_passiontide_solesmes_1961},} et in fine dicitur:}

\smallscore{dox_sept_dolor_bvm}

\invesperisminor{II}

%%different rubric but it's OK

\omniapraeter

\admagnificat

\gscore[]{8. G}{an_cum_vidisset_jesus_matrem_solesmes_1961}

\commferiae

\rubrique{Si hoc Festum habeat tantum II Vesperas, Hymnus \normaltext{Virgo Vírgium} dicitur ad II Vesperas.}

\gscore[]{6.}{hy_virgo_virginum_praeclara_solesmes}

\end{document}