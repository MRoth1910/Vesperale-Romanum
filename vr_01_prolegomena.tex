% !TEX TS-program = lualatexmk
% !TEX parameter =  --shell-escape

\documentclass[vesperale_romanum.tex]{subfiles}

\ifcsname preamble@file\endcsname
  \setcounter{page}{\getpagerefnumber{M-vr_01_prolegomena}}
\fi

%%this code when \customsubfiles is used should allow for continuous pagination when subfiles are compiled individually.

\begin{document}


\begin{titlepage}
 \begin{center}
\fontsize{36}{45}\selectfont\capspace{VESPERALE ROMANUM}

{\LARGE\scspace{a pio papa x restitutum et editum}}
 \end{center}
%\centering
%\includegraphics[width=10cm,height=10cm,keepaspectratio]{Dürer_The_Visitation} %% need arms of some kind…
 \begin{center}
{\LARGE\scspace{rhythmicis signis a solesmensibus monachis ornatum}}
 \end{center}
 
 \vfill
 
{\centering{\LARGE\scspace{sancta maria ab assumptione}\par}}
 {\centering{\capspace{\YEAR}\par}}
\end{titlepage}

%%I believe this should be blank and the other stuff begins on the right page (and with lowercase roman numerals?)

%%we need to consider any prefatory materials…

\thispagestyle{empty}
Decreta, quibus non derogatur per typicam Antiphonalis Romani editionem, hic addere in gratiam Le\-ctoris Editoribus visum est.

\myrule

{\centering\capspace{DECRETUM}\par} %%to replace with a better title command, but for now this will work

{\centering{\rubrique{seu declaratio super editione Vaticana ejusque reprodu\-ctione quoad libros liturgicos gregorianos.}}}

Cum postulatum fuerit, an Episcopi possint propriam approbationem donare libris cantus gregoriani, melodias Vaticanæ editionis adamussim reprodu\-ctas continentibus, sed cum signorum rhythmicorum indicatione, privata au\-ctoritate additorum?

Sacra Rituum Congregatio, ad majorem declarationem Decreti n. 4259, 25 Januarii vertentis anni, respondendum censuit:
Editionibus in subsidium scholarum cantorum, signis rhythmicis, uti vacant, privata au\-ctoritate ornatis, poterunt Ordinarii, in sua quisque Dioecesi, apponere \textit{Imprimatur,} dum modo constet, cetera, quæ in Decretis Sacræ Rituum Congregationis injun\-cta sunt, quoad cantus gregoriani restaurationem, fuisse servata.

Quam resolutionem San\-ctissimo Domino nostro Pio Papæ X, per Sacrorum Rituum Congregationis Secretarium relatam, San\-ctitas Sua ratam habuit et probavit.

Die 11 Aprilis 1911. {\hfill(n. 4263. Vol. VI., p. 114. Decret. authent. S.R.C.)}
\vspace{5ex}%% to replace later

{\centering\capspace{DECRETUM}\par} %%to replace with a better title command, but for now this will work

{\centering{\rubrique{circa modulandas monosyllabas vel hebraicas voces in le\-ctionibus, versiculis et psalmis.}\par}}

A quibusdam cantus gregoriani magistris Sacræ Rituum Congregationi sequens dubium pro opportuna solutione expositum fuit; nimirum:

An in cantandis Le\-ctionibus et Versiculis, præsertim vero in Psalmorum mediantibus ad asteriscum, quando vel di\-ctio monosyllaba vel hebraica vox occurrit, immutari possit clausula, vel cantilena proferri sub modulatione consueta?

Et Sacra eadem Congregatio, approbante San\-ctissimo Domino nostro Pio Papa X, rescribere statuit: \textit{Affimative ad utrumque.}

Die 8 Julii 1912.

\hfill{Fr. S. \scspace{Card. Martinelli,} \textit{S.R.C. Præfi\-ctis.}}

L. † S.

{\hfill † Petrus La Fontaine, Episc. Charystien., \textit{Secretarius.}}

(A\-cta Apost. Sedis, Vol. IV., p. 539)
\vspace{5ex}%% to replace later

{\centering\capspace{DECRETUM}\par} %%to replace with a better title command, but for now this will work

{\centering{\rubrique{circa syllabas hypermetricas in cantu hymnorum.}\par}}

\textit{Dubium de syllabis hypermetrici quoad cantum.} Sacra Rituum Congregatione pluries expostulatum fuit: « An regula descripta in Antiphonario Vaticano circa syllabas hypermetricas, quæ frequenter occurrunt in cantu hymnorum, scilicet quod ipsæ non elidantur, sed distin\-ctæ pronuncientur propriaque nota cantentur, stri\-cte et rigorose interpretanda sit, vel e contra liceat etiam ipsas syllabas elidere, præsertim si in praxi id facilius et convenientius censeatur »?

Et Sacra eadem Congregatio, audita specialis Commissionis pro cantu liturgico gregoriano sententia, propositæ quæstioni, re sedulo perpensa ita rescribendum censuit : «Negative ad primam partem, affirmative ad secundam ». Atque ita rescripsit et declaravit die 14 Maii 1915.

\hfill{A.  \scspace{Card. Vico,} \textit{Pro-Præfe\-ctus.}}

{\hfill† Petrus La Fontaine, Patriarcha ele\-ctus Venetiarum, \textit{Secretarius.}}

(Acta Apost. Sedis, 1915, fasc. 9, p. 237).

\end{document}