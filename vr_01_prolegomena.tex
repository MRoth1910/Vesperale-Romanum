 % !TEX TS-program = lualatexmk
% !TEX parameter =  --shell-escape

\documentclass[vesperale_romanum.tex]{subfiles}

\ifcsname preamble@file\endcsname
  \setcounter{page}{\getpagerefnumber{M-vr_01_prolegomena}}
\fi

%%this code when \customsubfiles is used should allow for continuous pagination when subfiles are compiled individually.

\begin{document}

%%we need to figure out blank pages here

\begin{titlepage}
 \begin{center}
\fontsize{40}{50}\selectfont\capspace{VESPERALE ROMANUM}

%%originally 36 and 45

{\LARGE\scspace{a pio papa x restitutum et editum}}
 \end{center}
%\centering
%\includegraphics[width=10cm,height=10cm,keepaspectratio]{Dürer_The_Visitation} %% need arms of some kind…
 \begin{center}
{\LARGE\scspace{rhythmicis signis a solesmensibus monachis ornatum}}
 \end{center}
 
 %%commission a drawing: something Marian and Benedictine like a lilium inter spinas (a classic day lilly to be precise)
 
 \vfill
 
 {\centering{\large{Editio Nova}\par}}
{\centering{\LARGE\scspace{sancta maria ab assumptione}\par}}
 {\centering{\capspace{\YEAR}\par}}
\end{titlepage}

\shipout\null
\shipout\null

\frontmatter

\thispagestyle{empty}
\section*{Proœmium.}
\addcontentsline{toc}{section}{Proœmium}
\phantomsection


\bigtitle{Preface to the \textit{Vesperale Romanum.}}

\begin{enpars}
The \textit{Vesperale Romanum} is the book containing the chants and texts of the evening offices, that is, of Vespers and of Compline. The Vatican commission established by Pope Saint Pius \textsc{x} to create a unified edition of chant for the entire Roman church published the official edition of the \textit{Antiphonale Romanum} for the day hours, excluding Matins, in 1912; a vesperale followed in 1913. This edition however suffers from some problems, namely that the psalter of 1911 is included, but the texts, calendar, and rubrics are still those of the office of Saint Pius \textsc{v} as amended; it does not therefore conform to the \textit{editio typica} issued by Saint Pius \textsc{x}.

The monks of the abbey of Saint-Pierre de Solesmes, whose research of medieval manuscripts led to the creation of the Vatican commission in the first place, published their own vesperale, including the rhythmic signs favored by Dom André Mocquereau, longtime choirmaster of the abbey, but this edition was apparently less popular than other Solesmes editions, namely the \textit{Liber Usualis} which contains the chant for Sundays, feasts, and Holy Week, and the offices of the Sacred Triduum, but for not ferial days otherwise.

\smalltitle{Why a new vesperale?}

Above all, this new edition is intended as a replacement for print copies, given that neither the Vatican nor Solesmes editions of the full antiphonal nor the vesperale are in print. Further, we take advantage of the Gregorio software package, which allows for computer typesetting of Gregorian chant via LuaLaTeX; this is superior to relying on scans, since even if a high-quality scan were available, the quality of the printed score would deterioate over time, which is the case with the republished editions of chant, where the neumes such as the \textit{quilisma} or the rhythmic signs, notably the \textit{ictus} or \textit{vertical episema,} are not always identifiable.

However, this is not simply a reproduction. We aim to improve upon the traditional books to make the book as easy to use as possible, although a certain familiarity with the office is required due to the nature of a printed book.

\smalltitle{Editorial Principles.}

The vesperale project follows at its heart the office before the changes of Venerable Pope Pius \scspace{xii}, while allowing nevertheless for some discretion, e.g. we include Ss.\@ Anthony and Lawrence Brindisi among the Doctors; one is thus free to choose the antiphon for a Doctor or the antiphon from the respective common, according to the rubrics.

The Common of Popes and the new offices of the Queenship of the Blessed Virgin Mary (May 31), of the Assumption (August 15), and of the Immaculate Heart Blessed Virgin Mary (August 22), introduced by Pope Pius \scspace{xii} before 1954 (obligatory as of 1955), of Saint Joseph the Worker (May 1) and Vespers of Holy Saturday introduced in 1955 to be celebrated in the spring 1956, and three minor feasts introduced by Saint John \scspace{xxiii}, to be celebrated according to the new \textit{Codex Rubricarum} are included in an appendix. Note, however, that the office of the Queenship of the Blessed Virgin Mary was only published in May 1955 and does not follow the Divino Afflatu rubrics, already with simplified rubrics that were used universally from January 1956 to July 1960.

For the most part, it is possible to use the \textit{Divino Afflatu} offices for the 1960 office, which is already the case for most people who must use a vintage \textit{Liber Usualis} published before 1960. There are two unusual cases: the feast of Saint Lawrence of Brindisi was added to the general calendar on July 21 after he was named a Doctor of the Church in 1960. The feast already on the calendar is commemorated.

This aims to be a practical book to be put out as soon as possible, and we do not include any melodic restitutions, which also avoids upseting the faithful. Therefore, while we include the ancient hymns in the appendix, due to great demand and interest, which is a decision that follows the Vatican Edition, we will keep mode 3 psalms with a dominant of Do, the newer mode 6 tone, the psalm termini and mode assignments as in the Vatican Edition and in the Solesmes editions. However, we deviate slightly because custom has changed with respect to endings for which A* indicates an alternative of a podatus (4A* and 8G*): the endings will be given at the end of the first verse. This is a change even from Solesmes practice: Dom Suñol decreed that the podatus was to be sung only at the doxology.

Since the psalm ending is provided, EUOUAE is not included for most psalm antiphons. But as this is a practical edition, the \textit{Magnificat} will be printed in one block by tone as in the \textit{Liber Usualis,} and so EUOUAE will be kept in the antiphons. It is also kept in some cases where a page turn is required.

%%do we really want to do this? est-cequ'on veut vraiment faire ça ?

The rhythmic signs are kept. As the scores from the Solesmes editions are already available, little editorial intervention is needed to make the scores usable for typesetting. Further, given the number of chanters who still use even the \textit{ictus,}  it is better to abstain from making more changes and to simply reproduce the signs, ideally up to the 1960 edition of the \textit{Liber antiphonarius,} wherein most but not all \textit{redundant cadences} such as those found in mode VIII antiphons have two dotted \textit{puncta} on the penultimate and ultimate syllables before a half bar and especially a full bar. In this case, the spondaic or paryoxtyonic syllable, that is with an accent on the penult and the final weak (unaccented) syllable are both marked with a dot by Dom Gajard. Dom Mocquerau treated the question in \textit{Le nombre musicale grégorien,} vol. 2, but did not add the dot to the note of the accented syllable in his published editions. This choice conforms with the work done to date with Gregobase.

In addition, although we have a slight preference for the rhythmic signs, the notation and even the melodies as found in the monastic antiphonal, we faithfully preserve the Roman version in those places where the Solesmes monks have changed the \textit{punctum mora} to a  \textit{punctum} with an \textit{episema} or a \textit{bistropha,} with one exception: in the antiphon of I Vespers of Saint Clothilde, which has a \textit{punctum mora} on a first syllable, introducing necessarily even a slight pause between the two syllables of the word, violating the Golden Rule which forbids a cæsura as Latin syllables determine the word's meaning just as much in writing as in speech in ecclesiastical Latin. We therefore insert a \textit{bistropha} instead of the \textit{punctum mora.}

Nevertheless, we do not reproduce the preface to the Solesmes edition presenting the rhythm, in order to not print too many pages, leaving you free to add or to even remove rythmic signs in following another school of Gregorian rhythm.

We wish to facilitate chanting, and the direction of chanters by choirmasters, by providing pointed psalms according to the different tones (with bold text for accents or syllables to be treated as one, italic for preparatory syllables, following the \textit{Liber Usualis}) with the first verse of the psalm notated under every antiphon or, for the common and proper of saints and for the variable psalms of major feasts of the temporal cycle, at least once per division of the book so as to reduce repetition but in minimizing the turns needed and the number of ribbons or bookmarks.

For the psalter, antiphons are given as in the antiphonal (and \textit{Liber Usualis}) for Sunday and Saturday Vespers and at Compline; otherwise, the psalms and antiphons are as in the \textit{Liber Usualis,} with the full antiphon at the beginning of the corresponding psalm, in order to chant on double feasts which use the ferial psalms and antiphons. In the psalter, we give the first verse of psalms which have a different first verse from the antiphon but which share an intonation (e.g. at Friday Compline). This allows for the book to be used by those using the 1960 office, called the ``Extraordinary Form''; that said, one must know the rubric directing the singing of the intonation of the psalm at the second verse when the entire first verse is skipped. Proper antiphons and the psalms in the other parts of the book are given before the psalm, as is done with Matins and during the Triduum in the \textit{Liber Usualis}.

With the \textit{Liber Usualis,} the singer is forced to turn delicate pages in order to find most of the psalms of feasts. Because of choices made over a century ago, the book is very unbalanced, particularly for Sunday Vespers, which is in the first third of the book. We aim to avoid this by moving the psalter to the middle, as in postconciliar editions. This should also maximize the life of the binding. Further, the size will hew as closely as possible to that of the modern Solesmes antiphonal.

The tones of the hymns are taken from the 1912 edition as reproduced by Solesmes; the elisions are given per the \textit{Liber antiphonarius,} that is to say that, except for newer hymns first edited by Solesmes, notes of hypermetric syllables are left in, to let those chanters who sing them sing them and to let those who omit them do so as well, according to custom.

As said above, the \textit{Magnificat} will be given in full with all of the tones necessary, as in the \textit{Liber Usualis;} the antiphonal gives the first half but presumes that the chanter memorized the second half of the text and can correctly apply the tones, even though dactyls pose a problem especially in mode IV.

The common tones will also be reproduced in a dedicated section (as in the antiphonal, and therefore not in the psalter); at the back, there is are chants needed for benediction and the proper of France.

The verse numbers beginning at ``1'' are kept according to the form of the \textit{Liber Usualis} as a reference for choirmasters working with singers and organists not sufficiently fluent in Latin, particularly useful in finding their place in rehearsals or keeping track of the alternations between two choirs or cantors and the full choir while singing the office; this is a primary practical consideration, even though the numbers are not biblical, but it is to be noted that the divisions do not correspond to the biblical divisions as it is.

Finally, we include the decrees not that they are still in force or that we do so with authority, but to give the context for the decisions made in the Solesmes editions.
\end{enpars}


\bigtitle{Préface du \textit{Vesperale Romanum.}}

\begin{frpars}
Le \textit{Vesperale Romanum} est le livre contenant les chants et les textes des offices du soir, c’est-à-dire des vêpres et des complies. Après la publication de l'antiphonaire pour les heures diurnes, excluant ainsi les matines, celle-ci qu'on appelle l'édition Vaticane,, dans l'année 1911, un vespéral le suivit l'année prochaine. Pourtant, celui-ci présente des difficultés ; on emploie le psautier reformé, mais les textes, les rubriques et le calendrier sont ceux du bréviaire de saint Pie textsc{v}, avec les modifications introduites jusqu'en 1910. L'antiphonaire et le vespéral ne conforment pas alors à la \textit{editio typica} de saint Pie \textsc{x}.

Les moines de l'abbaye Saint-Pierre de Solesmes, dont les recherches sur les manuscrits médiévaux inspira dans un premier temps la commission Vaticane, publièrent leur propre édition avec les signes rythmiques préférés par dom André Mocquereau, maître de chœur de l'abbaye pendant des décennies, mais cette édition était moins populaires que les autres éditions solesmiennes, notamment le \textit{Liber Usualis} (le \textit{Paroissien romain} ou encore le \textit{800}) qui inclut les chants des dimanches, des fêtes et de la Semaine sainte mais sans ceux des autres jours en semaines.

\smalltitle{Pourquoi un nouveau \textit{vespéral}?}

Avant tout, cette édition est conçu pour remplacer les copies imprimées déjà en circulation, vu que on ne publie plus ni l'antiphonaire, ni le vespéral, que ce soit l'édition Vaticane ou celle de Solesmes. En plus, nous profitons du logiciel Gregorio, qui permet la réalistion d'une partion grégorienne sur ordinateur bien supérieur à l'impression dépendant d'un scan, car même si une copie numérique de haute qualité était à notre disposition, la version imprimée se dégrade au fur et à mesure, comme on trouve déjà dans les réimpressions des éditions solesmiennes, dans lesquelles on peine à identifier des neumes tels que le \textit{quilisma} ou des signes rythmiques, notamment le \textit{ictus} ou la \textit{épisème verticale.}

Toutefois, ce livre n'est pas simplement une réproduction à l'identique. Nous avons pour but d'améliorer une édition par rapport aux livres classiques afin de faciliter son emploi le plus possible, mais une certaine familiarité avec l'office divin rest indispensable, selon la nature d'un livre.

\smalltitle{Principes d'édition.}

Au cœur, le projet Vesperale Romanum suit l'office avant les changements de Pie \scspace{xii}, en se permettant tout de même une certain discretion. Par exemple, les noms de saint Antoine et de saint Laurent de Brindes se trouvent parmi les Docteurs de l'Église. On est ainsi libre de choisir l'antiphon pour un Docteur ou celle du commun respectif, selon les rubriques.

Dans l'appendice, nous inclusns le commun des Souverains Pontiffes, les nouveaux offices de saint Joseph, ouvrier (1\up{er} mai), de la Très Sainte Vierge Marie, Reine (31 mai); de l'Assomption (15 août), et du Cœur Immaculée de la Très Sainte Vierge Marie (22 août), introduits par Pie \scspace{xii} avant 1954 (obligatoire à partir de l'année 1955), ainsi que les vêpres du Samedi saint obligatoire dès l'année 1956 et finalement deux nouvelles fêtes inscrites au calendrier par saint Jean \scspace{xxiii}, dans le \textit{Codex Rubricarum} de 1960.

Pour l'essentiel, il est tout à fait possible d'employer l'office \textit{Divino Afflatu} pour celui de 1960, une situation qui se présente déjà pour la plupart des personnes utilisant un \textit{Liber Usualis} antérieur aux réformes de 1960.

Cette édition est surtout pratique. Nous souhaitons la faire publier dès que possible. Nous n'avons pas l'intention de faire aucune restitution mélodique, ce qui évite aussi la gêne chez les fidèles. Ainsi, bien que nous pensions à l'includion des hymni antiqui, en raison de la demande immense et l'intérêt déjà suscité chez les fidèles, en suivant l'édition Vaticane, nous voulons garder les ton psalmodique associé avec le troisième mode sur la teneur de \textit{Do,} le ton plus récent du ton 6, celui du deuxième mode qui est plus familier, selon tant l'Édition Vaticane que celle de Solesmes. Néanmoins, il nous faut dévier car les coutumes évolèrent en ce qui concerne les terminus selon lesquels A* indiquent un terminus alternatif d'un podatus  (4A* and 8G*). Nous donnons le terminus à la fin du premier verset. En effet, dom Suñol avait écrit qu'on n'était à chanter le podatus qu'\kern 0.01 emà la doxologie.

Puisque nous fournissons le terminus du psaume, nous supprimons \frquote{EUOUAE} pour la plupart des antiennes. Mais nous gardons \frquote{EUOUAE} pour l'antienne du Magnificat, car il s'agit d'une édition, pratique dans laquelle nous insérons le cantique de la Très Sainte Vierge Marie dans une section seulement. Nous gardons également \frquote{EUOUAE} quand il faut tourner la page pour retrouver l'antienne après le chant du psaume.

Nous conservons aussi les signes rythmiques. Les partitions solesmniennes étant déjà disponibles, nous avons besoin de peu d'intervention rédactionelle pour rendre les partitions utilisables afin de permettre une composition correcte. En outre, avec le nombre de chanteurs qui emploient encore même \textit{ictus,} il vaut mieux s'abstenir et reproduire les signes, dans l'idéal en utilisant ceux de \textit{Liber antiphonarius,} dans lequel la plupart, mais pas tout, des cadences redondantes comme on le trouve dans les antiennes du mode VIII, sur les syllabes pénultièmes ou ultimes, avant la demie-barre ou surtout la barre finale, où les \textit{puncta} sont munis d'un point tous les deux. Dans ce cas, dom Gajard donna aussi le point à la syllabe spondaïque ou paryoxtyonique, avec l'accent sur l'avant-dernière syllabe. En revanche, dom Mocquereau l'avait fait seulement à l'égard de la dernière dans ses éditions publiées, malgré en avoir parlé dans \textit{Le nombre musicale grégorien,} t. 2.

Voulant rendre facile le chant et la direction d'un chœur par les maîtres de chœur, on retrouvera le texte du psaume préparé pour le chant selon les divers tons (le texte en gras pour les accents, ou sinon pour les syllabes qui y sont assimilées, le texte en italique pour les syllabes de préparation, en suivant \textit{le Liber Usualis}). Le premier verset de chaque psaume est notée en-dessous de chaque antienne. ou, pour ce qui est du commun et propre du saints et du propre du temps aux fêtes majeures, au moins une fois par division du livre, pour réduire les répétitions inutiles tout en évitant le nombre de rubans nécessaires.

En ce qui concerne le psautier, on retrouvera les antiennes comme dans le \textit{le Liber Usualis} aux vêpres du dimanche et aux complies ainsi qu'aux vêpres du samedi. Sinon, on fait de même en semaine, avec l'antienne complète avant le début du psaume, pour chanter aux fêtes doubles qui emploient les antiennes et les psaumes du jour. Ce permet aussi de pouvoir chanter l'office de 1960, dite la \frquote{forme extraordinaire}. Cela dit, il faut savoir qu'il existe une rubrique selon laquelle on chante l'intonation du psaume mais seulement si on saute le premier verset car on le chante intégralement dans l'antienne. Aux dimanches et aux fêtes avec des antiennes et des psaumes propres, les psaumes et les antiennes sont comme aux matines et aux offices du Triduum dans le \textit{le Liber Usualis}, avec l'antienne, puis le psaume, sans répéter la composition de l'antienne.

Pour retrouver les psaumes aux fêtes dans le \textit{Liber Usualis,} il faut feuiller des pages délicates. En conséquence des décisions prises il y a plus d'un siècle, le livre n'est pas très équilibré, surtout aux vêpres dominicales, se trouvant au premier tiers du livre. Nous souhaitons éviter cette gêne, en déplaçant le psautier au milieu du livre, ce qu'on trouve dans les éditions post-conciliaire. Ce choix devrait aider à mieux présérver la reliure.

Les tons des hymns sont repris de l'édition Vaticane, les mêmes trouvées dans celles de Solesmes; on trouve les élisions du  \textit{Liber antiphonarius,} c’est-à-dire que la note d'une syllabe hypermétrique est laissée dans la partition, sauf pour des hymnes plus récents édités dans un premier temps par les moines de Solesmes, pour que les chanteurs qui chantent ces syllabes puissent le faire et pour permettre également aux autres de ne pas le faire selon la coutume. 

Comme nous l'avons expliqué plus haut, les Magnificats sont regroupé dans une seule division, selon les différents tons, comme dans le \textit{Liber Usualis;} l'antiphonaire donne la première partie, surtout pour le premier versets, puis pour les tons solennels, mais il faudrait mémoriser la deuxième partie, en appliquant les terminus ordinaire, en dépit de la difficulté que posent les dactyles surtout en mode IV.

Les tons communs se trouvent dans une partie spéciale, comme dans l'antiphonaire, et non pas dans le psautier comme dans  le \textit{Liber Usualis;} vers le fond du livre, on trouvera les chants du Salut du Très Saint Sacrement ainsi que le propre de France.

Les versets sont numérotés à partir de \frquote{1} selon la forme du \textit{Liber Usualis} afin d'aider les maîtres de chœur qui travaillent avec des chanteurs ou des organistes moins expérimentés, dont le niveau en latin n'est pas toujours suffisant. La numérotation est très utile aux répétitions et aux offices, pour diriger les chanteurs et pour suivre l'alternance entre les deux chœurs ou autres divisions des chanteurs. Il s'agit d'une considération pratique primaire. Force est de constater que les divisions des psaumes ne correspondent pas aux divisions bibliques ordinaires non plus.
\end{frpars}

%%I believe this should be blank and the other stuff begins on the right page (and with lowercase roman numerals?)

%%we need to consider any other prefatory materials…

\newpage

\section*{Decreta S.R.C.}
\addcontentsline{toc}{section}{Decreta S.R.C.}
\phantomsection

\thispagestyle{empty}
Decreta, quibus non derogatur per typicam Antiphonalis Romani editionem, hic addere in gratiam Le\-ctoris Editoribus visum est.

\myrule

{\centering\capspace{DECRETUM}\par} %%to replace with a better title command, but for now this will work

{\centering{\rubrique{seu declaratio super editione Vaticana ejusque reprodu\-ctione quoad libros liturgicos gregorianos.}}}

Cum postulatum fuerit, an Episcopi possint propriam approbationem donare libris cantus gregoriani, melodias Vaticanæ editionis adamussim reprodu\-ctas continentibus, sed cum signorum rhythmicorum indicatione, privata au\-ctoritate additorum?

Sacra Rituum Congregatio, ad majorem declarationem Decreti n. 4259, 25 Januarii vertentis anni, respondendum censuit:
Editionibus in subsidium scholarum cantorum, signis rhythmicis, uti vacant, privata au\-ctoritate ornatis, poterunt Ordinarii, in sua quisque Dioecesi, apponere \textit{Imprimatur,} dum modo constet, cetera, quæ in Decretis Sacræ Rituum Congregationis injun\-cta sunt, quoad cantus gregoriani restaurationem, fuisse servata.

Quam resolutionem San\-ctissimo Domino nostro Pio Papæ \textsc{x}, per Sacrorum Rituum Congregationis Secretarium relatam, San\-ctitas Sua ratam habuit et probavit.

Die 11 Aprilis 1911. {\hfill(n. 4263. Vol. VI., p. 114. Decret. authent. S.R.C.)}
\vspace{5ex}%% to replace later

{\centering\capspace{DECRETUM}\par} %%to replace with a better title command, but for now this will work

{\centering{\rubrique{circa modulandas monosyllabas vel hebraicas voces in le\-ctionibus, versiculis et psalmis.}\par}}

A quibusdam cantus gregoriani magistris Sacræ Rituum Congregationi sequens dubium pro opportuna solutione expositum fuit; nimirum:

An in cantandis Le\-ctionibus et Versiculis, præsertim vero in Psalmorum mediantibus ad asteriscum, quando vel di\-ctio monosyllaba vel hebraica vox occurrit, immutari possit clausula, vel cantilena proferri sub modulatione consueta?

Et Sacra eadem Congregatio, approbante San\-ctissimo Domino nostro Pio Papa \textsc{x}, rescribere statuit: \textit{Affimative ad utrumque.}

Die 8 Julii 1912.

\hfill{Fr. S. \scspace{Card. Martinelli,} \textit{S.R.C. Præfi\-ctis.}}

L. † S.

{\hfill † Petrus La Fontaine, Episc. Charystien., \textit{Secretarius.}}

(A\-cta Apost. Sedis, Vol. IV., p. 539)
\vspace{5ex}%% to replace later

{\centering\capspace{DECRETUM}\par} %%to replace with a better title command, but for now this will work

{\centering{\rubrique{circa syllabas hypermetricas in cantu hymnorum.}\par}}

\textit{Dubium de syllabis hypermetrici quoad cantum.} Sacra Rituum Congregatione pluries expostulatum fuit: « An regula descripta in Antiphonario Vaticano circa syllabas hypermetricas, quæ frequenter occurrunt in cantu hymnorum, scilicet quod ipsæ non elidantur, sed distin\-ctæ pronuncientur propriaque nota cantentur, stri\-cte et rigorose interpretanda sit, vel e contra liceat etiam i\-psas syllabas elidere, præsertim si in praxi id facilius et convenientius censeatur »?

Et Sacra eadem Congregatio, audita specialis Commissionis pro cantu liturgico gregoriano sententia, propositæ quæstioni, re sedulo perpensa ita rescribendum censuit : «Negative ad primam partem, affirmative ad secundam ». Atque ita rescripsit et declaravit die 14 Maii 1915.

\hfill{A.  \scspace{Card. Vico,} \textit{Pro-Præfe\-ctus.}}

{\hfill† Petrus La Fontaine, Patriarcha ele\-ctus Venetiarum, \textit{Secretarius.}}

(Acta Apost. Sedis, 1915, fasc. 9, p. 237).

\newpage

\raggedbottom

\setlength{\defaultparskip}{\parskip}
\setlength{\parskip}{0.5\baselineskip plus 2pt}

\begingroup
\setlength{\parindent}{0pt}

%\setlength{\parfillskip}{30\p@ \@plus 1fil}
{\centering{\large \capspace{KALENDARIUM PERPETUUM}.}\par} \thispagestyle{empty} \markboth{Calendrier}{Calendrier} \nopagebreak \par \nopagebreak\vspace{5mm}\label{kalendarium} %%will need to fix the label later
\setlength\LTleft{0pt}
\setlength\LTright{0pt}
\setlength{\tabcolsep}{5pt}
\renewcommand{\arraystretch}{1.4}
\fontsize{8}{9}\selectfont

% !TEX TS-program = LuaLaTeX+se
% !TEX root = Kalendarium.tex

{\centering{\normalsize Januarius.}\par}

\begin{longtable}{>{\centering}p{0.025\textwidth}|>{\raggedright}p{0.040\textwidth}|>{\raggedleft}p{0.025\textwidth}|>{\raggedright\arraybackslash}p{0.80\textwidth}}
%\begin{longtable}{>{\centering}p{0.025\textwidth}|>{\raggedleft}p{0.025\textwidth}|>{\raggedright\arraybackslash}p{0.85\textwidth}}
A &Kal. & 1 & \hang \scspace{Circumcisio Domini} et Octava Nativitatis. \textit{Duplex II classis.}\\
 &  &  &  \hang \textit{Dominica inter Circumcisionem et Epiphaniam.}  \scspace{Ss}. \scspace{Nominis Jesu}.  \textit{Duplex II classis.}\\
b &iv & 2 & \hang Octava S. Stephani Protomartyr. \textit{Simplex.}\\
c &iij & 3 & \hang Octava S. Joannis Apost. et Evang. \textit{Simplex.}\\
d &Prid. & 4 & \hang Octava SS. Innocentum Martyrum. \textit{Simplex.}\\
e &Non. & 5 & \hang Vigiliæ Epiphianiæ.  \textit{Semiduplex.} \mem{S. Telesphori Papæ Martyris.}\\
f &viij & 6 & \hang \capspace{EPIPHANIA DOMINI}. \textit{Duplex I classis cum Octava privilegiata II ordinis.}\\
 &  &  & \hang \textit{Dominica infra Octavam Epiphianiæ.} S. Familiæ Jesu, Mariæ, Joseph.  \textit{Duplex majus.} \textit{Commem.} Dominicæ et Octavæ.\\
g & vij & 7 & \hang De Octava Epiphianæ. \textit{Semiduplex.}\\
A & vj & 8 & \hang De Octava. \textit{Semiduplex.}\\
b & v & 9 &  \hang De Octava. \textit{Semiduplex.}\\
c & iv & 10 &  \hang De Octava. \textit{Semiduplex.}\\
d & iij & 11 & \hang De Octava. \textit{Semiduplex.} \mem{S. Hygini Papæ Martyris.}\\
e & Prid. & 12 &  \hang De Octava.  \textit{Semiduplex.}\\
f & Idib. & 13 & \hang Octava Epiphianiæ. \textit{Duplex majus.}\\
g & xix & 14 & \hang S. Hilarii Episc. Conf. et Eccl. Doct. \textit{Duplex.} \mem{S. Felicis Presbyteri Martyris.}\\
A & xviij & 15 & \hang S. Pauli primi Eremitæ Conf. \textit{Duplex. \mem{S. Mauri Abb.}}\\
b & xvij & 16 & \hang S. Marcelli I Papæ Mart. \textit{Semiduplex.}\\
c & xvj & 17 & \hang S. Antonii Abbatis. \textit{Duplex.}\\
d & xv & 18 & \hang Cathedra S. Petri Romæ. \textit{Duplex majus.} \mem{S. Pauli Ap., ac S. Priscæ Virginis et Martyris.}\\
e & xiv & 19 &  \hang SS. Marii, Marthæ, Audifacis et Abachum Martyrum. \textit{Simplex.}\\
f & xiij & 20 & \hang SS. Fabiani Papæ et Sebastiani Martyr. \textit{Duplex.}\\
g & xij & 21 & \hang S. Agnetis Virginis et Martyris. \textit{Duplex.}\\
A & xj & 22 & \hang SS. Vincentii et Anastasii Martyrum. \textit{Semiduplex.}\\
b & x & 23 & \hang S. Raymundi de Peñafort Conf. \textit{Semiduplex.} \mem{S. Emerentianæ Virginis et Martyris.} \\
c & ix & 24 & \hang S. Timothei Episc. Martyris. \textit{Duplex.}\\
d & viij & 25 & \hang Conversio S. Pauli Ap. \textit{Duplex majus.} \mem{S. Petri Ap.}\\
e & vij & 26 & \hang S. Polycarpi Episc. Martyris. \textit{Duplex.}\\
f & vj & 27 & \hang S. Joannis Chrysostomi Episc. Conf. et Eccl. Doct. \textit{Duplex.}\\
g & v & 28 & \hang S. Petri Nolasci Conf. \textit{Duplex.}\\
A & iv & 29 & S. Francisci Salesii Episc. Conf. et Eccl. Doct. \textit{Duplex.}\\
b & iij & 30 & S. Martinæ Virginis et Martyris.  \textit{Semiduplex.}\\
c & Prid. & 31 & \hang S. Joannis Bosco Conf. \textit{Duplex.}
\end{longtable}
%
% !TEX TS-program = LuaLaTeX+se
% !TEX root = Kalendarium.tex

{\centering{{\normalsize Februarius.}\par}}
\begin{longtable}{>{\centering}p{0.025\textwidth}|>{\raggedright}p{0.040\textwidth}|>{\raggedleft}p{0.025\textwidth}|>{\raggedright\arraybackslash}p{0.80\textwidth}}

d & Kal. & 1 & S. Ignatii Episc. Martyr. \textit{Duplex.}\\
e & iv & 2 & \hang \scspace{Purificatio B}. \scspace{Mariæ Virginis}. \textit{Duplex II classis.}\\
f & iij & 3 & \hang S. Blasii Episc. Martyr. \textit{Simplex.}\\
g & Prid. & 4 & \hang S. Andreæ Corsini Episc. Conf. \textit{Duplex.}\\
A & Non. & 5 & \hang S. Agathæ Virginis et Martyris. \textit{Duplex.}\\
b & viij & 6 & \hang S. Titi Episc. Conf. \textit{Duplex.} \mem{S. Dorotheæ Virginis et Martyris.} \\
c & vij & 7 & \hang S. Romualdi Abbatis. \textit{Duplex.}\\
d & vj & 8 & \hang S. Joannis de Matha Conf. \textit{Duplex.}\\
e & v & 9 & S. Cyrilli Episc. Alexandrini Conf. et Eccl. Doct. \textit{Duplex.}\\
f & iv & 10 & \hang S. Scholasticæ Virginis. \textit{Duplex.}\\
g & iij & 11 & \hang Apparitionis B.M.V. Immaculatæ. \textit{Duplex majus.}\\
A & Prid. & 12 & SS. Septem Fundatorum Ord. Servorum B.V.M., Cc. \textit{Duplex.}\\
b & Ibid. & 13 & \\
c & xvj & 14 & \hang S. Valentini Presbyteri Martyris. \textit{Simplex.}\\
d & xv & 15 & SS. Faustini et Jovitæ Martyrum. \textit{Simplex.}\\
e & xiv & 16 & \\
f & xiij & 17 & \\
g & xij & 18 & S. Simeonis Episcopi Martyris. \textit{Simplex.}\\
A & xj & 19 & \\
b & x & 20 & \\
c & ix & 21 & \\
d & viij & 22 & \hang Cathedra S. Petri Antiochæ. \textit{Dupl. maj.} \mem{S. Pauli Ap.}\\
e & vij & 23 & \hang S. Petri Damiani Episc. Conf. et Eccl. Doct. \textit{Duplex.} \mem{Vigiliæ.}\\
f & vj & 24 & \scspace{S}. \scspace{Matthiæ Apostoli}. \textit{Duplex II classis.}\\
g & v & 25 & \\
A & iv & 26 & \\
b & iij & 27 & \hang S. Gabrielis a Virgine Perdolente Conf. \textit{Duplex.}\\
c & Prid. & 28 &
%%& & & \rubrique{In anno bisextili mensis Febrarius est dierum 29, et Festum S. Matthiæ celebratur die 25 Februarii, ac Festum S. Gabrielis a Virg. Perdolente die 28 Februarii, et bis dicitur Sexto Kalendas, id est die 24 et die 25; et littera Dominicalis, quæ assumpta fuit in mense Januario, mutatur in præcedentem ut, si in Januario littera Dominicalis fuerit A, mutetur in præcedentum, quæ est \normaltext{g,} etc., littera  \normaltext{f} bis servit 24 et 25.}
\end{longtable}
%%

\textes{In_anno_bisextili}%% contains ad hoc vspace command which could be removed, modified, or replaced with a custom spacing command to avoid magic numbers. rubrique macro is defined in commonheaders file
%%%% In the current configuration this is commented out and the text is inserted in the calendar table itself for reasons of space. Result TBD as the table of moveable feasts needs serious work and may have to cover two pages.
%%
% !TEX TS-program = LuaLaTeX+se
% !TEX root = Kalendarium.tex

{\centering{{\normalsize Martius.}\par}}
%
\begin{longtable}{>{\centering}p{0.025\textwidth}|>{\raggedright}p{0.040\textwidth}|>{\raggedleft}p{0.025\textwidth}|>{\raggedright\arraybackslash}p{0.80\textwidth}}
d & Kal. & 1 & \\
e & vj & 2 & \\
f & v & 3 & \\
g & iv & 4 & \hang S. Casimiri Conf. \textit{Semid.} \mem{S. Lucii I Papæ Mart.}\\
A & iij & 5 & \\
b & Prid. & 6 & SS. Perpetuæ et Felicitatis Martyrum. \textit{Duplex.}\\
c & Non. & 7 & \hang S. Thomæ de Aquino Conf. et Eccl. Doct. \textit{Duplex.}\\
d & viij & 8 & \hang S. Joannis de Deo Conf. \textit{Duplex.}\\
e & vij & 9 & \hang S. Franciscæ Romanæ, Viduæ. \textit{Duplex.}\\
f & vj & 10 & SS. Quadraginta Martyrum. \textit{Semiduplex.}\\
g & v & 11 & \\
A & iv & 12 & S. Gregorii Papæ, Conf. et Eccl. Doctoris. \textit{Duplex.}\\
b & iij & 13 & \\
c & Prid. & 14 & \\
d & Idib. & 15 & \\
e & xvij & 16 & \\
f & xvj & 17 & \hang S. Patricii Episc. Conf. \textit{Duplex.}\\
g & xv & 18 & \hang S. Cyrilli Ep. Hierosolymitani, Conf. et Eccl. Doct. \textit{Duplex.}\\
A & xiv & 19 & \hang \capspace{S. JOSEPH} Sponsi B.M.V., Conf. \textit{Duplex I classis.}\\
b & xiij & 20 & \\
c & xij & 21 & S. Benedicti Abbatis. \textit{Duplex majus.}\\
d & xj & 22 & \\
e & x & 23 & \\
f & ix & 24 & S. Gabrielis Archangeli. \textit{Duplex.}\\
g & viij & 25 & \hang \capspace{ANNUNTIATIO B}. \capspace{MARIÆ VIRGINIS}. \textit{Duplex I classis.}\\
A & vij & 26 & \\
b & vj & 27 & S. Joannis Damasceni Conf. et Ecclesiæ Doctoris. \textit{Duplex.}\\
c & v & 28 & S. Joannis a Capistrano Conf. \textit{Semiduplex.}\\
d & iv & 29 & \\
e & iij & 30 & \\
f & Prid. & 31 & \\
 &  &  & \hang \textit{Feria VI post Dom. Passionis.} Septum Dolorum B. Mariæ Virginis. \textit{Duplex majus.} \mem{Feriæ.}
\end{longtable}

% !TEX TS-program = LuaLaTeX+se
% !TEX root = Kalendarium.tex

{\centering{{\normalsize Aprilis.}\par}}

\begin{longtable}{>{\centering}p{0.025\textwidth}|>{\raggedright}p{0.040\textwidth}|>{\raggedleft}p{0.025\textwidth}|>{\raggedright\arraybackslash}p{0.80\textwidth}}
g & Kal. & 1 & \\
A & iv & 2 & \hang S. Francisci de Paula Conf. \textit{Duplex.}\\
b & iij. & 3 & \\
c & Prid. & 4 & \hang S. Isidori Episc. Conf. et Eccl. Doctoris. \textit{Duplex.}\\
d & Non. & 5 & \hang S. Vincentii Ferrerii Conf. \textit{Duplex.}\\
e & viij & 6 & \\
f & vij & 7 & \\
g & vj & 8 & \\
A & v & 9 & \\
b & iv & 10 & \\
c & iij &11 & \hang S. Leonis I Papæ, Conf. et Eccl. Doctoris. \textit{Duplex.}\\
d & Prid. & 12 & \\
e & Ibid. & 13 & \hang S. Hermenegildi Martyris. \textit{Semiduplex.}\\
f & xviij & 14 & \hang S. Justini Mart. \textit{Duplex.} \mem{SS. Tiburtii, Valeriani et Maximi Martyrum.} \\
g & xvij & 15 & \\
A & xvj & 16 & \\
b & xv & 17 & \hang S. Aniceti I, Papæ Martyris. \textit{Simplex.}\\
c & xiv & 18 & \\
d & xiij & 19 & \\
e & xij & 20 & \\
f & xj & 21 & \hang S. Anselmi Episc. Conf. et Eccl. Doctoris. \textit{Duplex.}\\
g & x & 22 & \hang SS. Soteris et Caii Pontif. Mart. \textit{Semiduplex.}\\
A & ix & 23 & \hang S. Georgii Martyris. \textit{Semiduplex.}\\
b & viij & 24 & \hang S. Fidelis a Sigmaringa Martyris.\textit{Duplex.}\\
c & vij & 25 & \hang S. \scspace{Marci Evangelistæ}. \textit{Duplex II classis.}\\
d & vj & 26 & \hang SS. Cleti et Marcellini Pontif. Martyrum. \textit{Semiduplex.}\\
e & v & 27 & \hang S. Petri Canisii Conf. et Eccl. Doct. \textit{Duplex.}\\
f & iv & 28 & \hang S. Pauli a Cruce Conf. \textit{Dupl.}\\
g & iij & 29 & \hang S. Petri Martyris. \textit{Duplex.}\\
A & Prid. & 30 & \hang S. Catharinæ Senensis Virginis. \textit{Duplex.}\\
 &  &  & \hang \textit{Feria IV infra Hebdomadam II post octavam Paschæ.} \capspace{SOLEMNITAS S}. \capspace{JOSEPH}, Sponsi B.M.V., Conf. et Eccl. univers. Patroni. \textit{Duplex I classis cum Octava communi.}\\
 &  &  & \hang \textit{Feria IV infra Hebdomadam III post octavam Paschæ.} Octava S. Joseph. \textit{Duplex majus.}
\end{longtable}

% !TEX root = Kalendarium.tex
% !TEX TS-program = LuaLaTeX+se

{\centering{{\normalsize Maius.}\par}}

\begin{longtable}{>{\centering}p{0.025\textwidth}|>{\raggedright}p{0.040\textwidth}|>{\raggedleft}p{0.025\textwidth}|>{\raggedright\arraybackslash}p{0.80\textwidth}}
b & Kal. & 1 & \hang \scspace{Ss}. \scspace{Philippi et Jacobi Apostolorum}. \textit{Duplex II classis.}\\
c & vj & 2 & \hang S. Athanasii Episc. Conf. et Eccl. Doctoris. \textit{Duplex.}\\
d & v & 3 & \hang \scspace{Inventio S}. \scspace{Crucis}. \textit{Dupl. II classis.} \mem{SS. Alexandri I Papæ et Soc. Martyrum, ac S. Juvenalis Episc. Conf.}\\
e & iv & 4 & \hang S. Monicæ Viduæ. \textit{Duplex.}\\
f & iij & 5 & S. Pii V Papæ Conf. \textit{Duplex.}\\
g & Prid. & 6 & \hang S. Joannis Ap. ante Portam Latinam. \textit{Duplex majus.}\\
A & Non. & 7 & S. Stanislai Episc. Martyris. \textit{Duplex.}\\
b & viij & 8 & \hang Apparitio S. Michaelis Archangeli. \textit{Duplex majus.}\\
c & vij & 9 & \hang S. Grgeorii Nazianzni Episc. Conf. et Eccl. Doct. \textit{Duplex.}\\
d & vj & 10 & \hang S. Antonini Episc. Conf. \textit{Duplex.}\\
e & v & 11 & \\
f & iv & 12 & \hang SS. Nerei, Achillei et Domitillæ Virginis, atque Pancratii Martyrum. \textit{Semiduplex.}\\
g & iij & 13 & \hang S. Roberti Bellarmino Episc. Conf. et Eccl. Doct. \textit{Duplex.}\\
A & Prid. & 14 & \hang S. Bonifatii Martyris. \textit{Simplex.}\\
b & Idib. & 15 & \hang S. Joannis Baptistæ de la Salle Conf. \textit{Duplex.}\\
c & xvij & 16 & S. Ubaldi Episc. Conf. \textit{Semiduplex.}\\
d & xvj & 17 & S. Paschalis Baylon Conf. \textit{Duplex.}\\
e & xv & 18 & \hang S. Venantii Martyris. \textit{Duplex.}\\
f & xiv & 19 & \hang S. Petri Cælestini Papæ Conf. \textit{Dupl.} \mem{S. Pudentianæ Virginis.}\\
g & xiij & 20 & \hang S. Bernardini Senensis Conf.\textit{Semiduplex.}\\
A & xij & 21 & \\
b & xj & 22 &  \\
c & x & 23 & \\
d & ix & 24 &\\
e & viij & 25 & \hang S. Gregorii VII Papæ Conf. \textit{Duplex.} \mem{S. Urbani I Papæ Martyris.}\\
f & vij & 26 & \hang S. Philippi Nerii Conf. \textit{Duplex.}\\
g & vj & 27 & \hang S. Bedæ Venerabilis Conf. et Eccl. Doct. \textit{Duplex.} \mem{S. Joannis I Papæ Martyris.}\\
A & v & 28 & S. Augustini Episc. Conf. \textit{Duplex.}\\
b & iv & 29 & \hang S. Mariæ Magdalenæ de Pazzis Virg. \textit{Semiduplex.}\\
c & iij & 30 & \hang S. Felicis I Papæ Martyris. \textit{Simplex.}\\
d & Prid. & 31 & \hang S. Angelæ Mericiæ Virg. \textit{Duplex.} \mem{S. Petronillæ Virg.}\\ 
 &  &  & \hang \textit{Vel.} \scspace{B}. \scspace{Mariæ Virginis Reginæ}.  \textit{Duplex II classis.}  \mem{S. Petronillæ Virg.}
\end{longtable}

% !TEX root = Kalendarium.tex
% !TEX TS-program = LuaLaTeX+se

{\centering{{\normalsize Junius.}\par}}

\begin{longtable}{>{\centering}p{0.025\textwidth}|>{\raggedright}p{0.040\textwidth}|>{\raggedleft}p{0.025\textwidth}|>{\raggedright\arraybackslash}p{0.80\textwidth}}
e & Kal. & 1 & \\
 &  &  & \hang \textit{Vel.} S. Angelæ Mericiæ Virg. \textit{Duplex.}\\
f & iv & 2 & SS. Marcellini, Petri atque Erasmi Martyrum. \hang \textit{Simplex.}\\
g & iij & 3 & \hang \\
A & Prid. & 4 & \hang S. Francisci Caracciolo Conf. \textit{Duplex.}\\
b & Non. & 5 & \hang S. Bonifatii Episc. Martyris. \textit{Duplex.}\\
c & viij & 6 & \hang S. Norberti Episc. Conf. \textit{Duplex.}\\
d & vij & 7 & \\
e & vj & 8 & \\
f & v & 9 & \hang SS. Primi et Feliciani Martyrum. \textit{Simplex.}\\
g & iv & 10 & S. Margaritæ Reginæ, Viduæ. \textit{Semiduplex.}\\
A & iij & 11 & \hang S. Barnabæ Apostoli. \textit{Duplex majus.}\\
b & Prid. & 12 & \hang S. Joannis a S. Facundo Conf.  \textit{Dupl.} \mem{Comm. SS. Basilidis, Cyrini, Naboris et Nazarii Martyrum.}\\
c & Idib. & 13 & \hang S. Antonii de Padua Conf. (et Eccl. Doct.) \textit{Duplex.}\\ \noalign{\penalty-5000} %%this marks pagebreak preferred location
d & xviij & 14 & S. Basilii Magni Episc. Conf. et Eccl. Doct. \textit{Duplex.}\\
e & xvij & 15 & SS. Viti, Modesti atque Crescentiæ Martyrum. \textit{Simplex.}\\
f & xvj & 16 & \\
g & xv & 17 & \\
A & xiv & 18 & S. Ephraem Syri Diac., Conf. et Eccl. Doct. \textit{Duplex.}\\
b & xiij & 19 & \hang S. Julianæ de Falconeriis Virginis. \textit{Dupl.} \mem{SS. Gervasii et Protasii Martyrum.}\\
c & xij & 20 & S. Silverii Papæ Martyris. \textit{Simplex.}\\
d & xj & 21 & \hang S. Aloisii Gonzagæ Conf. \textit{Duplex.}\\
e & x & 22 & \hang S. Paulini Episc. Conf. \textit{Duplex.}\\
f & ix & 23 & Vigilia.\\
g & viij & 24 & \hang \capspace{NATIVITAS S}. \capspace{JOANNIS BAPTISTÆ}. \textit{Duplex I classis cum Octava communi.}\\
A & vij & 25 & S. Gulielmi Abbatis. \textit{Duplex.} \mem{Octavæ.}\\
b & vj & 26 & SS. Joannis et Pauli Martyrum. \mem{Octavæ.}\\
c & v & 27 & \hang De Octava. \textit{Semiduplex.}\\
d & iv & 28 & \hang S. Irenæi Episc. et Martyr. \textit{Duplex.} \mem{Octavæ et Vigiliæ.}\\
e & iij & 29 & \hang \capspace{SS}. \capspace{PETRI ET PAULI APOSTOLORUM}. \textit{Duplex I classis cum Octava communi.}\\
f & Prid. & 30 & \hang Commemoratio S. Pauli Apostoli. \textit{Duplex majus.} \mem{S. Petri Apostoli et Octavæ S. Joannis Baptistæ.}
\end{longtable}

%%\pagebreak 

% !TEX TS-program = LuaLaTeX+se
% !TEX root = Kalendarium.tex

{\centering{{\normalsize Julius.}\par}}

\begin{longtable}{>{\centering}p{0.025\textwidth}|>{\raggedright}p{0.040\textwidth}|>{\raggedleft}p{0.025\textwidth}|>{\raggedright\arraybackslash}p{0.80\textwidth}}
g & Kal. & 1 & \hang \capspace{PRETIOSISSIMI SANGUINIS} D.N.J.C. \textit{Duplex I classis.} \mem{diei Octavæ S.~Joannis Baptistæ.}\\  %%~ needed with current settings
A & vj. & 2 & \scspace{Visitatio} B.M.V. \textit{Duplex II classis.} \mem{SS. Processi et Martiniani Martyrum} \\
b & v & 3 & \hang S. Leonis II Papæ et Conf. \textit{Semiduplex.} \mem{Octavæ.}\\
c & iv & 4 & \hang De Octava. \textit{Semiduplex.}\\
d & iij & 5 & \hang S. Antonii Mariæ Zaccaria Conf. \textit{Duplex.}\\
e & Prid. & 6 & \hang Octava SS. Petri et Pauli Apostolorum. \textit{Duplex majus.}\\
f & Non. & 7 & \hang SS. Cyrilli et Methodii Episc. Conf. \textit{Duplex.}\\
g & viij & 8 & \hang S. Elisabeth Reginæ, Viduæ. \textit{Semiduplex.}\\
A & vij & 9 & \\
b & vj & 10 &  \hang SS. Septem Fratrum Martyrum et SS. Rufinæ et Secundæ Virgimum et Martyrum. \textit{Semiduplex.}\\
c & v & 11 & \hang S. Pii I Papæ Martyris. \textit{Simplex.}\\
d & iv & 12 & S. Joannis Gualberti Abbatis. \textit{Duplex.} \mem{SS. Naboris et Felicis Martyrum.}\\
e & iij & 13 & \hang S. Anacleti Papæ Mart. \textit{Semiduplex.}\\
f & Prid. & 14 & \hang  S. Bonaventuræ Eepisc. Conf. et Eccl. Doct. \textit{Duplex.}\\
g & Idib. & 15 & \hang S. Henrici Imperatoris, Conf. \textit{Semiduplex.}\\
A & xvij &16 & \hang Commemoratio B. Mariæ Virginis de Monte Carmelo. \textit{Duplex majus.}\\
b & xvj & 17 &  \hang S. Alexii Conf. \textit{Semiduplex.}\\ \noalign{\penalty-5000} %%this marks pagebreak preferred location
c & xv & 18 & \hang S. Camilli de Lellis Conf. \textit{Dupl.} \mem{SS. Symphorosæ et septem Filiorum ejus Martyrum.}\\
d & xiv & 19 & \hang S. Vincentii a Paulo Conf. \textit{Duplex.}\\
e & xiij & 20 & \hang S. Hieronymi Æmiliani Conf. \textit{Dupl.} \mem{S. Margaritæ Virg. et Mart.}\\
f & xij & 21 & \hang S. Praxedis Virginis. \textit{Simplex.}\\
g & xj & 22 & \hang S. Mariæ Magdalenæ Pœnitentis. \textit{Duplex.}\\
A & x & 23 & \hang S. Apollinaris Episc. Mart. \textit{Dupl.} \mem{Comm. S. Liborii Ep. Conf.}\\
b & ix & 24 & \hang  Vigilia. \mem{S. Christinæ Virginis et Martyris.}\\
c & viij & 25 & \hang S. \scspace{Jacobi Apostoli}. \textit{Duplex II classis.} \mem{S. Christophori Martyr.}\\
d & vij & 26 & \hang S. \scspace{Annæ Matris} B.M.V. \textit{Duplex II classis.}\\
e & vj & 27 & \hang S. Pantaleonis Martyris. \textit{Simplex.}\\
f & v & 28 &  \hang SS. Nazarii et Celis Martyrum, Victoris I Papæ Mart. ac Innocentii I Papæ Conf. \textit{Semiduplex.}\\
g & iv & 29 & \hang S. Marthæ Virg. \textit{Semid.} \mem{SS. Felicis II Papæ, Simplicii, Faustini et Beatricis Martyrum.}\\
A & iij & 30 & \hang SS. Abdon et Sennen Martyrum. \textit{Simplex.}\\
b & Prid. & 31 & \hang S. Ignatii Conf. \textit{Duplex majus.}
\end{longtable}

% !TEX TS-program = LuaLaTeX+se
% !TEX root = Kalendarium.tex

{\centering{{\normalsize Augustus.}\par}}

\begin{longtable}{>{\centering}p{0.025\textwidth}|>{\raggedright}p{0.040\textwidth}|>{\raggedleft}p{0.025\textwidth}|>{\raggedright\arraybackslash}p{0.80\textwidth}}
c & Kal. & 1 & \hang S. Petri ad Vincula. \textit{Duplex majus.} \mem{S. Pauli Ap. ac SS. Machabæorum Martyrum.}\\
d & iv & 2 & \hang S. Alpphonsi Mariæ de Ligorio Episc. Conf. et Eccl. Doct. \textit{Duplex.} \mem{S. Stephani I Papæ Martyris.}\\
e & iij & 3 & \hang Inventio S. Stephani Protomartyris. \textit{Semiduplex.}\\
f & Prid. & 4 & \hang S. Dominici Conf. \textit{Duplex majus.}\\
g & Non. & 5 & \hang Dedicatio S. Mariæ ad Nives. \textit{Duplex majus.}\\
A & viij & 6 & \hang \scspace{Transfiguratio} D.N.J.C. \textit{Duplex II classis.} \mem{SS. Xysti II Papæ, Felicissimi et Agapiti Martyrum.}\\
b & vij & 7 & \hang S. Cajetani Conf. \textit{Duplex.} \mem{S. Donatii Episc. Mart.}\\
c & vj & 8 & \hang SS. Cyriaci, Largi et Smaragdi Martyrum. \textit{Semiduplex.}\\
d & v & 9 & \hang S. Joannis Mariæ Vianney Conf. \textit{Duplex.} \mem{Vigiliæ et S. Romani Martyris.}\\
e & iv & 10 & \hang S. \scspace{Laurentii Martyris}. \textit{Duplex II classis cum Octava simplici.}\\
f & iij & 11 & \hang SS. Tiburtii et Susannæ Virg., Martyrum. \textit{Simplex.}\\
 g & Prid. & 12 & \hang S. Claræ Virginis. \textit{Duplex.}\\
A & Ibid. & 13 & \hang SS. Hippolyti et Cassiani Martyrum. \textit{Simplex.}\\
b & xix & 14 & \hang Vigilia. \mem{S. Eusebii Conf.}\\
c & xviij & 15 & \hang \capspace{ASSUMPTIO B}. \capspace{MARIÆ VIRGINIS}. \textit{Duplex I classis cum Octava communi.}\\
d & xvij & 16 & \hang S. \scspace{Joachim Patris} B.M.V. \textit{Duplex II classis.}\\
e & xvj & 17 & \hang S. Hyacinthi Conf. \textit{Duplex.} \mem{Octavæ Assumptionis ac diei Octavæ S. Laurentii Mart.} \\
f & xv & 18 & \hang De Octava Assumptionis. \textit{Semiduplex.} \mem{S. Agapiti Mart.}\\
g & xiv & 19 & \hang S. Joannis Eudes. \textit{Duplex.} \mem{Octavæ.}\\
A & xiij & 20 & \hang S. Bernardi Abbatis Conf. et Eccl. Doct. \textit{Duplex.} \mem{Octavæ.}\\
b & xij & 21 & \hang S. Joannæ Franciscæ Fremiot de Chantal Viduæ. \textit{Duplex.} \mem{Octavæ.}\\
c & xj & 22 & \hang Octava Assumptionis B.M.V. \textit{Duplex majus.} \mem{SS Timothei, Hippolyti et Symphoriani Martyrum.}\\
& & & \hang \textit{Vel.} \scspace{Immaculati Cordis} B.M.V. \textit{Duplex II classis.}  \mem{SS Timothei, Hippolyti et Symphoriani Martyrum.}\\
d & x & 23 & \hang S. Philippi Benitii Conf. \textit{Duplex.} \mem{Vigiliæ.}\\
e & ix & 24 & \hang S. \scspace{Bartholomæi Apostoli}. \textit{Duplex II classis.}\\
f & vijj & 25 & \hang S. Ludovici Regis, Conf. \textit{Semiduplex.}\\
g & vij & 26 &  \hang S. Zephyrini Papæ Martyris. \textit{Simplex.}\\
A & vj & 27 & \hang S. Josephi Calasanctii Conf. \textit{Duplex.}\\
b & v & 28 & \hang S. Augustini Episc. Conf. et Eccl. Doct. \textit{Dupl.} \mem{S. Hermetis Martyris.}\\
c & iv & 29 & \hang Decollatio S. Joannis Baptistæ. \textit{Duplex majus.} \mem{S. Sabinæ Martyris.}\\
d & iij & 30 &  \hang S. Rosæ Limanæ Virginis. \textit{Duplex.} \mem{SS. Felicis et Adaucti Martyrum.}\\
e & Prid. & 31 &  \hang S. Raymundi Nonnati Conf. \textit{Duplex.}\\
\end{longtable}


% !TEX TS-program = LuaLaTeX+se
% !TEX root = Kalendarium.tex

{\centering{{\normalsize September.}\par}}

\begin{longtable}{>{\centering}p{0.025\textwidth}|>{\raggedright}p{0.040\textwidth}|>{\raggedleft}p{0.025\textwidth}|>{\raggedright\arraybackslash}p{0.80\textwidth}}
 f & Kal. & 1 & S. Ægidii Abbatis. \textit{Simplex.} \mem{SS. Duodecim Fratrum Martyrum.}\\
g & iv & 2 & \hang S. Stephani Hungariæ Regis, Conf. \textit{Semiduplex.}\\
A & iij & 3 & \\
& & & \hang \textit{Vel.} S. Pii X Papæ et Conf. \textit{Duplex.}\\
b & Prid. & 4 & \\
c & Non. & 5 & S. Laurentii Justiniani Episc. Conf. \textit{Duplex.}\\
d & viij & 6 &  \\
e & vij & 7 & \\
f & vj & 8 & \hang \scspace{Nativitas B}. \scspace{Mariæ Virginis}. \textit{Duplex II classis cum Octava simplici.} \mem{S. Hadriani Martyris.}\\
g & v & 9 & \hang S. Gorgonii Martyris. \textit{Simplex.}\\
A & iv & 10 & \hang S. Nicolai a Tolentino. \textit{Duplex.}\\
b & iij & 11 & SS. Proti et Hyacinthi Mart.  \textit{Simplex.}\\
c & Prid. & 12 & \hang Ss. Nominis Mariæ. \textit{Duplex majus.}\\
d & Idib. & 13 & \hang \\
e & xviij & 14 & \hang Exaltatio S. Crucis. \textit{Duplex majus.}\\
f & xvij & 15 & \hang \scspace{Septum Dolorum B}. \scspace{Mariæ Virginis}. \textit{Duplex II classis.} \mem{S. Nicomedis Mart.}\\
g & xvj & 16 & \hang SS. Cornelii Papæ et Cypriani Episc., Mart. \textit{Semid.} \mem{SS. Euphemiæ et Sociorum Martyrum.}\\
A & xv & 17 & \hang Impressio sacrorum Stigmatum S. Francisci Conf. \textit{Duplex.}\\
b & xiv & 18 & \hang S. Josephi a Cupertino Conf. \textit{Duplex.}\\
c & xiij & 19 & \hang SS. Januarii Episcopi et Soc. Martyrum. \textit{Duplex.}\\
d & xij & 20 & \hang SS. Eustachii et Soc. Martyrum. \textit{Duplex.} \mem{Vigiliæ.}\\
e & xj & 21 & \hang S. \scspace{Matthæi Apostoli et Evangelistæ}. \textit{Duplex II classis.}\\
f & x & 22 &  \hang S. Thomæ de Villanova Episc. Conf. \textit{Dupl.} \mem{SS. Mauritii et Sociorum Martyrum.}\\
g & ix & 23 & \hang S. Lini Papæ Mart. \textit{Semid.} \mem{S. Theclæ Virg. et Mart.}\\
A & viij & 24 &  \hang B. Mariæ Virginis de Mercede. \textit{Duplex majus.}\\
b & vij & 25 &  \\
c & vj & 26 & \hang SS. Cypriani et Justinæ Virginis, Martyrum. \textit{Simplex.}\\
d & v & 27 & \hang SS. Cosmæ et Damiani Martyrum. \textit{Semiduplex.}\\
e & iv & 28 & \hang S. Wenceslai Ducis, Martyris. \textit{Semiduplex.}\\
f & iij & 29 & \hang \capspace{DEDICATIO S}. \capspace{MICHAELIS ARCHANGELI}. \textit{Dupl. I classis.}\\
g & Prid. & 30 & \hang S. Hieronymi Presbyteri, Conf. et Eccl. Doctoris. \textit{Duplex.}
\end{longtable}


% !TEX root = Kalendarium.tex

{\centering{{\normalsize October.}\par}}

\begin{longtable}{>{\centering}p{0.025\textwidth}|>{\raggedright}p{0.040\textwidth}|>{\raggedleft}p{0.025\textwidth}|>{\raggedright\arraybackslash}p{0.80\textwidth}}
A & Kal. & 1 & \hang S. Remigii Episc. Conf. \textit{Simplex.}\\
b & vj & 2 & \hang SS. Angelorum Custodum. \textit{Duplex majus.}\\
c & v &3 & S. Teresiæ a Jesu Infante Virg. \textit{Duplex.}\\
d & iv & 4 & \hang  S. Francisci Confessoris. \textit{Duplex majus.}\\
e & iij & 5 & \hang SS. Placidi et Sociorum Martyrum. \textit{Simplex.}\\
f & Prid. & 6 & \hang S. Brunonis Confessoris. \textit{Duplex.}\\
g & Non. & 7 & \hang \scspace{Sacratissimi Rosarii} B.M.V. \textit{Duplex II classis.} \mem{S. Marci Papæ Conf. ac SS. Sergii et Sociorum Mart.}\\
A & viij & 8 & S. Birgittæ Viduæ. \textit{Duplex.} \mem{SS. Dionysii, Episc., Rustici et Eleutherii Mart.}\\
b & vij & 9 & \hang S. Joannis Leonardi Conf. \textit{Duplex.}\\
c & vj & 10 & S. Francisci Borgiæ Confessoris. \textit{Semiduplex.}\\
d & v & 11 & \hang \scspace{Maternitatis B}. \scspace{Mariæ Virginis.} \textit{Duplex II classis.}\\
e & iv & 12 & \\
f & iij & 13 & S. Eduardi Regis, Conf. \textit{Semiduplex.}\\
g & Prid. & 14 & \hang S. Callisti I Papæ Martyris. \textit{Duplex.}\\
A & Idib. & 15 & \hang S. Teresiæ Virginis. \textit{Duplex.}\\
b & xvij & 16 & \hang S. Hedwigis Viduæ. \textit{Semiduplex.}\\
c & xvj & 17 & \hang  S. Margaritæ Mariæ Alacoque Virginis. \textit{Duplex.}\\
d & xv & 18 & \hang S. \scspace{Lucæ Evanglistæ}. \textit{Duplex II classis.}\\
e & xiv &19 & \hang S. Petri de Alcantara Conf. \textit{Duplex.}\\
f & xiij & 20 & \hang S. Joannis Cantii Conf. \textit{Duplex.}\\
g & xij & 21 & \hang S. Hilarionis Abbatis. \textit{Simplex.} \mem{SS. Ursulæ ac Sociarum Virginum et Martyrum.}\\
A & xj & 22 & \\
b & x & 23 & \hang \\
c & ix & 24 & \hang S. Raphaelis Archangelis. \textit{Duplex majus.}\\
d & viij & 25 & \hang SS. Chrystani et Dariæ Martyrum. \textit{Simplex.}\\
e & vij & 26 & \hang S. Evaristi Papæ Martyris. \textit{Simplex.}\\
f & vj & 27 & \hang Vigilia.\\
g & v & 28 & \hang \scspace{Ss}. \scspace{Simonis et Judæ Apostolorum}. \textit{Duplex II classis.}\\
A & iv & 29 & \\
b & iij & 30 & \\
c & Prid. & 31 & \hang Vigilia Omnium Sanctorum.\\
& & & \hang \textit{Dominica ultima Octobris.} \capspace{FESTUM} D.N. \capspace{JESU CHRISTI REGIS}. \textit{Duplex I classis.}
\end{longtable}


% !TEX root = Kalendarium.tex

{\centering{{\normalsize November.}\par}}

\begin{longtable}{>{\centering}p{0.025\textwidth}|>{\raggedright}p{0.040\textwidth}|>{\raggedleft}p{0.025\textwidth}|>{\raggedright\arraybackslash}p{0.80\textwidth}}
d & Kal. & 1 & \hang \capspace{OMNIUM SANCTORUM}. \textit{Duplex I classis cum Octava communi.}\\
e & iv & 2 & \hang Commemoratio Omnium Fidelium Defunctorum. \textit{Duplex.}\\
f & iij & 3 & \hang De Octava Omnium Sanctorum. \textit{Semiduplex.} \mem{Octavæ ac SS. Vitalis et Agricolæ Martyrum.}\\
g & Prid. & 4 & \hang S. Caroli Episc. Conf. \textit{Duplex.}\\
A & Non. & 5 & \hang De Octava. \textit{Semiduplex.}\\
b & viij & 6 & \hang De Octava. \textit{Semiduplex.}\\
c & vij & 7 & \hang  De Octava. \textit{Semiduplex.}\\
d & vj & 8 & \hang Octava Omnium Sanctorum.  \textit{Duplex majus.} \mem{SS. Quatuor Coronatorum Martyrum.}\\
e & v & 9 & \hang \scspace{Dedicatio Archibasilicæ Ss. Salvatoris}. \textit{Duplex II classis.} \mem{S.~Theodori Martyris.}\\ %%~ needed with current settings
f & iv & 10 & \hang  S. Andreæ Avellini Conf. \textit{Duplex.} \mem{SS. Tryphonis et Sociorum Martyrum.}\\
g & iij & 11 & \hang  S. Martini Episc. Conf. \textit{Duplex.} \mem{S. Mennæ Mart.}\\
A & Prid. & 12 & \hang S. Martini I Papæ Martyris. \textit{Semiduplex.}\\
b & Idib. & 13 & S. Didaci Conf. \textit{Semiduplex.}\\
c & xviij & 14 & S. Josaphat Episc. Martyris. \textit{Duplex.}\\
d & xvij & 15 & \hang S. Alberti Magni Ep., Conf. et Eccl. Doct. \textit{Duplex.}\\
e & xvj & 16 & \hang S. Gertrudis Virginis. \textit{Duplex.}\\
f & xv &17 & \hang S. Gregorii Thaumaturgi Episc. Conf. \textit{Semiduplex.}\\
g & xiv & 18 & \hang Dedicatio Basilicarum SS. Petri et Pauli Apost. \textit{Dupl. majus.}\\
A & xiij & 19 & \hang S. Elisabeth Viduæ. \textit{Dupl.} \mem{S. Pontiani Papæ Mart.}\\
b & xij & 20 & \hang S. Felicis de Valois Conf. \textit{Duplex.}\\
c & xj & 21 & \hang Præsentatio B. Mariæ Virginis. \textit{Duplex majus.}\\
d & x & 22 & \hang S. Cæciliæ Virginis et Martyris. \textit{Duplex.}\\
e & ix & 23 & \hang S. Clementis I Papæ Martyris. \textit{Duplex.} \mem{S. Felicitatis Martyris.}\\
f & viij & 24 & \hang S. Joannis a Cruce Conf. et Eccl. Doct. \textit{Duplex.} \mem{S. Chrysogoni Mart.}\\
g & vij & 25 & \hang S. Catharinæ Virginis et Martyris. \textit{Duplex.}\\
A & vj & 26 & \hang S. Silvestri Abbatis. \textit{Duplex.} \mem{S. Petri Alexandrini Episc. Martyris.}\\
b & v & 27 & \\
c & iv & 28 & \\
d & iij & 29 & Vigilia. \mem{S. Saturnini Martyris.}\\
e & Prid. & 30 & \hang \scspace{S}. \scspace{Andreæ Apostoli}. \textit{Duplex II classis.}
\end{longtable}

% !TEX root = Kalendarium.tex

{\centering{{\normalsize December.}\par}}

\begin{longtable}{>{\centering}p{0.025\textwidth}|>{\raggedright}p{0.040\textwidth}|>{\raggedleft}p{0.025\textwidth}|>{\raggedright\arraybackslash}p{0.80\textwidth}}
f & Kal. & 1 & \\
g & iv & 2 & S. Bibianæ Virginis et Martyris. \textit{Semiduplex.}\\
A & iij & 3 & \hang S. Francisci Xaverii Conf. \textit{Duplex majus.}\\
b & Prid. & 4 & \hang S. Petri Chrysologi Episc. Conf. et Eccl. Doctoris. \textit{Duplex.} \mem{S. Barbaræ Virginis et Martyris.}\\
c & Non. & 5 & \hang \mem{S. Sabbæ Abbatis.}\\
d & viij & 6 & \hang S. Nicolai Episc. Conf. \textit{Duplex.}\\
e & vij & 7 & \hang S. Ambrosii Episc. Conf. et Eccl. Doct. \textit{Duplex.}\\ %% the sources disagree: LA1949 has no mention of the Vigil. AM1934 mentions it but has a note "de qua nihil in Officio" AR 1912 has (Vigilia.)
f & vj & 8 & \hang \capspace{CONCEPTIO IMMACULATA B}. \capspace{MARIÆ VIRGINIS}. \textit{Duplex I classis cum Octava communi.}\\ 
g & v & 9 & \hang De Octava Conceptionis. \textit{Semiduplex.}\\
A & iv & 10 & \hang De Octava. \textit{Semiduplex.} \mem{S. Melchiadis Papæ Martyr.}\\
b & iij & 11 & \hang S. Damasi I Papæ Conf. \textit{Semiduplex.} \mem{Octavæ.}\\
c & Prid. & 12 & \hang De Octava. \textit{Semiduplex.}\\
d & Idib. & 13 & \hang S. Luciæ Virginis et Martyris. \textit{Duplex.} \mem{Octavæ.}\\
e & xix & 14 & \hang De Octava. \textit{Semiduplex.}\\
f & xviij & 15 & Octava Conceptionis Immaculatæ B.M.V. \textit{Duplex majus.}\\
g & xvij & 16 & S. Eusebii Episc. Martyris.  \textit{Semiduplex.}\\
A & xvj & 17 & \\
b & xv & 18 & \\
c & xiv & 19 & \\
d & xiij & 20 & Vigilia.\\
e & xij & 21 & \hang S. \scspace{Thomæ Apostoli}. \textit{Duplex II classis.}\\
f & xj & 22 & \\
g & x & 23 & \\
A & ix & 24 & Vigilia.\\
b & viij & 25 & \hang \capspace{NATIVITAS} D.N. \capspace{JESU CHRISTI}. \textit{Duplex I classis cum Octava privilegiata III ordinis.}\\
c & vij & 26 & \hang S. \scspace{Stephani Protomartyris}. \textit{Duplex II classis cum Octava simplici.} \mem{Octavæ Nativitatis.}\\ 
d & vj & 27 & \hang S. \scspace{Joannis Apostoli et Evangelistæ}. \textit{Duplex II classis cum Octava simplici.} \mem{Octavæ Nativitatis.}\\ \noalign{\penalty-5000} %%this marks pagebreak preferred location
e & v & 28 & \hang \scspace{Ss}. \scspace{Innocentium Martyrum}. \textit{Duplex II classis cum Octava simplici.} \mem{Octavæ Nativitatis.}\\
f & iv & 29 & \hang S. Thomæ Episcopi Martyris. \textit{Duplex.}\\
g & iij & 30 & De Octava Nativitatis.  \textit{Semiduplex.}\\
A & Prid. & 31 & \hang S. Silvestri I Papæ Conf. \textit{Duplex.} \mem{Comm. Octavæ Nativitatis.}
\end{longtable}

\endgroup

\setlength{\parskip}{\defaultparskip}
\normalsize

\flushbottom

\mainmatter

\end{document}