% !TEX TS-program = lualatexmk
% !TEX parameter =  --shell-escape
\documentclass[vesperale_romanum.tex]{subfiles}

\ifcsname preamble@file\endcsname
  \setcounter{page}{\getpagerefnumber{M-vr_NN_variationes}}
\fi

%%this code when \customsubfiles is used should allow for continuous pagination when subfiles are compiled individually.

\begin{document}

\chapter*{MUTATIONES IN ANTIPHONALI.}\header{mutationes in antiphonali.}

\addcontentsline{toc}{chapter}{Mutationes in Antiphonali.}
%\thispagestyle{empty}

\bigtitle{Excerpti de « Codice Rubricarum » et « Variationibus in Breviario et Missali Romano ad Normam novi Codicis Rubricarum »} %%we don't want this halfway down the page
%Antiphonæ duplicantur ad Vesperas etiam ad Completorium.
\smalltitle{De Dominicis.}

Dominicæ sunt I aut II classis. Dominicæ I classis sunt:

\begin{enumerate}[nosep,label=\alph*.]
\item I – IV Adventus;
\item I – IV Quadragesimæ;
\item I – II Passionis;
\item  Dominica Resurrectionis seu Paschatis;
\item Dominica in albis;
\item Dominica Pentecostes. %%holy smokes the f) is badly kerned
\end{enumerate}

Dominicæ Paschatis et Pentecostes sunt pariter festa I classis cum octava. Omnes aliæ dominicæ sunt II classis.

Dominica celebratur suo die, juxta rubricas. Officium et Missa dominicæ impeditæ nec anticipantur nec resumuntur.

Festum tamen Immaculatæ Conceptionis B. Mariæ Virg. præfertur occurrenti dominicæ Adventus.
Ad concurrentiam vero quod attinet, servetur norma, juxta rubricas.

Si in dominicis II classis per annum occurrent festum cujusvis tituli vel mysterii Domini, festum ipsum locum tenet dominicæ, de qua nulla fit commemoratio.

Hoc in casu, festum ipsum, « cum locum tenet dominicæ », acquirit I Vesperas, etiam si agatur olim de duplici majori: ita Festum S. Familiæ, Exaltationis S. Crucis (14 Sept.), Dedicationis basilicarum Ss. Petri et Pauli App. (18 nov.) %%this rubric needs to be rewritten

Dominica prima mensis ea intellegitur, quæ prima occurrit in mense, scilicet a die primo ad septimum mensis; dominica autem ultima, quæ diem primum mensis sequentis proxime præcedit.

Item ad computandam primam dominicam mensium augusti, septembris, octobris et novembris, ratione lectionum Scripturæ occurrentis, ea dicitur prima dominica mensis, quæ cadit a die primo ad septimum mensis.

\smalltitle{De festis.}

Festa sunt primæ, secundæ aut tertiæ classis.

Festa I classis inter dies solemniores adnumerantur, quorum Officium incipit a I Vesperis, die præcedenti;

Festa II et III classis Officium habent quod per se decurrit a Matutino ad Completorium ipsius diei; festa vero Domini II classis I Vesperas acquirunt, quoties, in occurrentia, locum tenent dominicæ II classis.

\smalltitle{De feriis.}

Feriæ sunt primæ, secundæ, tertiæ aut quartæ classis. Feriæ I classis sunt:

\begin{enumerate}[nosep,label=\alph*.]
\item 
Feria IV cinerum;
\item omnes feriæ Hebdomadæ sanctæ.
\end{enumerate}

Hæ feriæ festis quibuslibet præferuntur, et nullam admittunt commemorationem, nisi unam privilegiatam.

Feriæ II classis sunt:
\begin{enumerate}[nosep,label=\alph*.]

\item feriæ Adventus a die 17 ad diem 23 decembris;
\item feriæ Quatuor Temporum Adventus, Quadragesimæ et mensis septembris.
\end{enumerate}

Hæ feriæ præferuntur festis particularibus II classis; si vero impediuntur, commemorari debent.

Feriæ III classis sunt:
\begin{enumerate}[nosep,label=\alph*.]
\item feriæ Quadragesimæ et Passionis, a feria V post cineres usque ad sabbatum ante dominicam II Passionis inclusive, superius non nominatæ, quæ præferuntur festis III classis;

\item feriæ Adventus usque ad diem 16 decembris inclusive, superius non nominatæ, quæ cedunt festis III classis.
\end{enumerate}

Hæ feriæ, si impediuntur, commemorari debent.

Omnes feriæ, numeris 23–25 non nominatæ, sunt feriæ IV classis; quæ numquam commemorantur. %% to fix.

Officium feriæ incipit a Matutino et explicit per se post Completorium; Officium vero sabbati, excepto Officio Sabbati sancti, explicit post Nonam.

\smalltitle{De vigiliis.}

Vigiliæ sunt primæ, secundæ aut tertiæ classis. Vigiliæ I classis sunt:

\begin{enumerate}[nosep,label=\alph*.]
\item  vigilia Nativitatis Domini quæ, in occurrentia, locum tenet dominicæ IV Adventus, de qua, proinde, nulla fit commemoratio;
\item vigilia Pentecostes.
\end{enumerate}

Hæ vigiliæ festis quibuslibet præferuntur, et nullam admittunt commemorationem.

Vigiliæ II classis sunt:

\begin{enumerate}[nosep,label=\alph*.]
\item vigilia Ascensionis Domini;

\item vigilia Assumptionis B. Mariæ Virg.;

\item vigilia Nativitatis S. Joannis Baptistæ;

\item vigilia Ss. Petri et Pauli Apostolorum.
\end{enumerate}

Hæ vigiliæ præferuntur diebus liturgicis III et IV classis; et, si impediuntur, commemorantur, iuxta rubricas.

Vigilia III classis est vigilia S. Laurentii.

Hæc vigilia præfertur diebus liturgicis IV classis; et, si impeditur, commemoratur, iuxta rubricas. Omnes aliæ vigiliæ supprimuntur.

Vigilia II aut III classis penitus omittitur, si occurrat in dominica quavis, aut in festo I classis, vel si festum cui præmittitur in alium diem transferri aut ad commemorationem reduci contingat.

Officium vigiliæ incipit a Matutino et explicit quando initium habet Officium festi subsequentis.

\smalltitle{De octavis.}
Celebrantur tantum octavæ Nativitatis Domini, Paschatis et Pentecostes, suppressis omnibus aliis, sive in calendario universali, sive in calendariis particularibus occurrentibus. 

Octavæ sunt I aut II classis. Octavæ I classis sunt octavæ Paschatis et Pentecostes. Dies infra has octavas sunt I classis. Octava II classis est octava Nativitatis Domini. Dies infra octavam sunt II classis; dies autem octavus est I classis.

*Octava Nativitatis Domini modo peculiari ordinatur, scilicet:
\begin{enumerate}[nosep,label=\alph*.]
\item die 26 decembris, fit festum S. Stephani Protomart. (II classis);

\item die 27 decembris, fit festum S. Joannis Ap. et Evang. (II classis);

\item die 28 decembris, fit festum Ss. Innocentium Mart. (II classis);

\item die 29 decembris, fit commemoratio S. Thomæ Episcopi et Mart.;

\item die 31 decembris, fit commemoratio S. Silvestri I Papæ et Conf.;

\item ex festis particularibus ea tantum admittuntur quæ sunt I classis et in honorem Sanctorum qui in calendario universali his diebus inscribuntur, etsi tantum ad modum commemorationis; cetera transferuntur post octavam.
\end{enumerate}

De dominica infra octavam Nativitatis Domini, quæ scilicet a die 26 ad 31 decembris occurrit, semper fit Officium cum commemoratione festi forte occurrentis, iuxta rubricas, nisi dominica incidat in festum I classis: quo in casu, fit de festo cum commemoratione dominicæ.

Normæ peculiares pro ordinandis Officio et Missa infra octavam Nativitatis Domini inveniuntur in rubricis Breviarii et Missalis.

\smalltitle{De anni temporibus.}

Tempus natalicium decurrit a I Vesperis Nativitatis Domini usque ad diem 13 januarii inclusive.

Huiusmodi autem temporis spatium comprehendit:

a) tempus Nativitatis, quod decurrit a I Vesperis Nativitatis Domini usque ad Nonam inclusive diei 5 januarii;

b) tempus Epiphaniæ, quod decurrit a I Vesperis Epiphaniæ Domini usque ad diem 13 januarii inclusive.

\smalltitle{De sancta Maria in sabbato.}

In sabbatis, in quibus occurrit Officium de feria IV classis, fit de sancta Maria in sabbato.

Officium sanctæ Mariæ in sabbato incipit a Matutino et explicit post Nonam

\smalltitle{De concurrentia dierum liturgicorum.}
 
In concurrentia præferuntur Vesperæ diei liturgici classis superioris, et alteræ commemorantur vel non, iuxta rubricas.

Quando vero dies liturgici, quorum Vesperæ concurrunt, sunt ejusdem classis, dicuntur integræ secundæ Vesperæ de Officio currenti et fit commemoratio sequentis, iuxta rubricas.

\smalltitle{TABELLA DIERUM LITURGICORUM SECUNDUM ORDINEM PRÆCEDENTIÆ DISPOSITA.} %%cannot be separated from table, nor each class from following days
%%table makes reference to WHOLE of rubrics…
\smalltitle{Dies liturgici I classis.}

 \begin{enumerate}[nosep]
\item
Festum Nativitatis Domini, dominica Resurrectionis et dominica Pentecostes (I classis cum octava).
\item Triduum sacrum.
\item Festa Epiphaniæ et Ascensionis Domini, Ss.mæ Trinitatis, Corporis Christi, Cordis Jesu et Christi Regis.
\item Festa Immaculatæ Conceptionis et Assumptionis B. Mariæ Virg.
\item Vigilia et dies octavus Nativitatis Domini.
\item
 Dominicæ Adventus, Quadragesimæ et Passionis, et dominica in albis.
\item
Feriæ I classis superius non nominatæ, nempe: IV cinerum et II, III et IV Hebdomadæ sanctæ.
\item
Commemoratio omnium Fidelium defunctorum, quæ tamen locum cedit dominicæ occurrenti.
\item
Vigilia Pentecostes.
\item
Dies infra octavas Paschatis et Pentecostes.
\item
Festa I classis Ecclesiæ universæ superius non nominata.
\item
Festa I classis propria, nempe:
\begin{enumerate}[nosep,label=\arabic*.]
\item
Festum Patroni principalis rite constituti:
\begin{enumerate}[nosep,label=\alph*.]
\item
nationis,
\item
 regionis seu provinciæ sive ecclesiasticæ sive civilis,
\item
diœcesis.
\end{enumerate}
\item
Anniversarium Dedicationis ecclesiæ cathedralis.
\item
Festum Patroni principalis rite constituti loci seu oppidi vel civitatis.
\item
Festum et anniversarium Dedicationis ecclesiæ propriæ, aut oratorii publici vel semipublici, quod locum tenet ecclesiæ.
\item
Titulus ecclesiæ propriæ.
\item
Festum Tituli Ordinis seu Congregationis.
\item
Festum Fundatoris canonizati Ordinis seu Congregationis.
\item
Festum Patroni principalis rite constituti Ordinis seu Congregationis, et provinciæ religiosæ.
\end{enumerate}
\item
Festa indulta I classis, primum mobilia, deinde fixa.
\end{enumerate}

\smalltitle{Dies liturgici II classis.}
 \begin{enumerate}[nosep]
\item Festa Domini II classis, primum mobilia, deinde fixa.
\item Dominicæ II classis.
\item Festa II classis Ecclesiæ universæ, quæ non sunt Domini.
\item Dies infra octavam Nativitatis Domini.
\item Feriæ II classis, nempe: Adventus a die 17 ad diem 23 decembris inclusive, et feriæ Quatuor Temporum Adventus, Quadragesimæ et mensis septembris.
\item Festa propria II classis, nempe:
\begin{enumerate}[nosep,label=\arabic*.]
\item Festum Patroni secundarii rite constituti:
\begin{enumerate}[nosep,label=\alph*.]
\item nationis,
\item regionis seu provinciæ sive ecclesiasticæ sive civilis,
\item diœcesis,
\item loci seu oppidi vel civitatis.
\end{enumerate}
\item Festa Sanctorum aut Beatorum, de quibus n. 43  Codicis rubricarum.
\item Festa Sanctorum alicui ecclesiæ propria (n. 45  Codicis rubricarum).
\item Festum Fundatoris beatificati Ordinis seu Congregationis (n. 46~b  Codicis rubricarum).
\item Festum Patroni secundarii rite constituti Ordinis seu Congregationis, et provinciæ religiosæ (n. 46~d).
\item Festa Sanctorum aut Beatorum, de quibus n. 46~e  Codicis rubricarum.
\end{enumerate}
\item Festa indulta II classis, primum mobilia, deinde fixa.
\item Vigiliæ II classis.
\end{enumerate}

\smalltitle{Dies liturgici III classis.}
 \begin{enumerate}[nosep]
\item Feriæ Quadragesimæ et Passionis, a feria V post cineres usque ad sabbatum ante dominicam II Passionis inclusive, exceptis feriis Quatuor Temporum.
\item Festa III classis, in calendariis particularibus inscripta, et quidem primum festa propria, nempe:
\begin{enumerate}[nosep,label=\arabic*.]
\item Festa Sanctorum aut Beatorum, de quibus n. 43~d  Codicis rubricarum.
\item Festa Beatorum alicui ecclesiæ propria (n. 45~d  Codicis rubricarum).
\item Festa Sanctorum aut Beatorum, de quibus n. 46~e  Codicis rubricarum; deinde festa indulta, primum mobilia, deinde fixa.
\end{enumerate}
\item Festa III classis, in calendario Ecclesiæ universæ inscripta, primum mobilia, deinde fixa.
\item Feriæ Adventus usque ad diem 16 decembris inclusive, exceptis feriis Quatuor Temporum.
\item Vigilia III classis.
\end{enumerate}

\smalltitle{Dies liturgici IV classis.}

 \begin{enumerate}[nosep]
\item
Officium sanctæ Mariæ in sabbato.
\item
Feriæ IV classis.
\end{enumerate}

\smalltitle{De commemorationibus.}

Commemorationes sunt aut privilegiatæ aut ordinariæ. Commemorationes privilegiatæ fiunt in Laudibus et in Vesperi; commemorationes vero ordinariæ fiunt tantum in Laudibus,

Commemorationes privilegiatæ sunt commemora­tiones:
\begin{enumerate}[nosep,label=\alph*.]
\item de dominica;
\item de die liturgico I classis;
\item de diebus infra octavam Nativitatis Domini;
\item de feriis Quatuor Temporum mensis septembris;
\item de feriis Adventus, Quadragesimæ et Passionis;
\end{enumerate}

Omnes aliæ commemorationes sunt commemora­tiones ordinariæ.

In Officio S. Petri semper fit commemoratio S. Pauli, et vicissim. Hæc commemoratio dicitur inseparabilis; et duæ orationes adeo in unam coalescere censentur ut in numero orationum computando, pro unica habeantur.

Proinde:

\begin{enumerate}[nosep,label=\alph*.]
\item in Officio S. Petri aut S. Pauli, oratio alterius Apostoli additur, ad Laudes et ad Vesperas, sub unica conclusione, orationi diei, absque antiphona et versu;
\item
quoties vero oratio unius Apostoli addenda est ad modum commemorationis, huic orationi additur altera immediate, ante omnes alias commemorationes.
\end{enumerate}

Ratio admittendi commemorationes hæc est:
\begin{enumerate}[nosep,label=\alph*.]
\item 
in diebus liturgicis I classis et in Missis in cantu non conventualibus, nulla admittitur commemoratio, præter unam privilegiatam;
\item
in dominicis II classis, una tantum admittitur commemoratio, scilicet de festo II classis, quæ tamen omittitur si commemoratio privilegiata facienda sit;
\item
in aliis diebus liturgicis II classis, una tantum admittitur commemoratio, scilicet aut una privilegiata aut una ordinaria;
\item
in diebus liturgicis III et IV classis, duæ tantum admittuntur commemorationes.
\end{enumerate}

Ad commemorationes et orationes quod attinet, hæc insuper serventur:
\begin{enumerate}[nosep,label=\alph*.]
\item Officium, Missa aut commemoratio de aliquo festo vel mysterio unius Divinæ Personæ excludit commemorationem aut orationem de alio festo vel mysterio ejusdem Divinæ Personæ;
\item
Officium, Missa aut commemoratio de dominica excludit commemorationem aut orationem de festo vel mysterio Domini, et vicissim;
\item
Officium, Missa aut commemoratio de Tempore excludit aliam commemorationem de Tempore;
\end{enumerate}

Commemoratio de Tempore fit primo loco. In admittendis et ordinandis aliis commemorationibus, servetur ordo tabellæ præcedentiæ. Quælibet commemoratio, quæ numerum pro singulis diebus liturgicis statutum superet, omittitur.

\smalltitle{De initio et fine Horarum.}

\begin{enumerate}[nosep]
\item Horæ canonicæ, tam in publica quam in privata recitatione, omissis \textit{Pater,} \textit{Ave,} et respective \textit{Credo,} inchoantur absolute, hoc modo:

\begin{enumerate}[nosep,label=\null]
\item Vesperæ: a versu \textit{Deus, in adjutórium.}
\item Completorium: a versu \hspace{0.03em}\textit{Jube, domne, benedícere.}
\end{enumerate}

\item In officio tridui sacri et in officio defunctorum omnes Horæ, \textit{Pater,} \textit{Ave,} et respective \textit{Credo,} incipiunt ut in Antiphonali notatur.

\item Horæ canonicæ, tam in publica quam in privata recitatione, absolvuntur hoc modo:
\begin{enumerate}[nosep,label=\null]
\item Vesperæ: versu \textit{Fidélium áinimæ.}

\item Completorium: benedictione \textit{Benedícat et custódiat.}
\end{enumerate}
\end{enumerate}

\smalltitle{De conclusione officii.}

Cursus cotidjanus divini offici concluditur post Completorium, sueta antiphona B.M.V., cum versiculo \textit{Divínum auxílium.} Indultum et indulgentiæ, pro recitatione orationis \textit{Sacrosanctæ} concessa, eidem antiphonæ finali adnectuntur.

\smalltitle{De quibusdam partibus in officio.}

\begin{enumerate}[nosep]
\item Antiphonæ dicuntur semper integræ ante et post psalmos et cantica, ad omnes Horas, tam maiores quam minores.
\item Hymni proprii quorumdam sanctorum certis Horis assignati non transferuntur. In hymno \textit{Iste conféssor} numquam mutatur tertius versus, qui erit semper \textit{Méruit suprémos laudis honóres.}
\item Quilibet hymnus semper dicitur sub conclusione quæ ipsi in Breviario assignatur, exclusa quavis conclusionis mutatione ratione festi vel Temporis.
\item Officium commemoratum nunquam doxologiam propriam inducit in fine hymnorum Officii diei.
\item Antiphonæ ad \textit{Magníficat} feriarum tempore Septuagesimæ forte prætermissæ non resumuntur.
\item Ad Vesperas feriæ VI, tempore paschali, pro antiphona ad Magnificat resumitur antiphona ad Magnificat e II Vesperis dominicæ præcedentis.
\item Preces feriales dicuntur tantum in Vesperis officii feriarum IV et VI tempore Adventus, Quadragesimæ et Passionis, necnon feriarum IV et VI, et sabbati Quatuor Temporum, excepta octava Pentecostes, quando officium fit de feria.
\item Omnes aliæ preces omittuntur.
 \item Suffragium Sanctorum et commemoratio de Cruce omittuntur.
\item Symbolum Athanasjanum recitatur in Festo Ss. Trinitatis tantum.
\end{enumerate}

\smalltitle{De aliiis variationibus.}

Festum cujusvis tituli vel mysterii Domini occurrens in Dominicam acquirit primas Vesperas.

Ad singulas partes officii quod attinet hæc serventur:
\begin{enumerate}[nosep,label=\alph*.]
\item In dominicis et festis I classis nihil innovatur.
\item In festis II classis ad Vesperas fit ut in proprio et in communi; ad Completorium de dominica.
\item In ceteris festis, vigiliis vel feriis, per omnes Horas fit ut in psalterio et proprio loco, nisi in Vesperis antiphonæ et psalmi specialiter assignati habeantur.
\end{enumerate}

\smalltitle{Variationes in calendario.}.

\begin{enumerate}[nosep]
\item Festa, quæ tamquam \textit{duplicia I classis} in calendariis indicantur, abhinc fiunt \textit{festa I classis.}

\item Festa, quæ tamquam \textit{duplicia II classis} in calendariis indicantur, abhinc fiunt \textit{festa II classis.}

\item Festa, quæ tamquam \textit{duplicia majora} aut \textit{minora;} et festa, quæ tamquam semiduplicia (ab anno 1955 tamquam \textit{simplicia}) in calendariis indicantur, abhinc fiunt \textit{festa III classis.}

\item Festa, quæ tamquam \textit{simplicia} in calendariis indicantur, et anno 1955 ad \textit{commemorationem} sunt reducta, inscribuntur tamquam \textit{commemorationes}.

\item Ad commemorationem insuper reducuntur:
\begin{enumerate}[nosep,label=\alph*.]
\item festum S, Georgii Mart. (23 aprilis);
\item festum B. Mariæ Virg. de Monte Carmelo (16 julii);
\item festum S. Alexii Conf. (17 julii);
\item festum Ss. Cyriaci, Largi et Smaragdi Mm. (8 augusti);
\item festum impressionis Stigmatum S. Francisci (17 septembris);
\item festum Ss. Eustachii et Sociorum, Mm. (20 septembris);
\item festum B. Mariæ Virg. de Mercede (24 septembris);
\item festum S. Thomæ Ep. et Mart. (29 decembris);
\item festum S. Silvestri I Papæ et Conf. (31 decembris);
\item festum septem dolorum B. Mariæ Virg. (feria VI post dominicam I Passionis).
\end{enumerate}

\item Fiunt dies liturgici I classis:
\begin{enumerate}[nosep,label=\alph*.]
\item
Dies octavus Nativitatis Domini (1 januarii);
\item
 Commemoratio omnium Fidelium defunctorum (2 novembris),
quæ tamen locum cedere pergit dominicæ occurrenti.
\end{enumerate}

\item Fiunt dies liturgici II classis:
\begin{enumerate}[nosep,label=\alph*.]
\item festum S. Familiæ Jesu, Mariæ, Joseph (dominica I post Epiphaniam);
\item festum Cathedræ S. Petri Ap. (22 februarii); c) festum Exaltationis S. Crucis (14 septembris).
\end{enumerate}

\item E calendario expunguntur festa:
\begin{enumerate}[nosep,label=\alph*.]
\item Cathedræ S. Petri Romæ (18 januarii);
\item Inventionis S. Crucis (3 maii);
\item S. Joannis ante Portam Latinam (6 maii);
\item Apparitionis S.. Michælis Archangeli (8 maii); e) S. Leonis II (3 julii);
\item S. Anacleti (13 julii);
\item S. Petri ad Vincula (1 augusti);
\item Inventionis S. Stephani (3 augusti).
\end{enumerate}
\end{enumerate}

Item, e calendario expungitur commemoratio S. Vitalis Mart. (28 aprilis).

\begin{enumerate}[nosep,resume]

\item In calendario inscribuntur festa:
\begin{enumerate}[nosep,label=\alph*.]
\item Commemorationis Baptismatis D.N.J.C. (13 januarii);
\item S. Gregorii Barbadici Ep, et Conf. (17 junii);
\item S. Antonii Mariæ Claret Ep. et Conf. (23 octobris).
\end{enumerate}

\item Transferuntur festa:
\begin{enumerate}[nosep,label=\alph*.]
\item S. Irenæi, a die 28 junii ad diem 3 julii;
\item S. Joannis Mariæ Vianney, a die 9 ad diem 8 augusti.
\end{enumerate}
\end{enumerate}

\begin{enumerate}[nosep,resume]
\item Commemoratio Ss. Sergii, Bacchi, Marcelli et Apuleii Mm. transfertur a die 7 ad diem 8 octobris.

\item Mutatur denominatio:
\begin{enumerate}[nosep,label=\alph*.]
\item festi Circumcisionis Domini, in « Octava Nativitatis Domini »
(1 januarii);
\item festi Cathedræ S. Petri Ap. Antiochiæ, in « Festum Cathedræ S. Petri Ap. » (22 februarii);
\item festi Ssmi Rosarii B. Mariæ Virg., in « Festum B. Mariæ Virg.
a Rosario » (7 octobris).
\end{enumerate}\end{enumerate}

Circa festa quæ vi n.\@ 8 \textit{Variationum in Breviario et Missali romano ad normam Codicis rubricarum} e calendario universali expuncta sunt, pro calendariis particularibus hæc pressius statuuntur:
\begin{enumerate}[nosep,label=\alph*.]
\item  festum S. Anacleti, quolibet titulo et gradu celebretur, transfertur in diem 26 aprilis, sub recto nomine S. Cleti;
\item festum S. Vitalis transfertur in diem 4 novembris, una cum S. Agricola;
\item festum Cathedræ S. Petri unice die 22 februarii celebrandum est.
\item præstat ut festa sub n. 8 b, c, d, g, et h recensita, etsi alicubi tamquam Patronus principalis vel Titulus ecclesiæ habeantur, transferantur ad festa principalia, scilicet:
\begin{enumerate}[nosep,label=\arabic*.]
\item festum Inventionis S. Crucis, a die 3 maii ad diem 14 septembris;
\item festum S. Joannis ante Portam Latinam a die 6 maii ad diem 27 decembris;
\item festum Apparitionis S. Michælis Arch., a die 8 maii ad diem 29 septembris;
\item festum S. Petri ad Vincula, a die 1 augusti ad diem 29 junii;
\item festum Inventionis S. Stephani, a die 3 augusti ad diem 26 decembris.
\end{enumerate}
\end{enumerate}


…

\smalltitle{Variationes in Proprium Tempore.}
\begin{enumerate}[nosep,start=19]
%\item Orationes pro diversitate Temporum abolentur.
\item Si vigilia Nativitatis Domini venerit in dominica, Officium ordinatur hoc modo:  sabbato præcedenti, ad Vesperas, omnia dicuntur ut in sabbato ante dominicam IV Adventus. …
\end{enumerate}

\begin{enumerate}[nosep,start=21]
\item In festo Ss. Innocentium Mart. (28 decembris): a) adhibetur color ruber paramentorum…
\item Dies a 2 ad 5 ianuarii sunt feriæ temporis Nativitatis Domini. Ad Officium et Missam horum dierum quod attinet, hæc animadvertantur:
\begin{enumerate}[nosep,label=\alph*.]
\item \textit{In Officio feriali,} antiphonæ et psalmi, ad omnes Horas ut in Psalterio… reliqua, una cum versu in responsorio brevi ad Primam, ut die 1 januarii.
\item \textit{In festis,} … nulla fit commemoratio feriæ.
\end{enumerate}

 \item Dies a 7 ad 12 ianuarii sunt feriæ temporis Epiphaniæ Domini. Ad Officium et Missam horum dierum quod attinet, hæc animadvertantur:
 \begin{enumerate}[nosep,label=\alph*.]
 \item \textit{In Officio feriali,} antiphonæ et psalmi, ad omnes Horas ut in Psalterio… reliqua, una cum versu in responsorio brevi ad Primam, ut in festo Epiphaniæ. Oratio dicitur ut in festo Epiphaniæ; in feriis autem post dominicam I occurrentibus, de eadem dominica. Diebus a 7 ad 12 januarii sumuntur etiam antiphonæ ad \textit{Magnificat,} quæ singulis diebus propriæ assignantur; die vero 12 januarii, ad Magnificat, resumitur antiphona ad \textit{Magnificat} e II Vesperis Epiphaniæ.
 \item \textit{In festis,} … nulla fit commemoratio feriæ.
 \end{enumerate}
 
\item Die 13 januarii fit \textit{Commemoratio Baptismatis D.N. Jesu Christi} (II classis). Ad…Vesperas et Completorium omnia dicuntur ut in festo Epiphaniæ,  … et oratio dicuntur ut die 13 januarii. Quod si occurrat eodem die dominica I post Epiphaniam, fit Officium de S. Familia, sine commemoratione Baptismatis Domini et sine commemoratione dominicæ…
\end{enumerate}

…

\begin{enumerate}[nosep,start=29]
\item Dies a feria VI post Ascensionem Domini usque ad vigiliam Pentecostes exclusive sunt feriæ temporis Ascensionis. Ad Officium et Missam horum dierum quod attinet, hæc animadvertantur:
\begin{enumerate}[nosep,label=\alph*.]
\item \textit{In Officio feriali,} antiphonæ et psalmi, ad omnes Horas ut in Psalterio pro tempore paschali; reliqua… ut in festo Ascensionis.
 \item \textit{In festis,} … nulla fit commemoratio feriæ.
 \end{enumerate}
 
 \item Dominica olim infra octavam Ascensionis inscribitur tantum « Dominica post Ascensionem ». Eius Officium ordinatur hoc modo:
 \begin{enumerate}[nosep,label=\alph*.]
\item partes Ordinarii sumuntur e festo Ascensionis sicut in feriis huius temporis; capitula vero, antiphonæ ad \textit{Magnificat,} et oratio sunt propria;
\item ad I Vesperas, psalmi de sabbato dicuntur sub antiphona \textit{Alleluia, alleluia, alleluia};
\item ad Vesperas [dominicæ] psalmi de dominica dicuntur sub antiphona \textit{Alleluia, alleluia, alleluia} de tempore paschali;
\end{enumerate}\end{enumerate}

…

\begin{enumerate}[nosep,start=33]
\item … Dominicæ et dies, olim infra octavam Ssmi Corporis Christi et Ssmi Cordis Iesu, celebrantur per omnia sicut reliquæ dominicæ et feriæ per annum.
\end{enumerate}

…

\begin{enumerate}[nosep,start=37]
\item Ad Vesperas feriæ IV et VI Quatuor Temporum mensis septembris, pro antiphona ad Magnificat repetitur antiphona quæ, in iisdem feriis, habetur ad Benedictus.


\item Sabbato ante dominicam primam octobris, ponatur sequens rubrica: 

\quad Si prima dominica incidit in dies a 1 ad 3 octobris, mensis habet quinque dominicas, et Scriptura occurrens absolvitur integra, ut in Breviario habetur.

\quad Si vero prima dominica incidit in dies a 4 ad 7 octobris, tunc mensis habet quatuor dominicas tantum, et de Scriptura occurrenti omittitur illa pars, quae tertiae habdomadae assignata est.
%

\item Sabbato ante primam dominicam novembris, ponatur sequens rubrica:

\quad Si prima dominica incidit in dies 1 vel 2 novembris, mensis habet quidem quinque dominicas, ultima tamen est prima Adventus, ita ut pro Scriptura occurrenti maneant quatuor tantum hebdomadae. Item quatuor dominicas tantum habet mensis, si prima dominica incidit in dies a 3 ad 5 novembris. His in casibus, de Scriptura occurrenti omittitur illa pars, quae secundae hebdomadae assignata est. 

\quad Si vero prima dominica incidit in dies 6 vel 7 novembris, mensis habet quidem quatuor dominicas, ultima tamen est prima Adventus, ita ut pro Scriptura occurrenti maneant tres hebdomadae tantum. Hoc in casu, de Scriptura occurrenti omittitur illa pars quae primae et secundae heb- domadae assignata est.


\end{enumerate}
%%47 in official document

 Die 17 junii, pro festo S. Gregorii Barbadici, ponatur hæc pars propria:
 
\noindent Deus, qui beátum Gregórium Confessórem tuum atque Pontíficem pastoráli sollicitúdine,~† et páuperum miseratióne claréscere voluísti: concéde propítius; ut cuius mérita celebrámus,~* caritátis imitémur exémpla.
Per Dóminum.

%%55
Die 23 octobris, pro festo S. Antonii Mariæ Claret, ponatur hæc pars propria:

\noindent Deus, qui beátum Antónium Maríam Confessórem tuum atque Pontíficem, apostólicis virtútibus sublimásti,~† et per eum novas in Ecclésia clericórum ac vírginum famílias collegísti: concéde, quǽsumus; ut ejus dirigéntibus mónitis ac suffragántibus mentis,~* animárum salútem quǽrere júgiter studeámus.
Per Dóminum.

%%58
Vesperæ defunctorum, cum suo Completorio proprio, diei 1 novembris olim assignatæ, transferuntur in Commemorationem omnium Fidelium defunctorum; sed si concurrant cum dominica, vel festo I classis, Officium Commemorationis omnium Fidelium defunctorum cessat post Nonam.
 
Celebratio tamen Vesperarum defunctorum post II Vesperas diei 1 novembris, quæ pro pietate fidelium peragi consuevit, continuari potest, una cum aliis piis exercitiis forsitan consuetudine traditis, tam- quam peculiare pietatis obsequium.

\smalltitle{Variationes in Commune Sanctorum.}

%%60

In Communi Dedicationis ecclesiæ, in initio, ponatur sequens rubrica: Festum Dedicationis ecclesiæ est festum Domini.

In ipso die Dedicationis ecclesiæ, dicitur Officium de die liturgico occurrenti usque ad Nonam.

…

A Vesperis ipsius diei Dedicationis, dicitur Officium de Dedicatione ecclesiæ, quod protrahitur usque ad Completorium diei sequentis, et dicitur ad modum Officii festi I classis.



\end{document}