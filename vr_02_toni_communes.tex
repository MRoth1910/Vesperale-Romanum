% !TEX TS-program = lualatexmk
% !TEX parameter =  --shell-escape

\documentclass[vesperale_romanum.tex]{subfiles}

\ifcsname preamble@file\endcsname
  \setcounter{page}{\getpagerefnumber{M-vr_02_toni_communes}}
\fi

\begin{document}

%%may need to adjust ALT continuously until "final" version
%\chapter{TONI COMMUNES.}\cheader{toni communes.}

\chapter*{TONI COMMUNES.}
\addcontentsline{toc}{chapter}{Toni Communes.}
%\cheader{toni communes.} 
%% Need to redefine chapter to allow for optional argument to be used in TOC

%\thispagestyle{empty}

\section[I. In Principio Vesperarum]{I. IN PRINCIPIO VESPERARUM.}\cheader{i. in principio vesperarum.}

\smalltitle{Tonus festivus.}

\initialscore{Deus_In_Adjutorium_Festivus}
%\medskip
\rubrique{A Septuagesima usque ad Pascha, loco \normaltext{Allelúia,} dicitur:}
%\vspace{-8pt}

\smallscore{Laus_tibi_Domine}
%\vspace{2mm}
\rubrique{Hoc tono utendum est in Duplicibus, Semiduplicibus et Dominicis.}

\smalltitle{Tonus ferialis.}

\gscore[]{}{Deus_in_adiutorium_Tonus_ferialis}
%\vspace{2mm}
\rubrique{Hoc tono utendum est supradictis diebus ad Completorium, et in Festis Simplicibus et Feriis etiam ad Vesperas.}

%\vspace{2mm}
\rubrique{Adhiberi potest ad libitum in Vesperis Festorum quæ cum majori solemnitate celebrantur.}
%\vspace{2mm}
\smalltitle{Tonus solemnis.}

\gscore[]{}{Deus_in_adjutorium_tonus_solemnior} %% includes content from Laus_tibi_Domine_tonus_solemnior to include Vel as in LA1949.

%\smallscore{Laus_tibi_Domine_tonus_solemnior}

\section[II. Toni Psalmorum]{II. TONI PSALMORUM.}\cheader{ii. toni psalmorum.}

Tot sunt Toni regulares Psalmorum, quot modi cantionum, id est octo. Est etiam Tonus Pcregrinus, et Tonus in directum, de quibus infra.

In singulis Tonis habetur initium seu inceptio, flexa (si in versiculo locus sit flexæ), mediatio, et terminatio.

In Tonis 1, 3, 4, 7 et 8 habentur variæ terminationes seu \textit{Differentiæ,} quæ respondent variis modis quibus inchoantur Antiphonæ istorum tonorum, ita ut ex fine versiculi facilior evadat transitus ad Antiphonam quando resumi debet.

Psalmus quilibet cum Antiphona cantandus, in i\-pso tono Antiphonæ cantari debet, et terminari in differentia quam Antiphona postulat.

In eodem tono variæ diiferentiæ designantur per litteram ultima notæ cujusque differentiæ respondentem; id est A pro \textit{la;} B pro \textit{si;}
C pro \textit{ut;} D pro \textit{re;} E pro \textit{mi;} F pro \textit{fa;} G pro \textit{sol.}

Littera conveniens inseribitur cum i\-pso tono Antiphonæ: ponitur majuscula, quando respondet chordæ finali modi; secus, minuscula.

Si plures diiferentiæ ejusdem Toni terminantur in eadem nota, distinguuntur per numerum litteræ adjunctum: v. g. in Tono 1., g, g2, g3.

Psalmus quilibet intonandus est a Cantore cum initio proprii Toni, ad omnes Horas, etiam in Officio feriali vel pro Defunctis. Versiculi sequentes incipiuntur in chorda tenoris.

Si duo vel plures Psalmi, vel plures ejusdem Psalmi divisiones substant eidem Antiphonæ, et dicendi sunt cum distinctis \textit{Gloria Patri,} intonatio rursus fieri debet a Cantore cum inceptione in initio cujusque Psalmi vel divisionis. Sed si dicuntur sub eodem Gloria Patri, non fit nova intonatio.

In Canticis ex Evangelio (\textit{Magnificat, Nunc dimittis}) inceptio fit ad singulos versus, etiam in Officio feriali vel pro Defunctis.

Pro Cantico \textit{Magnificat} in singulis Tonis, antiqui utebantur quotidie modulatione solemni, nempe magis ornata, quæ secundum usum nunc communiorem, usurpari potest saltem in Festis majoribus, seu primæ vel secundæ classis.

Quando Antiphona est tantum inchoanda ante Psalmum (ut in Semiduplicibus et infra et ad Horas minores), si ejus prima verba sunt eadem ac prima verba Psalmi et ex i\-pso Psalmo desum\-pta, Psalmus incipi debet in chorda tenoris, ab eo verbo in quo desinit inchoatio Antiphonæ.%
%In Antiphonario post prima verba Psalmi quæ non repeti debent, ponitur signum ].%
Si Antiphona componitur ex primo versu Psalmi, ut Ant.\@ \textit{Qui habitat }in Officio Dedicationis, Cantor in Officio Duplici incipit Psalmum cum initio a secundo versu (nisi in fine Antiphonæ addatur Alleluia; tunc enim resumitur Psalmus a primo versu).

\altnormal
\smalltitle{Primus Tonus.}

\smallscore{primus_tonus}

\rubrique{(Loco diiferentiæ \normaltext{g2} potest ad libitum usurpari Diff. \normaltext{g;} et pro diiferentia \normaltext{a3,} Diff. \normaltext{a2.)}}

In omnibus Tonis Psalmorum flexa fit deprimendo ultimam syllabam ac etiam penultimam si sit brevis. Sed in monosyllaba vel hebraica dictione fit tantum sustinendo vocem in tenore, cum aliquantula pausa.\footnote{Ex decretis S.R.C. 8 Julii et 8 Decembris 1912, quando in Lectionibus et Versiculis et in Psalmorum mediationibus monosyilaba vel hebraica dictio occurrit, potest ad libitum immutari clausula, vel proferri sub modulatione consueta.}

\smalltitle{Secondus Tonus.}

\smallscore{secundus_tonus}

\rubrique{Vel:}
\smallscore{secundus_tonus_a}

\smalltitle{Tertius Tonus.}

\grechangenextscorelinedim{2}{spacebeneathtext}{0.1cm}{scalable}
\smallscore{tertius_tonus}
\rubrique{(Loco hujus ultimæ diiferentiæ quæ cum solis Antiphonis \normaltext{Confitémini Dómino, Dómine probásti} et \normaltext{Allelúia} in Vesperis ferialibus usurpatur, adhiberi potest diiferentia \normaltext{g.)}}

\smalltitle{Quartus Tonus.}

\smallscore{quartus_tonus}

\rubrique{Altera Positio ejusdem toni. (Quando Antiphona notatur cum \normaltext{A*,} potest ad libitum adhiberi Differentia sequens.)}

\smallscore{quartus_tonus_c_a}

\smalltitle{Quintus Tonus.}

\smallscore{quintus_tonus}

\smalltitle{Sextus Tonus.}

\smallscore{sextus_tonus}
Apud antiquos Mediatio fiebat ut in 1 Tono : sed a sæculo XVI sensim hujus invaluit alterius Mediationis, quæ jam in antiquis Tonalibus reperitur ut secunda Mediatio pro longioribus versibus.

\smallscore{sextus_tonus_alt}
\smalltitle{Septimus Tonus.}

\smallscore{septimus_tonus}

\smalltitle{Octavus Tonus.}

\rubrique{(Quando Antiphona notatur cum \normaltext{G*,} potest ad libitum adhiberi Differentia sequens.)}

\smallscore{octavus_tonus}

\altitshape

\smalltitle{Tonus peregrinus.}

Hic Tonus adhibetur tantum pro Psalmo \textit{In éxitu Israel} quando cantatur in Vesperis Dominicæ, et aliquoties pro Psalmo \textit{Laudáte púeri} ad Vesperas.

\gscore[]{}{tonus_per_113}

\gscore[]{}{tonus_per_112}

\smalltitle{Tonus in directum.}

Hic Tonus usurpatur pro Psalmis qui dicendi prescribuntur in Precibus Officii sine Antiphona: ut pro Ps. 145. in Vesperis pro Defunctis et pro Ps. 69 in Litaniis Sanctorum, etc.

\gscore[]{}{tonus_in_directum}

In voce monosyllaba vel hebraica, flexa fit eodem modo quo supra, vel ad libitum recta voce et protracta cum pausa, sicut in aliis Tonis Psalmorum.
%\newpage
Sabbato Sancto ad Completorium pro Psalmis, et in Officio Resurrectionis Domini usque ad Vesperas Sabbati in Albis, pro Psalmis qui ad Horas cantantur sine Antiphona et pro Cantico \textit{Nunc dimíttis,} potest usurpari Tonus sequens:
\grechangenextscorelinedim{2}{spacebeneathtext}{0.1cm}{scalable}
\gscore[]{}{tonus_ad_compl_in_sab_san} %%hspace needed for italic touching bold

\rubrique{Flexa ut in Tono 2 Psalmodiæ.}

In Commemoratione Omnium Fidelium Defunctorum, Psalmi ad Completorium, cantantur in Tono in directum ut supra, vel ad libitum in tono sequenti:

\gscore[]{}{tonus_irregularis}

\section[III. Toni Versiculorum]{III. TONI VERSICULORUM.}\cheader{iii. toni versiculorum.}

\smalltitle{Tonus cum neuma.}

\smallscore{versiculus_dom_pa_tc}

\rubrique{Vel juxta recentiorem usum.}

\smallscore{versiculus_dom_t_recentior_tc}

\rubrique{Sic cantatur \vv cum \rr suo post Hymnum, vel Responsorium breve. In quibusdam Festis solenmioribus, \vv  et \rr cantantur in tono magis ornato, ut notatur propriis locis.}

\smalltitle{Tonus simplex.}

\rubrique{Omnes Versiculi præter eos qui supra memorantur, cantantur in tono simplici (nisi aliter notetur), ut infra:}

\smallscore{versiculus_simplex_tc}

\rubrique{Si in fine occurrat vox monosyllaba vel hehraica indeclinabilis:}

\smallscore{versiculus_simplex_monosyllabica_tc}

\rubrique{Nota vocem \normaltext{\vv Allelúia} in fine Versiculi semper tractandam esse latino modo. Nomen vero \normaltext{Jesus,} etiamsi declinetur, semper acui in ultima syllaba. Et hoc in omnibus Tonis communibus.}

\rubrique{Quando Versiculus est solito longior, potest in eo fieri flexa~† et metrum~*, eodem modo quo fiunt flexa et mediatio Tono Psalmorum in directum, ut supra.}

%%.to add later: at least tonus indirectus…

\rubrique{\normaltext{\vv Dóminus vobíscum} ante vel post Orationem cantatur semper recta voce, nisi adhibeatur pro Oratione tonus antiquus ad libitum.}

\rubrique{Pro Versiculis post Orationem in fine Officii confer quæ infra notantur ad Tonum Orationis.}

\rubrique{Preces ad Completorium, cantantur in tono simplici Versiculorum; quibus præmittitur:}

\gscore[]{}{kyrie_preces_ad_compl_tc}

\rubrique{Confessio vero, sive in principio Completorii, sive intra Preces, non cantatur unquam, sed tota dicitur, cum \normaltext{\vv Misreátur }et \normaltext{Indulgéntiam,} voce recta et paulisper depressa.}

\rubrique{Item, Preces feriales non cantantur, sed dicuntur recto tono, nisi contraria adsit consuetudo.}

\section[IV. Tonus Capituli]{IV. TONUS CAPITULI.}\cheader{iv. tonus capituli.}

\gscore[]{}{capitulum_tc}

\smallscore{exempla_capitulum_tc}

%\smallscore{exempla_pro_puncto_tc}

\rubrique{Punctum non mutatur ad vocem monosyllabam vel hebraicam. Omittitur flexa si textus brevior sit vei alio modo non permittat.}

\rubrique{Si loco flexæ vel metri, vel alibi in Capitulo occurrit interrogatio, modulatur ut in Lectione; sed si venerit in fine, servartur tonus puncti.}

\section[V. Toni Orationum]{V. TONI ORATIONUM.}\cheader{v. toni orationum.}

\smalltitle{Tonus festivus.}

Hic tonus servatur quando Officium est Duplex, vel Semiduplex, vel de Dominica, in Vesperis ad Orationem principalem et ad Orationes Suffragiorum et Commemorationum.

\gscore[]{}{or_dominus_vobiscum_toni_orationum_tonus_festivus}

\gscore[]{}{or_toni_orationum_tonus_festivus}

\smallscore{or_metrum_flexa_festivus}

In i\-psa Oratione fit primo metrum, deinde flexa. In conclusione vero prius flexa, deinde metrum. Metrum in Oratione fit plerumque ubi in textu habetur duplex punctum; flexa, ubi habetur punctum cum virgula, vel si non adsit, ad primam virgulam post metrum ubi permittit sensus; secus, omittitur.

In conclusione \textit{Qui vivis} vel \textit{Qui tecum vivit,} fit solummodo metrum.

Advertendum est verba \textit{Jesum Christum Fílium tuum,} aliquando in fine Orationis posita, pertinere ad corpus orationis, ut in Festo et in Octava S. Stephani. Conclusio tunc incipit ad verba \textit{Qui lecum.}

(Diligenter notandum est, in hoc Vesperali, signa †, *, Orationibus interjecta, non posita esse pro tono festivo supra descripto, sed pro \textit{tonis antiquis ad libitum} qui infra traduntur. Porro præfata signa etsi aliquando, non semper concordant cum divisionibus istius toni festivi.)

\smalltitle{Tonus ferialis.}

A) Diebus supra memoratis ad Horas minores, in Festis Simplicibus et Feriis ad omnes Horas, Orationes cantantur in tono, ut aiunt, \textit{Feriali;} hoc est: recta voce a principia ad finem, solummodo sustentando tenorem ubi alias fieret metrum et flexa, et in fine.

B)  Est etiam in usu alius tonus ferialis, qui assignatur pro Orationibus positis in fine Psalterii post Antiphonas B.M.V. et pro Officio Defunctorum quando cum minori clausula dicuntur. Inservit etiam pro Orationibus Litaniarum.

Hic alter tonus in omnibus convenit cum prima tono feriali (A), præterquam quod in fine Orationis, et conclusionis, fit punctum per semiditonum

\smalltitle{Tonus antiqui ad libitum.}


Sed olim pro Orationibus duplex usurpabatur tonus, quorum usus in quibusdam Ecclesiis, et apud veteres Ordines perseverat cum variationibus quæ ad essentiam non pertinent: unus, \textit{solemnis} dictus; alter, \textit{simplex.}

Tonus \textit{solemnis} adhibetur pro Oratione principali, pro Orationibus Suffragiorum et Commemorationum in Vesperis (idque totum sine distinctione ritus festivi vel ferialis.

Tonus \textit{simplex} inservit pro Oratione diei ad Horas minore, pro Oratione post Antiphonam B.M.V. in fine Officii, et ceteris Orationibus.

%%Solesmes text
%%Tonus \textit{simplex} inservit pro Oratione diei ad Horas minores, pro Oratione post Antiphonam B.M.V. in fine Officii, et ceteris Orationibus, vel in Aspersione, aut in Benedictionibus, Litaniis, et in quibuscumque Functionibus dicendis (præter illas quas præcedit monitio Fleddmus genua, ut supra).

Adhibetur etiam in Officio Defunctorum, etiam ad Vesperas, in Exsequiis et Absolutionibus (non autem in Missa).

\smalltitle{Tonus solemnis.}

\grechangenextscorelinedim{10,12,13}{spacebeneathtext}{0.13cm}{scalable}
\gscore[]{}{or_tonus_antiqui_solemnis}

In i\-psa Oratione fit fiexa tantum, in fine primæ distinctionis.

Post flexam, et post pausam quamlibet, tenor non statim, sed mediante unius toni intervallo, resumi debet.

Si Oratio sit solito longior, ut Oratio \textit{A cun\-ctis, Omnípotens sempitérne Deus} punctum fieri potest in i\-pso corpore Orationis semel vel pluries, prout fert textus, sed ita ut inter punctum et punctum fiat semper flexa.

Sic pro \textit{Dóminus vobiscum} dicendus sit \vv \textit{Dómine exáudi,} sic cantatur:

\smallscore{or_tonus_domine_exaudi}

\smalltitle{Tonus simplex.}

\gscore[]{}{or_toni_antiqui_tonus_simplex_solesmes}

%%to insert examples of flex etc.

Punctum in fine Orationis ante conclusionem fit per semiditonum vel per diapente, juxta receptum usum. Ante \rr \textit{Amen,} punctum semper fieri debet in semiditono, etiam si desit conclusio proprie dicta, ut in Oratione \textit{Deus qui salútis ætérnæ,} quando post Ant.\@ \textit{Alma redemptóris} dicitur.

Flexa regulariter fieri debet in fine primæ distinctionis; omittitur tantum quando Oratio est brevior. Metrum numquam omittendum est.

In Orationibus quæ longiores sunt, ut in Benedictionibus solemnibus, et in Pontificalibus Functionibus, alternantur flexa et metrum. Si vero textus in plures periodos dividatur, in fine cujusque periodi fit punctum ut in fine Orationis.

\rubrique{Nota in hoc Antiphonario Orationes his signis muniri quæ tonis antiquis conveniunt. Scilicet ad signum †, debet fieri flexa tonis simplicis; ad*, metrum. Si deest signum †, locus non est flexæ, quæ omittenda est.}

\rubrique{In tono solemni, flexa fit ad signum †; ad signum *, pausa tantum. Si desit, flexa fit ad *.}

\rubrique{Orationes quæ in Vesperis tantum dicendæ sunt (Orationes Suffragiorum et Vesperarum in Feriis Quadragesimæ) unicum exhibent signum *, pro flexa toni solemnis, in quo sunt cantandæ.}

\section[VI. In fine Horarum]{VI. IN FINE HORARUM.}\cheader{vi. in fine horarum.}

In fine Completorii, dicta Oratione, et post Orationem repetito \vv \textit{Dóminus vobíscum} \rr \textit{Et cum spíritu tuo,} dicitur:
\smallscore{BD_ad_completorium}

Post \vv \textit{Benedicámus Dómino,} Benedictio \textit{Benedícat et custódiat} dicitur ab eo qui præest, recta quidem, sed gravi et protracta voce.

Tunc dicitur Antiphona B.M.V. pro tempore cum \vv et Oratione in altero tono feriali (vel antiquo simplici), si sit cantanda. Deinde, voce depressa et recta ut supra:

\vv Divínum auxílium máneat semper nobíscum. \rr Amen.

Isti Versiculi \textit{Fidélium ánimæ, Dóminus det nobis, Divínum auxílium} dicuntur eodem modo ad Vesperas (si tunc dicendi sunt), post \vv \textit{Benedicámus Dómino} decantatum in tono competenti, ut infra.

\section[VII. Toni ℣. Benedicamus Domino]{VII. TONI ℣. BENEDICAMUS DOMINO.}\cheader{vii. toni ℣. benedicamus domino.}

%\smalltitle{Ad Laudes et Vesperas.}

\textes{BD_rubrique}

\smalltitle{In Festibus Solemnibus.} %% this is bigger.

\smalltitle{In I Vesperis.} %% this is smaller %% also this is in Roman text, then switches to italics like in the AM1934.
% possibly need a vspace here

\gscore[]{2.}{BD_Solemnis_I_Vesp}

%\smalltitle{ Ad Laudes.}
%
%\gscore[]{5.}{BD_Solemnis_Laudes}

\smalltitle{In II Vesperis.}

\gscore[]{6.}{BD_Solemnis_mode_6}
% the rubric contained as alt is too high if they're on the same page.

\gscore[]{5.}{BD_Solemnis_mode_5}

\newpage
\smalltitle{In Festibus Duplicibus.}

\smalltitle{In I Vesperis.} %% switches to italics in LA1949 and in AM1934, it's all italicized.

\gscore[]{2.}{BD_Duplex_I_Vesp}

%\smalltitle{ Ad Laudes.}
% %% \smalltitle{In Laudibus.} 
% 
% \gscore[]{5.}{BD_Duplex_Laudes}

\smalltitle{In II Vesperis.}

\gscore[]{8.}{BD_Duplex_II_Vesp}
%this alignment needs work
\smalltitle{In Festibus Semiduplicibus. \\ In Vigilia Epiphaniæ, in Dominicis infra Octavas Nativitatis et Corporis Christi, et diebus infra Octavas quæ non sunt de B.M.V.\\ (præter Octavas Paschæ, Ascensionis et Pentecostes)} %%the seconds lines are all smaller headings. %% \\ should be considered a placeholder, this is not good syntax

%\smalltitle{ Ad Laudes.}
%
%\gscore[]{1.}{BD_Semiduplex_Laudes}

\smalltitle{In utrisque Vesperis.}

\gscore[]{2.}{BD_Semiduplex_Vesp}

\smalltitle{In Festibus Beatæ Mariæ Virginis.}

\textes{tc_BMV_rubrique}

\gscore[]{1.}{BD_BVM}

\smalltitle{In Festis Simplicibus.}

\gscore[]{1.}{BD_Simplex}

\smalltitle{In Officium de B.M.V. in Sabbato.}

\gscore[]{1.}{BD_BMV_in_sabbato}

\smalltitle{In Dominicis per annum \\ et in Dom. Septuagesimæ, Sexagesimæ et Quinquagesimæ.}
\textes{tc_dominica_rubrique}

\gscore[]{1.}{BD_per_annum}

\smalltitle{In Dominicis Adventus et Quadragesimæ.}

\textes{tc_dominica_rubrique}

\gscore[]{6.}{BD_Dominicis_Adventus_Quadragesimae}

\smalltitle{In Dominicis et in Feriis Temporis Paschalis\\ (Quando Officium fit de Tempore.)} %% see page 1284 of IA edition of Antiphonale Monasticum

\gscore[]{7.}{BD_TP}

\smalltitle{In Feriis.}

\gscore[]{4.}{BD_In_Feriis}

\textes{tc_ascension_rubrique} %% need to replace seq. with full word./

\section[VIII. De Cantu Hymnorum]{VIII. DE CANTU HYMNORUM.}\cheader{viii. de cantu hymnorum.}

1. Hymni semper cantandi sunt in tono assignato, vel in uno ex assignatis si plures ponuntur ad libitum. Excipe tantum Tempus a Nativitate ad Epiphaniam decurrens, et Tempus Paschale usque ad Pentecosten: in quibus omnes Hymni (ejusdem metri) etiam in Officio Sanctorum cantandi sunt in tonis pro tempore propriis, ut in suis locis ponuntur, nisi aliter notetur.

2. Hymnus Completorii cantatur per annum in uno ex tribus tonis conmunibus in Psalterio positis, pro qualitate diei vel Festi, nisi aliter notetur.

Ratione vero Temporis currentis, i.e. Adventus, Nativitatis Domini, Quadragesimæ, Passionis, Paschalis, cantatur in tonis pro tempore assignatis, etiam in Festis occurrentibus, nisi aliter notetur.

In Festis qua habent doxologiam propriam, tonus specialis assignatur pro Completorio. Ad Vesperas Festorum qua occurrunt, tonus solitus servatur, etiam si mutanda sit doxologia.

\textes{decretum_hymni}

In fine Hymnorum sic cantatur \textit{Amen} per octo Tonos.

\altnormal

\smallscore{amen_per_8_tonos_tc}

\newpage
\section[IX. Toni « Alleluia »]{IX. TONI ALLELUIA.}\cheader{ix. toni « alleluia ».}

\smalltitle{quomodo sit cantandurn tempore Paschali per octo tonos, nisi jam aliter notetur.}

\smalltitle{In fine Antiphonarum.}

\smallscore{alleluia_per_8_tonos_tc}

\altitshape

\end{document}